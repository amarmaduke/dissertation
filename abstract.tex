\begin{abstract}

Cedille is a dependently typed programming language with a significant history of encodeable features.
All of this originating from a small type theory that extends the Calculus of Constructions with just three new features: erased function spaces, dependent intersections, and untyped equality.
However, the last extension, untyped equality, is a source of undecidability.
Perhaps worse from the perspective of modern type theory research, the untyped nature of equality is responsible for refuting function extensionality.
This work attempts to refine the system of Cedille by specifically modifying the equality type into a new system: $\ced$.
A philosophy borrowed from proof theory is applied to the design of Cedille, where equality is modelled after a standard identity type.
This philosophy and design result in a partial success: a notion of proof reduction is strongly normalizing and supports all expected properties.
Moreover, the system $\ced$ is shown to be consistent by a syntactic translation into the Calculus of Dependent Lambda Eliminations, the core type theory of Cedille.

Unfortunately, object reduction, a critical component of $\ced$ required for deciding conversion, is not strongly normalizing.
Caused by the cast rule, casts are the last bastion of undecidability within the type theory.
Thus, with the cast rule removed, the system $\ced$ is a proof theory in the spirit of Kreisel and Gentzen.
The cast rule is, however, critical to the expressiveness of Cedille.
Its benefits outweigh the costs.
Thus, an external condition delineating when casts admit strong object normalization is described.
Therefore, the full type theory $\ced$ is a proof theory relative to an oracle deciding this condition.

\end{abstract}