
\begin{figure}
    \centering
    \begin{align*}
        t &::= x_K\ |\ \mathfrak{b}(\kappa_1, x : t_1, t_2)\ |\ \mathfrak{c}(\kappa_2, t_1, \ldots, t_{\mathfrak{a}(\kappa_2)}) \\
        \kappa_1 &::= \lambda_m\ |\ \Pi_m\ |\ \cap \\
        \kappa_2 &::=\diamond\ |\ \star\ |\ \kind\ | \bullet_m |\ \text{pair}\ |\ \text{proj}_1\ |\ \text{proj}_2\ |\ \text{eq}\ |\ \text{refl}\ |\ \psi\ |\ \vartheta\ |\ \delta\ |\ \varphi  \\
        m &::= \omega\ |\ 0\ |\ \tau
    \end{align*}
    \vspace{-.3in}
    \begin{minipage}{0.5\textwidth}
        \begin{align*}
            &\mathfrak{a}(\diamond) = \mathfrak{a}(\star) = \mathfrak{a}(\kind) = 0 \\
            &\mathfrak{a}(\text{proj}_1) = \mathfrak{a}(\text{proj}_2) = \mathfrak{a}(\delta) = 1 \\
            &\mathfrak{a}(\bullet_m) = \mathfrak{a}(\text{refl}) = 2
        \end{align*}
    \end{minipage}%
    \begin{minipage}{0.5\textwidth}
        \begin{align*}
            &\mathfrak{a}(\text{pair}) = \mathfrak{a}(\text{eq}) = \mathfrak{a}(\varphi) = 3 \\
            &\mathfrak{a}(\vartheta) = 4 \\
            &\mathfrak{a}(\psi) =  5
        \end{align*}
    \end{minipage}
    \begin{minipage}{0.5\textwidth}
        \begin{align*}
            \diamond &:= \mathfrak{c}(\diamond) \\
            \star &:= \mathfrak{c}(\star) \\
            \kind &:= \mathfrak{c}(\kind) \\
            \abs{\lambda_m}{x}{A}{t} &:= \mathfrak{b}(\lambda_m, x : A, t) \\
            (x : A) \to_m B &:= \mathfrak{b}(\Pi_m, x : A, B) \\
            (x : A) \cap B &:= \mathfrak{b}(\cap, x : A, B) \\
            f \bullet_m a &:= \mathfrak{c}(\bullet_m, f, a) \\
            \psi(e, a, b; A, P) &:= \mathfrak{c}(\psi, e, a, b, A, P)
        \end{align*}
    \end{minipage}%
    \begin{minipage}{0.5\textwidth}
        \begin{align*}
            [t_1, t_2; A] &:= \mathfrak{c}(\text{pair}, t_1, t_2, A) \\
            t.1 &:= \mathfrak{c}(\text{proj}_1, t) \\
            t.2 &:= \mathfrak{c}(\text{proj}_2, t) \\
            a =_{A} b &:= \mathfrak{c}(\text{eq}, a, A, b) \\
            \text{refl}(t; A) &:= \mathfrak{c}(\text{refl}, t, A) \\
            \vartheta(e, a, b; T) &:= \mathfrak{c}(\vartheta, e, a, b, T) \\
            \varphi(a, b, e) &:= \mathfrak{c}(\varphi, a, b, e, A, T) \\
            \delta(e) &:= \mathfrak{c}(\delta, e)
        \end{align*}
    \end{minipage}
    %$$J(t_1, t_2, t_3, t_4, t_5, t_6) := \mathfrak{c}(J, t_1, t_2, t_3, t_4, t_5, t_6)$$
    \caption{
        Generic syntax, there are three constructors, variables, a generic binder, and a generic non-binder.
        Each is parameterized with a constant tag to specialize to a particular syntactic construct.
        The non-binder constructor has a vector of subterms determined by an arity function computed on tags.
        Standard syntactic sugar is defined in terms of the generic forms.
    }
    \label{fig:syntax}
\end{figure}
