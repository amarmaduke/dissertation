

\begin{figure}
    \centering
    \begin{align*}
        \parp{\app{(\abs{\lambda_m}{x}{t_1}{t_2})}{m}{t_3}} &= [x := \parp{t_3}]\parp{t_2} \\
        \parp{\app{\psi(\text{refl}(t_1; t_2), t_3, t_4; t_5, t_6)}{\omega}{t_7}} &= \parp{t_7} \\
        \parp{[t_1, t_2; t_3].1} &= \parp{t_1} \\
        \parp{[t_1, t_2; t_3].2} &= \parp{t_2} \\
        \parp{\vartheta(\text{refl}(t_1; t_2), t_3, t_4; t_5)} &= \text{refl}(\parp{t_3}; \parp{t_5}) \\
        \parp{\mathfrak{c}(\kappa, t_1, \ldots, t_{\mathfrak{a}(\kappa)})} &= \mathfrak{c}(\kappa, \parp{t_1}, \ldots, \parp{t_{\mathfrak{a}(\kappa)}}) \\
        \parp{\mathfrak{b}(\kappa, (x : t_1), t_2)} &= \mathfrak{b}(\kappa, (x : \parp{t_1}), \parp{t_2}) \\
        \parp{x_K} &= x_K
    \end{align*}
    \caption{
        Definition of a reduction completion function $\parp{-}$ for parallel reduction.
        Note that this function is defined by pattern matching, applying cases from top to bottom.
        Thus, the cases at the very bottom are catch-all for when the prior cases are not applicable.
    }
    \label{fig:par-triangle}
\end{figure}
