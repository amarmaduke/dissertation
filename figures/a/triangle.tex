

\begin{figure}
    \centering
    \begin{align*}
        \parp{\app{(\abs{\lambda_m}{x}{t_1}{t_2})}{m}{t_3}} &= [x := \parp{t_3}]\parp{t_2} \\
        \parp{\psi(\text{refl}(t_1), t_2)} &= \abs{\lambda_\omega}{x}{\app{\parp{t_2}}{\tau}{\parp{t_1}}}{x} \\
        \parp{[t_1, t_2; t_3].1} &= \parp{t_1} \\
        \parp{[t_1, t_2; t_3].2} &= \parp{t_2} \\
        \parp{\vartheta_1(\text{refl}(t_1), t_2, t_3)} &= \text{refl}(\parp{t_2}) \\
        \parp{\vartheta_2(\text{refl}(t_1), t_2, t_3)} &= \text{refl}(\parp{t_2}) \\
        \parp{\mathfrak{c}(\kappa, t_1, \ldots, t_{\mathfrak{a}(\kappa)})} &= \mathfrak{c}(\kappa, \parp{t_1}, \ldots, \parp{t_{\mathfrak{a}(\kappa)}}) \\
        \parp{\mathfrak{b}(\kappa, (x : t_1), t_2)} &= \mathfrak{b}(\kappa, (x : \parp{t_1}), \parp{t_2}) \\
        \parp{x} &= x
    \end{align*}
    \caption{
        Definition of a reduction completion function $\parp{-}$ for parallel reduction.
        Note that this function is defined by pattern matching, applying cases from top to bottom.
        Thus, the cases at the very bottom are catch-all for when the prior cases are not applicable.
    }
\end{figure}
