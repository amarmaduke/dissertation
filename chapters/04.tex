\chapter{Consistency}

\Rule{\CedAxiomRule}
    {
        \textcolor{white}{\_}
    }
    {\Gamma \vdash \star}
    {}

\Rule{\CedTypeVarRule}
    {
        \der{\D{1}}{(x : \kappa) \in \Gamma}
    }
    {\Gamma \vdash x \infr \kappa}
    {}

\Rule{\CedTypeKindPiRule}
    {
        \der{\D{1}}{\Gamma \vdash A \infr \star}
        \\ \der{\D{2}}{\Gamma; x : A \vdash \kappa}
    }
    {\Gamma \vdash \abs{\Pi}{x}{A}{\kappa}}
    {}

\Rule{\CedKindKindPiRule}
    {
        \der{\D{1}}{\Gamma \vdash \kappa^\prime}
        \\ \der{\D{2}}{\Gamma; x : \kappa^\prime \vdash \kappa}
    }
    {\Gamma \vdash \abs{\Pi}{x}{\kappa^\prime}{\kappa}}
    {}

\Rule{\CedTypeTypePiRule}
    {
        \der{\D{1}}{\Gamma \vdash A \infr \star}
        \\ \der{\D{2}}{\Gamma; x : A \vdash B \infr \star}
    }
    {\Gamma \vdash \abs{\Pi}{x}{A}{B} \infr \star}
    {}

\Rule{\CedKindForallRule}
    {
        \der{\D{1}}{\Gamma \vdash \kappa}
        \\ \der{\D{2}}{\Gamma; x : \kappa \vdash B \infr \star}
    }
    {\Gamma \vdash \abs{\forall}{x}{\kappa}{B} \infr \star}
    {}

\Rule{\CedTypeForallRule}
    {
        \der{\D{1}}{\Gamma \vdash A \infr \star}
        \\ \der{\D{2}}{\Gamma; x : A \vdash B \infr \star}
    }
    {\Gamma \vdash \abs{\forall}{x}{A}{B} \infr \star}
    {}

\Rule{\CedKindTypeLambdaRule}
    {
        \der{\D{1}}{\Gamma \vdash \kappa^\prime}
        \\ \der{\D{2}}{\Gamma; x : \kappa^\prime \vdash t \infr \kappa}
    }
    {\Gamma \vdash \abs{\lambda}{x}{\kappa^\prime}{t} \infr \abs{\Pi}{x}{\kappa^\prime}{\kappa}}
    {}

\Rule{\CedTypeTypeLambdaRule}
    {
        \der{\D{1}}{\Gamma \vdash A \infr \star}
        \\ \der{\D{2}}{\Gamma; x : A \vdash t \infr \kappa}
    }
    {\Gamma \vdash \abs{\lambda}{x}{A}{t} \infr \abs{\Pi}{x}{A}{\kappa}}
    {}

\Rule{\CedTypeTypeApplyRule}
    {
        \der{\D{1}}{\Gamma \vdash f \infr \abs{\Pi}{x}{\kappa_1}{\kappa_2}}
        \\ \der{\D{2}}{\Gamma \vdash a \infr \kappa_1^\prime}
        \\ \der{\D{3}}{\kappa_1 \cong \kappa_1^\prime}
    }
    {\Gamma \vdash f \cdot a \infr [x := a]\kappa_2}
    {}

\Rule{\CedTermTypeApplyRule}
    {
        \der{\D{1}}{\Gamma \vdash f \infr \abs{\Pi}{x}{A}{\kappa}}
        \\ \der{\D{2}}{\Gamma \vdash a \chck A}
    }
    {\Gamma \vdash f\ a \infr [x := \chi\ A - a]\kappa}
    {}

\Rule{\CedIotaRule}
    {
        \der{\D{1}}{\Gamma \vdash A \infr \star}
        \\ \der{\D{2}}{\Gamma; x : A \vdash B \infr \star}
    }
    {\Gamma \vdash \abs{\iota}{x}{A}{B} \infr \star}
    {}

\Rule{\CedEqRule}
    {
        \der{\D{1}}{FV(a\ b) \subseteq dom(\Gamma)}
    }
    {\Gamma \vdash \{ a \simeq b \} \infr \star}
    {}

\Rule{\CedTermVarRule}
    {
        \der{\D{1}}{(x : A) \in \Gamma}
    }
    {\Gamma \vdash x \infr A}
    {}

\Rule{\CedTypeTermLambdaRule}
    {
        \der{\D{1}}{T \betastar_n \abs{\Pi}{x}{A}{B}}
        \\ \der{\D{2}}{\Gamma; x : A \vdash t \chck B}
    }
    {\Gamma \vdash \absu{\lambda}{x}{t} \chck T}
    {}

\Rule{\CedKindTermLambdaRule}
    {
        \der{\D{1}}{T \betastar_n \abs{\forall}{x}{\kappa}{B}}
        \\ \der{\D{2}}{\Gamma; x : \kappa \vdash t \chck B}
    }
    {\Gamma \vdash \absu{\Lambda}{x}{t} \chck T}
    {}

\Rule{\CedErasedLambdaRule}
    {
        \der{\D{1}}{T \betastar_n \abs{\forall}{x}{A}{B}}
        \\ \der{\D{2}}{\Gamma; x : A \vdash t \chck B}
        \\ \der{\D{3}}{x \notin FV(|t|)}
    }
    {\Gamma \vdash \absu{\Lambda}{x}{t} \chck T}
    {}

\Rule{\CedPairRule}
    {
        \der{\D{1}}{T \betastar_n \abs{\iota}{x}{A}{B}}
        \\ \der{\D{2}}{\Gamma \vdash t_1 \chck A}
        \\ \der{\D{3}}{\Gamma \vdash t_2 \chck [x := \chi\ A - t_1]B}
        \\ \der{\D{4}}{|t_1| \betaconv_\eta |t_2|}
    }
    {\Gamma \vdash [t_1, t_2] \chck T}
    {}

\Rule{\CedSecondProjectRule}
    {
        \der{\D{1}}{\Gamma \vdash t \cinfr \abs{\iota}{x}{A}{B}}
    }
    {\Gamma \vdash t.2 \infr [x := t.1]B}
    {}

\Rule{\CedDeltaRule}
    {
        \der{\D{1}}{\Gamma \vdash t \infr A}
        \\ \der{\D{2}}{A \cong \{ \absu{\lambda}{x\ y}{x} \simeq \absu{\lambda}{x\ y}{y} \}}
    }
    {\Gamma \vdash \delta - t \chck T}
    {}

\Rule{\CedChiRule}
    {
        \der{\D{1}}{\Gamma \vdash A \infr \star}
        \\ \der{\D{2}}{\Gamma \vdash t \chck A}
    }
    {\Gamma \vdash \chi\ A - t \infr A}
    {}

\Rule{\CedCheckRule}
    {
        \der{\D{1}}{\Gamma \vdash t \infr A}
        \\ \der{\D{2}}{A \cong B}
    }
    {\Gamma \vdash t \chck B}
    {}

\Rule{\CedTermTermApplyRule}
    {
        \der{\D{1}}{\Gamma \vdash f \cinfr \abs{\Pi}{x}{A}{B}}
        \\ \der{\D{2}}{\Gamma \vdash a \chck A}
    }
    {\Gamma \vdash f\ a \chck [x := \chi\ A - a]B}
    {}

\Rule{\CedTypeTermApplyRule}
    {
        \der{\D{1}}{\Gamma \vdash f \cinfr \abs{\Pi}{x}{\kappa_1}{B}}
        \\ \der{\D{2}}{\Gamma \vdash a \infr \kappa_2}
        \\ \der{\D{3}}{\kappa_1 \cong \kappa_2}
    }
    {\Gamma \vdash f \cdot a \chck [x := a]B}
    {}

\Rule{\CedErasedApplyRule}
    {
        \der{\D{1}}{\Gamma \vdash f \cinfr \abs{\forall}{x}{A}{B}}
        \\ \der{\D{2}}{\Gamma \vdash a \chck A}
    }
    {\Gamma \vdash f\ \edash a \chck [x := \chi\ A - a]B}
    {}

\Rule{\CedFirstProjectRule}
    {
        \der{\D{1}}{\Gamma \vdash f \cinfr \abs{\iota}{x}{A}{B}}
    }
    {\Gamma \vdash t.1 \infr A}
    {}

\Rule{\CedBetaRule}
    {
        \der{\D{1}}{T \betastar_n \{ a \simeq b \}}
        \\ \der{\D{2}}{FV(t) \subseteq dom(\Gamma)}
        \\ \der{\D{3}}{|a| \betaconv_\eta |b|}
    }
    {\Gamma \vdash \beta \{ t \} \chck T}
    {}

\Rule{\CedRewriteRule}
    {
        \der{\D{1}}{\Gamma \vdash e \cinfr \{ a \simeq b^\prime \}}
        \\ \der{\D{2}}{FV(b^\prime) \subseteq dom(\Gamma)}
        \\ \der{\D{3}}{|b^\prime| \betaconv_\eta |b|}
        \\ \der{\D{4}}{\Gamma \vdash [x := b]A \infr \star}
        \\ \der{\D{5}}{\Gamma \vdash t \chck [x := b]A }
        \\ \der{\D{6}}{[x := a]A \cong T}
    }
    {\Gamma \vdash \rho\ e\ @\ x\ \langle b \rangle.\ A - t \chck T}
    {}

\Rule{\CedPhiInferRule}
    {
        \der{\D{1}}{\Gamma \vdash e \chck \{ a \simeq b \}}
        \\ \der{\D{2}}{\Gamma \vdash a \infr A}
        \\ \der{\D{3}}{FV(b) \subseteq dom(\Gamma)}
    }
    {\Gamma \vdash \varphi\ e - a\ \{ b \} \infr A}
    {}

\Rule{\CedPhiCheckRule}
    {
        \der{\D{1}}{\Gamma \vdash e \chck \{ a \simeq b \}}
        \\ \der{\D{2}}{\Gamma \vdash a \chck A}
        \\ \der{\D{3}}{FV(b) \subseteq dom(\Gamma)}
    }
    {\Gamma \vdash \varphi\ e - a\ \{ b \} \chck A}
    {}

\Rule{\CedSymRule}
    {
        \der{\D{1}}{\Gamma \vdash e \cinfr \{ a \simeq b \}}
    }
    {\Gamma \vdash \text{\c{c}}\ e \infr \{ b \simeq a \}}
    {}

\Rule{\CedTypeCongRedRule}
    {
        \der{\D{1}}{A \betastar_n A^\prime \not\betared_n}
        \\ \der{\D{2}}{B \betastar_n B^\prime \not\betared_n}
        \\ \der{\D{2}}{A^\prime \cong^t B^\prime}
    }
    {A \cong B}
    {}

\Rule{\CedTypeCongVarRule}
    {
        \textcolor{white}{\_}
    }
    {x \cong^t x}
    {}

\Rule{\CedTypeCongTypeForallRule}
    {
        \der{\D{1}}{A_1 \cong A_2}
        \\ \der{\D{2}}{B_1 \cong B_2}
    }
    {\abs{\forall}{x}{A_1}{B_1} \cong^t \abs{\forall}{x}{A_2}{B_2}}
    {}

\Rule{\CedTypeCongTypeLambdaRule}
    {
        \der{\D{1}}{A_1 \cong A_2}
        \\ \der{\D{2}}{B_1 \cong B_2}
    }
    {\abs{\lambda}{x}{A_1}{B_1} \cong^t \abs{\lambda}{x}{A_2}{B_2}}
    {}

\Rule{\CedTypeCongIotaRule}
    {
        \der{\D{1}}{A_1 \cong A_2}
        \\ \der{\D{2}}{B_1 \cong B_2}
    }
    {\abs{\iota}{x}{A_1}{B_1} \cong^t \abs{\iota}{x}{A_2}{B_2}}
    {}

\Rule{\CedTypeCongTypeApplyRule}
    {
        \der{\D{1}}{A_1 \cong^t A_2}
        \\ \der{\D{2}}{B_1 \cong B_2}
    }
    {A_1 \cdot B_1 \cong^t A_2 \cdot B_2}
    {}

\Rule{\CedTypeCongKindForallRule}
    {
        \der{\D{1}}{\kappa_1 \cong \kappa_2}
        \\ \der{\D{2}}{B_1 \cong B_2}
    }
    {\abs{\forall}{x}{\kappa_1}{B_1} \cong^t \abs{\forall}{x}{\kappa_2}{B_2}}
    {}

\Rule{\CedTypeCongPiRule}
    {
        \der{\D{1}}{A_1 \cong A_2}
        \\ \der{\D{2}}{B_1 \cong B_2}
    }
    {\abs{\Pi}{x}{A_1}{B_1} \cong^t \abs{\Pi}{x}{A_2}{B_2}}
    {}
    
\Rule{\CedTypeCongKindLambdaRule}
    {
        \der{\D{1}}{\kappa_1 \cong \kappa_2}
        \\ \der{\D{2}}{B_1 \cong B_2}
    }
    {\abs{\lambda}{x}{\kappa_1}{B_1} \cong^t \abs{\lambda}{x}{\kappa_2}{B_2}}
    {}

\Rule{\CedTypeCongTermApplyRule}
    {
        \der{\D{1}}{A_1 \cong^t A_2}
        \\ \der{\D{2}}{|b_1| \betaconv_\eta |b_2|}
    }
    {A_1\ b_1 \cong^t A_2\ b_2}
    {}

\Rule{\CedTypeCongEqRule}
    {
        \der{\D{1}}{|a_1| \betaconv_\eta |a_2|}
        \\ \der{\D{2}}{|b_1| \betaconv_\eta |b_2|}
    }
    {\{ a_1 \simeq b_1 \} \cong^t \{ a_2 \simeq b_2 \}}
    {}

\Rule{\CedKindCongAxiomRule}
    {
        \textcolor{white}{\_}
    }
    {\star \cong \star}
    {}

\Rule{\CedKindCongTypePiRule}
    {
        \der{\D{1}}{A_1 \cong A_2}
        \\ \der{\D{2}}{\kappa_1 \cong \kappa_2}
    }
    {\abs{\Pi}{x}{A_1}{\kappa_1} \cong \abs{\Pi}{x}{A_2}{\kappa_2}}
    {}

\Rule{\CedKindCongKindPiRule}
    {
        \der{\D{1}}{\kappa^\prime_1 \cong \kappa^\prime_2}
        \\ \der{\D{2}}{\kappa_1 \cong \kappa_2}
    }
    {\abs{\Pi}{x}{\kappa^\prime_1}{\kappa_1} \cong \abs{\Pi}{x}{\kappa^\prime_2}{\kappa_2}}
    {}

\newcommand{\cBool}{\textnormal{Bool}}
\newcommand{\ctt}{\textnormal{tt}}
\newcommand{\cff}{\textnormal{ff}}

\newcommand{\cId}{\textnormal{Id}}
\newcommand{\crefl}{\textnormal{refl}}
\newcommand{\ctheta}{\textnormal{theta}}
\newcommand{\csubst}{\textnormal{subst}}
\newcommand{\cdelta}{\textnormal{delta}}
\newcommand{\cirrel}{\textnormal{irrel}}

The Calculus of Dependent Lambda Eliminations (abbreviated as CDLE or $\oldced$) was first introduced in 2017 \cite{stump2017_cdle} as the core system for the in progress Cedille tool.
At that time, CDLE included complicated machinery for lifting lambda terms to the type-level enabling some large eliminations.
Over the years, the core system for the Cedille tool was still referred to as CDLE as it evolved culminating in the current core system used in Cedille version 1.1.2 \cite{stump2021_cedillecore}.
The ideas leading to CDLE grew over time with work on efficient lambda encodings in total theories \cite{stump2016_encodings}; self-types for encodings \cite{fu2014self}; and experiments involving irrelevance \cite{sjoberg2012irrelevance,sjoberg2011equality}.
Ultimately, the modern version of CDLE, as presented in this chapter, is the culmination of these efforts.

CDLE is an affirmative answer to the question: is lambda-encoded data enough for a proof assistant?
While there may be other philosophical objections, Mendler-style encodings have been shown to be efficient and support course-of-values induction \cite{firsov2018_mendler,firsov2018course}.
The $\varphi$ construct, an idea borrowed from the direct computation rule of Nuprl \cite{allen2000}, enabled several encodings including efficient Mendler encodings.
A non-exhaustive list of the successes of CDLE include: quotient subtypes \cite{marmaduke2020quotients}; coinductive data \cite{jenkins2020efficient}; zero-cost constructor subtypes \cite{marmaduke2020_constructor_subtyping}; monotonic recursive types \cite{jenkins2021monotone}; simulated large eliminations \cite{jenkins2021_large_elim}; and inductive-inductive data \cite{marmaduke2023_indind}.

CDLE commits to impredicative (i.e. parametric in the sense of $F^\omega$) quantification.
With that in mind the well-studied reader may not be surprised at the power and versatility of CDLE.
However, taming impredicative quantification without losing logical consistency is a difficult task.
Indeed, this is why several proof assistants have discarded impredicative quantification or relegated it to a universe of propositions.
A core philosophy behind both CDLE and $\ced$ is to try a different path and embrace impredicative quantification.
To achieve that goal a realizability model was developed for CDLE to demonstrate logical consistency \cite{stump2021_cedillecore}.
This chapter will describe a model of $\ced$ in CDLE to prove consistency.

\section{Calculus of Dependent Lambda Eliminations}

CDLE is described using an intrinsic style where syntax is presented directly with the typing derivation.
However, erasure is still a crucial part of CDLE which gives it an extrinsic philosophy.
Whether a system is intrinsic or extrinsic is perhaps not an interesting distinction.
Technically, $\ced$ is described extrinsically because syntax is defined independently of the typing relation, but there is no essential reason for this choice.
Moreover, any intrinsic system necessarily admits a projection of its raw syntax which enables an extrinsic presentation.
It is better to think about these details via their philosophical import.
An intrinsic system wishes to say that raw syntax has no meaning, or at the very least no meaning the designer cares about.
Alternatively, an extrinsic system wishes to say that types are in some sense only annotations with the raw syntax being primary.


\begin{figure}
    \centering
    $$\CedAxiomRule \CedTypeKindPiRule \CedKindKindPiRule$$
    \caption{
        TODO
    }
    \label{fig:4:kind}
\end{figure}


\begin{figure}
    \centering
    \begin{minipage}{0.5\textwidth}
        $$\CedTypeVarRule$$
        $$\CedTypeForallRule$$
        $$\CedTypeTypeLambdaRule$$
        $$\CedTermTypeApplyRule$$
        $$\CedIotaRule$$
    \end{minipage}%
    \begin{minipage}{0.5\textwidth}
        $$\CedKindForallRule$$
        $$\CedTypeTypePiRule$$
        $$\CedKindTypeLambdaRule$$
        $$\CedTypeTypeApplyRule$$
        $$\CedEqRule$$
    \end{minipage}%
    \caption{
        TODO
    }
    \label{fig:4:type}
\end{figure}


\begin{figure}
    \centering
    $$\Gamma \vdash t \cinfr A \text{ iff } \exists\ T.\ (\Gamma \vdash t \infr T) \wedge (T \betastar_n A)$$
    \begin{minipage}{0.5\textwidth}
        $$\CedTermVarRule$$
        $$\CedTypeTermLambdaRule$$
        $$\CedKindTermLambdaRule$$
        $$\CedErasedLambdaRule$$
        $$\CedPairRule$$
        $$\CedSecondProjectRule$$
        $$\CedDeltaRule$$
        $$\CedChiRule$$
        $$\CedSymRule$$
    \end{minipage}%
    \begin{minipage}{0.5\textwidth}
        $$\CedCheckRule$$
        $$\CedTermTermApplyRule$$
        $$\CedTypeTermApplyRule$$
        $$\CedErasedApplyRule$$
        $$\CedFirstProjectRule$$
        $$\CedBetaRule$$
        $$\CedRewriteRule$$
        $$\CedPhiInferRule$$
        $$\CedPhiCheckRule$$
    \end{minipage}%
    \caption{
        TODO
    }
    \label{fig:4:term}
\end{figure}


As one might guess these philosophical positions are not entirely black and white.
For example, Pfenning demonstrates how both methods can be combined \cite{pfenning2008church}.
Cedille has been historically described as an extrinsic system.
The type theory $\ced$ might best be described as a \textit{combined} system, both intrinsic and extrinsic.
That is, a \textit{proof} has no meaning as just syntax, but an \textit{object} discards the extra information as mere annotations.

The CLDE type system kind formation rules are presented in Figure~\ref{fig:4:kind}, type formation rules in Figure~\ref{fig:4:type}, and term annotation rules in Figure~\ref{fig:4:term}.
Lowercase letters are used to refer to metavariables of terms, uppercase letters for metavariables of types, and variations of $\kappa$ for metavariables of kinds.
Call-by-name reduction of the $\lambda$-calculus fragment is used in the rules for types and is written $A \betared_n B$.
The purpose of this relation is only to reveal a constructor for a type, thus weak-head normal form is sufficient.
Conversion for types is presented in Figure~\ref{fig:4:type_cong} and kind conversion in Figure~\ref{fig:4:kind_cong}.
Note that these conversion relations correspond to $\beta$-conversion for types and kinds.
Finally, erasure of terms is presented in Figure~\ref{fig:4:erasure}.
Erasure is only meaningful for terms in CDLE unlike in $\ced$ where it is defined for all syntax.

The presentation in this work deviates from other descriptions of CDLE by adding a symmetry rule for equality (\c{c}).
This rule is admissible using the rewrite rule ($\rho$), but it is convenient to have available for the model.
Otherwise, the presentation is identical to the one by Stump and Jenkins \cite{stump2021_cedillecore}.

\begin{figure}
    \centering
    $$\CedTypeCongRedRule$$
    \begin{minipage}{0.5\textwidth}
        $$\CedTypeCongVarRule$$
        $$\CedTypeCongTypeForallRule$$
        $$\CedTypeCongTypeLambdaRule$$
        $$\CedTypeCongIotaRule$$
        $$\CedTypeCongTypeApplyRule$$
    \end{minipage}%
    \begin{minipage}{0.5\textwidth}
        $$\CedTypeCongKindForallRule$$
        $$\CedTypeCongPiRule$$
        $$\CedTypeCongKindLambdaRule$$
        $$\CedTypeCongTermApplyRule$$
        $$\CedTypeCongEqRule$$
    \end{minipage}%
    \caption{
        Definition of conversion for types in CDLE.
    }
    \label{fig:4:type_cong}
\end{figure}

\begin{figure}
    \centering
    $$\CedKindCongAxiomRule$$
    \begin{minipage}{0.5\textwidth}
        $$\CedKindCongTypePiRule$$
    \end{minipage}%
    \begin{minipage}{0.5\textwidth}
        $$\CedKindCongKindPiRule$$
    \end{minipage}%
    \caption{
        Defintion of conversion for kinds in CDLE.
    }
    \label{fig:4:kind_cong}
\end{figure}



\begin{figure}
    \centering
    \begin{minipage}{0.5\textwidth}
        \begin{align*}
            |x| &= x \\
            |f\ a| &= |f|\ |a| \\
            |\absu{\Lambda}{x}{t}| &= |t| \\
            |[t_1, t_2]| &= |t_1| \\
            |t.2| &= |t| \\
            |\delta - t| &= \absu{\lambda}{x}{x} \\
            |\varphi\ e - a\ \{ b \}| &= |b|
        \end{align*}
    \end{minipage}%
    \begin{minipage}{0.5\textwidth}
        \begin{align*}
            |\absu{\lambda}{x}{t}| &= \absu{\lambda}{x}{|t|} \\
            |f \cdot a| &= |f| \\
            |f\ \edash a| &= |f| \\
            |t.1| &= |t| \\
            |\beta \{ t \}| &= |t| \\
            |\rho\ e\ @\ x\ \langle a \rangle.\ A - t| &= |t| \\
            |\chi\ A - t| &= |t|
        \end{align*}
    \end{minipage}%
    \caption{
        TODO
    }
    \label{fig:4:erasure}
\end{figure}

A few useful facts about CDLE are needed before defining the model.
First, some helpful terms are defined below.
Note that an annotation rule ($\chi$) is added to guarantee that each definition always infers a type.
The $\cBool$ definition is a standard Church encoded boolean type, with its two associated values: $\ctt$ and $\cff$.
An identity type, $\cId$, is defined as a desired output of the model for the equality of $\ced$.
Indeed, CDLEs equality is very flexible in comparison to $\ced$.
Not only is it untyped, but it allows for any well-scoped term to serve as the erasure (or object) of a reflexivity proof.
\begin{align*}
    \cBool &:= \abs{\forall}{X}{\star}{X \to X \to X} \\
    \ctt &:= \chi\ \cBool\ - \absu{\Lambda}{X}{
        \absu{\lambda}{x\ y}{x}
    } \\
    \cff &:= \chi\ \cBool\ - \absu{\Lambda}{X}{
        \absu{\lambda}{x\ y}{y}
    } \\
    \cId &:= \abs{\lambda}{A}{\star}{
        \abs{\lambda}{a\ b}{A}{
            \abs{\iota}{e}{\{ a \simeq b \}}{
                \abs{\iota}{y}{\{ (\absu{\lambda}{x}{x}) \simeq e \}}{
                    \abs{\forall}{X}{\star}{
                        X \to X
                    }
                }
            }
        }
    } \\
    \crefl &:= \chi\ \abs{\forall}{A}{\star}{
            \abs{\forall}{a}{A}{
                \cId \cdot A\ a\ a
            }
        }\ -
        \\ &\absu{\Lambda}{A\ a}{
        [\beta\{ \absu{\lambda}{x}{x} \}, [\beta\{ \absu{\lambda}{x}{x} \}, \absu{\Lambda}{X}{\absu{\lambda}{x}{x}}]]
    } \\
    \cdelta &:= \chi\ \cId \cdot \cBool\ \ctt\ \cff \to \abs{\forall}{X}{\star}{X}\ -
        \\ &\absu{\lambda}{e}{
        (\delta - e.1) \cdot (\cId \cdot \cBool\ \ctt\ \cff \to \abs{\forall}{X}{\star}{X})\ e
    }
\end{align*}
Aside from the previous terms it is also useful to have terms representing the target output of the substitution and promotion rules of $\ced$.
All of these terms are constructed to obtain specific erasures.
\begin{align*}
    \ctheta &:= \chi\ 
        \abs{\forall}{A}{\star}{
            \abs{\forall}{B}{A \to \star}{
                \abs{\forall}{a\ b}{(\abs{\iota}{x}{A}{B\ x})}{
                    \\ &\cId \cdot A\ a.1\ b.1 \to \cId \cdot (\abs{\iota}{x}{A}{B\ x})\ a\ b
                }
            }
        }\ -
        \\ &\absu{\Lambda}{A\ B\ a\ b}{
            \absu{\lambda}{e}{
                \\ &\varphi\ (\rho\ e.2.1\ @\ x\ \langle e \rangle.\ \{ x \cong e \}\ -\ \beta\{ \absu{\lambda}{x}{x} \})\ -
                \\ &(\rho\ e.1\ @\ x\ \langle b \rangle.\ \cId \cdot (\abs{\iota}{x}{A}{B\ x}))\ x\ b\ -\ \crefl \cdot (\abs{\iota}{x}{A}{B\ x})\ \edash b)
                \\ &\{ e \} 
            }
        }
\end{align*}
\begin{align*}
    \csubst &:= \chi\ 
        \abs{\forall}{A}{\star}{
            \abs{\forall}{a\ b}{A}{
                \abs{\forall}{P}{ (\abs{\Pi}{y}{A}{\cId \cdot A\ a\ y \to \star}) }{
                    \\ &\abs{\Pi}{e}{\cId \cdot A\ a\ b}{
                        P\ a\ (\crefl \cdot A\ \edash a) \to P\ b\ e
                    }
                }
            }
        }\ -
        \\ &\absu{\Lambda}{A\ a\ b\ P}{
            \absu{\lambda}{e}{
                \\ &\rho\ e.2.1\ @\ x\ \langle e \rangle.\ P\ a\ x \to P\ b\ e\ -\
                \\ &\rho\ e.1\ @\ x\ \langle b \rangle.\ P\ x\ e \to P\ b\ e\ -\
                \\ &e.2.2 \cdot (P\ b\ e)
            }
    }
\end{align*}
The erasure of each term is designed to match with the erasure of the associated construct in $\ced$.
While this might not be strictly necessary to obtain a model of $\ced$ inside CDLE it makes the process easier.
Moreover, carefully crafting terms with specific erasures is a trivial matter in CDLE because of the $\varphi$ rule.
\begin{align*}
    |\ctt| &= \absu{\lambda}{x\ y}{x} \\
    |\cff| &= \absu{\lambda}{x\ y}{y} \\
    |\crefl \cdot A\ \edash a| &= \absu{\lambda}{x}{x} \\
    |\cdelta\ e| &= |e| \\
    |\ctheta \cdot A \cdot B\ \edash a\ \edash b\ e| &= |e| \\
    |\csubst \cdot A\ \edash a\ \edash b \cdot P\ e| &= |e|
\end{align*}
Finally, each of these terms is shown to infer the desired type.
Note that for syntax that is type-like, such as $\cId$ and $\cBool$, there is no type-checking rule, only an inference judgment.
Moreover, the $\chi$ rule only works with term-like syntax.
Thus, for these definitions more care is needed to infer the correct kind.
\begin{lemma}
    \label{lem:4:c1_stuff}
    \textcolor{white}{\_}
    \begin{enumerate}
        \item $\vdash_{\oldced} \cBool \infr \star$
        \item $\vdash_{\oldced} \ctt \infr \cBool$
        \item $\vdash_{\oldced} \cff \infr \cBool$
        \item $\vdash_{\oldced} \cId \infr \abs{\Pi}{A}{\star}{A \to A \to \star}$
        \item {
            $
                \vdash_{\oldced} \crefl \infr \abs{\forall}{A}{\star}{
                    \abs{\forall}{a}{A}{
                        \cId \cdot A\ a\ a
                    }
                }
            $
        }
        \item $\vdash_{\oldced} \cdelta \infr \cId \cdot \cBool\ \ctt\ \cff \to \abs{\forall}{X}{\star}{X}$
        \item {
            $\vdash_{\oldced} \ctheta \infr
            \begin{aligned}
                 &\abs{\forall}{A}{\star}{
                    \abs{\forall}{B}{A \to \star}{
                        \abs{\forall}{a\ b}{(\abs{\iota}{x}{A}{B\ x})}{
                            \\ &\cId \cdot A\ a.1\ b.1 \to \cId \cdot (\abs{\iota}{x}{A}{B\ x})\ a\ b
                        }
                    }
                }
            \end{aligned}
            $
        }
        \item {
            $\vdash_{\oldced} \csubst \infr 
            \begin{aligned}
                &\abs{\forall}{A}{\star}{
                    \abs{\forall}{a\ b}{A}{
                        \abs{\forall}{P}{ (\abs{\Pi}{y}{A}{\cId \cdot A\ a\ y \to \star}) }{
                            \\ &\abs{\Pi}{e}{\cId \cdot A\ a\ b}{
                                P\ a\ (\crefl \cdot A\ \edash a) \to P\ b\ e
                            }
                        }
                    }
                }
            \end{aligned}
            $
        }
    \end{enumerate}
\end{lemma}
\begin{proof}
    Straightforward by applying a short sequence of CDLE rules in each case.
    These inferences are trivially formalized in the Cedille tool.
\end{proof}

A small collection of additional lemmas about CDLE is needed to prove soundness of the model and presented next.
These lemmas are standard: weakening, symmetry of conversion, and transitivity of conversion.
The only real difficulty is the bidirectional presentation which requires stating the desired lemma for each variation of judgment and using mutual recursion in the proof.

\begin{lemma}
    \label{lem:4:c1_weakening}
    Suppose $\Gamma \vdash_{\oldced} T \infr K$ and $x$ fresh
    \begin{enumerate}
        \item If $t$ is a kind and $\Gamma, \Delta \vdash_{\oldced} t$ then $\Gamma, x : T, \Delta \vdash_{\oldced} t$
        \item If $t$ is a type and $\Gamma, \Delta \vdash_{\oldced} t \infr K$ then $\Gamma, x : T, \Delta \vdash_{\oldced} t \infr K$
        \item If $t$ is a term and $\Gamma, \Delta \vdash_{\oldced} t \infr A$ then $\Gamma, x : T, \Delta \vdash_{\oldced} t \infr A$
        \item If $t$ is a term and $\Gamma, \Delta \vdash_{\oldced} t \chck A$ then $\Gamma, x : T, \Delta \vdash_{\oldced} t \chck A$
    \end{enumerate}
\end{lemma}
\begin{proof}
    Straightforward by mutual recursion on the associated judgments.
\end{proof}

\begin{lemma}
    \label{lem:4:c1_sym}
    \textcolor{white}{\_}
    \begin{enumerate}
        \item If $a, b$ are terms and $|a| \betaconv_\eta |b|$ then $|b| \betaconv_\eta |a|$
        \item If $A, B$ are types and values and $A \cong^t B$ then $B \cong^t A$
        \item If $A, B$ are types and $A \cong B$ then $B \cong A$
        \item If $A, B$ are kinds and $A \cong B$ then $B \cong A$
    \end{enumerate}
\end{lemma}
\begin{proof}
    Note that \textit{1.} holds because $|a|$ and $|b|$ are untyped $\lambda$-calculus terms.
    For \textit{2.} through \textit{4.} mutual recursion and pattern match on $A$ is sufficient.
\end{proof}

\begin{lemma}
    \label{lem:4:c1_trans}
    \textcolor{white}{\_}
    \begin{enumerate}
        \item If $a, b, c$ are terms, $|a| \betaconv_\eta |b|$, and $|b| \betaconv_\eta |c|$ then $|a| \betaconv_\eta |c|$
        \item If $A, B, C$ are types and values, $A \cong^t B$, and $B \cong^t C$ then $A \cong^t C$
        \item If $A, B, C$ are types, $A \cong B$, and $B \cong C$ then $A \cong C$
        \item If $A, B, C$ are kinds, $A \cong B$, and $B \cong C$ then $A \cong C$
    \end{enumerate}
\end{lemma}
\begin{proof}
    Note that \textit{1.} holds because $|a|$ and $|b|$ are untyped $\lambda$-calculus terms and reduction is confluent.
    The remainder are proved by mutual recursion.
    Note that in \textit{3.} the types $A, B$, and $C$ are reduced using call-name to a weak-head normal form.
    In particular, this reduction strategy is deterministic, thus $B \betastar_n B^\prime$ for a unique $B^\prime$.
    This combined with using \textit{2.} is sufficient for the \textit{3.} case.
    The other two cases follow by pattern matching on $B$, inversion on the respective conversions, and applying the IH.
\end{proof}

\section{Counterexamples to Decidability of Type Checking in CDLE}

It is well-known that Cedille does not enjoy decidability of type checking.
However, it might not be clear exactly how this property fails.
Below is a series of formalized examples that will loop when attempting to check using the Cedille tool.
\begin{enumerate}
    \item {
        First, there is an obvious problem caused by equality and reflexivity witnesses being untyped $\lambda$-calculus terms.
        Under this regime $\Omega = (\absu{\lambda}{x}{x\ x}) (\absu{\lambda}{x}{x\ x})$ is easily inserted.
\begin{minted}{Lean}
    bad : { (λ x. x x) (λ x. x x) ≃ λ x. x } = β.

    omega : { λ x. x ≃ λ x. x } = β{(λ x. x x) (λ x. x x)}.
    bad : { omega ≃ λ x. x } = β.
\end{minted}
        This is an unsurprising consequence of the design of the CDLE equality type and certainly the least interesting instance of non-termination.
    }
    \item {
        To fix the previous case the equality could instead be \textit{annotated}.
        Note, this does not mean that equality is typed, but only that the indices and reflexivity witnesses must infer a type.
        With this change the rewrite rule ($\rho$) allows annotated substitutions.
        As it turns out, these kinds of rewrites also enable non-termination.
\begin{minted}{Lean}
    Id : Π A:★. Π B:★. A ➔ B ➔ ★ = λ A:★. λ B:★. λ x:A. λ y:B. {x ≃ y}.
    Unit : ★ = ∀ X:★. X ➔ X.
    unit : Unit = Λ X. λ x. x.
    self : Unit ➔ Unit = λ u. u u.
    False : ★ = ∀ X:★. X.
    bad : ∀ P:False ➔ ★. Π f:False. P f
    = Λ P. λ f. {e1 = f·(Id·False·(Unit ➔ Unit) f self)}
        - {e2 = f·(Id·False·False f (f·(False ➔ False) f))}
        - ρ e2 - ρ e1 - (f·(P f)).
\end{minted}
        The equality described in the example should be typed at False, but an equality casting False to $\textnormal{Unit} \to \textnormal{Unit}$ is used to generate $\Omega$ in the resulting predicate.
        Thus, when the tool attempts to check the convertibility of f with $\Omega$ the type checking algorithm loops.
    }
    \item {
        Even when equality is typed the rewrite rule can cause non-termination.
        For this example, it is necessary that there is some method to encode equality of types.
        In the formalization below a simulated large elimination on booleans is used.
        Note that the only way to construct this large elimination that is currently known is by using the $\varphi$ construct.
        However, any extension or feature that enables discussing type equalities would unlock this example (e.g. universe hierarchies with an equality type at each level).
\begin{minted}{Lean}
    False : ★ = ∀ X:★. X.
    Not : ★ ➔ ★ = λ A:★. A ➔ False.
    True : ★ = Not·False.
    self : True = λ f. f·(False ➔ False) f.
    Bool : ★ = ∀ X:★. X ➔ X ➔ X.
    tt : Bool = Λ X. λ x. λ y. x.
    ff : Bool = Λ X. λ x. λ y. y.
    Id : Π A:★. A ➔ A ➔ ★ = λ A:★. λ a:A. λ b:A. {b ≃ a}.
    subst : ∀ A:★. ∀ a:A. ∀ b:A. ∀ P:A ➔ ★. P a ➔ Id·A a b ➾ P b
    = Λ A. Λ a. Λ b. Λ P. λ p. Λ i. ρ i - p.
    
    elim : ★ ➔ ★ ➔ Bool ➔ ★
    = λ A:★. λ B:★. λ x:Bool. ι _:{x ≃ tt} ➾ A. {x ≃ ff} ➾ B.
    in1 : ∀ A:★. ∀ B:★. A ➔ elim·A·B tt
    = Λ A. Λ B. λ a. [Λ e. a
        , Λ e. {f:False = δ - e} - φ (f·{f ≃ a}) - (f·B) {a}].
    
    cast : ∀ A:★. ∀ B:★. ∀ a:Bool. ∀ b:Bool.
        Id·Bool a b ➾ elim·A·B a ➔ elim·A·B b
    = Λ A. Λ B. Λ a. Λ b. Λ e. λ p. subst·Bool -a -b ·(elim·A·B) p -e.
    omega : Not·(∀ a:Bool. ∀ b:Bool. Id·Bool a b)
    = λ x. (cast·True·False -tt -ff -(x -tt -ff) (in1 self)).2 -β.
    Omega : Not·(∀ a:Bool. ∀ b:Bool. Id·Bool a b)
    = λ x. self (omega x).
    bad : {Omega ≃ λ x. x} = β.
\end{minted}
        This example is a direct adaptation of Abel's work \cite{abel2020_normalization}.
        It depends on the elimination form of equality being irrelevant, an impredicative universe, and some method of discussing equality of types.
        This is the first of three examples of irrelevance in equality eliminators causing non-termination.
    }
    \item {
        The separation rule ($\delta$) is also irrelevant and can cause non-termination.
        Constructing an $\Omega$ is easy as $\delta$ erases to an identity function, thus applying two self terms directly to a $\delta$ immediately yields $\Omega$.
\begin{minted}{Lean}
    False : ★ = ∀ X:★. X.
    Unit : ★ = ∀ X:★. X ➔ X.
    self : Unit ➔ Unit = λ u. u u.
    Bool : ★ = ∀ X:★. X ➔ X ➔ X.
    tt : Bool = Λ X. λ x. λ y. x.
    ff : Bool = Λ X. λ x. λ y. y.
    Id : Π A:★. A ➔ A ➔ ★ = λ A:★. λ x:A. λ y:A. {y ≃ x}.
    omega : Id·Bool tt ff ➾ False
    = Λ e. (δ - e)·((Unit ➔ Unit) ➔ (Unit ➔ Unit) ➔ False) self self.
    bad : {omega ≃ λ x. x} = β.
\end{minted}
    }
    \item {
        Finally, the $\varphi$ construct is capable of giving a term a recursive type in an inconsistent context.
        Constructing $\Omega$ is a trivial consequence.
\begin{minted}{Lean}
    False : ★ = ∀ X:★. X.
    Unit : ★ = ∀ X:★. X ➔ X.
    self : Unit ➔ Unit = λ u. u u.
    b : False ➔ ι _:Unit ➔ Unit. Unit = λ f. [f·(Unit ➔ Unit), f·Unit].
    e : Π f:False. {b f ≃ self} = λ f. f·{b f ≃ self}.
    omega : False ➾ Unit = Λ f. self (φ (e f) - (b f).2 {self}).
    bad : {omega ≃ λ x. x} = β.
\end{minted}
    }
\end{enumerate}

\section{Model}

Figure~\ref{fig:4:model} describes the model of $\ced$ in CDLE.
This model is straightforward: abstractions to abstractions, applications to applications, pairs to pairs, etc.
The complicated part is the equality type, however all the necessary work to find suitable terms for the rules of equality was already completed above.
There is one hiccup involving the promotion ($\vartheta$) rule.
In order to have a fully applied $\ctheta$ it must be the case that the annotation for $\vartheta$ is an intersection type.
For proofs this will always be the case, but for arbitrary syntax it is not necessarily true.
To work around this a catch-all case is defined where the model only interprets the equality proof $e$.
This choice is largely arbitrary, but it is picked to make sure that one critical property is preserved: erasure.


\begin{figure}
    \centering
    \begin{minipage}{0.5\textwidth}
        \begin{align*}
            \sema{(x : A) \to_\tau B} &= \abs{\Pi}{x}{\sema{A}}{\sema{B}} \\
            \sema{(x : A) \to_\omega B} &= \abs{\Pi}{x}{\sema{A}}{\sema{B}} \\
            \sema{(x : A) \to_0 B} &= \abs{\forall}{x}{\sema{A}}{\sema{B}} \\
            \sema{\abs{\lambda_\tau}{x}{A}{t}} &= \abs{\lambda}{x}{\sema{A}}{\sema{t}} \\
            \sema{\abs{\lambda_\omega}{x}{A}{t}} &= \absu{\lambda}{x}{\sema{t}}\\
            \sema{\abs{\lambda_0}{x}{A}{t}} &= \absu{\Lambda}{x}{\sema{t}}
        \end{align*}
    \end{minipage}%
    \begin{minipage}{0.5\textwidth}
        \begin{align*}
            \sema{\star} &= \star \\
            \sema{x_K} &= x \\
            \sema{\app{f}{\tau}{a}} &= \sema{f}\ \sema{a} \quad&\text{if }& a\pterm \\
            \sema{\app{f}{\tau}{a}} &= \sema{f} \cdot \sema{a} \quad&\text{if }& a\ptype \\
            \sema{\app{f}{\omega}{a}} &= \sema{f}\ \sema{a} \\
            \sema{\app{f}{0}{a}} &= \sema{f}\ \edash \sema{a} \quad&\text{if }& a\pterm \\
            \sema{\app{f}{0}{a}} &= \sema{f} \cdot \sema{a} \quad&\text{if }& a\ptype
        \end{align*}
    \end{minipage}
    \begin{minipage}{0.5\textwidth}
        \begin{align*}
            \sema{(x : A) \cap B} &= \abs{\iota}{x}{\sema{A}}{\sema{B}} \\
            \sema{[t_1, t_2, A]} &= [\sema{t_1}, \sema{t_2}]
        \end{align*}
    \end{minipage}%
    \begin{minipage}{0.5\textwidth}
        \begin{align*}
            \sema{t.1} &= \sema{t}.1 \\
            \sema{t.2} &= \sema{t}.2
        \end{align*}
    \end{minipage}
    \begin{align*}
        \sema{a =_A b} &= \cId \cdot \sema{A}\ \sema{a}\ \sema{b} \\
        \sema{\text{refl}(t; A)} &= \crefl \cdot \sema{A}\ \edash \sema{t} \\
        \sema{\vartheta(e, a, b; (x : A) \cap B)} &= \ctheta \cdot \sema{A} \cdot \sema{B}\ \edash \sema{a}\ \edash \sema{b}\ \sema{e} \\
        \sema{\vartheta(e, a, b; T)} &= \sema{e} \\
        \sema{\psi(e, a, b; A, P)} &= \csubst  \cdot \sema{A}\ \edash \sema{a}\ \edash \sema{b} \cdot \sema{P}\ \sema{e} \\
        \sema{\varphi(a, b, e)} &= \varphi\ \text{\c{c}}\ \sema{e}.1 - \sema{b}\ \{ \sema{a} \} \\
        \sema{\delta(e)} &= \cdelta\ \sema{e}
    \end{align*}
    \begin{align*}
        \sema{\varepsilon} &= \varepsilon \\
        \sema{\Gamma, x : A} &= \sema{\Gamma}, x : \sema{A}
    \end{align*}
    \caption{
        Model definition interpreting $\ced$ in CDLE.
    }
    \label{fig:4:model}
\end{figure}



\begin{lemma}
    \label{lem:4:sema_erasure}
    If $t\pterm$ then $\sema{|t|} = |\sema{t}|$
\end{lemma}
\begin{proof}
    By induction on $t$ and inversion on $t\pterm$.
    The case of first projection and first equality promotion cases are omitted.

    $\text{Case: }t = x_\star$
    \begin{proofcase}
        Have $\sema{|x_\star|} = \sema{x_\star} = x$ and $|\sema{x_\star}| = |x| = x$, hence trivial.
    \end{proofcase}

    $\text{Case: }t = \abs{\lambda_0}{x}{A}{b}$
    \begin{proofcase}
        Have $\sema{|\abs{\lambda_0}{x}{A}{b}|} = \sema{|b|}$ and $|\sema{\abs{\lambda_0}{x}{A}{b}}| = |\absu{\Lambda}{x}{\sema{b}}| = |\sema{b}|$.
        Note that $b\pterm$, hence by the IH $\sema{|b|} = |\sema{b}|$.
    \end{proofcase}

    $\text{Case: }t = \abs{\lambda_\omega}{x}{A}{b}$
    \begin{proofcase}
        Have $\sema{|\abs{\lambda_\omega}{x}{A}{b}|} = \absu{\lambda}{x}{\sema{|b|}}$ and $|\sema{\abs{\lambda_\omega}{x}{A}{b}}| = |\absu{\lambda}{x}{\sema{b}}| = \absu{\lambda}{x}{|\sema{b}|}$.
        Note that $b\pterm$, hence by the IH $\sema{|b|} = |\sema{b}|$.
    \end{proofcase}

    $\text{Case: }t = \app{f}{0}{a}$
    \begin{proofcase}
        Have $\sema{|\app{f}{0}{a}|} = \sema{|f|}$ and $|\sema{\app{f}{0}{a}}| = |\sema{f}\ \edash \sema{a}| = |\sema{f}|$.
        Given $\app{f}{0}{a}\pterm$ it is always the case that $f\pterm$.
        Thus, by the IH $\sema{|f|} = |\sema{f}|$.
    \end{proofcase}

    $\text{Case: }t = \app{f}{\omega}{a}$
    \begin{proofcase}
        Have $\sema{|\app{f}{\omega}{a}|} = \sema{|f|}\ \sema{|a|}$ and $|\sema{\app{f}{\omega}{a}}| = |\sema{f}|\ |\sema{a}|$.
        Note that $f, a\pterm$ because the mode is $\omega$ there is no possibility of $a\ptype$.
        Hence, by the IH $\sema{|f|} = |\sema{f}|$ and $\sema{|a|} = |\sema{a}|$.
    \end{proofcase}

    $\text{Case: }t = [t_1, t_2; A]$
    \begin{proofcase}
        Have $\sema{|[t_1, t_2; A]|} = \sema{|t_1|}$ and $|\sema{[t_1, t_2; A]}| = |[\sema{t_1}, \sema{t_2}]| = |\sema{t_1}|$.
        By the IH applied to $t_1\pterm$: $\sema{|t_1|} = |\sema{t_1}|$.
    \end{proofcase}

    $\text{Case: }t = t.2$
    \begin{proofcase}
        Have $\sema{|t.2|} = \sema{|t|}$ and $|\sema{t.2}| = |\sema{t}.2| = |\sema{t}|$.
        By the IH applied to $t\pterm$: $\sema{|t|} = |\sema{t}|$.
    \end{proofcase}

    $\text{Case: }t = \prefl(a; A)$
    \begin{proofcase}
        Have $\sema{|\prefl(a; A)|} = \sema{\abs{\lambda}{x}{\diamond}{x_\star}} = \absu{\lambda}{x}{x}$ and $|\sema{\prefl(a; A)}| = |\crefl \cdot \sema{A}\ \edash \sema{a}| = \absu{\lambda}{x}{x}$.
    \end{proofcase}

    $\text{Case: }t = \vartheta(e, a, b; T)$
    \begin{proofcase}
        Have $\sema{|\vartheta(e, a, b; T)|} = \sema{|e|}$.
        Suppose $T = (x : A) \cap B$ then $|\sema{\vartheta(e, a, b; (x : A) \cap B)}| = |\ctheta \cdot \sema{A} \cdot \sema{B}\ \edash \sema{a}\ \edash \sema{b}\ \sema{e}| = |\sema{e}|$.
        Otherwise, $|\sema{\vartheta(e, a, b; T)}| = |\sema{e}|$.
        By the IH applied to $e\pterm$: $\sema{|e|} = |\sema{e}|$.
    \end{proofcase}

    $\text{Case: }t = \psi(e, a, b; A, P)$
    \begin{proofcase}
        Have $\sema{|\psi(e, a, b; A, P)|} = \sema{|e|}$ and $|\sema{\psi(e, a, b; A, P)}| = |\csubst \cdot \sema{A}\ \edash \sema{a}\ \edash \sema{b} \cdot \sema{P}\ \sema{e}| = |\sema{e}|$.
        By the IH applied to $e\pterm$: $\sema{|e|} = |\sema{e}|$.
    \end{proofcase}

    $\text{Case: }t = \varphi(a, b, e)$
    \begin{proofcase}
        Have $\sema{|\varphi(a, b, e)|} = \sema{|a|}$ and $|\sema{\varphi(a, b, e)}| = |\varphi\ \text{\c{c}}\ \sema{e}.1 - \sema{b}\ \{ \sema{a} \}| = |\sema{a}|$.
        By the IH applied to $a\pterm$: $\sema{|a|} = |\sema{a}|$.
    \end{proofcase}

    $\text{Case: }t = \delta(e)$
    \begin{proofcase}
        Have $\sema{|\delta(e)|} = \sema{|e|}$ and $|\sema{\delta(e)}| = |\cdelta\ \sema{e}| = |\sema{e}|$.
        By the IH applied to $e\pterm$: $\sema{|e|} = |\sema{e}|$.
    \end{proofcase}
\end{proof}

To obtain soundness we first need to know that conversion is preserved.
Luckily, because $\ced$ terms are closely matched with CDLE terms lemmas involving reduction can be precise.

\begin{lemma}
    \label{lem:4:sema_subst}
    $\sema{[x := b]t} = [x := \sema{b}]\sema{t}$
\end{lemma}
\begin{proof}
    Straightforward by induction on $t$, substitution is structural with the only exception being variables, but $\sema{x_K} = x$.
\end{proof}

\begin{lemma}
    \label{lem:4:sema_term_step}
    If $t\pterm$ and $|t| \betared t^\prime$ then $|\sema{t}| \betared \sema{t^\prime}$
\end{lemma}
\begin{proof}
    By induction on $t$ and inversion on $t\pterm$.
    The cases: erased lambda, pair, first projection, second projection, promotion ($\vartheta$), substitution ($\psi$), and separation ($\delta$) all erase to a subexpression that is a term.
    Hence, these cases are very similar to the erased application case and omitted.
    The erasure of the variable, reflexivity, and cast cases are values and thus do not reduce.

    $\text{Case: }t = \abs{\lambda_\omega}{x}{A}{b}$
    \begin{proofcase}
        Have $|\abs{\lambda_\omega}{x}{A}{b}| = \abs{\lambda_\omega}{x}{\diamond}{|b|}$ which means $\abs{\lambda_\omega}{x}{\diamond}{|b|} \betared \abs{\lambda_\omega}{x}{\diamond}{b^\prime}$.
        Now $b\pterm$ and $|b| \betared b^\prime$, applying the IH gives $|\sema{b}| \betared \sema{b^\prime}$.
        Note that $|\sema{\abs{\lambda_\omega}{x}{A}{b}}| = \absu{\lambda}{x}{|\sema{b}|} \betared \absu{\lambda}{x}{|\sema{b^\prime}|}$.
        By Lemma~\ref{lem:4:sema_subst}: $|\sema{b^\prime}| = \sema{|b^\prime|}$.
        However, $b^\prime$ is the result of a contracted redex in an already erased term, hence $|b^\prime| = b^\prime$.
        Thus, $|\sema{\abs{\lambda_\omega}{x}{A}{b}}| \betared \sema{\abs{\lambda_\omega}{x}{\diamond}{b^\prime}}$.
    \end{proofcase}

    $\text{Case: }t = \app{f}{0}{a}$
    \begin{proofcase}
        Have $|\app{f}{0}{a}| = |f|$, thus $|f| \betared t^\prime$.
        Applying the IH gives $|\sema{f}| \betared \sema{t^\prime}$.
        Note that $|\sema{\app{f}{0}{a}}| = |\sema{f}\ \edash \sema{a}| = |\sema{f}|$.
        Thus, $|\sema{\app{f}{0}{a}}| \betared \sema{t^\prime}$.
    \end{proofcase}

    $\text{Case: }t = \app{f}{\omega}{a}$
    \begin{proofcase}
        Have $|\app{f}{\omega}{a}| = \app{|f|}{\omega}{|a|}$.
        Suppose $|f| = \abs{\lambda_\omega}{x}{\diamond}{b}$ and $\app{|f|}{\omega}{|a|} \betared [x := |a|]b$.
        Now $|\sema{\app{f}{\omega}{a}}| = |\sema{f}|\ |\sema{a}|$.
        By Lemma~\ref{lem:4:sema_erasure}: $|\sema{f}| = \sema{|f|} = \absu{\lambda}{x}{\sema{b}}$.
        Thus, $(\absu{\lambda}{x}{\sema{b}})\ |\sema{a}| \betared [x := |\sema{a}|]\sema{b}$.
        Using Lemma~\ref{lem:4:sema_erasure} and Lemma~\ref{lem:4:sema_subst} gives $[x := |\sema{a}|]\sema{b} = \sema{[x := |a|]b}$.
        \\ \\
        Suppose w.l.o.g. that $|f| \betared f^\prime$ (the case of $|a| \betared a^\prime$ is very similar).
        Note that $f\pterm$, applying the IH gives $|\sema{f}| \betared \sema{f^\prime}$.
        Now $|\sema{\app{f}{\omega}{a}}| = |\sema{f}|\ |\sema{a}| \betared \sema{f^\prime}\ |\sema{a}| = \sema{\app{f^\prime}{\omega}{|a|}}$.
        The final equality uses Lemma~\ref{lem:4:sema_erasure}.
    \end{proofcase}
\end{proof}

\begin{lemma}
    \label{lem:4:sema_term_multistep}
    If $t\pterm$ and $|t| \betastar t^\prime$ then $|\sema{t}| \betastar \sema{t^\prime}$
\end{lemma}
\begin{proof}
    By induction on $|t| \betastar t^\prime$ using Lemma~\ref{lem:4:sema_term_step}, Lemma~\ref{lem:2:classify_preservation}, and Lemma~\ref{lem:2:classify_erase}.
\end{proof}

\begin{lemma}
    \label{lem:4:sema_term_conv}
    If $a, b\pterm$ and $|a| \betaconv |b|$ then $|\sema{a}| \betaconv |\sema{b}|$
\end{lemma}
\begin{proof}
    Deconstructing $|a| \betaconv |b|$ gives $|a| \betastar z$ and $|b| \betastar z$.
    Applying Lemma~\ref{lem:4:sema_term_multistep} gives $|\sema{a}| \betastar \sema{z}$ and $|\sema{b}| \betastar \sema{z}$.
    Thus, $|\sema{a}| \betaconv |\sema{b}|$.
\end{proof}

\begin{lemma}
    \label{lem:4:sema_type_step}
    If $s\ptype$ and $s \betared_n t$ then $\sema{s} \betared_n \sema{t}$
\end{lemma}
\begin{proof}
    By induction on $s$ and inversion on $s\ptype$.
    Note that only the case where $s$ is a redex is important as all other cases are in weak-head normal form.
    Thus, suppose $s = \app{f}{\tau}{a}$,  $f = \abs{\lambda_\tau}{x}{A}{b}$, and $\app{f}{\tau}{a} \betared_n [x := a]b$.
    Suppose w.l.o.g. that $a\pterm$.
    Now $\sema{\app{f}{\tau}{a}} = \sema{f}\ \sema{a} = (\abs{\lambda}{x}{\sema{A}}{\sema{b}})\ \sema{a} \betared [x := \sema{a}]\sema{b}$.
    Using Lemma~\ref{lem:4:sema_subst} gives $[x := \sema{a}]\sema{b} = \sema{[x := a]b}$.
\end{proof}

\begin{lemma}
    \label{lem:4:sema_type_multistep}
    If $s\ptype$ and $s \betastar_n t$ then $\sema{s} \betastar_n \sema{t}$
\end{lemma}
\begin{proof}
    By induction on $s \betastar_n t$ using Lemma~\ref{lem:4:sema_type_step} and Lemma~\ref{lem:2:classify_preservation}.
\end{proof}

\begin{lemma}
    \label{lem:4:sema_type_conv}
    \textcolor{white}{\_}
    \begin{enumerate}
        \item If $A, B\ptype$, $A\ B$ are values, and $A \equiv B$ then $\sema{A} \cong^t \sema{B}$
        \item If $A, B\ptype$ and $A \equiv B$ then $\sema{A} \cong \sema{B}$
        \item If $A, B\pkind$ and $A \equiv B$ then $\sema{A} \cong \sema{B}$
    \end{enumerate}
\end{lemma}
\begin{proof}
    By mutual recursion.

    \noindent \textbf{1.}
    By induction on $A$ and inversion on $A$ being a value and $A \equiv B$ (hence $B$ must match $A$).
    Conversion in $\oldced$ is structural over weak-head normal forms and in this case $A$ and $B$ must be weak-head normal.
    Thus, a combination of \textit{1.}, \textit{2.}, \textit{3.}, and Lemma~\ref{lem:4:sema_term_conv} on subexpressions in each case is sufficient.

    \noindent \textbf{2.}
    By Theorem~\ref{lem:3:proof_normalization}, $\exists\ A^\prime, B^\prime$ such that $A \betastar A^\prime$, $B \betastar B^\prime$ and $A^\prime, B^\prime$ are values.
    Lemma~\ref{lem:2:classify_preservation} gives that $A^\prime, B^\prime\ptype$.
    Lemma~\ref{lem:2:conv_red_f} gives that $A^\prime \equiv B^\prime$.
    Thus, applying \textit{1.} concludes.

    \noindent \textbf{3.}
    By induction on $A$ and inversion on $A \equiv B$.
    Again, conversion of kinds is structural in $\oldced$.
    Thus, a combination of \textit{2.} and \textit{3.} on subexpressions in each case is sufficient.
\end{proof}

\begin{theorem}[Soundness of $\sema{-}$]
    \label{lem:4:soudness}
    Suppose $\Gamma \vdash_{\ced} t : A$
    \begin{enumerate}
        \item if $A = \kind$ then $\sema{\Gamma} \vdash_{\oldced} \sema{t}$
        \item if $\Gamma \vdash_{\ced} A : \kind$ then $\sema{\Gamma} \vdash_{\oldced} \sema{t} \infr T$ and $T \cong \sema{A}$
        \item if $\Gamma \vdash_{\ced} A : \star$ then $\sema{\Gamma} \vdash_{\oldced} \sema{t} \chck \sema{A}$
    \end{enumerate}
\end{theorem}
\begin{proof}
    By induction on $\Gamma \vdash_{\ced} t : A$.
    Note that each case is mutually exclusive by classification.

    $\text{Case: }\begin{array}{c} \AxiomRule[*] \end{array}$
    \begin{proofcase}
        Have $A = \kind$ and $\Gamma \vdash_{\oldced} \star$, hence trivial.
    \end{proofcase}

    $\text{Case: }\begin{array}{c} \VarRule[*] \end{array}$
    \begin{proofcase}
        Let $\Gamma = \Gamma_1; x : A; \Gamma_2$.
        Have $(x : \sema{A}) \in \sema{\Gamma}$.
        Now $\sema{\Gamma_1} \vdash_{\oldced} x \infr \sema{A}$ by the IH and $\sema{\Gamma} \vdash_{\oldced} x \infr \sema{A}$ by Lemma~\ref{lem:4:c1_weakening}.
        Suppose $K = \kind$ then $\sema{A} \cong \sema{A}$ and $\sema{\Gamma} \vdash_{\oldced} x \infr \sema{A}$.
        Suppose $K = \star$ then $\sema{\Gamma} \vdash_{\oldced} x \chck \sema{A}$.
    \end{proofcase}

    $\text{Case: }\begin{array}{c} \PiRule[*] \end{array}$
    \begin{proofcase}
        Suppose $m = \tau$, then $\pdom(m, K) = K$ and $\pcodom(m) = \kind$.
        Applying the IH gives:
        \begin{enumerate}
            \item[$\D{1}$.] $\sema{\Gamma} \vdash_{\oldced} \sema{A}$ if $K = \kind$
            \item[$\D{1}$.] $\sema{\Gamma} \vdash_{\oldced} \sema{A} \infr \star$ if $K = \star$
            \item[$\D{2}$.] $\sema{\Gamma}, x : \sema{A} \vdash_{\oldced} \sema{B}$
        \end{enumerate}
        The corresponding $\Pi$ rule for the two possibilities of $K$ concludes the case.
        \\ \\
        Suppose $m = 0$, then $\pdom(m, K) = K$ and $\pcodom(m) = \star$.
        Applying the IH gives:
        \begin{enumerate}
            \item[$\D{1}$.] $\sema{\Gamma} \vdash_{\oldced} \sema{A}$ if $K = \kind$
            \item[$\D{1}$.] $\sema{\Gamma} \vdash_{\oldced} \sema{A} \infr \star$ if $K = \star$
            \item[$\D{2}$.] $\sema{\Gamma}, x : \sema{A} \vdash_{\oldced} \sema{B} \infr \star$
        \end{enumerate}
        The corresponding $\forall$ rule for the two possibilities of $K$ concludes the case.
        \\ \\
        Suppose $m = \omega$, then $\pdom(m, K) = \star$ and $\pcodom(m) = \star$.
        Applying the IH gives:
        \begin{enumerate}
            \item[$\D{1}$.] $\sema{\Gamma} \vdash_{\oldced} \sema{A} \infr \star$
            \item[$\D{2}$.] $\sema{\Gamma}, x : \sema{A} \vdash_{\oldced} \sema{B} \infr \star$
        \end{enumerate}
        The corresponding $\Pi$ rule concludes the case.
    \end{proofcase}

    $\text{Case: }\begin{array}{c} \LambdaRule[*] \end{array}$
    \begin{proofcase}
        Suppose $m = \tau$, then $\pcodom(m) = \kind$.
        Note that this means that $t\ptype$.
        Applying the IH gives:
        \begin{enumerate}
            \item[$\D{1}$.] $\sema{\Gamma} \vdash_{\oldced} \abs{\Pi}{x}{\sema{A}}{\sema{B}}$
            \item[$\D{2}$.] $\sema{\Gamma}, x : \sema{A} \vdash_{\oldced} \sema{t} \infr T$ and $T \cong \sema{B}$
        \end{enumerate}
        Suppose $\sema{\Gamma} \vdash_{\oldced} \sema{A}$, then $\sema{\Gamma} \vdash_{\oldced} \abs{\lambda}{x}{\sema{A}}{\sema{t}} \infr \abs{\Pi}{x}{\sema{A}}{T}$.
        By rules of conversion for kinds yields $\abs{\Pi}{x}{\sema{A}}{T} \cong \abs{\Pi}{x}{\sema{A}}{\sema{B}}$.
        The case where $\sema{A}$ is a type instead of a kind is similar.
        \\ \\
        Suppose $m = 0$, then $\pcodom(m) = \star$.
        Note that this means $t\pterm$.
        Applying the IH gives:
        \begin{enumerate}
            \item[$\D{1}$.] $\sema{\Gamma} \vdash_{\oldced} \abs{\Pi}{x}{\sema{A}}{\sema{B}} \infr \star$
            \item[$\D{2}$.] $\sema{\Gamma}, x : \sema{A} \vdash_{\oldced} \sema{t} \chck \sema{B}$
        \end{enumerate}
        Note that $FV(|\sema{t}|) \subseteq FV(|t|)$, thus $x \notin FV(|\sema{t}|)$.
        Using the corresponding $\Lambda$ rule based on the classification of $\sema{A}$ concludes the case.
        \\ \\
        Suppose $m = \omega$, then $\pcodom(m) = \star$.
        This case is omitted because the previous case is a more general version of it.
    \end{proofcase}

    $\text{Case: }\begin{array}{c} \AppRule[*] \end{array}$
    \begin{proofcase}
        Suppose $m = \tau$.
        Classification forces $f\ptype$, but $a$ is either a term or a type.
        Applying the IH gives:
        \begin{enumerate}
            \item[$\D{1}$.] $\sema{\Gamma} \vdash_{\oldced} \sema{f} \infr T$ with $T \cong \abs{\Pi}{x}{\sema{A}}{\sema{B}}$
            \item[$\D{2}$.] $\sema{\Gamma} \vdash_{\oldced} \sema{a} \infr T_2$ with $T_2 \cong \sema{A}$ if $a\ptype$
            \item[$\D{2}$.] $\sema{\Gamma} \vdash_{\oldced} \sema{a} \chck \sema{A}$ if $a\pterm$
        \end{enumerate}
        Note that because kinds cannot reduce, it must be the case that $\exists\ C, D$ such that $T = \abs{\Pi}{x}{C}{D}$.
        Moreover, $C \cong \sema{A}$ and $D \cong \sema{B}$ by the conversion rules.
        Suppose $a\ptype$ then using the associated rule yields $\sema{\Gamma} \vdash_{\oldced} \sema{f} \cdot \sema{a} \infr [x := \sema{a}]D$.
        Now, $[x := \sema{a}]D \cong [x := \sema{a}]\sema{B}$ and the case is concluded.
        Suppose $a\pterm$ then using the associated rule yields $\sema{\Gamma} \vdash_{\oldced} \sema{f}\ \sema{a} \infr [x := \chi\ C - \sema{a}]D$.
        Again, $[x := \chi\ C - \sema{a}]D \cong [x := \chi\ C - \sema{a}]\sema{B}$ and the case is concluded.
        Note that $[x := \chi\ C - \sema{a}]\sema{B} \cong [x := \sema{a}]\sema{B}$ because the $\chi$ is only well-typed in term positions, where it is promptly erased during conversion checking.
        \\ \\
        Suppose $m = 0$.
        Classification forces $f\pterm$, but $a$ is either a term or a type.
        Applying the IH gives:
        \begin{enumerate}
            \item[$\D{1}$.] $\sema{\Gamma} \vdash_{\oldced} \sema{f} \chck \abs{\forall}{x}{\sema{A}}{\sema{B}}$
            \item[$\D{2}$.] $\sema{\Gamma} \vdash_{\oldced} \sema{a} \infr T_2$ with $T_2 \cong \sema{A}$ if $a\ptype$
            \item[$\D{2}$.] $\sema{\Gamma} \vdash_{\oldced} \sema{a} \chck \sema{A}$ if $a\pterm$
        \end{enumerate}
        Deconstructing the checking judgment for $\sema{f}$ yields $\exists\ C, D$ such that $\sema{\Gamma} \vdash_{\oldced} \sema{f} \cinfr \abs{\forall}{x}{C}{D}$ and $C \cong \sema{A}$ and $D \cong \sema{B}$.
        Suppose $a\ptype$ then the associated judgment gives $\sema{\Gamma} \vdash_{\oldced} \sema{f} \cdot \sema{a} \infr [x := \sema{a}]D$.
        Now, $[x := \sema{a}]D \cong [x := \sema{a}]\sema{B}$ and the case is concluded.
        Suppose $a\pterm$ then the associated judgment gives $\sema{\Gamma} \vdash_{\oldced} \sema{f}\ \edash \sema{a} \infr [x := \chi\ C - \sema{a}]D$.
        Again, $[x := \chi\ C - \sema{a}]D \cong [x := \chi\ C - \sema{a}]\sema{B}$ and the case is concluded.
        \\ \\
        Suppose $m = \omega$
        Classification forces $f, a\pterm$.
        Applying the IH gives:
        \begin{enumerate}
            \item[$\D{1}$.] $\sema{\Gamma} \vdash_{\oldced} \sema{f} \chck \abs{\Pi}{x}{\sema{A}}{\sema{B}}$
            \item[$\D{2}$.] $\sema{\Gamma} \vdash_{\oldced} \sema{a} \chck \sema{A}$ if $a\pterm$
        \end{enumerate}
        As with the previous case, $\exists\ C, D$ such that $\sema{\Gamma} \vdash_{\oldced} \sema{f} \cinfr \abs{\Pi}{x}{C}{D}$ and $C \cong \sema{A}$ and $D \cong \sema{B}$.
        Applying the associated rule yields $\sema{\Gamma} \vdash_{\oldced} \sema{f}\ \sema{a} \infr [x := \chi\ C - \sema{a}]D$.
        Now, $[x := \chi\ C - \sema{a}]D \cong [x := \chi\ C - \sema{a}]\sema{B}$ and the case is concluded.
    \end{proofcase}

    $\text{Case: }\begin{array}{c} \IntersectionRule[*] \end{array}$
    \begin{proofcase}
        Applying the IH gives:
        \begin{enumerate}
            \item[$\D{1}$.] $\sema{\Gamma} \vdash_{\oldced} \sema{A} \infr \star$
            \item[$\D{2}$.] $\sema{\Gamma}, x : \sema{A} \vdash_{\oldced} \sema{B} \infr \star$
        \end{enumerate}
        Thus, $\sema{\Gamma} \vdash_{\oldced} \abs{\iota}{x}{\sema{A}}{\sema{B}} \infr \star$ as required.
    \end{proofcase}

    $\text{Case: }\begin{array}{c} \PairRule[*] \end{array}$
    \begin{proofcase}
        Note by classification and $\D{1}$: $\Gamma \vdash A : \star$ and $\Gamma, x : A \vdash B : \star$.
        Applying the IH gives:
        \begin{enumerate}
            \item[$\D{1}$.] $\sema{\Gamma} \vdash_{\oldced} \abs{\iota}{x}{\sema{A}}{\sema{B}} \infr \star$
            \item[$\D{2}$.] $\sema{\Gamma} \vdash_{\oldced} \sema{t} \chck \sema{A}$
            \item[$\D{3}$.] $\sema{\Gamma} \vdash_{\oldced} \sema{s} \chck [x := \sema{t}]\sema{B}$
        \end{enumerate}
        Note that $\sema{\Gamma} \vdash_{\oldced} \sema{t} \chck \sema{A}$ so clearly $\sema{\Gamma} \vdash_{\oldced} \sema{s} \chck [x := \chi\ \sema{A}\ -\ \sema{t}]\sema{B}$ as the $\chi$ merely adds extra typing information.
        Lemma~\ref{lem:4:sema_term_conv} applied to $\D{4}$ and using the fact that $t, s\pterm$ gives $|\sema{t}| \betaconv |\sema{s}|$.
        Combining this information yields $\sema{\Gamma} \vdash \sema{[t, s; A]} \chck \sema{(x : A) \cap B}$.
    \end{proofcase}

    $\text{Case: }\begin{array}{c} \FirstRule[*] \end{array}$
    \begin{proofcase}
        By classification $t\pterm$.
        Applying the IH to $\D{1}$ gives $\sema{\Gamma} \vdash_{\oldced} \sema{t} \chck \abs{\iota}{x}{\sema{A}}{\sema{B}}$.
        Deconstruct this checking rule and notice that either the inferred type is already an intersection or it must reduce to an intersection.
        Thus, $\exists\ C\ D$ such that $\sema{\Gamma} \vdash_{\oldced} \sema{t} \cinfr \abs{\iota}{x}{\sema{C}}{\sema{D}}$ and $\abs{\iota}{x}{\sema{C}}{\sema{D}} \cong \abs{\iota}{x}{\sema{A}}{\sema{B}}$.
        Deconstructing the congruence yields $\sema{C} \cong \sema{A}$.
        Thus, $\sema{\Gamma} \vdash_{\oldced} \sema{t}.1 \chck \sema{A}$
    \end{proofcase}

    $\text{Case: }\begin{array}{c} \SecondRule[*] \end{array}$
    \begin{proofcase}
        By classification $t\pterm$.
        Applying the IH to $\D{1}$ gives $\sema{\Gamma} \vdash_{\oldced} \sema{t} \chck \abs{\iota}{x}{\sema{A}}{\sema{B}}$.
        Deconstruct this checking rule and notice that either the inferred type is already an intersection or it must reduce to an intersection.
        Thus, $\exists\ C\ D$ such that $\sema{\Gamma} \vdash_{\oldced} \sema{t} \cinfr \abs{\iota}{x}{\sema{C}}{\sema{D}}$ and $\abs{\iota}{x}{\sema{C}}{\sema{D}} \cong \abs{\iota}{x}{\sema{A}}{\sema{B}}$.
        Deconstructing the congruence yields $\sema{D} \cong \sema{B}$ and thus $[x := \sema{t}.1]\sema{D} \cong [x := \sema{t}.1]\sema{B}$.
        Now $\sema{\Gamma} \vdash_{\oldced} \sema{t}.2 \infr [x := \sema{t}.1]\sema{D}$.
        Thus, $\sema{\Gamma} \vdash_{\oldced} \sema{t}.2 \chck [x := \sema{t}.1]\sema{B}$.
    \end{proofcase}

    $\text{Case: }\begin{array}{c} \EqualityRule[*] \end{array}$
    \begin{proofcase}
        Note that $a, b\pterm$ by $\D{1}$.
        Applying the IH gives:
        \begin{enumerate}
            \item[$\D{1}$.] $\sema{\Gamma} \vdash_{\oldced} \sema{A} \infr \star$
            \item[$\D{2}$.] $\sema{\Gamma} \vdash_{\oldced} \sema{a} \chck \sema{A}$
            \item[$\D{3}$.] $\sema{\Gamma} \vdash_{\oldced} \sema{b} \chck \sema{A}$
        \end{enumerate}
        By Lemma~\ref{lem:4:c1_stuff}, Lemma~\ref{lem:4:c1_weakening}, and the application rule for $\oldced$: $\sema{\Gamma} \vdash_{\oldced} \cId \cdot \sema{A}\ \sema{a}\ \sema{b} \infr \star$.
    \end{proofcase}

    $\text{Case: }\begin{array}{c} \ReflRule[*] \end{array}$
    \begin{proofcase}
        Note that $t\pterm$ by $\D{1}$.
        Applying the IH gives:
        \begin{enumerate}
            \item[$\D{1}$.] $\sema{\Gamma} \vdash_{\oldced} \sema{A} \infr \star$
            \item[$\D{2}$.] $\sema{\Gamma} \vdash_{\oldced} \sema{t} \chck \sema{A}$
        \end{enumerate}
        By Lemma~\ref{lem:4:c1_stuff}, Lemma~\ref{lem:4:c1_weakening}, and the application rule for $\oldced$: $\sema{\Gamma} \vdash_{\oldced} \crefl \cdot \sema{A}\ \edash \sema{t} \infr \cId \cdot \sema{A}\ \sema{t}\ \sema{t}$.
    \end{proofcase}

    \begin{minipage}{.8\textwidth}$\text{Case: }\begin{array}{c} \SubstRule[*] \end{array}$\end{minipage}
    \begin{proofcase}
        Note by $\D{1}$ that $a, b\pterm$ and by classification $e\pterm$ with $A, P\ptype$.
        Applying the IH gives:
        \begin{enumerate}
            \item[$\D{1}$.] $\sema{\Gamma} \vdash_{\oldced} \sema{A} \infr \star$
            \item[$\D{2}$.] $\sema{\Gamma} \vdash_{\oldced} \sema{a} \chck \sema{A}$
            \item[$\D{3}$.] $\sema{\Gamma} \vdash_{\oldced} \sema{b} \chck \sema{A}$
            \item[$\D{4}$.] $\sema{\Gamma} \vdash_{\oldced} \sema{e} \chck \cId \cdot \sema{A}\ \sema{a}\ \sema{b}$
            \item[$\D{5}$.] $\sema{\Gamma} \vdash_{\oldced} \sema{P} \infr T$ and $T \cong \abs{\forall}{y}{\sema{A}}{\cId \cdot \sema{A}\ \sema{a}\ \sema{y} \to \star}$
        \end{enumerate}
        By Lemma~\ref{lem:4:c1_stuff}, Lemma~\ref{lem:4:c1_weakening}, and the application rule for $\oldced$:
            $\sema{\Gamma} \vdash_{\oldced} \csubst \cdot \sema{A}\ \edash \sema{a}\ \edash \sema{b} \cdot \sema{P}\ \sema{e} \infr \sema{P}\ \sema{a}\ (\crefl \cdot \sema{A}\ \edash \sema{a}) \to \sema{P}\ \sema{b}\ \sema{e}$.
    \end{proofcase}

    $\text{Case: }\begin{array}{c} \PromoteRule[*] \end{array}$
    \begin{proofcase}
        Note by $\D{1}$ that $a, b\pterm$ and by classification $e\pterm$ with $(x : A) \cap B\ptype$.
        Applying the IH gives:
        \begin{enumerate}
            \item[$\D{1}$.] $\sema{\Gamma} \vdash_{\oldced} \abs{\iota}{x}{\sema{A}}{\sema{B}} \infr \star$
            \item[$\D{2}$.] $\sema{\Gamma} \vdash_{\oldced} \sema{a} \chck \abs{\iota}{x}{\sema{A}}{\sema{B}}$
            \item[$\D{3}$.] $\sema{\Gamma} \vdash_{\oldced} \sema{b} \chck \abs{\iota}{x}{\sema{A}}{\sema{B}}$
            \item[$\D{4}$.] $\sema{\Gamma} \vdash_{\oldced} \sema{e} \chck \cId \cdot \sema{A}\ \sema{a}.1\ \sema{b}.1$
        \end{enumerate}
        Note that $\sema{\Gamma}, x : \sema{A} \vdash_{\oldced} \sema{B} \infr \star$ which means $\sema{\Gamma} \vdash_{\oldced} \sema{B} \infr A \to \star$.
        By Lemma~\ref{lem:4:c1_stuff}, Lemma~\ref{lem:4:c1_weakening}, and the application rule for $\oldced$:
            $\sema{\Gamma} \vdash_{\oldced} \ctheta \cdot \sema{A} \cdot \sema{B}\ \edash \sema{a}\ \edash \sema{b}\ \sema{e} \infr \cId \cdot (\abs{\iota}{x}{\sema{A}}{\sema{B}})\ \sema{a}\ \sema{b}$.
    \end{proofcase}

    $\text{Case: }\begin{array}{c} \CastRule[*] \end{array}$
    \begin{proofcase}
        Note by soundness of classification that $a, b, e\pterm$.
        Applying the IH gives:
        \begin{enumerate}
            \item[$\D{1}$.] $\sema{\Gamma} \vdash_{\oldced} \sema{a} \chck \sema{A}$
            \item[$\D{2}$.] $\sema{\Gamma} \vdash_{\oldced} \sema{b} \chck \abs{\iota}{x}{\sema{A}}{\sema{B}}$
            \item[$\D{3}$.] $\sema{\Gamma} \vdash_{\oldced} \sema{e} \chck \cId \cdot \sema{A}\ \sema{a}\ \sema{b}.1$
        \end{enumerate}
        By the application and first projection rule and some maneuvering of type conversion: $\sema{\Gamma} \vdash_{\oldced} \text{\c{c}}\ \sema{e}.1 \chck \{ \sema{b} \cong \sema{a}\}$.
        Note that $FV(\sema{a}) \subseteq dom(\Gamma)$ because otherwise $\D{1}$ is not a proof.
        Thus, the goal is obtained by the $\varphi$ rule of $\oldced$.
    \end{proofcase}

    $\text{Case: }\begin{array}{c} \SeparationRule[*] \end{array}$
    \begin{proofcase}
        Applying the IH to $\D{1}$ yields $\sema{\Gamma} \vdash_{\oldced} \cId \cBool \ctt \cff$.
        By Lemma~\ref{lem:4:c1_stuff}, Lemma~\ref{lem:4:c1_weakening}, and the application rule for $\oldced$:
            $\sema{\Gamma} \vdash_{\oldced} \cdelta\ \sema{e} \infr \abs{\forall}{X}{\star}{X}$.
    \end{proofcase}

    $\text{Case: }\begin{array}{c} \ConvRule[*] \end{array}$
    \begin{proofcase}
        Suppose $K = \kind$.
        Then by classification and $\D{3}$: $\Gamma \vdash B : \kind$.
        Applying the IH to $\D{2}$ gives $\sema{\Gamma} \vdash \sema{t} \infr T$ with $T \cong \sema{B}$.
        By Lemma~\ref{lem:4:sema_type_conv}: $\sema{A} \cong \sema{B}$.
        Now by Lemma~\ref{lem:4:c1_trans} and Lemma~\ref{lem:4:c1_sym}: $T \cong \sema{B}$.
        \\ \\
        Suppose $K = \star$.
        Then by classification and $\D{3}$: $\Gamma \vdash B : \star$.
        Applying the IH to $\D{2}$ gives $\sema{\Gamma} \vdash \sema{t} \infr \sema{B}$.
        By Lemma~\ref{lem:4:sema_type_conv} and Lemma~\ref{lem:4:c1_sym}: $\sema{B} \cong \sema{A}$.
        Applying the checking rule of $\oldced$ yields $\sema{\Gamma} \vdash \sema{t} \chck \sema{A}$.
    \end{proofcase}
\end{proof}

\begin{theorem}[Logical Consistency]
    \label{lem:4:logical_consistency}
    $\neg (\vdash_{c_2} t : (X : \star) \to_0 X_\square)$
\end{theorem}
\begin{proof}
    Proceed using proof by negation.
    Suppose $\vdash_{c_2} t : (X : \star) \to_0 X_\square$.
    By Theorem~\ref{lem:4:soudness}: $\vdash_{\oldced} \sema{t} \chck \abs{\forall}{X}{\star}{X}$.
    However, this is impossible by consistency of $\oldced$.
\end{proof}

\begin{corollary}[Equational Consistency]
    \label{lem:4:eq_consistency}
    $\neg (\vdash_{c_2} t : \ptt =_{\pBool} \pff)$
\end{corollary}
