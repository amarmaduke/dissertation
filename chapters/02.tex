\chapter{Theory Description and Basic Metatheory}

\Rule{\AxiomRule}
    {\textcolor{white}{\_}}
    {\Gamma \vdash \star : \kind}
    {Axiom}

\Rule{\VarRule}
    {
        \der{\D{1}}{{x \notin FV(\Gamma_1; \Gamma_2)}} \\
        \der{\D{2}}{\Gamma_1 \vdash A : K}
    }
    {\Gamma_1; x_m : A; \Gamma_2 \vdash x_K : A}
    {Var}

\Rule{\WeakenRule}
    {
        \der{\D{1}}{x \notin FV(\Gamma)} \\
        \der{\D{2}}{\Gamma \vdash t : A} \\
        \der{\D{3}}{\Gamma \vdash B : K}
    }
    {\Gamma, x : B \vdash t : A}
    {Weaken}

\Rule{\ConvRule}
    {
        \der{\D{1}}{\Gamma \vdash A : K} \\
        \der{\D{2}}{\Gamma \vdash t : B} \\
        \der{\D{3}}{A \equiv B}
    }
    {\Gamma \vdash t : A}
    {Conv}

\Rule{\PiRule}
    {
        \der{\D{1}}{\Gamma \vdash A : \pdom(m, K)}
        \\ \der{\D{2}}{\Gamma; x_m : A \vdash B : \pcodom(m)}
    }
    {\Gamma \vdash (x : A) \to_m B : \pcodom(m)}
    {Pi}

\Rule{\LambdaRule}
    {
        \der{\D{1}}{\Gamma \vdash (x : A) \to_m B : \pcodom(m)}
        \\ \der{\D{2}}{\Gamma; x_m : A \vdash t : B}
        \\ \der{\D{3}}{x \notin FV(|t|)\text{ if } m = 0}
    }
    {\Gamma \vdash \abs{\lambda_m}{x}{A}{t} : (x : A) \to_m B}
    {Lam}

\Rule{\AppRule}
    {
        \der{\D{1}}{\Gamma \vdash f : (x : A) \to_m B} \\
        \der{\D{2}}{\Gamma \vdash a : A}
    }
    {\Gamma \vdash \app{f}{m}{a} : [x := a]B}
    {App}

\Rule{\IntersectionRule}
    {
        \der{\D{1}}{\Gamma \vdash A : \star} \\
        \der{\D{2}}{\Gamma; x_\tau : A \vdash B : \star}
    }
    {\Gamma \vdash (x : A) \cap B : \star}
    {Int}

\Rule{\PairRule}
    {
        \der{\D{1}}{\Gamma \vdash (x : A) \cap B : \star} \\
        \der{\D{2}}{\Gamma \vdash t : A} \\
        \der{\D{3}}{\Gamma \vdash s : [x := t]B} \\
        \der{\D{4}}{t \equiv s}
    }
    {\Gamma \vdash [t, s; (x : A) \cap B] : (x : A) \cap B }
    {Pair}

\Rule{\FirstRule}
    {\der{\D{1}}{\Gamma \vdash t : (x:A) \cap B}}
    {\Gamma \vdash t.1 : A}
    {Fst}

\Rule{\SecondRule}
    {\der{\D{1}}{\Gamma \vdash t : (x:A) \cap B}}
    {\Gamma \vdash t.2 : [x := t.1]B}
    {Snd}

\Rule{\EqualityRule}
    {
        \der{\D{1}}{\Gamma \vdash A : \star} \\
        \der{\D{2}}{\Gamma \vdash a : A} \\
        \der{\D{2}}{\Gamma \vdash b : A}
    }
    {\Gamma \vdash a =_A b : \star}
    {Eq}

\Rule{\ReflRule}
    {
        \der{\D{1}}{\Gamma \vdash A : \star} \\
        \der{\D{2}}{\Gamma \vdash t : A}
    }
    {\Gamma \vdash \text{refl}(t; A) : t =_A t}
    {Refl}

\Rule{\SubstRule}
    {
        \der{\D{1}}{\Gamma \vdash A : \star} \\
        \der{\D{2}}{\Gamma \vdash a : A} \\
        \der{\D{3}}{\Gamma \vdash b : A} \\
        \der{\D{4}}{\Gamma \vdash e : a =_A b} \\
        \der{\D{5}}{\Gamma \vdash P : (y : A) \to_\tau (p : a =_A y_\star) \to_\tau \star}
    }
    {\Gamma \vdash \psi(e, a, b; A, P) : \apptwo{P}{\tau}{a}{\tau}{\text{refl}(a; A)} \to_\omega \apptwo{P}{\tau}{b}{\tau}{e}}
    {Subst}

\Rule{\PromoteRule}
    {
        \der{\D{1}}{\Gamma \vdash (x:A) \cap B : \star} \\
        \der{\D{2}}{\Gamma \vdash a : (x:A) \cap B} \\
        \der{\D{3}}{\Gamma \vdash b : (x:A) \cap B} \\
        \der{\D{4}}{\Gamma \vdash e : a.1 =_A b.1} \\
    }
    {\Gamma \vdash \vartheta(e, a, b; (x : A) \cap B) : a =_{(x:A) \cap B} b}
    {Prm}

% \Rule{\PromoteSndRule}
%     {
%         \der{\D{1}}{\Gamma \vdash A : \star} \\
%         \der{\D{2}}{\Gamma \vdash B : \star} \\
%         \der{\D{3}}{\Gamma \vdash a : (x:A) \cap B} \\
%         \der{\D{4}}{\Gamma \vdash b : (x:A) \cap B} \\
%         \der{\D{5}}{\Gamma \vdash e : a.2 =_B b.2} \\
%     }
%     {\Gamma \vdash \vartheta_2(e, a, b; (x : A) \cap B) : a =_{(x:A) \cap B} b}
%     {PrmSnd}

% \Rule{\CastRule}
%     {
%         \der{\D{1}}{\Gamma \vdash b \cinfr (x:A) \cap B} \\
%         \der{\D{2}}{\Gamma \vdash a \chck A} \\
%         \der{\D{3}}{\Gamma \vdash e \chck a =_A b.1} \\
%         \der{\D{4}}{\text{FV}(|e|) \subseteq \text{FV}(|a|)}
%     }
%     {\Gamma \vdash \varphi(a, b, e) \infr (x:A) \cap B}
%     {Cast}

\Rule{\CastRule}
    {
        \der{\D{1}}{\Gamma \vdash a : A} \\
        \der{\D{2}}{\Gamma \vdash b : (x:A) \cap B} \\
        \der{\D{3}}{\Gamma \vdash e : a =_A b.1}
    }
    {\Gamma \vdash \varphi(a, b, e) : (x:A) \cap B}
    {Cast}

\Rule{\SeparationRule}
    {
        \der{\D{1}}{\Gamma \vdash e : \text{ctt} =_{\text{cBool}} \text{cff}}
    }
    {\Gamma \vdash \delta(e) : (X : \star) \to_0 X_\square}
    {Sep}


The theory described in this chapter is a variation of the core theory of Cedille \cite{stump2021_cedillecore}.
It is closely related with the significant differences occurring with the equality type.
This variation has two primary goals.
First, to have decidable type checking (and thus decidable conversion checking).
Second, to retain as many constructions as possible from Cedille.
This chapter focuses on the description of the theory and some basic metatheory.
By basic, we mean properties that are provable by induction on the various derivations or are otherwise provable using straightforward methods.

Syntax for the theory is described in Figure~\ref{fig:syntax}.
Unlike other presentations a generic syntax tree is used with a tag to indicate different syntactic forms.
There are three basic syntactic constructs: variables, binders, and constructors.
A generic presentation enables occasional economic benefits in presenting other derivations.
However, a more standard syntax is defined in terms of the generic one.
The specific syntactic forms and the generic forms are used interchangeably whichever is more convenient.


\begin{figure}
    \centering
    \begin{align*}
        t &::= x\ |\ \mathfrak{b}(\kappa_1, x : t_1, t_2)\ |\ \mathfrak{c}(\kappa_2, t_1, \ldots, t_{\mathfrak{a}(\kappa_2)}) \\
        \kappa_1 &::= \lambda_m\ |\ \Pi_m\ |\ \cap \\
        \kappa_2 &::=\diamond\ |\ \star\ |\ \kind\ | \bullet_m |\ \text{pair}\ |\ \text{proj}_1\ |\ \text{proj}_2\ |\ \text{eq}\ |\ \text{refl}\ |\ \psi\ |\ \vartheta_1\ |\ \vartheta_2\ |\ \delta\ |\ \varphi  \\
        m &::= \omega\ |\ 0\ |\ \tau \\
        &\mathfrak{a}(\diamond) = \mathfrak{a}(\star) = \mathfrak{a}(\kind) = 0 \\
        &\mathfrak{a}(\text{proj}_1) = \mathfrak{a}(\text{proj}_2) = \mathfrak{a}(\text{refl}) = \mathfrak{a}(\delta) = 1 \\
        &\mathfrak{a}(\bullet_m) = \mathfrak{a}(\psi) = \mathfrak{a}(\varphi) = 2 \\
        &\mathfrak{a}(\text{pair}) = \mathfrak{a}(\text{eq}) = \mathfrak{a}(\vartheta_1) = \mathfrak{a}(\vartheta_2) = 3
    \end{align*}
    \vspace{-.4in}
    \begin{minipage}{0.5\textwidth}
        \begin{align*}
            \diamond &:= \mathfrak{c}(\diamond) \\
            \star &:= \mathfrak{c}(\star) \\
            \kind &:= \mathfrak{c}(\kind) \\
            \abs{\lambda_m}{x}{t_1}{t_2} &:= \mathfrak{b}(\lambda_m, x : t_1, t_2) \\
            (x : t_1) \to_m t_2 &:= \mathfrak{b}(\Pi_m, x : t_1, t_2) \\
            (x : t_1) \cap t_2 &:= \mathfrak{b}(\cap, x : t_1, t_2) \\
            t_1 \bullet_m t_2 &:= \mathfrak{c}(\bullet_m, t_1, t_2) \\
            \varphi(t_1, t_2) &:= \mathfrak{c}(\varphi, t_1, t_2) \\
            \psi(t_1, t_2) &:= \mathfrak{c}(\psi, t_1, t_2)
        \end{align*}
    \end{minipage}%
    \begin{minipage}{0.5\textwidth}
        \begin{align*}
            [t_1, t_2; t_3] &:= \mathfrak{c}(\text{pair}, t_1, t_2, t_3) \\
            t.1 &:= \mathfrak{c}(\text{proj}_1, t) \\
            t.2 &:= \mathfrak{c}(\text{proj}_2, t) \\
            t_1 =_{t_2} t_3 &:= \mathfrak{c}(\text{eq}, t_1, t_2, t_3) \\
            \text{refl}(t) &:= \mathfrak{c}(\text{refl}, t) \\
            \vartheta_1(t_1, t_2, t_3) &:= \mathfrak{c}(\vartheta_1, t_1, t_2, t_3) \\
            \vartheta_2(t_1, t_2, t_3) &:= \mathfrak{c}(\vartheta_2, t_1, t_2, t_3) \\
            \delta(t) &:= \mathfrak{c}(\delta, t)
        \end{align*}
    \end{minipage}
    \vspace{-.05in}
    %$$J(t_1, t_2, t_3, t_4, t_5, t_6) := \mathfrak{c}(J, t_1, t_2, t_3, t_4, t_5, t_6)$$
    \caption{Generic syntax, there are three constructors, variables, a generic binder, and a generic non-binder. Each are parameterized with a constant tag to specialize to a particular syntactic consruct. The non-binder constructor has a vector of subterms determined by an arity function computed on tags. Standard syntactic constructors are defined in terms of the generic forms.}
    \label{fig:syntax}
\end{figure}


Formally, syntax is worked with as a locally nameless set following the axioms of Pitts \cite{2023pitts_lns}.
For the sake of presentation these details are elided.
This means that freshness of variables and capture avoiding substitution are largely taken for granted in the exposition of the theory.
Moreover, identity of syntactic terms is assumed to be alpha equivalence.
Meaning that, again, bureaucracy around variables is taken for granted.
Thus, substitution is defined simply:
\begin{align*}
    [x := v]y &= v \text{ if }x = y \\
    [x := v]y &= y \text{ if }x \neq y \\
    [x := v]\mathfrak{b}(\kappa_1, x : t_1, t_2) &= \mathfrak{b}(\kappa_1, x : [x := v]t_1, [x := v]t_2) \\
    [x := v]\mathfrak{c}(\kappa_2, t_1, \ldots, t_{\mathfrak{a}(\kappa_2)}) &= \mathfrak{c}(\kappa_2, [x := v]t_1, \ldots, [x := v]t_{\mathfrak{a}(\kappa_2)})
\end{align*}




\begin{figure}
    \centering
    \begin{minipage}{0.5\textwidth}
        \begin{align*}
            |x_K| &= x_K \\
            |\!\star\!| &= \star \\
            |\kind| &= \kind \\
            |\abs{\lambda_0}{x}{A}{t}| &= |t| \\
            |\abs{\lambda_\omega}{x}{A}{t}| &= \abs{\lambda_\omega}{x}{\diamond}{|t|} \\
            |\abs{\lambda_\tau}{x}{A}{t}| &= \abs{\lambda_\tau}{x}{|A|}{|t|} \\
            |(x:A) \to_m B| &= (x:|A|) \to_m |B| \\
            |(x:A) \cap B| &= (x:|A|) \cap |B| \\
            |\app{f}{0}{a}| &= |f| \\
            |\app{f}{\omega}{a}| &= \app{|f|}{\omega}{|a|} \\
            |\app{f}{\tau}{a}| &= \app{|f|}{\tau}{|a|}
        \end{align*}
    \end{minipage}%
    \begin{minipage}{0.5\textwidth}
        \begin{align*}
            |\!\diamond\!| &= \diamond \\
            |[t_1, t_2; A]| &= |t_1| \\
            |t.1| &= |t| \\
            |t.2| &= |t| \\
            |x =_A y| &= |x| =_{|A|} |y| \\
            |\text{refl}(t; A)| &= \abs{\lambda_\omega}{x}{\diamond}{x_\star} \\
            %|J(A, P, x, y, e, w)| &= \app{|e|}{\omega}{|w|} \\
            |\psi(e, a, b; A, P)| &= |e| \\
            |\vartheta_1(e, a, b; T)| &= |e| \\
            |\vartheta_2(e, a, b; T)| &= |e| \\
            |\delta(e)| &= |e| \\
            |\varphi(f, e; A, T)| &= \abs{\lambda_\omega}{x}{\diamond}{x_\star}
        \end{align*}
    \end{minipage}
    \caption{Erasure of syntax, for type-like and kind-like syntax erasure is homomorphic, for term-like syntax erasure reduces to the untyped lambda calculus.}
    \label{fig:2:erasure}
\end{figure}


\begin{figure}
    \centering
    \begin{minipage}{0.5\textwidth}
        $$\BinderReductionOne$$
    \end{minipage}%
    \begin{minipage}{0.5\textwidth}
        $$\BinderReductionTwo$$
    \end{minipage}%
    $$\NonbinderReduction$$
    \begin{align*}
        \app{(\abs{\lambda_m}{x}{A}{b})}{m}{t} &\betared [x := t]b \\
        [t_1, t_2, A].1 &\betared t_1 \\
        [t_1, t_2, A].2 &\betared t_2 \\
        J(A, P, x, y, \text{refl}(z), w) &\betared \app{w}{0}{z} \\
        \vartheta(\text{refl}(t.1)) &\betared \text{refl}(t) \\
        \vartheta(\text{refl}(t.2)) &\betared \text{refl}(t) \\
    \end{align*}
    \caption{Reduction rules for arbitrary syntax.}
\end{figure}




\begin{figure}
    \centering
    \begin{minipage}{0.5\textwidth}
        \begin{align*}
            \text{dom}_\Pi(\omega, K) &= \star \\
            \text{dom}_\Pi(\tau, K) &= K \\
            \text{dom}_\Pi(0, K) &= K
        \end{align*}
    \end{minipage}%
    \begin{minipage}{0.5\textwidth}
        \begin{align*}
            \text{codom}_\Pi(\omega) &= \star \\
            \text{codom}_\Pi(\tau) &= \kind \\
            \text{codom}_\Pi(0) &= \star
        \end{align*}
    \end{minipage}
    \caption{Domain and codomains for function types. The metavariable $K$ is either $\star$ or $\kind$.}
\end{figure}


\begin{figure}
    \centering
    \begin{minipage}{0.5\textwidth}
        $$\AxiomRule$$
        $$\HeadInferenceRule$$
        $$\ContextEmptyRule$$
    \end{minipage}%
    \begin{minipage}{0.5\textwidth}
        $$\VarRule$$
        $$\CheckRule$$
        $$\ContextAppendRule$$
    \end{minipage}%
    $$\PiRule$$
    $$\LambdaRule$$
    $$\AppRule$$
    \caption{
        Inference rules for function types, including erased functions.
        The variable $K$ is either $\star$ or $\kind$.
    }
\end{figure}


\begin{figure}
    \centering
    \begin{minipage}{0.5\textwidth}
        $$\IntersectionRule$$
        $$\FirstRule$$
    \end{minipage}%
    \begin{minipage}{0.5\textwidth}
        $$\PairRule$$
        $$\SecondRule$$
    \end{minipage}
    \caption{
        Inference rules for intersection types.
    }
\end{figure}


\begin{figure}
    \centering
    \begin{minipage}{0.5\textwidth}
        $$\EqualityRule$$
    \end{minipage}%
    \begin{minipage}{0.5\textwidth}
        $$\ReflRule$$
    \end{minipage}%
    $$\SubstRule$$
    $$\PromoteRule$$
    $$\CastRule$$
    $$\SeparationRule$$
    \caption{
        Inference rules for equality types where
        $\text{cBool} := (X : \star) \to_0 (x : X_\square) \to_\omega (y : X_\square) \to_\omega X_\square$;
        $\text{ctt} := \abs{\lambda_0}{X}{\star}{
            \abs{\lambda_\omega}{x}{X_\square}{
                \abs{\lambda_\omega}{y}{X_\square}{x_\star}
            }
        }$;
        and
        $\text{cff} := \abs{\lambda_0}{X}{\star}{
            \abs{\lambda_\omega}{x}{X_\square}{
                \abs{\lambda_\omega}{y}{X_\square}{y_\star}
            }
        }$.
    }
    \label{fig:2:typ3}
\end{figure}


