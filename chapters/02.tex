\chapter{Theory Description and Basic Metatheory}

\Rule{\AxiomRule}
    {\textcolor{white}{\_}}
    {\Gamma \vdash \star : \kind}
    {Axiom}

\Rule{\VarRule}
    {
        \der{\D{1}}{{x \notin FV(\Gamma_1; \Gamma_2)}} \\
        \der{\D{2}}{\Gamma_1 \vdash A : K}
    }
    {\Gamma_1; x_m : A; \Gamma_2 \vdash x_K : A}
    {Var}

\Rule{\WeakenRule}
    {
        \der{\D{1}}{x \notin FV(\Gamma)} \\
        \der{\D{2}}{\Gamma \vdash t : A} \\
        \der{\D{3}}{\Gamma \vdash B : K}
    }
    {\Gamma, x : B \vdash t : A}
    {Weaken}

\Rule{\ConvRule}
    {
        \der{\D{1}}{\Gamma \vdash A : K} \\
        \der{\D{2}}{\Gamma \vdash t : B} \\
        \der{\D{3}}{A \equiv B}
    }
    {\Gamma \vdash t : A}
    {Conv}

\Rule{\PiRule}
    {
        \der{\D{1}}{\Gamma \vdash A : \pdom(m, K)}
        \\ \der{\D{2}}{\Gamma; x_m : A \vdash B : \pcodom(m)}
    }
    {\Gamma \vdash (x : A) \to_m B : \pcodom(m)}
    {Pi}

\Rule{\LambdaRule}
    {
        \der{\D{1}}{\Gamma \vdash (x : A) \to_m B : \pcodom(m)}
        \\ \der{\D{2}}{\Gamma; x_m : A \vdash t : B}
        \\ \der{\D{3}}{x \notin FV(|t|)\text{ if } m = 0}
    }
    {\Gamma \vdash \abs{\lambda_m}{x}{A}{t} : (x : A) \to_m B}
    {Lam}

\Rule{\AppRule}
    {
        \der{\D{1}}{\Gamma \vdash f : (x : A) \to_m B} \\
        \der{\D{2}}{\Gamma \vdash a : A}
    }
    {\Gamma \vdash \app{f}{m}{a} : [x := a]B}
    {App}

\Rule{\IntersectionRule}
    {
        \der{\D{1}}{\Gamma \vdash A : \star} \\
        \der{\D{2}}{\Gamma; x_\tau : A \vdash B : \star}
    }
    {\Gamma \vdash (x : A) \cap B : \star}
    {Int}

\Rule{\PairRule}
    {
        \der{\D{1}}{\Gamma \vdash (x : A) \cap B : \star} \\
        \der{\D{2}}{\Gamma \vdash t : A} \\
        \der{\D{3}}{\Gamma \vdash s : [x := t]B} \\
        \der{\D{4}}{t \equiv s}
    }
    {\Gamma \vdash [t, s; (x : A) \cap B] : (x : A) \cap B }
    {Pair}

\Rule{\FirstRule}
    {\der{\D{1}}{\Gamma \vdash t : (x:A) \cap B}}
    {\Gamma \vdash t.1 : A}
    {Fst}

\Rule{\SecondRule}
    {\der{\D{1}}{\Gamma \vdash t : (x:A) \cap B}}
    {\Gamma \vdash t.2 : [x := t.1]B}
    {Snd}

\Rule{\EqualityRule}
    {
        \der{\D{1}}{\Gamma \vdash A : \star} \\
        \der{\D{2}}{\Gamma \vdash a : A} \\
        \der{\D{2}}{\Gamma \vdash b : A}
    }
    {\Gamma \vdash a =_A b : \star}
    {Eq}

\Rule{\ReflRule}
    {
        \der{\D{1}}{\Gamma \vdash A : \star} \\
        \der{\D{2}}{\Gamma \vdash t : A}
    }
    {\Gamma \vdash \text{refl}(t; A) : t =_A t}
    {Refl}

\Rule{\SubstRule}
    {
        \der{\D{1}}{\Gamma \vdash A : \star} \\
        \der{\D{2}}{\Gamma \vdash a : A} \\
        \der{\D{3}}{\Gamma \vdash b : A} \\
        \der{\D{4}}{\Gamma \vdash e : a =_A b} \\
        \der{\D{5}}{\Gamma \vdash P : (y : A) \to_\tau (p : a =_A y_\star) \to_\tau \star}
    }
    {\Gamma \vdash \psi(e, a, b; A, P) : \apptwo{P}{\tau}{a}{\tau}{\text{refl}(a; A)} \to_\omega \apptwo{P}{\tau}{b}{\tau}{e}}
    {Subst}

\Rule{\PromoteRule}
    {
        \der{\D{1}}{\Gamma \vdash (x:A) \cap B : \star} \\
        \der{\D{2}}{\Gamma \vdash a : (x:A) \cap B} \\
        \der{\D{3}}{\Gamma \vdash b : (x:A) \cap B} \\
        \der{\D{4}}{\Gamma \vdash e : a.1 =_A b.1} \\
    }
    {\Gamma \vdash \vartheta(e, a, b; (x : A) \cap B) : a =_{(x:A) \cap B} b}
    {Prm}

% \Rule{\PromoteSndRule}
%     {
%         \der{\D{1}}{\Gamma \vdash A : \star} \\
%         \der{\D{2}}{\Gamma \vdash B : \star} \\
%         \der{\D{3}}{\Gamma \vdash a : (x:A) \cap B} \\
%         \der{\D{4}}{\Gamma \vdash b : (x:A) \cap B} \\
%         \der{\D{5}}{\Gamma \vdash e : a.2 =_B b.2} \\
%     }
%     {\Gamma \vdash \vartheta_2(e, a, b; (x : A) \cap B) : a =_{(x:A) \cap B} b}
%     {PrmSnd}

% \Rule{\CastRule}
%     {
%         \der{\D{1}}{\Gamma \vdash b \cinfr (x:A) \cap B} \\
%         \der{\D{2}}{\Gamma \vdash a \chck A} \\
%         \der{\D{3}}{\Gamma \vdash e \chck a =_A b.1} \\
%         \der{\D{4}}{\text{FV}(|e|) \subseteq \text{FV}(|a|)}
%     }
%     {\Gamma \vdash \varphi(a, b, e) \infr (x:A) \cap B}
%     {Cast}

\Rule{\CastRule}
    {
        \der{\D{1}}{\Gamma \vdash a : A} \\
        \der{\D{2}}{\Gamma \vdash b : (x:A) \cap B} \\
        \der{\D{3}}{\Gamma \vdash e : a =_A b.1}
    }
    {\Gamma \vdash \varphi(a, b, e) : (x:A) \cap B}
    {Cast}

\Rule{\SeparationRule}
    {
        \der{\D{1}}{\Gamma \vdash e : \text{ctt} =_{\text{cBool}} \text{cff}}
    }
    {\Gamma \vdash \delta(e) : (X : \star) \to_0 X_\square}
    {Sep}


% The theory described in this chapter is a variation of the core theory of Cedille \cite{stump2021_cedillecore}.
% It is closely related with the significant differences occurring with the equality type.
% This variation has two primary goals.
% First, to have decidable type checking (and thus decidable conversion checking).
% Second, to retain as many constructions as possible from Cedille.
% This chapter focuses on the description of the theory and some basic metatheory.
% By basic, we mean properties that are provable by induction on the various derivations or are otherwise provable using straightforward methods.

\outline{Talk about the design idea of Cedille2, we keep a tight understanding of a proof, and erasure is more a translation to this object thingy. Types depend on objects, not proofs.}


\begin{figure}
    \centering
    \begin{align*}
        t &::= x\ |\ \mathfrak{b}(\kappa_1, x : t_1, t_2)\ |\ \mathfrak{c}(\kappa_2, t_1, \ldots, t_{\mathfrak{a}(\kappa_2)}) \\
        \kappa_1 &::= \lambda_m\ |\ \Pi_m\ |\ \cap \\
        \kappa_2 &::=\diamond\ |\ \star\ |\ \kind\ | \bullet_m |\ \text{pair}\ |\ \text{proj}_1\ |\ \text{proj}_2\ |\ \text{eq}\ |\ \text{refl}\ |\ \psi\ |\ \vartheta_1\ |\ \vartheta_2\ |\ \delta\ |\ \varphi  \\
        m &::= \omega\ |\ 0\ |\ \tau \\
        &\mathfrak{a}(\diamond) = \mathfrak{a}(\star) = \mathfrak{a}(\kind) = 0 \\
        &\mathfrak{a}(\text{proj}_1) = \mathfrak{a}(\text{proj}_2) = \mathfrak{a}(\text{refl}) = \mathfrak{a}(\delta) = 1 \\
        &\mathfrak{a}(\bullet_m) = \mathfrak{a}(\psi) = \mathfrak{a}(\varphi) = 2 \\
        &\mathfrak{a}(\text{pair}) = \mathfrak{a}(\text{eq}) = \mathfrak{a}(\vartheta_1) = \mathfrak{a}(\vartheta_2) = 3
    \end{align*}
    \vspace{-.4in}
    \begin{minipage}{0.5\textwidth}
        \begin{align*}
            \diamond &:= \mathfrak{c}(\diamond) \\
            \star &:= \mathfrak{c}(\star) \\
            \kind &:= \mathfrak{c}(\kind) \\
            \abs{\lambda_m}{x}{t_1}{t_2} &:= \mathfrak{b}(\lambda_m, x : t_1, t_2) \\
            (x : t_1) \to_m t_2 &:= \mathfrak{b}(\Pi_m, x : t_1, t_2) \\
            (x : t_1) \cap t_2 &:= \mathfrak{b}(\cap, x : t_1, t_2) \\
            t_1 \bullet_m t_2 &:= \mathfrak{c}(\bullet_m, t_1, t_2) \\
            \varphi(t_1, t_2) &:= \mathfrak{c}(\varphi, t_1, t_2) \\
            \psi(t_1, t_2) &:= \mathfrak{c}(\psi, t_1, t_2)
        \end{align*}
    \end{minipage}%
    \begin{minipage}{0.5\textwidth}
        \begin{align*}
            [t_1, t_2; t_3] &:= \mathfrak{c}(\text{pair}, t_1, t_2, t_3) \\
            t.1 &:= \mathfrak{c}(\text{proj}_1, t) \\
            t.2 &:= \mathfrak{c}(\text{proj}_2, t) \\
            t_1 =_{t_2} t_3 &:= \mathfrak{c}(\text{eq}, t_1, t_2, t_3) \\
            \text{refl}(t) &:= \mathfrak{c}(\text{refl}, t) \\
            \vartheta_1(t_1, t_2, t_3) &:= \mathfrak{c}(\vartheta_1, t_1, t_2, t_3) \\
            \vartheta_2(t_1, t_2, t_3) &:= \mathfrak{c}(\vartheta_2, t_1, t_2, t_3) \\
            \delta(t) &:= \mathfrak{c}(\delta, t)
        \end{align*}
    \end{minipage}
    \vspace{-.05in}
    %$$J(t_1, t_2, t_3, t_4, t_5, t_6) := \mathfrak{c}(J, t_1, t_2, t_3, t_4, t_5, t_6)$$
    \caption{Generic syntax, there are three constructors, variables, a generic binder, and a generic non-binder. Each are parameterized with a constant tag to specialize to a particular syntactic consruct. The non-binder constructor has a vector of subterms determined by an arity function computed on tags. Standard syntactic constructors are defined in terms of the generic forms.}
    \label{fig:syntax}
\end{figure}


\outline{Briefly introduce the syntax, recalling it is the same generic setup as with F omega.}

\begin{lemma}
    $|[x := t]b| = [x := |t|]|b|$
    \label{lem:2:erase_subst}
\end{lemma}
\begin{proof}
    By induction on the size of $b$.

    $\text{Case: }\mathfrak{b}(\kappa, (x : t_1), t_2)$
    \begin{proofcase}
        If $b = \abs{\lambda_0}{y}{A}{b^\prime}$, then $|b| = |b^\prime|$ which is a smaller term.
        Then, by the IH $|[x := t]b^\prime| = [x := |t|]|b^\prime|$.
        Thus,
        \begin{align*}
            &|[x := t]\abs{\lambda_0}{y}{A}{b^\prime}| = |\abs{\lambda_0}{y}{[x := t]A}{[x := t]b^\prime}| \\
            &= |[x := t]b^\prime| = [x := |t|]|b^\prime| = [x := |t|]|\abs{\lambda_0}{y}{A}{b^\prime}|
        \end{align*}
        For the remaining tags, assume w.l.o.g. $\kappa = \cap$.
        Then $b = (y : A) \cap B$, and by the IH $|[x := t]A| = [x := |t|]|A|$ and $|[x := t]B| = [x := |t|]|B|$.
        Thus,
        \begin{align*}
            &|[x := t]((y : A) \cap B)| = |(y : [x := t]A) \cap [x := t]B| \\
            &= (y : |[x := t]A|) \cap |[x := t]B| = (y : [x := |t|]|A|) \cap [x := |t|]|B|
        \end{align*}
        And, $[x := |t|]|(y : A) \cap B| = (y : [x := |t|]|A|) \cap [x := |t|]|B|$.
        Thus, both sides are equal.
    \end{proofcase}

    $\text{Case: }\mathfrak{c}(\kappa, t_1, \ldots, t_{\mathfrak{a}(\kappa)})$
    \begin{proofcase}
        If $\kappa \in \{ \star, \kind \}$ then the equality is trivial.

        If $\kappa \in \{ \bullet_0, \text{pair}, \text{proj}_1, \text{proj}_2, \psi, \vartheta, \delta \}$ then $|\mathfrak{c}(\kappa, t_1, \ldots)| = |t_1|$.
        Moreover, substitution commutes and both sides of the equality are equal.

        If $\kappa \in \{ \text{refl}, \varphi \}$ then the equality is trivial.

        If $\kappa \in \{ \bullet_\omega, \bullet_\tau, \text{eq} \}$ then w.l.o.g. assume $\kappa = \text{eq}$.
        Now $|[x := t](a =_A b)| = |[x := t]a| =_{|[x := t]A|} |[x := t]b|$.
        By the IH this becomes $[x := |t|]|a| =_{[x := |t|]|A|} [x := |t|]|b|$.
        On the right-hand side, $[x := |t|]|a =_A b| = [x := |t|]|a| =_{[x := |t|]|A|} [x := |t|]|b|$.
        Thus, both sides are equal.

    \end{proofcase}

    $\text{Case: }b \text{ variable}$
    \begin{proofcase}
        Suppose $b = x$, then $|[x := t]x| = |t|$ and $[x := |t|]|x| = |t|$.
        Suppose $b = y$, then $|[x := t]y| = y$ and $[x := |t|]|y| = y$.
        Thus, both sides are equal.
    \end{proofcase}
\end{proof}

\begin{lemma}
    If $x \neq y$ then $[x := a][y := b]t = [y := [x := a]b][x := a]t$
    \label{lem:2:subst_commute}
\end{lemma}
\begin{proof}
    By induction on $t$.
    If $t$ is a binder or a constructor, then substitution unfolds and the IH applied to subterms concludes those cases.
    Suppose $t$ is a variable, $z$.
    If $z = x$, then $z \neq y$ and $t = a$ on both sides.
    If $z = y$, then $z \neq x$ and $t = [x := a]b$ on both sides.
    If $z \neq x$ and $z \neq y$, then $t = z$ on both sides.
\end{proof}

\begin{lemma}
    If $a \betared b$ then $[x := t]a \betared [x := t]b$
    \label{lem:2:betared_subst}
\end{lemma}
\begin{proof}
    By induction on $a \betared b$.

    $\text{Case: }\begin{array}{c} \app{(\abs{\lambda_m}{x}{A}{b})}{m}{t} \betared [x := t]b \end{array}$
    \begin{proofcase}
        $[x := s](\app{(\abs{\lambda_m}{y}{A}{b})}{m}{t}) = \app{(\abs{\lambda_m}{x}{[x := s]A}{[x := s]b})}{m}{[x := s]t} \betared [y := [x := s]t][x := s]b = [x := s][y := t]b$ \\
        Note that the final equality holds by Lemma~\ref{lem:2:subst_commute}.
    \end{proofcase}

    $\text{Case: }\begin{array}{c} [t_1, t_2; A].1 \betared t_1 \end{array}$
    \begin{proofcase}
        $[x := t][t_1, t_2, A].1 = [[x := t]t_1, [x := t]t_2, [x := ]A].1 \betared [x := t]t_1$
    \end{proofcase}

    $\text{Case: }\begin{array}{c} [t_1, t_2; A].2 \betared t_2 \end{array}$
    \begin{proofcase}
        $[x := t][t_1, t_2, A].2 = [[x := t]t_1, [x := t]t_2, [x := ]A].2 \betared [x := t]t_2$
    \end{proofcase}

    $\text{Case: }\begin{array}{c} \psi(\text{refl}(t),P) \betared \abs{\lambda_\omega}{x}{\app{P}{\tau}{t}}{x} \end{array}$
    \begin{proofcase}
        $[x := s]\psi(\text{refl(t)}, P) = \psi(\text{refl([x := s]t)}, [x := s]P) \betared \abs{\lambda_\omega}{y}{\app{[x := s]P}{\tau}{[x := s]t}}{y} = [x := s](\abs{\lambda_\omega}{y}{\app{P}{\tau}{t}}{y})$
    \end{proofcase}

    $\text{Case: }\begin{array}{c} \vartheta_1(\text{refl}(t_1), t_2, t_3) \betared \text{refl}(t_2) \end{array}$
    \begin{proofcase}
        $[x := s]\vartheta_1(\text{refl}(t_1), t_2, t_3) = \vartheta_1(\text{refl}(([x := s]t_1)), [x := s]t_2, [x := s]t_3) \betared \text{refl}([x := s]t_2) = [x := s]\text{refl}(t_2)$
    \end{proofcase}

    $\text{Case: }\begin{array}{c} \vartheta_2(\text{refl}(t_1), t_2, t_3) \betared \text{refl}(t_2) \end{array}$
    \begin{proofcase}
        $[x := s]\vartheta_2(\text{refl}(t_1), t_2, t_3) = \vartheta_2(\text{refl}(([x := s]t_1)), [x := s]t_2, [x := s]t_3) \betared \text{refl}([x := s]t_2) = [x := s]\text{refl}(t_2)$
    \end{proofcase}

    $\text{Case: }\begin{array}{c} \NonbinderReduction[*] \end{array}$
    \begin{proofcase}
        By the IH, $[x := t]t_i \betared [x := t]t_i^\prime$.
        Note that $$[x := t]\mathfrak{c}(\kappa, t_1, \ldots, t_{\mathfrak{a}}(\kappa)) = \mathfrak{c}(\kappa, [x := t]t_1, \ldots, [x := t]t_{\mathfrak{a}}(\kappa))$$
        Applying the constructor reduction rule and reversing the previous equality concludes the case.
    \end{proofcase}

    $\text{Case: }\begin{array}{c} \BinderReductionOne[*] \end{array}$
    \begin{proofcase}
        By the IH, $[x := t]t_1 \betared [x := t]t_1^\prime$.
        Note that $$[x := t]\mathfrak{b}(\kappa, (y : t_1), t_2) = \mathfrak{b}(\kappa, (y : [x := t]t_1), [x := t]t_2)$$
        Applying the first binder reduction rule and reversing the previous equality concludes the case.
    \end{proofcase}

    $\text{Case: }\begin{array}{c} \BinderReductionTwo[*] \end{array}$
    \begin{proofcase}
        Same as previous case but applying the second binder reduction rule.
    \end{proofcase}
\end{proof}

\begin{lemma}
    If $a \betastar b$ then $[x := t]a \betastar [x := t]b$
    \label{lem:2:betastar_subst}
\end{lemma}
\begin{proof}
    By induction on $a \betastar b$.
    The reflexivity case is trivial.

    $\text{Case: }\begin{array}{c} \MultiStep[*] \end{array}$
    \begin{proofcase}
        Let $z = t^\prime$.
        By the IH applied to $\D{2}$: $[x := t]z \betastar [x := t]b$.
        By Lemma~\ref{lem:2:betared_subst} applied to $\D{1}$: $[x := t]a \betared [x := t]b$.
        Applying the transitivity rule yields $[x := t]a \betastar [x := t]b$.
    \end{proofcase}
\end{proof}

\begin{lemma}
    If $a \betaconv b$ then $[x := t]a \betaconv [x := t]b$
    \label{lem:2:betaconv_subst}
\end{lemma}
\begin{proof}
    By definition $\exists\ z$ such that $a \betastar z$ and $b \betastar z$.
    Applying Lemma~\ref{lem:2:betastar_subst} twice yields $[x := t]a \betastar [x := t]z$ and $[x := t]b \betastar [x := z]$.
\end{proof}

\begin{lemma}
    If $|a| \betaconv |b|$ then $|[x := t]a| \betaconv |[x := t]b|$
    \label{lem:2:betaconv_erased_subst}
\end{lemma}
\begin{proof}
    By definition $\exists\ z$ such that $|a| \betastar z$ and $|b| \betastar z$.
    By Lemma~\ref{lem:2:betastar_subst} applied twice $[x := |t|]|a| \betastar [x := |t|]z$ and $[x := |t|]|b| \betastar [x := |t|]z$.
    Finally, by Lemma~\ref{lem:2:erase_subst} $[x := |t|]|a| = |[x := t]a|$ and $[x := |t|]|b| = |[x := t]b|$.
\end{proof}



\begin{figure}
    \centering
    \begin{minipage}{0.5\textwidth}
        \begin{align*}
            |x_K| &= x_K \\
            |\!\star\!| &= \star \\
            |\kind| &= \kind \\
            |\abs{\lambda_0}{x}{A}{t}| &= |t| \\
            |\abs{\lambda_\omega}{x}{A}{t}| &= \abs{\lambda_\omega}{x}{\diamond}{|t|} \\
            |\abs{\lambda_\tau}{x}{A}{t}| &= \abs{\lambda_\tau}{x}{|A|}{|t|} \\
            |(x:A) \to_m B| &= (x:|A|) \to_m |B| \\
            |(x:A) \cap B| &= (x:|A|) \cap |B| \\
            |\app{f}{0}{a}| &= |f| \\
            |\app{f}{\omega}{a}| &= \app{|f|}{\omega}{|a|} \\
            |\app{f}{\tau}{a}| &= \app{|f|}{\tau}{|a|}
        \end{align*}
    \end{minipage}%
    \begin{minipage}{0.5\textwidth}
        \begin{align*}
            |\!\diamond\!| &= \diamond \\
            |[t_1, t_2; A]| &= |t_1| \\
            |t.1| &= |t| \\
            |t.2| &= |t| \\
            |x =_A y| &= |x| =_{|A|} |y| \\
            |\text{refl}(t; A)| &= \abs{\lambda_\omega}{x}{\diamond}{x_\star} \\
            %|J(A, P, x, y, e, w)| &= \app{|e|}{\omega}{|w|} \\
            |\psi(e, a, b; A, P)| &= |e| \\
            |\vartheta_1(e, a, b; T)| &= |e| \\
            |\vartheta_2(e, a, b; T)| &= |e| \\
            |\delta(e)| &= |e| \\
            |\varphi(f, e; A, T)| &= \abs{\lambda_\omega}{x}{\diamond}{x_\star}
        \end{align*}
    \end{minipage}
    \caption{Erasure of syntax, for type-like and kind-like syntax erasure is homomorphic, for term-like syntax erasure reduces to the untyped lambda calculus.}
    \label{fig:2:erasure}
\end{figure}


\outline{Introduce the projection from proofs to objects. Note that for type-level stuff the projection is homomorphic. Note that equality evidence is never erased. Mention why this is important.}


\begin{figure}
    \centering
    \begin{minipage}{0.5\textwidth}
        $$\BinderReductionOne$$
    \end{minipage}%
    \begin{minipage}{0.5\textwidth}
        $$\BinderReductionTwo$$
    \end{minipage}%
    $$\NonbinderReduction$$
    \begin{align*}
        \app{(\abs{\lambda_m}{x}{A}{b})}{m}{t} &\betared [x := t]b \\
        [t_1, t_2, A].1 &\betared t_1 \\
        [t_1, t_2, A].2 &\betared t_2 \\
        J(A, P, x, y, \text{refl}(z), w) &\betared \app{w}{0}{z} \\
        \vartheta(\text{refl}(t.1)) &\betared \text{refl}(t) \\
        \vartheta(\text{refl}(t.2)) &\betared \text{refl}(t) \\
    \end{align*}
    \caption{Reduction rules for arbitrary syntax.}
\end{figure}




\outline{Describe reduction, note that it has the same non-axiom rules as F omega. Describe the additional axioms}

\outline{Note that conversion is defined differently than just reduction, define both proof conversion and object conversion}

\begin{lemma}
    If $t_i \betastar t_i^\prime$ for any $i$ then,
    \begin{enumerate}
        \item $\mathfrak{b}(\kappa, (x : t_1), t_2) \betastar \mathfrak{b}(\kappa, (x : t_1^\prime), t_2^\prime)$
        \item $\mathfrak{c}(\kappa, t_1,\ldots,t_{\mathfrak{a}(\kappa)}) \betastar \mathfrak{c}(\kappa, t_1^\prime,\ldots,t_{\mathfrak{a}(\kappa)}^\prime)$
    \end{enumerate}
    \label{lem:2:beta_par}
\end{lemma}
\begin{proof}
    Pick any $i$ and apply the reductions to the associate subterm.
    A straightforward induction on $t_i \betastar t_i^\prime$ demonstrates that the reductions apply only to the associated subterm.
    Repeat until all $i$ reductions are applied.
\end{proof}

\begin{lemma}[Confluence]
    If $s \betastar t_1$ and $s \betastar t_2$ then $\exists\ t^\prime$ such that $t_1 \betastar t^\prime$ and $t_2 \betastar t^\prime$
\end{lemma}
\begin{proof}
    See Appendix~\ref{ap:a}.
\end{proof}

\begin{lemma}
    For any $s$ and $t$ the relation $s \betaconv t$ is an equivalence.
    \label{lem:2:beta_conv_equivalence}
\end{lemma}
\begin{proof}
    Reflexivity is immediate because $s \betastar s$.
    Symmetry is also immediate because if $s \betaconv t$ then $\exists\ z$ such that $s \betastar z$ and $t \betastar z$, but logical conjunction is commutative.
    Transitivity is a consequence of confluence, see Theorem~\ref{thm:1:trans}.
\end{proof}

\begin{corollary}
    If $t_i \betaconv t_i^\prime$ for any $i$ then,
    \begin{enumerate}
        \item $\mathfrak{b}(\kappa, (x : t_1), t_2) \betaconv \mathfrak{b}(\kappa, (x : t_1^\prime), t_2^\prime)$
        \item $\mathfrak{c}(\kappa, t_1,\ldots,t_{\mathfrak{a}(\kappa)}) \betaconv \mathfrak{c}(\kappa, t_1^\prime,\ldots,t_{\mathfrak{a}(\kappa)}^\prime)$
    \end{enumerate}
    \label{cor:2:conv_congr}
\end{corollary}

\outline{Note how confluence is \textbf{not} enough to prove transitivity of conversion}


\Rule{\BinderPseObj}
    {
        \der{\D{1}}{t_1 \pseobj} \\
        \der{\D{2}}{t_2 \pseobj} \\
        \der{\D{3}}{\kappa \neq \lambda_0}
    }
    {\mathfrak{b}(\kappa, x : t_1, t_2) \pseobj}
    {}

\Rule{\LambdaPseObj}
    {
        \der{\D{1}}{A\pseobj} \\
        \der{\D{2}}{t \pseobj} \\
        \der{\D{3}}{x \notin FV(|t|)}
    }
    {\abs{\lambda_0}{x}{A}{t} \pseobj}
    {}

\Rule{\CtorPseObj}
    {
        \der{\D{1}}{\forall\ i \in {1, \ldots, \mathfrak{a}(\kappa)}.\ t_i \pseobj} \\
        \der{\D{2}}{\kappa \neq \text{pair}}
    }
    {\mathfrak{c}(\kappa, t_1, \ldots, t_{\mathfrak{a}(\kappa)}) \pseobj}
    {}

\Rule{\PairPseObj}
    {
        \der{\D{1}}{t_1 \pseobj} \\
        \der{\D{2}}{t_2 \pseobj} \\
        \der{\D{3}}{t_3 \pseobj} \\
        \der{\D{4}}{|t_1| \betaconv |t_2|}
    }
    {[t_1, t_2; t_3] \pseobj}
    {}

\begin{figure}
    \centering
    \begin{minipage}{0.5\textwidth}
        $$\BinderPseObj$$
        $$\LambdaPseObj$$
    \end{minipage}%
    \begin{minipage}{0.5\textwidth}
        $$\CtorPseObj$$
        $$\PairPseObj$$
    \end{minipage}%
    $$x_K \pseobj$$
    \caption{Definition of Pseudo Objects.}
    \label{fig:2:pseobj}
\end{figure}


\outline{Introduce the notion of pseudo objects}

\begin{lemma}
    If $s\pseobj$ and $s \betared t$ then $|s| \betaconv |t|$
    \label{lem:2:erase_pseobj_red}
\end{lemma}
\begin{proof}
    By induction on $s\pseobj$.

    $\text{Case: }\begin{array}{c} \BinderPseObj[*] \end{array}$
    \begin{proofcase}
        By cases on $s \betared t$, applying the IH and Corollary~\ref{cor:2:conv_congr}.
    \end{proofcase}

    $\text{Case: }\begin{array}{c} \LambdaPseObj[*] \end{array}$
    \begin{proofcase}
        By cases on $s \betared t$, applying the IH and Corollary~\ref{cor:2:conv_congr}.
    \end{proofcase}

    $\text{Case: }\begin{array}{c} \CtorPseObj[*] \end{array}$
    \begin{proofcase}
        By cases on $s \betared t$.

        $\text{Case: }\begin{array}{c} \app{(\abs{\lambda_m}{x}{A}{b})}{m}{t} \betared [x := t]b \end{array}$
        \begin{proofcase}
            Note that $\abs{\lambda_m}{x}{A}{b}\pseobj$.
            If $m = 0$ then $x \notin FV(b)$ and $|[x := t]b| = |b|$.
            Thus, $|\app{(\abs{\lambda_0}{x}{A}{b})}{0}{t}| = |\abs{\lambda_0}{x}{A}{b}| = |b|$.
            If $m = \omega$, then $|\app{(\abs{\lambda_\omega}{x}{A}{b})}{\omega}{t}| = \app{(\absu{\lambda_\omega}{x}{b})}{\omega}{|t|}$.
            By definition of reduction $\app{(\absu{\lambda_\omega}{x}{b})}{\omega}{|t|} \betaconv [x := |t|]|b|$.
            Finally, by Lemma~\ref{lem:2:erase_subst} the goal is obtained.
            The case of $m = \tau$ is almost exactly the same.
        \end{proofcase}

        $\text{Case: }\begin{array}{c} [t_1, t_2; A].1 \betared t_1 \end{array}$
        \begin{proofcase}
            $|[t_1, t_2; A].1| = |[t_1, t_2; A]| = |t_1|$
        \end{proofcase}

        $\text{Case: }\begin{array}{c} [t_1, t_2; A].2 \betared t_2 \end{array}$
        \begin{proofcase}
            Observe that $|[t_1, t_2; A].2| = |t_1|$ and $[t_1, t_2; A]\pseobj$. \\
            Thus, $|s| = |t_1| \betaconv |t_2|$.
        \end{proofcase}

        $\text{Case: }\begin{array}{c} \psi(\text{refl}(t),P) \betared \abs{\lambda_\omega}{x}{\app{P}{\tau}{t}}{x} \end{array}$
        \begin{proofcase}
            $|\psi(\text{refl}(t),P)| = |\text{refl}(t)| = \absu{\lambda_\omega}{x}{x} = |\abs{\lambda_\omega}{x}{\app{P}{\tau}{t}}{x}|$
        \end{proofcase}

        $\text{Case: }\begin{array}{c} \vartheta_1(\text{refl}(t_1), t_2, t_3) \betared \text{refl}(t_2) \end{array}$
        \begin{proofcase}
            $|\vartheta_1(\text{refl}(t_1), t_2, t_3)| = |\text{refl}(t_1)| = \absu{\lambda_\omega}{x}{x} = |\text{refl}(t_2)|$
        \end{proofcase}

        $\text{Case: }\begin{array}{c} \vartheta_2(\text{refl}(t_1), t_2, t_3) \betared \text{refl}(t_2) \end{array}$
        \begin{proofcase}
            Same as previous case.
        \end{proofcase}

        $\text{Case: }\begin{array}{c} \NonbinderReduction[*] \end{array}$
        \begin{proofcase}
            By the IH, $|t_i| \betaconv |t_i^\prime|$. The goal is achieved by Corollary~\ref{cor:2:conv_congr}
        \end{proofcase}

    \end{proofcase}

    $\text{Case: }\begin{array}{c} \PairPseObj[*] \end{array}$
    \begin{proofcase}
        By cases on $s \betared t$, applying the IH and Corollary~\ref{cor:2:conv_congr}.
    \end{proofcase}

    $\text{Case: }\begin{array}{c} s\text{ variable} \end{array}$
    \begin{proofcase}
        By cases on $s \betared t$, $t$ must be a variable.
        Thus, $|s| = |t|$.
    \end{proofcase}
\end{proof}

\begin{lemma}
    If $s\pseobj$, $|s| \betaconv |b|$, and $s \betared t$ then $|t| \betaconv |b|$
    \label{lem:2:pseobj_red_f_step}
\end{lemma}
\begin{proof}
    By Lemma~\ref{lem:2:erase_pseobj_red} $|s| \betaconv |t|$ and by Lemma~\ref{lem:2:beta_conv_equivalence} $|t| \betaconv |b|$.
\end{proof}

\begin{lemma}
    If $b\pseobj$ and $t\pseobj$ then $[x := t]b\pseobj$
    \label{lem:2:pseobj_subst}
\end{lemma}
\begin{proof}
    By induction on $b\pseobj$. The $\lambda_0$ and pair cases are no different from the respective $\mathfrak{b}$ and $\mathfrak{c}$ cases.

    $\text{Case: }\begin{array}{c} \BinderPseObj[*] \end{array}$
    \begin{proofcase}
        By the IH $[x := t]t_1\pseobj$ and $[x := t]t_2\pseobj$.
        Thus, $\mathfrak{b}(\kappa, (y : [x := t]t_1), [x := t]t_2)\pseobj$.
    \end{proofcase}

    $\text{Case: }\begin{array}{c} \CtorPseObj[*] \end{array}$
    \begin{proofcase}
        By the IH $[x := t]t_i\pseobj$. \\
        Thus, $\mathfrak{c}(\kappa, [x := t]t_1, \ldots [x := t]t_{\mathfrak{a}(\kappa)})\pseobj$.
    \end{proofcase}

    $\text{Case: }\begin{array}{c} s\text{ variable} \end{array}$
    \begin{proofcase}
        If $s = x$ then $[x := t]x = t$, and $t\pseobj$.
        Otherwise, $s = y$ with $y$ a variable and $y\pseobj$.
    \end{proofcase}
\end{proof}

\begin{lemma}
    If $s\pseobj$ and $s \betared t$ then $t\pseobj$
    \label{lem:2:pseobj_preservation_step}
\end{lemma}
\begin{proof}
    By induction on $s\pseobj$.

    $\text{Case: }\begin{array}{c} \BinderPseObj[*] \end{array}$
    \begin{proofcase}
        By cases on $s \betared t$. Suppose w.l.o.g. that $t_2 \betared t_2^\prime$.
        Observe that $t_2\pseobj$ because it is a subterm of $s$.
        Then by the IH $t_2^\prime\pseobj$.
        Thus, $\mathfrak{b}(\kappa, x : t_1, t_2^\prime) \pseobj$.
    \end{proofcase}

    $\text{Case: }\begin{array}{c} \LambdaPseObj[*] \end{array}$
    \begin{proofcase}
        By cases on $s \betared t$. Suppose w.l.o.g that $t \betared t^\prime$.
        Note that if $x \notin FV(|t|)$ then $x \notin FV(|t^\prime|)$, reduction only reduces the amount of free variables.
        Observe that $t \pseobj$.
        Then by the IH $t^\prime\pseobj$.
        Thus, $\abs{\lambda_0}{x}{A}{t^\prime}\pseobj$.
    \end{proofcase}

    $\text{Case: }\begin{array}{c} \CtorPseObj[*] \end{array}$
    \begin{proofcase}
        By cases on $s \betared t$.

        $\text{Case: }\begin{array}{c} \app{(\abs{\lambda_m}{x}{A}{b})}{m}{t} \betared [x := t]b \end{array}$
        \begin{proofcase}
            Observe that $b\pseobj$ and $t\pseobj$ because both are subterms of $s$.
            By Lemma~\ref{lem:2:pseobj_subst} $[x := t]b\pseobj$.
        \end{proofcase}

        $\text{Case: }\begin{array}{c} [t_1, t_2; A].1 \betared t_1 \end{array}$
        \begin{proofcase}
            Observe that $t_1\pseobj$ because it is a subterm of $s$.
        \end{proofcase}

        $\text{Case: }\begin{array}{c} [t_1, t_2; A].2 \betared t_2 \end{array}$
        \begin{proofcase}
            Observe that $t_2\pseobj$.
        \end{proofcase}

        $\text{Case: }\begin{array}{c} \psi(\text{refl}(t),P) \betared \abs{\lambda_\omega}{x}{\app{P}{\tau}{t}}{x} \end{array}$
        \begin{proofcase}
            Observe that $t\pseobj$ and $P\pseobj$.
            By application of constructor and binder rules $\abs{\lambda_\omega}{x}{\app{P}{\tau}{t}}{x}\pseobj$.
        \end{proofcase}

        $\text{Case: }\begin{array}{c} \vartheta_1(\text{refl}(t_1), t_2, t_3) \betared \text{refl}(t_2) \end{array}$
        \begin{proofcase}
            Observe that $t_2\pseobj$.
            By application of constructor rule $\text{refl}(t_2)\pseobj$.
        \end{proofcase}

        $\text{Case: }\begin{array}{c} \vartheta_2(\text{refl}(t_1), t_2, t_3) \betared \text{refl}(t_2) \end{array}$
        \begin{proofcase}
            Same as previous case.
        \end{proofcase}

        $\text{Case: }\begin{array}{c} \NonbinderReduction[*] \end{array}$
        \begin{proofcase}
            By the IH $t_i^\prime\pseobj$.
            By application of the constructor rule the goal is obtained.
        \end{proofcase}

    \end{proofcase}

    $\text{Case: }\begin{array}{c} \PairPseObj[*] \end{array}$
    \begin{proofcase}
        By cases on $s \betared t$.
        Suppose w.l.o.g. $t_1 \betared t_1^\prime$.
        Note that $t_1\pseobj$ because it is a subterm of $s$.
        By the IH $t_1^\prime\pseobj$.
        By Lemma~\ref{lem:2:pseobj_red_f_step} $|t_1^\prime| \betaconv |t_2|$.
        Thus, $[t_1^\prime, t_2; A]\pseobj$.
    \end{proofcase}

    $\text{Case: }\begin{array}{c} s\text{ variable} \end{array}$
    \begin{proofcase}
        By cases on $s \betared t$, $t$ must be a variable.
        Thus, $t\pseobj$.
    \end{proofcase}
\end{proof}

\begin{lemma}
    If $s\pseobj$, $|s| \betaconv |b|$, and $s \betastar t$ then $|t| \betaconv |b|$
    \label{lem:2:pseobj_red_f}
\end{lemma}
\begin{proof}
    By induction on $s \betastar t$.
    The reflexivity case is trivial.
    The transitivity case is obtained from Lemma~\ref{lem:2:pseobj_red_f_step} and Lemma~\ref{lem:2:pseobj_preservation_step}
    and applying the IH.
\end{proof}

\begin{theorem}
    If $s\pseobj$ and $s \betastar t$ then $t\pseobj$
    \label{lem:2:pseobj_preservation}
\end{theorem}
\begin{proof}
    By induction on $s \betastar t$.
    The reflexivity case is trivial.
    The transitivity case is obtained from Lemma~\ref{lem:2:pseobj_preservation_step} and applying the IH.
\end{proof}

\begin{lemma}
    If $s\pseobj$, $|t| \betaconv |b|$, and $s \betastar t$ then $|s| \betaconv |b|$
    \label{lem:2:pseobj_red_b}
\end{lemma}
\begin{proof}
    By induction on $s \betastar t$.
    Consequence of Lemma~\ref{lem:2:erase_pseobj_red} and Lemma~\ref{lem:2:pseobj_preservation}.
\end{proof}

\begin{lemma}
    If $s\pseobj$, $s \equiv b$, and $s \betastar t$ then $t \equiv b$
    \label{lem:2:conv_red_f}
\end{lemma}
\begin{proof}
    Note that $\exists\ z_1, z_2$ such that $s \betastar z_1$, $b \betastar z_2$, and $|z_1| \betaconv |z_2|$.
    By confluence $\exists\ z_1^\prime$ such that $z_1 \betastar z_1^\prime$ and $t \betastar z_1^\prime$.
    Then, by Lemma~\ref{lem:2:pseobj_preservation} $z_1\pseobj$.
    Finally, by Lemma~\ref{lem:2:pseobj_red_f} $|z_1^\prime| \betaconv |z_2|$.
    Therefore, $t \equiv b$.
\end{proof}

\begin{theorem}
    If $b\pseobj$, $a \equiv b$, and $b \equiv c$ then $a \equiv c$
    \label{thm:2:conv_trans}
\end{theorem}
\begin{proof}
    Note that $\exists\ u_1, u_2$ such that $a \betastar u_1$, $b \betastar u_2$, and $|u_1| \betaconv |u_2|$.
    Additionally, $\exists\ v_1, v_2$ such that $b \betastar v_1$, $c \betastar v_2$, and $|v_1| \betaconv |v_2|$.
    By confluence, $\exists\ z$ such that $u_2 \betastar z$ and $v_1 \betastar z$.
    Then, by Lemma~\ref{lem:2:pseobj_preservation} $u_2\pseobj$ and $v_1\pseobj$.
    Next, by Lemma~\ref{lem:2:pseobj_red_f} $|u_1| \betaconv |z|$ and $|z| \betaconv |v_2|$.
    Thus, $|u_1| \betaconv |v_2|$ by Lemma~\ref{lem:2:beta_conv_equivalence} and $a \equiv c$.
\end{proof}

\begin{theorem}
    Suppose $s\pseobj$ and $t\pseobj$, then $|s| \betaconv |t|$ iff $s \equiv t$
    \label{thm:2:beta_iff_conv}
\end{theorem}
\begin{proof}
    Case $(\Rightarrow)$:
    Suppose $|s| \betaconv |t|$.
    By definition $s \betastar s$ and $t \betastar t$.
    Thus, $s \equiv t$.
    Case $(\Leftarrow)$:
    Suppose $s \equiv t$, then $\exists\ z_1, z_2$ such that $s \betastar z_1$, $t \betastar z_2$, and $|z_1| \betaconv |z_2|$.
    By two applications of Lemma~\ref{lem:2:pseobj_red_b} $|s| \betaconv |t|$.
\end{proof}

\begin{lemma}
    If $a\pseobj$, $b\pseobj$, and $a \equiv b$ then $[x := t]a \equiv [x := t]b$
    \label{lem:2:conv_subst}
\end{lemma}
\begin{proof}
    By Lemma~\ref{thm:2:beta_iff_conv} $|a| \betaconv |b|$.
    Then, by Lemma~\ref{lem:2:betaconv_erased_subst} $|[x := t]a| \betaconv |[x := t]b|$.
    Finally, by Lemma~\ref{thm:2:beta_iff_conv} again, $[x := t]a \equiv [x := t]b$.
\end{proof}

\outline{Introduce the typing rules, explain a lot of them in detail in the prose}


\begin{figure}
    \centering
    \begin{minipage}{0.5\textwidth}
        \begin{align*}
            \text{dom}_\Pi(\omega, K) &= \star \\
            \text{dom}_\Pi(\tau, K) &= K \\
            \text{dom}_\Pi(0, K) &= K
        \end{align*}
    \end{minipage}%
    \begin{minipage}{0.5\textwidth}
        \begin{align*}
            \text{codom}_\Pi(\omega) &= \star \\
            \text{codom}_\Pi(\tau) &= \kind \\
            \text{codom}_\Pi(0) &= \star
        \end{align*}
    \end{minipage}
    \caption{Domain and codomains for function types. The metavariable $K$ is either $\star$ or $\kind$.}
\end{figure}


\begin{figure}
    \centering
    \begin{minipage}{0.5\textwidth}
        $$\AxiomRule$$
        $$\HeadInferenceRule$$
        $$\ContextEmptyRule$$
    \end{minipage}%
    \begin{minipage}{0.5\textwidth}
        $$\VarRule$$
        $$\CheckRule$$
        $$\ContextAppendRule$$
    \end{minipage}%
    $$\PiRule$$
    $$\LambdaRule$$
    $$\AppRule$$
    \caption{
        Inference rules for function types, including erased functions.
        The variable $K$ is either $\star$ or $\kind$.
    }
\end{figure}


\begin{figure}
    \centering
    \begin{minipage}{0.5\textwidth}
        $$\IntersectionRule$$
        $$\FirstRule$$
    \end{minipage}%
    \begin{minipage}{0.5\textwidth}
        $$\PairRule$$
        $$\SecondRule$$
    \end{minipage}
    \caption{
        Inference rules for intersection types.
    }
\end{figure}


\begin{figure}
    \centering
    \begin{minipage}{0.5\textwidth}
        $$\EqualityRule$$
    \end{minipage}%
    \begin{minipage}{0.5\textwidth}
        $$\ReflRule$$
    \end{minipage}%
    $$\SubstRule$$
    $$\PromoteRule$$
    $$\CastRule$$
    $$\SeparationRule$$
    \caption{
        Inference rules for equality types where
        $\text{cBool} := (X : \star) \to_0 (x : X_\square) \to_\omega (y : X_\square) \to_\omega X_\square$;
        $\text{ctt} := \abs{\lambda_0}{X}{\star}{
            \abs{\lambda_\omega}{x}{X_\square}{
                \abs{\lambda_\omega}{y}{X_\square}{x_\star}
            }
        }$;
        and
        $\text{cff} := \abs{\lambda_0}{X}{\star}{
            \abs{\lambda_\omega}{x}{X_\square}{
                \abs{\lambda_\omega}{y}{X_\square}{y_\star}
            }
        }$.
    }
    \label{fig:2:typ3}
\end{figure}


\begin{lemma}
    \textcolor{white}{\_}
    \begin{enumerate}
        \item If $\Gamma \vdash t \infr A$ then $t\pseobj$
        \item If $\Gamma \vdash t \cinfr A$ then $t\pseobj$
        \item If $\Gamma \vdash t \chck A$ then $t\pseobj$
    \end{enumerate}
    \label{lem:2:infer_is_pseobj}
\end{lemma}
\begin{proof}
    By mutual induction.
    There are only two non-trivial, the rest hold immediately by the IH and applying the associated rule.
    Well-formed context cases are omitted because they are not used in the induction.

    $\text{Case: }\begin{array}{c}\begin{minipage}{0.7\textwidth}$\PairRule[*]$ \end{minipage}\end{array}$
    \begin{proofcase}
        By the IH $t\pseobj$, $s\pseobj$, and $T\pseobj$.
        By Theorem~\ref{thm:2:beta_iff_conv} $|t| \betaconv |s|$.
        Applying the pair rule yields $[t,s;T]\pseobj$.
    \end{proofcase}

    $\text{Case: }\begin{array}{c}\begin{minipage}{0.7\textwidth} $\LambdaRule[*]$ \end{minipage}\end{array}$
    \begin{proofcase}
        Consider only the case when $m = 0$ as the other cases are trivial.
        By the IH $A\pseobj$ and $t\pseobj$.
        Because $m = 0$, $x \notin FV(|t|)$.
        Thus, applying the lambda rule yields $\abs{\lambda_0}{x}{A}{t}\pseobj$.
    \end{proofcase}
\end{proof}

\outline{Define the lifting of reduction to contexts}

\begin{lemma}
    If $\Gamma \vdash t \infr A$ or $\Gamma \vdash t \cinfr A$ or $\Gamma \vdash t \chck A$ then $\vdash \Gamma$
    \label{lem:2:ctx_wf}
\end{lemma}
\begin{proof}
    Straightforward by induction, all leaves of a derivation have $\vdash \Gamma$ as a premise.
\end{proof}

\begin{lemma}[Weakening]
    Suppose $\Gamma \vdash B \cinfr K$ for $K = \kind$ or $K = \star$.
    \begin{enumerate}
        \item If $\Gamma, \Delta \vdash t \infr A$ then $\Gamma, x : B, \Delta \vdash t \infr A$
        \item If $\Gamma, \Delta \vdash t \cinfr A$ then $\Gamma, x : B, \Delta \vdash t \cinfr A$
        \item If $\Gamma, \Delta \vdash t \chck A$ then $\Gamma, x : B, \Delta \vdash t \chck A$
        \item if $\vdash \Gamma, \Delta$ then $\vdash \Gamma, x : B, \Delta$
    \end{enumerate}
    \label{lem:2:weaken}
\end{lemma}
\begin{proof}
    By mutual induction.
    Omitted cases follow immediately from application of the IH to all subderivations and applying the associated rule of the case.

    $\text{Case: }\begin{array}{c} \AxiomRule[*] \end{array}$
    \begin{proofcase}
        By the IH on $\D{1}$: $\vdash \Gamma, x : B, \Delta$.
        Applying the \textsc{Axiom} rule concludes this case.
    \end{proofcase}

    $\text{Case: }\begin{array}{c} \VarRule[*] \end{array}$
    \begin{proofcase}        
        By the IH on $\D{1}$: $\vdash \Gamma, x : B, \Delta$.
        If $(y : A) \in \Gamma$ then clearly $(y : A) \in \Gamma, x : B, \Delta$.
        Thus, applying the \textsc{Var} rule concludes.
    \end{proofcase}

    $\text{Case: }\begin{array}{c} \PiRule[*] \end{array}$
    \begin{proofcase}
        Applying the IH to $\D{2}$ gives $\Gamma, x : B, \Delta, y : A \vdash B \cinfr \pcodom(m)$.
        Notice that $\Delta$ is generalized in the induction hypothesis, thus allowing for the capture of the additional context assumption introduced in $\D{2}$.
        Applying the IH to $\D{1}$ and the \textsc{Pi} rule concludes.
    \end{proofcase}

    $\text{Case: }\begin{array}{c} \ContextEmptyRule[*] \end{array}$
    \begin{proofcase}
        By the \textsc{CtxEm} rule.
    \end{proofcase}

    $\text{Case: }\begin{array}{c} \ContextAppendRule[*] \end{array}$
    \begin{proofcase}
        By the IH on $\D{2}$: $\vdash \Gamma, x : B, \Delta$.
        By the IH on $\D{3}$: $\Gamma, x : B, \Delta \vdash A \cinfr K$.
        Recall that renaming is applied implicitly to preserve meaning.
        In this case, either $\Gamma$ and $\Delta$ could be renamed so that $x$ is unique, or $x$ itself renamed so that $x \notin FV(\Gamma, \Delta)$.
        Either way some conclusion of the following form is obtained: Pick $y \notin FV(\Gamma, \Delta)$ and $y \neq x$.
        Thus, $\vdash \Gamma, x : B, \Delta, y : A$.
    \end{proofcase}
\end{proof}

\begin{lemma}
    If $\vdash \Gamma$ and $(x : A) \in \Gamma$ then $\exists\ K$ such that $\Gamma \vdash A \cinfr K$, where $K = \kind$ or $K = \star$
    \label{lem:2:ctx_get}
\end{lemma}
\begin{proof}
    By definition, $\exists\ \Delta_1, \Delta_2$ such that $\Gamma = \Delta_1, x : A, \Delta_2$ and $\Delta_1 \vdash A \cinfr K$.
    By weakening, $\Delta_1, x : A \vdash A \cinfr K$.
    Induction on $\Delta_2$ and repeated application of weakening concludes the proof.
\end{proof}

\begin{lemma}
    \textcolor{white}{\_}
    \begin{enumerate}
        \item If $\Gamma \vdash t \infr A$ then $A\pseobj$
        \item If $\Gamma \vdash t \cinfr A$ then $A\pseobj$
        \item If $\Gamma \vdash t \chck A$ then $A\pseobj$
    \end{enumerate}
    \label{lem:2:infer_type_is_pseobj}
\end{lemma}
\begin{proof}
    By mutual induction.
    Cases where $A\pseobj$ trivially by application of a finite number of rules are omitted.
    The well-formed context cases are omitted because they are not used in the mutual induction.

    $\text{Case: }\begin{array}{c} \VarRule[*] \end{array}$
    \begin{proofcase}
        By Lemma~\ref{lem:2:ctx_get} and Lemma~\ref{lem:2:infer_is_pseobj}.
    \end{proofcase}

    $\text{Case: }\begin{array}{c} \LambdaRule[*] \end{array}$
    \begin{proofcase}
        Applying the \textsc{Pi} rule with $\D{1}$ and $\D{3}$ gives $\Gamma \vdash (x : A) \to_m B \infr \pcodom(m)$.
        By Lemma~\ref{lem:2:infer_is_pseobj} the case concludes.
    \end{proofcase}

    $\text{Case: }\begin{array}{c} \AppRule[*] \end{array}$
    \begin{proofcase}
        Deconstruction and inversion on $\D{1}$ yields the judgment $\Gamma, x : A \vdash B \cinfr \pcodom(m)$.
        Thus, by Lemma~\ref{lem:2:infer_is_pseobj}: $B\pseobj$ and $a\pseobj$.
        Finally, by Lemma~\ref{lem:2:pseobj_subst} $[x := a]B\pseobj$.
    \end{proofcase}

    $\text{Case: }\begin{array}{c} \PairRule[*] \end{array}$
    \begin{proofcase}
        Applying the \textsc{Int} rule with $\D{2}$ and $\D{3}$ gives $\Gamma \vdash (x : A) \cap B \infr \star$.
        Now by Lemma~\ref{lem:2:infer_is_pseobj} the case concludes.
    \end{proofcase}

    $\text{Case: }\begin{array}{c} \FirstRule[*] \end{array}$
    \begin{proofcase}
        By the IH $(x : A) \cap B \pseobj$ which means $A\pseobj$.
    \end{proofcase}

    $\text{Case: }\begin{array}{c} \SecondRule[*] \end{array}$
    \begin{proofcase}
        By the IH $(x : A) \cap B \pseobj$ which means $B\pseobj$.
        The \textsc{Fst} rule applied to $\D{1}$ gives $\Gamma \vdash t.1 \infr A$.
        By Lemma~\ref{lem:2:infer_is_pseobj}: $t.1\pseobj$.
        Finally, by Lemma~\ref{lem:2:pseobj_subst} $[x := t.1]B\pseobj$.
    \end{proofcase}

    $\text{Case: }\begin{array}{c} \ReflRule[*] \end{array}$
    \begin{proofcase}
        By Lemma~\ref{lem:2:infer_is_pseobj} $t\pseobj$ and $A\pseobj$.
        Applying the constructor rule concludes the case.
    \end{proofcase}

    $\text{Case: }\begin{array}{c} \SubstRule[*] \end{array}$
    \begin{proofcase}
        By the IH $a =_A b\pseobj$ which means $a\pseobj$, $b\pseobj$, and $A\pseobj$.
        By Lemma~\ref{lem:2:infer_is_pseobj}: $P\pseobj$.
        The constructor rule yields $\app{P}{\tau}{a}\pseobj$ and $\app{P}{\tau}{b}\pseobj$.
        The binder rule concludes the case.
    \end{proofcase}

    $\text{Case: }\begin{array}{c} \PromoteFstRule[*] \end{array}$
    \begin{proofcase}
        By the IH $a.1 =_A b.1\pseobj$ and $(x : A) \cap B\pseobj$. Which means $a\pseobj$ and $b\pseobj$.
        Applying the constructor rule finishes the case.
    \end{proofcase}

    $\text{Case: }\begin{array}{c} \PromoteSndRule[*] \end{array}$
    \begin{proofcase}
        By the IH $a.2 =_A b.2\pseobj$ and $(x : A) \cap B\pseobj$. Which means $a\pseobj$ and $b\pseobj$.
        Applying the constructor rule finishes the case.
    \end{proofcase}

    $\text{Case: }\begin{array}{c} \CastRule[*] \end{array}$
    \begin{proofcase}
        Immediate by the IH applied to $\D{1}$.
    \end{proofcase}

    $\text{Case: }\begin{array}{c} \HeadInferenceRule[*] \end{array}$
    \begin{proofcase}
        By the IH applied to $\D{1}$: $A\pseobj$.
        By Lemma~\ref{lem:2:pseobj_preservation}: $B\pseobj$.
    \end{proofcase}

    $\text{Case: }\begin{array}{c} \CheckRule[*] \end{array}$
    \begin{proofcase}
        By Lemma~\ref{lem:2:infer_is_pseobj} on $\D{2}$: $B\pseobj$.
    \end{proofcase}
\end{proof}

\begin{lemma}[Substitution of Inference]
    Suppose $\Gamma \vdash b \infr B$.
    \begin{enumerate}
        \item If $\Gamma, x : B, \Delta \vdash t \infr A$ then $\Gamma, [x := b]\Delta \vdash [x := b]t \infr [x := b]A$
        \item If $\Gamma, x : B, \Delta \vdash t \cinfr A$ then $\Gamma, [x := b]\Delta \vdash [x := b]t \cinfr [x := b]A$
        \item If $\Gamma, x : B, \Delta \vdash t \chck A$ then $\Gamma, [x := b]\Delta \vdash [x := b]t \chck [x := b]A$
        \item If $\vdash \Gamma, x : B, \Delta$ then $\vdash \Gamma, [x := b]\Delta$
    \end{enumerate}
    \label{lem:2:subst_infer}
\end{lemma}
\begin{proof}
    By mutual induction.
    Omitted cases are obtained by applying the IH to all subderivations and applying the associated rule.

    $\text{Case: }\begin{array}{c} \AxiomRule[*] \end{array}$
    \begin{proofcase}
        Applying the IH to $\D{1}$ gives $\vdash \Gamma, [x := b]\Delta$.
        Using the \textsc{Axiom} rule concludes the case.
    \end{proofcase}

    $\text{Case: }\begin{array}{c} \VarRule[*] \end{array}$
    \begin{proofcase}
        Rename the case to $\Gamma \vdash y \infr A$.
        By the IH applied to $\D{1}$: $\vdash \Gamma, [x := b]\Delta$.
        Suppose $y \neq x$.
        Then $(y : A) \in \Gamma, x : B, \Delta$ implies $(y : A) \in \Gamma, [x := b]\Delta$.
        Applying the \textsc{Var} rule concludes this case.
        Suppose $y = x$.
        Then $[x := b]y = b$ and $A = B$.
        Recall that $\Gamma \vdash b \infr B$.
        It must be the case that $x \notin FV(\Gamma)$ by $\D{1}$.
        Thus, $x \notin FV(B)$ and $[x := b]B = B$.
        Finally, by weakening $\Gamma, [x := b]\Delta \vdash b \infr B$.
    \end{proofcase}

    $\text{Case: }\begin{array}{c} \PiRule[*] \end{array}$
    \begin{proofcase}
        By the IH applied to $\D{1}$: $\Gamma, [x := b]\Delta \vdash [x := b]A \cinfr \pdom(m, K)$.
        By the IH applied to $\D{2}$: $\Gamma, [x := b]\Delta, y : [x := b]A \vdash [x := b]B \cinfr \pcodom(m)$.
        Applying the \textsc{Pi} rule gives $\Gamma, [x := b]\Delta \vdash (y : [x := b]A) \to_m [x := b]B \infr \pcodom(m)$.
        Folding substitution concludes the case.
    \end{proofcase}

    $\text{Case: }\begin{array}{c} \LambdaRule[*] \end{array}$
    \begin{proofcase}
        As with the previous case, applying the IH to $\D{1}$, $\D{2}$, and $\D{3}$ yield the necessary subderivations to build the desired term with the \textsc{Lam} rule.
        Note that the substituted variable, $x$, cannot be equal to the bound variable, $y$, by implicit renaming.
        Moreover, $y \notin FV([x := b]|t|)$ if $m = 0$.
    \end{proofcase}

    $\text{Case: }\begin{array}{c} \AppRule[*] \end{array}$
    \begin{proofcase}
        Applying the IH gives, for $\D{1}$: $\Gamma, [x := b]\Delta \vdash [x := b]f \cinfr (y : [x := b]A) \to_m [x := b]B$, and for $\D{2}$: $\Gamma, [x := b]\Delta \vdash [x := b]a \chck [x := b]A$.
        Thus, applying the \textsc{App} rule gives $\Gamma, [x := b]\Delta \vdash \app{([x := b]f)}{m}{([x := b]a)} \infr [y := [x := b]a][x := b]B$.
        By Lemma~\ref{lem:2:subst_commute} $[y := [x := b]a][x := b] = [x := b][y := a]$.
        Thus, $\Gamma, [x := b]\Delta \vdash [x := b](\app{f}{m}{a}) \infr [x := b][y := a]B$.
    \end{proofcase}

    $\text{Case: }\begin{array}{c} \PairRule[*] \end{array}$
    \begin{proofcase}
        Applying the IH to $\D{5}$ gives $\Gamma, [x := b]\Delta \vdash [x := b]s \chck [x := b][y := t]B$.
        Notice that the bound variable is renamed to $y$ implicitly.
        By Lemma~\ref{lem:2:subst_commute} $[x := b][y := t]B = [y := [x := b]t][x := b]B$.
        Applying Lemma~\ref{lem:2:infer_is_pseobj} to $\D{4}$ and $\D{5}$ gives $t\pseobj$ and $s\pseobj$.
        Then, by Lemma~\ref{lem:2:conv_subst} applied to $\D{6}$: $[x := b]t \equiv [x := b]s$.
        The IH applied to the remaining subderivations and the application of the \textsc{Pair} rule concludes the case.
    \end{proofcase}

    $\text{Case: }\begin{array}{c} \SecondRule[*] \end{array}$
    \begin{proofcase}
        Applying the IH to $\D{1}$ gives $\Gamma, [x := b]\Delta \vdash [x := b]t \cinfr (y : [x := b]A) \cap [x := b]B$.
        The \textsc{Snd} rule and some folding of substitution yields $\Gamma, [x := b]\Delta \vdash [x := b]t.2 \infr [y := [x := b]t.1][x := b]B$.
        Finally, by Lemma~\ref{lem:2:subst_commute} $[y := [x := b]t.1][x := b]B = [x := b][y := t.1]B$.
    \end{proofcase}

    $\text{Case: }\begin{array}{c} \CastRule[*] \end{array}$
    \begin{proofcase}
        Deconstructing $\D{1}$ and using inversion yields $\Gamma \vdash A \cinfr \star$ and $\Gamma \vdash A^\prime \cinfr \star$.
        Applying Lemma~\ref{lem:2:infer_is_pseobj} means that $A\pseobj$ and $A^\prime\pseobj$.
        Then, by Lemma~\ref{lem:2:conv_subst} applied to $\D{2}$: $[x := b]A \equiv [x := b]A^\prime$.
        To see why $FV(|[x := b]e|)$ is empty, consider that if $x$ is in a free position in $e$, then necessarily $x \in FV(|e|)$, which is not true.
        Thus, $x$ can only be in an erased position in $e$, but then $|[x := b]e| = |e|$, because $x$ is erased, so its substituted term must also be erased.
        Applying the IH to $\D{1}$ and $\D{3}$ and using the \textsc{Cast} rule concludes the case.
    \end{proofcase}

    $\text{Case: }\begin{array}{c} \SeparationRule[*] \end{array}$
    \begin{proofcase}
        By the IH applied to $\D{1}$: $\Gamma, [x := b]\Delta \vdash [x := b]e \chck [x := b]\text{ctt} =_{[x := b]\text{cBool}} [x := b]\text{cff}$.
        However, ctt, cff and cBool are all closed terms, thus $[x := b]\text{ctt} = \text{ctt}$, etc.
        Moreover, $[x := b]((X : \star) \to_0 X) = (X : \star) \to_0 X$.
        Thus, applying the \textsc{Sep} rule concludes.
    \end{proofcase}

    $\text{Case: }\begin{array}{c} \HeadInferenceRule[*] \end{array}$
    \begin{proofcase}
        By Lemma~\ref{lem:2:betastar_subst} $[x := b]A \betastar [x := b]B$.
    \end{proofcase}

    $\text{Case: }\begin{array}{c} \CheckRule[*] \end{array}$
    \begin{proofcase}
        By Lemma~\ref{lem:2:infer_type_is_pseobj}: $A\pseobj$.
        By Lemma~\ref{lem:2:infer_is_pseobj}: $B\pseobj$.
        Then, by Lemma~\ref{lem:2:conv_subst} $[x := b]A \equiv [x := b]B$.
        Note that $[x := b]K = K$ because $K = \star$ or $K = \kind$.
        The case concludes by applying the IH to $\D{1}$ and $\D{2}$.
    \end{proofcase}

    $\text{Case: }\begin{array}{c} \ContextEmptyRule[*] \end{array}$
    \begin{proofcase}
        Impossible by inversion, context is not empty.
    \end{proofcase}

    $\text{Case: }\begin{array}{c} \ContextAppendRule[*] \end{array}$
    \begin{proofcase}
        By the IH applied to $\D{2}$: $\vdash \Gamma, [x := b]\Delta$.
        The IH applied to $\D{3}$: $\Gamma, [x := b]\Delta \vdash [x := b]A \cinfr K$.
        Pick $y \notin FV(\Gamma)$.
        Applying the \textsc{CtxApp} rule gives $\vdash \Gamma, [x := b]\Delta, y : [x := b]A$.
    \end{proofcase}
\end{proof}

% \begin{lemma}[Substitution of Reduction-Inference]
%     Suppose $\Gamma \vdash b \cinfr B$.
%     \begin{enumerate}
%         \item If $\Gamma, x : B, \Delta \vdash t \infr A$ then $\Gamma, [x := b]\Delta \vdash [x := b]t \cinfr [x := b]A$
%         \item If $\Gamma, x : B, \Delta \vdash t \cinfr A$ then $\Gamma, [x := b]\Delta \vdash [x := b]t \cinfr [x := b]A$
%         \item If $\Gamma, x : B, \Delta \vdash t \chck A$ then $\Gamma, [x := b]\Delta \vdash [x := b]t \chck [x := b]A$
%         \item If $\vdash \Gamma, x : B, \Delta$ then $\vdash \Gamma, [x := b]\Delta$
%     \end{enumerate}
%     \label{lem:2:subst_cinfer}
% \end{lemma}

% \begin{lemma}[Substitution of Checking]
%     Suppose $\Gamma \vdash b \chck B$.
%     \begin{enumerate}
%         \item If $\Gamma, x : B, \Delta \vdash t \infr A$ then $\Gamma, [x := b]\Delta \vdash [x := b]t \chck [x := b]A$
%         \item If $\Gamma, x : B, \Delta \vdash t \cinfr A$ then $\Gamma, [x := b]\Delta \vdash [x := b]t \chck [x := b]A$
%         \item If $\Gamma, x : B, \Delta \vdash t \chck A$ then $\Gamma, [x := b]\Delta \vdash [x := b]t \chck [x := b]A$
%         \item If $\vdash \Gamma, x : B, \Delta$ then $\vdash \Gamma, [x := b]\Delta$
%     \end{enumerate}
%     \label{lem:2:subst_check}
% \end{lemma}

% \begin{lemma}
%     If $\Gamma \vdash f \chck (x : A) \to_m B$ and $\Gamma \vdash a \chck A$ then $\Gamma \vdash \app{f}{m}{a} \chck [x := a]B$
%     \label{lem:2:check_app}
% \end{lemma}
% \begin{proof}
%     TODO(buggy)
%     By definition $\exists\ T$ such that $\Gamma \vdash f \infr T$, $\Gamma \vdash (x : A) \to_m B \cinfr K$, and $T \equiv (x : A) \to_m B$.
%     Now by inversion on the internal inference rule for \textsc{Pi} it is the case that $\Gamma \vdash (x : A) \to_m B \infr \pcodom(m)$.
%     Thus, $\Gamma \vdash A \infr \pdom(m, K)$ and $\Gamma, x : A \vdash B \infr \pcodom(m)$.

%     By definition of conversion, $\exists\ u_1, u_2$ such that $T \betastar u_1$, $(x : A) \to_m B \betastar u_2$ and $|u_1| \betaconv |u_2|$.
%     Inversion of reduction means $\exists\ A^\prime, B^\prime$ such that $u_2 = (x : A^\prime) \to_m B^\prime$, but then $|u_2| = |(x : A^\prime) \to_m B^\prime| = (x : |A^\prime|) \to_m |B^\prime|$.
%     This forces $\exists\ C, D$ such that $|u_1| = (x : |C|) \to_m |D|$ with $|C| \betaconv |A^\prime|$ and $|D| \betaconv |B^\prime|$.
%     However, this means that $T \betastar (x : C) \to_m D$.

%     Now, $\Gamma \vdash f \cinfr (x : C) \to_m D$ and $\Gamma \vdash a \chck C$ because $A \equiv A^\prime \equiv C$ and $A$ well-typed.
%     Applying the \textsc{App} rule yields $\Gamma \vdash \app{f}{m}{a} \infr [x := a]D$.
%     Note that $B \equiv B^\prime \equiv D$ and by Lemma~\ref{lem:2:conv_subst} $[x := a]B \equiv [x := a]D$.
%     By Lemma~\ref{lem:2:subst_check}, $\Gamma \vdash [x := a]B \chck \pcodom(m)$ and by Lemma~\ref{lem:2:check_k_to_cinfer_k}, $\Gamma \vdash [x := a]B \cinfr \pcodom(m)$.
%     Therefore, $\Gamma \vdash \app{f}{m}{a} \chck [x := a]B$.
% \end{proof}

\begin{lemma}
    \textcolor{white}{\_}
    \begin{enumerate}
        \item If $\Gamma \vdash s \infr A$ and $s \betared t$ then $\exists\ B$ such that $A \betastar B$ and $\Gamma \vdash t \infr B$
        \item If $\Gamma \vdash s \infr A$ and $\Gamma \betared \Delta$ then $\exists\ B$ such that $A \betastar B$ and $\Delta \vdash s \infr B$
        \item If $\Gamma \vdash s \cinfr A$ and $s \betared t$ then $\exists\ B$ such that $A \betastar B$ and $\Gamma \vdash t \cinfr B$
        \item If $\Gamma \vdash s \cinfr A$ and $\Gamma \betared \Delta$ then $\exists\ B$ such that $A \betastar B$ and $\Delta \vdash s \cinfr B$
        \item If $\Gamma \vdash s \chck A$ and $s \betared t$ then $\Gamma \vdash t \chck A$
        \item If $\Gamma \vdash s \chck A$ and $\Gamma \betared \Delta$ then $\Delta \vdash s \chck A$
        \item If $\vdash \Gamma$ and $\Gamma \betared \Delta$ then $\vdash \Delta$
    \end{enumerate}
    \label{lem:2:preservation_step}
\end{lemma}

\begin{theorem}[Preservation]
    \textcolor{white}{\_}
    \begin{enumerate}
        \item If $\Gamma \vdash s \infr A$ and $s \betastar t$ then $\exists\ B$ such that $A \betastar B$ and $\Gamma \vdash t \infr B$
        \item If $\Gamma \vdash s \infr A$ and $\Gamma \betastar \Delta$ then $\exists\ B$ such that $A \betastar B$ and $\Delta \vdash s \infr B$
        \item If $\Gamma \vdash s \cinfr A$ and $s \betastar t$ then $\exists\ B$ such that $A \betastar B$ and $\Gamma \vdash t \cinfr B$
        \item If $\Gamma \vdash s \cinfr A$ and $\Gamma \betastar \Delta$ then $\exists\ B$ such that $A \betastar B$ and $\Delta \vdash s \cinfr B$
        \item If $\Gamma \vdash s \chck A$ and $s \betastar t$ then $\Gamma \vdash t \chck A$
        \item If $\Gamma \vdash s \chck A$ and $\Gamma \betastar \Delta$ then $\Delta \vdash s \chck A$
        \item If $\vdash \Gamma$ and $\Gamma \betastar \Delta$ then $\vdash \Delta$
    \end{enumerate}
    \label{thm:2:preservation}
\end{theorem}
\begin{proof}
    By induction on either $s \betastar t$ or $\Gamma \betastar \Delta$ using Lemma~\ref{lem:2:preservation_step}.
\end{proof}

% \begin{lemma}
%     If $|t| \betastar \star$ then $t \betastar \star$
% \end{lemma}

% \begin{lemma}
%     If $\Gamma \vdash t \chck K$ then $\Gamma \vdash t \cinfr K$, where $K = \kind$ or $K = \star$
%     \label{lem:2:check_k_to_cinfer_k}
% \end{lemma}
% \begin{proof}
%     By definition $\exists\ T$ such that $\Gamma \vdash t \infr T$, $\Gamma \vdash K \infr K^\prime$, and $T \equiv K$.
%     Notice that, actually, $K \neq \kind$ because there is no rule where $\kind$ appears as a proof, thus $K = \star$ and $K^\prime = \kind$.
%     By Lemma~\ref{thm:2:beta_iff_conv}: $|T| \betaconv \star$.
%     Thus, $|T| \betastar \star$.
% \end{proof}

\begin{theorem}[Classification]
    If $\Gamma \vdash t \infr A$ then one (and only one) of the following statements holds:
    \begin{enumerate}
        \item $A$ is $\kind$ (i.e. $t$ is a kind)
        \item $\Gamma \vdash A \cinfr \kind$ (i.e. $t$ is a $\Gamma$-constructor)
        \item $\Gamma \vdash A \cinfr \star$ (i.e. $t$ is a $\Gamma$-term)
    \end{enumerate}
\end{theorem}
\begin{proof}
    By induction on the proof derivation of $t$ with motives
    \begin{itemize}
        \item {
            $\Gamma \vdash t \infr A$, $\Gamma \vdash t \cinfr A$, and $\Gamma \vdash t \chck A$ motives is
            $$(A = \kind) \vee (\Gamma \vdash A \cinfr \kind) \vee (\Gamma \vdash A \cinfr \star)$$
        }
        \item {
            $\vdash \Gamma$ motive is $\top$
        }
    \end{itemize}

    $\text{Case: }\begin{array}{c} \AxiomRule[*] \end{array}$
    \begin{proofcase}
        $A = \kind$, trivial.
    \end{proofcase}

    $\text{Case: }\begin{array}{c} \VarRule[*] \end{array}$
    \begin{proofcase}
        Obtained from Lemma~\ref{lem:2:ctx_get}.
    \end{proofcase}

    $\text{Case: }\begin{array}{c} \PiRule[*] \end{array}$
    \begin{proofcase}
        If $\pcodom(m) = \kind$ then trivial.
        Otherwise, $\pcodom(m) = \star$ and the \textsc{Axiom} rule with Lemma~\ref{lem:2:ctx_wf} applied to $\D{1}$ conclude the case.
    \end{proofcase}

    $\text{Case: }\begin{array}{c} \LambdaRule[*] \end{array}$
    \begin{proofcase}
        Apply the \textsc{Pi} rule with $\D{1}$ and $\D{3}$.
    \end{proofcase}

    $\text{Case: }\begin{array}{c} \AppRule[*] \end{array}$
    \begin{proofcase}
        By the IH applied to $\D{1}$ it is the case that $\exists\ K$ such that $\Gamma \vdash (x : A) \to_m B \cinfr K$.
        Note that it cannot be the case that it is equal $\kind$ by inversion on syntax.
        However, by inversion on the internal inference rule $\Gamma \vdash (x : A) \to_m B \infr \pcodom(m)$.
        By the \textsc{Pi} rule $\Gamma, x : A \vdash B \cinfr \pcodom(m)$.
        Applying Lemma~\ref{lem:2:subst_check} yields $\Gamma \vdash [x := a]B \chck \pcodom(m)$.
        Thus, by Lemma~\ref{lem:2:check_k_to_cinfer_k}: $\Gamma \vdash [x := a]B \cinfr \pcodom(m)$.
    \end{proofcase}

    $\text{Case: }\begin{array}{c} \IntersectionRule[*] \end{array}$
    \begin{proofcase}
        Immediate by the \textsc{Axiom} rule and Lemma~\ref{lem:2:ctx_get}.
    \end{proofcase}

    $\text{Case: }\begin{array}{c} \PairRule[*] \end{array}$
    \begin{proofcase}
        Apply the \textsc{Int} rule with $\D{2}$ and $\D{3}$.
    \end{proofcase}

    $\text{Case: }\begin{array}{c} \FirstRule[*] \end{array}$
    \begin{proofcase}
        Apply the IH to $\D{1}$.
        Inversion on syntax gives $\exists\ K$ such that $(x : A) \cap B \cinfr K$.
        Inversion on the internal inference rule yields $(x : A) \cap B \infr \star$.
        Thus, $K = \star$.
        Deconstructing the previous rule concludes the proof with $\Gamma \vdash A \cinfr \star$.
    \end{proofcase}

    $\text{Case: }\begin{array}{c} \SecondRule[*] \end{array}$
    \begin{proofcase}
        Exactly as the previous case $(x : A) \cap B \infr \star$ by the IH and two inversions.
        Deconstructing the previous rule gives $\Gamma, x : A \vdash B \cinfr \star$.
        By the \textsc{Fst} rule, $\Gamma \vdash t.1 \infr A$.
        Then, by Lemma~\ref{lem:2:subst_infer} it is the case that $\Gamma \vdash [x := t.1]B \cinfr \star$.
    \end{proofcase}

    $\text{Case: }\begin{array}{c} \EqualityRule[*] \end{array}$
    \begin{proofcase}
        Immediate by the \textsc{Axiom} rule and Lemma~\ref{lem:2:ctx_wf}.
    \end{proofcase}

    $\text{Case: }\begin{array}{c} \ReflRule[*] \end{array}$
    \begin{proofcase}
        Note that $\Gamma \vdash t \chck A$, thus by the \textsc{Eq} rule $\Gamma \vdash t =_A t \infr \star$.
    \end{proofcase}

    $\text{Case: }\begin{array}{c} \SubstRule[*] \end{array}$
    \begin{proofcase}
        Apply the IH to $\D{1}$.
        By inversion on syntax $\exists\ K$ such that $\Gamma \vdash a =_A b \cinfr K$.
        By inversion on the internal inference rule, $\Gamma \vdash a =_A b \infr \star$, thus $K = \star$.
        Deconstructing the previous rule yields $\Gamma \vdash A \cinfr \star$, $\Gamma \vdash a \chck A$, and $\Gamma \vdash b \chck A$.
        By Lemma~\ref{lem:2:check_app} $\Gamma \vdash \app{P}{\tau}{a} \chck \star$ and $\Gamma \vdash \app{P}{\tau}{b} \chck \star$.
        However, this means that $\Gamma \vdash \app{P}{\tau}{a} \cinfr \star$ and $\Gamma \vdash \app{P}{\tau}{b} \cinfr \star$ by Lemma~\ref{lem:2:check_k_to_cinfer_k}.
        Using weakening, $\Gamma, x : \app{P}{\tau}{a} \vdash \app{P}{\tau}{b} \cinfr \star$.
        Applying the \textsc{Pi} rule concludes the proof.
    \end{proofcase}

    $\text{Case: }\begin{array}{c} \PromoteFstRule[*] \end{array}$
    \begin{proofcase}
        Exactly as the \textsc{Fst} and \textsc{Snd} cases, $\Gamma \vdash (x : A) \cap B \cinfr \star$ by the IH and two inversions.
        By $\D{2}$, $\Gamma \vdash a \chck (x : A) \cap B$.
        Applying the \textsc{Eq} rule yields $\Gamma \vdash a =_{(x : A) \cap B} b \infr \star$.
    \end{proofcase}

    $\text{Case: }\begin{array}{c} \CastRule[*] \end{array}$
    \begin{proofcase}
        By the IH on $\D{1}$.
    \end{proofcase}

    $\text{Case: }\begin{array}{c} \SeparationRule[*] \end{array}$
    \begin{proofcase}
        It is clear that $\Gamma \vdash (X : \star) \to_0 X \infr \star$ by applying a short sequence of inference rules.
    \end{proofcase}

    $\text{Case: }\begin{array}{c} \HeadInferenceRule[*] \end{array}$
    \begin{proofcase}
        Apply the IH on $\D{1}$.
        By inversion on $\D{2}$, $A \neq \kind$.
        Thus, $\exists\ K$ such that $\Gamma \vdash A \cinfr K$.
        By preservation $\Gamma \vdash B \cinfr K$.
    \end{proofcase}

    $\text{Case: }\begin{array}{c} \CheckRule[*] \end{array}$
    \begin{proofcase}
        Immediate by $\D{2}$.
    \end{proofcase}

    $\text{Case: }\begin{array}{c} \ContextEmptyRule[*] \end{array}$
    \begin{proofcase}
        Trivial.
    \end{proofcase}

    $\text{Case: }\begin{array}{c} \ContextAppendRule[*] \end{array}$
    \begin{proofcase}
        Trivial.
    \end{proofcase}
\end{proof}
