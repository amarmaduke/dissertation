\section{Erasure and Pseudo-objects}
\label{sec:2:erasure}

Cedille has a notion of erasure of syntax that transforms a subset of syntax into the untyped $\lambda$-calculus.
This concept is generalized in $\ced$ to operate on general syntax.
It still called erasure mostly as a holdover, but erasure no longer actually erases all type annotations.
Instead, erasure should be thought of as computing the raw syntactic forms of objects.
In Section~\ref{sec:2:judgments} the notion of proof will be defined.
An object is the erasure of a proof.
Erasure is defined in Figure~\ref{fig:2:erasure}.
With erasure the desired conversion relation is also definable.
This definition enables equating objects in a dependent quantification instead of proofs.

\begin{definition}
    \label{def:2:conv}
    $s_1 \equiv s_2$ iff $\exists\ t_1, t_2.\ s_1 \betastar t_1, s_2 \betastar t_2, \text{ and } |t_1| \betaconv |t_2|$
\end{definition}



\begin{figure}
    \centering
    \begin{minipage}{0.5\textwidth}
        \begin{align*}
            |x_K| &= x_K \\
            |\!\star\!| &= \star \\
            |\kind| &= \kind \\
            |\abs{\lambda_0}{x}{A}{t}| &= |t| \\
            |\abs{\lambda_\omega}{x}{A}{t}| &= \abs{\lambda_\omega}{x}{\diamond}{|t|} \\
            |\abs{\lambda_\tau}{x}{A}{t}| &= \abs{\lambda_\tau}{x}{|A|}{|t|} \\
            |(x:A) \to_m B| &= (x:|A|) \to_m |B| \\
            |(x:A) \cap B| &= (x:|A|) \cap |B| \\
            |\app{f}{0}{a}| &= |f| \\
            |\app{f}{\omega}{a}| &= \app{|f|}{\omega}{|a|} \\
            |\app{f}{\tau}{a}| &= \app{|f|}{\tau}{|a|}
        \end{align*}
    \end{minipage}%
    \begin{minipage}{0.5\textwidth}
        \begin{align*}
            |\!\diamond\!| &= \diamond \\
            |[t_1, t_2; A]| &= |t_1| \\
            |t.1| &= |t| \\
            |t.2| &= |t| \\
            |x =_A y| &= |x| =_{|A|} |y| \\
            |\text{refl}(t; A)| &= \abs{\lambda_\omega}{x}{\diamond}{x_\star} \\
            %|J(A, P, x, y, e, w)| &= \app{|e|}{\omega}{|w|} \\
            |\psi(e, a, b; A, P)| &= |e| \\
            |\vartheta_1(e, a, b; T)| &= |e| \\
            |\vartheta_2(e, a, b; T)| &= |e| \\
            |\delta(e)| &= |e| \\
            |\varphi(f, e; A, T)| &= \abs{\lambda_\omega}{x}{\diamond}{x_\star}
        \end{align*}
    \end{minipage}
    \caption{Erasure of syntax, for type-like and kind-like syntax erasure is homomorphic, for term-like syntax erasure reduces to the untyped lambda calculus.}
    \label{fig:2:erasure}
\end{figure}


Note that the only purpose of the syntactic constructor $\diamond$ is to be a placeholder for erased type annotations of $\lambda_\omega$ syntactic forms.
However, for $\lambda_\tau$ variants, the annotation is \textit{not} erased.
This is partly why calling this transformation \textit{erasure} is a slight lie, because it does not always erase.
Regardless, it is faithful to the interpretation from Cedille when focused on non-type-like syntax.
Indeed, any form that is not type-like does reduce to the untyped $\lambda$-calculus.
For type-like syntax, erasure is instead locally homomorphic.
Erasure of raw syntax does not possess much structure, but it is idempotent and commutes with substitution.
Additionally, as a consequence an extension of Lemma~\ref{lem:2:betaconv_subst} is possible.

\begin{lemma}
    \label{lem:2:erasure_idempotent}
    $||t|| = |t|$
\end{lemma}
\begin{proof}
    By induction on $t$.
\end{proof}

\begin{lemma}
    \label{lem:2:erase_subst}
    $|[x := t]b| = [x := |t|]|b|$
\end{lemma}
\begin{proof}
    By induction on the size of $b$.

    $\text{Case: }\mathfrak{b}(\kappa, (x : t_1), t_2)$
    \begin{proofcase}
        If $b = \abs{\lambda_0}{y}{A}{b^\prime}$, then $|b| = |b^\prime|$ which is a smaller term.
        Then, by the IH $|[x := t]b^\prime| = [x := |t|]|b^\prime|$.
        Thus,
        \begin{align*}
            &|[x := t]\abs{\lambda_0}{y}{A}{b^\prime}| = |\abs{\lambda_0}{y}{[x := t]A}{[x := t]b^\prime}| \\
            &= |[x := t]b^\prime| = [x := |t|]|b^\prime| = [x := |t|]|\abs{\lambda_0}{y}{A}{b^\prime}|
        \end{align*}
        For the remaining tags, assume w.l.o.g. $\kappa = \cap$.
        Then $b = (y : A) \cap B$, and by the IH $|[x := t]A| = [x := |t|]|A|$ and $|[x := t]B| = [x := |t|]|B|$.
        Thus,
        \begin{align*}
            &|[x := t]((y : A) \cap B)| = |(y : [x := t]A) \cap [x := t]B| \\
            &= (y : |[x := t]A|) \cap |[x := t]B| = (y : [x := |t|]|A|) \cap [x := |t|]|B|
        \end{align*}
        And, $[x := |t|]|(y : A) \cap B| = (y : [x := |t|]|A|) \cap [x := |t|]|B|$.
        Thus, both sides are equal.
    \end{proofcase}

    $\text{Case: }\mathfrak{c}(\kappa, t_1, \ldots, t_{\mathfrak{a}(\kappa)})$
    \begin{proofcase}
        If $\kappa \in \{ \diamond, \star, \kind \}$ then the equality is trivial.
        \\ \\
        If $\kappa \in \{ \bullet_0, \text{pair}, \text{proj}_1, \text{proj}_2, \psi, \vartheta, \delta, \varphi \}$ then $|\mathfrak{c}(\kappa, t_1, \ldots)| = |t_1|$.
        Moreover, substitution commutes and both sides of the equality are equal.
        \\ \\
        If $\kappa \in \{ \text{refl} \}$ then the equality is trivial.
        \\ \\
        If $\kappa \in \{ \bullet_\omega, \bullet_\tau, \text{eq} \}$ then w.l.o.g. assume $\kappa = \text{eq}$.
        Now $|[x := t](a =_A b)| = |[x := t]a| =_{|[x := t]A|} |[x := t]b|$.
        By the IH this becomes $[x := |t|]|a| =_{[x := |t|]|A|} [x := |t|]|b|$.
        On the right-hand side, $[x := |t|]|a =_A b| = [x := |t|]|a| =_{[x := |t|]|A|} [x := |t|]|b|$.
        Thus, both sides are equal.
    \end{proofcase}

    $\text{Case: }b \text{ variable}$
    \begin{proofcase}
        Suppose $b = x_K$, then $|[x := t]x_K| = |t|$ and $[x := |t|]|x_K| = |t|$.
        Suppose $b = y_K$ where $x \neq y$, then $|[x := t]y_K| = y_K$ and $[x := |t|]|y_K| = y_K$.
        Thus, both sides are equal.
    \end{proofcase}
\end{proof}

\begin{lemma}
    \label{lem:2:betaconv_erased_subst}
    If $|s| \betaconv |t|$ and $|a| \betaconv |b|$ then $|[x := s]a| \betaconv |[x := t]b|$
\end{lemma}
\begin{proof}
    By definition $\exists\ z_1, z_2$ such that $|s| \betastar z_1$, $|t| \betastar z_1$, $|a| \betastar z_2$ and $|b| \betastar z_2$.
    By Lemma~\ref{lem:2:betastar_subst} applied twice $[x := |s|]|a| \betastar [x := |z_1|]z_2$ and $[x := |t|]|b| \betastar [x := |z_1|]z_2$.
    Finally, by Lemma~\ref{lem:2:erase_subst} $[x := |s|]|a| = |[x := s]a|$ and $[x := |t|]|b| = |[x := t]b|$.
\end{proof}

Beyond these lemmas more structure needs to be imposed on raw syntax to obtain better behavior with erasure.
In particular, the pair case and the $\lambda_0$ case are problematic.
Indeed, for pairs there is an assumption that the first and second component are convertible.
This restriction is what transforms pairs into something more, an element of an intersection.
Likewise, the $\lambda_0$ binder is meant to signify that the bound variable does not appear free in the erasure of the body.
Imposing these restrictions on syntax retains the spirit of what it means to be an object.
However, because syntax is still not a proof, this restriction on syntax instead forms a set of \textit{pseudo-objects}.
The inductive definition of pseudo-objects is presented in Figure~\ref{fig:2:pseobj}.


\Rule{\BinderPseObj}
    {
        \der{\D{1}}{t_1 \pseobj} \\
        \der{\D{2}}{t_2 \pseobj} \\
        \der{\D{3}}{\kappa \neq \lambda_0}
    }
    {\mathfrak{b}(\kappa, x : t_1, t_2) \pseobj}
    {}

\Rule{\LambdaPseObj}
    {
        \der{\D{1}}{A\pseobj} \\
        \der{\D{2}}{t \pseobj} \\
        \der{\D{3}}{x \notin FV(|t|)}
    }
    {\abs{\lambda_0}{x}{A}{t} \pseobj}
    {}

\Rule{\CtorPseObj}
    {
        \der{\D{1}}{\forall\ i \in {1, \ldots, \mathfrak{a}(\kappa)}.\ t_i \pseobj} \\
        \der{\D{2}}{\kappa \neq \text{pair}}
    }
    {\mathfrak{c}(\kappa, t_1, \ldots, t_{\mathfrak{a}(\kappa)}) \pseobj}
    {}

\Rule{\PairPseObj}
    {
        \der{\D{1}}{t_1 \pseobj} \\
        \der{\D{2}}{t_2 \pseobj} \\
        \der{\D{3}}{t_3 \pseobj} \\
        \der{\D{4}}{|t_1| \betaconv |t_2|}
    }
    {[t_1, t_2; t_3] \pseobj}
    {}

\begin{figure}
    \centering
    \begin{minipage}{0.5\textwidth}
        $$\BinderPseObj$$
        $$\LambdaPseObj$$
    \end{minipage}%
    \begin{minipage}{0.5\textwidth}
        $$\CtorPseObj$$
        $$\PairPseObj$$
    \end{minipage}%
    $$x_K \pseobj$$
    \caption{Definition of Pseudo Objects.}
    \label{fig:2:pseobj}
\end{figure}


Note that the restriction for pairs is $|t_1| \betaconv |t_2|$ as opposed to $t_1 \equiv t_2$.
The distinction here is subtle, but it enables proving one of the important properties for the structure of pseudo-objects, that $|t_1| \betaconv |t_2|$ if and only if $t_1 \equiv t_2$.
To reach that goal requires a series of technical lemmas about pseudo-objects and the concepts introduced so far.
The critical property about pseudo-objects is that reduction preserves the equivalence class of a pseudo-object after erasure.

\begin{lemma}
    \label{lem:2:erase_pseobj_red}
    If $s\pseobj$ and $s \betared t$ then $|s| \betaconv |t|$
\end{lemma}
\begin{proof}
    By induction on $s\pseobj$.

    $\text{Case: }\begin{array}{c} \BinderPseObj[*] \end{array}$
    \begin{proofcase}
        By cases on $s \betared t$, applying the IH and Lemma~\ref{lem:2:conv_congr}.
    \end{proofcase}

    $\text{Case: }\begin{array}{c} \LambdaPseObj[*] \end{array}$
    \begin{proofcase}
        By cases on $s \betared t$, applying the IH and Lemma~\ref{lem:2:conv_congr}.
    \end{proofcase}

    $\text{Case: }\begin{array}{c} \CtorPseObj[*] \end{array}$
    \begin{proofcase}
        By cases on $s \betared t$.

        $\text{Case: }\begin{array}{c} \app{(\abs{\lambda_m}{x}{A}{b})}{m}{t} \betared [x := t]b \end{array}$
        \begin{proofcase}
            Note that $\abs{\lambda_m}{x}{A}{b}\pseobj$.
            If $m = 0$ then $x \notin FV(b)$ and $|[x := t]b| = |b|$.
            Thus, $|\app{(\abs{\lambda_0}{x}{A}{b})}{0}{t}| = |\abs{\lambda_0}{x}{A}{b}| = |b|$.
            If $m = \omega$, then $|\app{(\abs{\lambda_\omega}{x}{A}{b})}{\omega}{t}| = \app{(\abs{\lambda_\omega}{x}{\diamond}{|b|})}{\omega}{|t|}$.
            By definition of reduction $\app{(\abs{\lambda_\omega}{x}{\diamond}{|b|})}{\omega}{|t|} \betaconv [x := |t|]|b|$.
            Finally, by Lemma~\ref{lem:2:erase_subst} the goal is obtained.
            The case of $m = \tau$ is almost exactly the same.
        \end{proofcase}

        $\text{Case: }\begin{array}{c} [t_1, t_2; A].1 \betared t_1 \end{array}$
        \begin{proofcase}
            $|[t_1, t_2; A].1| = |[t_1, t_2; A]| = |t_1|$
        \end{proofcase}

        $\text{Case: }\begin{array}{c} [t_1, t_2; A].2 \betared t_2 \end{array}$
        \begin{proofcase}
            Observe that $|[t_1, t_2; A].2| = |t_1|$ and $[t_1, t_2; A]\pseobj$. \\
            Thus, $|s| = |t_1| \betaconv |t_2|$.
        \end{proofcase}

        $\text{Case: }\begin{array}{c} \app{\psi(\text{refl}(z; Z), a, b; A, P)}{\omega}{t} \betared t \end{array}$
        \begin{proofcase}
            $|\app{\psi(\text{refl}(z; Z), a, b; A, P)}{\omega}{t}| = \app{|\text{refl}(z;Z)|}{\omega}{|t|} \betaconv |t|$
        \end{proofcase}

        $\text{Case: }\begin{array}{c} \vartheta(\text{refl}(z; Z), a, b; A) \betared \text{refl}(a; A) \end{array}$
        \begin{proofcase}
            $|\vartheta(\text{refl}(z; Z), a, b; A)| = |\text{refl}(z; Z)| = \abs{\lambda_\omega}{x}{\diamond}{x} = |\text{refl}(a; A)|$
        \end{proofcase}

        $\text{Case: }\begin{array}{c} \ConstructorReduction[*] \end{array}$
        \begin{proofcase}
            By the IH, $|t_i| \betaconv |t_i^\prime|$. The goal is achieved by Lemma~\ref{lem:2:conv_congr}
        \end{proofcase}

    \end{proofcase}

    $\text{Case: }\begin{array}{c} \PairPseObj[*] \end{array}$
    \begin{proofcase}
        By cases on $s \betared t$, applying the IH and Lemma~\ref{lem:2:conv_congr}.
    \end{proofcase}

    $\text{Case: }\begin{array}{c} s\text{ variable} \end{array}$
    \begin{proofcase}
        By cases on $s \betared t$, $t$ must be a variable.
        Thus, $|s| = |t|$.
    \end{proofcase}
\end{proof}

\begin{lemma}
    \label{lem:2:pseobj_red_f_step}
    If $s\pseobj$, $|s| \betaconv |b|$, and $s \betared t$ then $|t| \betaconv |b|$
\end{lemma}
\begin{proof}
    By Lemma~\ref{lem:2:erase_pseobj_red} $|s| \betaconv |t|$ and by Theorem~\ref{lem:2:beta_conv_equivalence} $|t| \betaconv |b|$.
\end{proof}

Of course, pseudo-objects also preserve substitution.
There is nothing inherently tricky about pseudo-objects in this respect.
It may be easier for the reader to observe that the pseudo-object predicate imposes a quotient on raw syntax.
With this intuition, the purpose of pseudo-objects is merely to filter syntactic forms that break the intended meaning of conversion.

\begin{lemma}
    \label{lem:2:pseobj_subst}
    If $b\pseobj$ and $t\pseobj$ then $[x := t]b\pseobj$
\end{lemma}
\begin{proof}
    By induction on $b\pseobj$. The $\lambda_0$ and pair cases are no different from the respective $\mathfrak{b}$ and $\mathfrak{c}$ cases.

    $\text{Case: }\begin{array}{c} \BinderPseObj[*] \end{array}$
    \begin{proofcase}
        By the IH $[x := t]t_1\pseobj$ and $[x := t]t_2\pseobj$.
        Thus, $\mathfrak{b}(\kappa, (y : [x := t]t_1), [x := t]t_2)\pseobj$.
    \end{proofcase}

    $\text{Case: }\begin{array}{c} \CtorPseObj[*] \end{array}$
    \begin{proofcase}
        By the IH $[x := t]t_i\pseobj$. \\
        Thus, $\mathfrak{c}(\kappa, [x := t]t_1, \ldots [x := t]t_{\mathfrak{a}(\kappa)})\pseobj$.
    \end{proofcase}

    $\text{Case: }\begin{array}{c} s\text{ variable} \end{array}$
    \begin{proofcase}
        If $s = x_K$ then $[x := t]x_K = t$, and $t\pseobj$.
        Otherwise, $s = y_K$ with $y$ a variable and $y_K\pseobj$.
    \end{proofcase}
\end{proof}

\begin{lemma}
    \label{lem:2:pseobj_preservation_step}
    If $s\pseobj$ and $s \betared t$ then $t\pseobj$
\end{lemma}
\begin{proof}
    By induction on $s\pseobj$.

    $\text{Case: }\begin{array}{c} \BinderPseObj[*] \end{array}$
    \begin{proofcase}
        By cases on $s \betared t$. Suppose w.l.o.g. that $t_2 \betared t_2^\prime$.
        Observe that $t_2\pseobj$ because it is a subterm of $s$.
        Then by the IH $t_2^\prime\pseobj$.
        Thus, $\mathfrak{b}(\kappa, x : t_1, t_2^\prime) \pseobj$.
    \end{proofcase}

    $\text{Case: }\begin{array}{c} \LambdaPseObj[*] \end{array}$
    \begin{proofcase}
        By cases on $s \betared t$. Suppose w.l.o.g that $t \betared t^\prime$.
        Note that if $x \notin FV(|t|)$ then $x \notin FV(|t^\prime|)$, reduction only reduces the amount of free variables.
        Observe that $t \pseobj$.
        Then by the IH $t^\prime\pseobj$.
        Thus, $\abs{\lambda_0}{x}{A}{t^\prime}\pseobj$.
    \end{proofcase}

    $\text{Case: }\begin{array}{c} \CtorPseObj[*] \end{array}$
    \begin{proofcase}
        By cases on $s \betared t$.
        The first and second projection cases are very similar to the substitution case.

        $\text{Case: }\begin{array}{c} \app{(\abs{\lambda_m}{x}{A}{b})}{m}{t} \betared [x := t]b \end{array}$
        \begin{proofcase}
            Observe that $b\pseobj$ and $t\pseobj$ because both are subterms of $s$.
            By Lemma~\ref{lem:2:pseobj_subst} $[x := t]b\pseobj$.
        \end{proofcase}

        $\text{Case: }\begin{array}{c} \app{\psi(\text{refl}(z;Z), a, b; A, P)}{\omega}{t} \betared t \end{array}$
        \begin{proofcase}
            Immediate by the IH: $t\pseobj$.
        \end{proofcase}

        $\text{Case: }\begin{array}{c} \vartheta_1(\text{refl}(z; Z), a, b; A) \betared \text{refl}(a; A) \end{array}$
        \begin{proofcase}
            Observe that $a\pseobj$ and $A\pseobj$.
            By application of constructor rule $\text{refl}(a; A)\pseobj$.
        \end{proofcase}

        $\text{Case: }\begin{array}{c} \ConstructorReduction[*] \end{array}$
        \begin{proofcase}
            By the IH $t_i^\prime\pseobj$.
            By application of the constructor rule the goal is obtained.
        \end{proofcase}
    \end{proofcase}

    $\text{Case: }\begin{array}{c} \PairPseObj[*] \end{array}$
    \begin{proofcase}
        By cases on $s \betared t$.
        Suppose w.l.o.g. $t_1 \betared t_1^\prime$.
        Note that $t_1\pseobj$ because it is a subterm of $s$.
        By the IH $t_1^\prime\pseobj$.
        By Lemma~\ref{lem:2:pseobj_red_f_step} $|t_1^\prime| \betaconv |t_2|$.
        Thus, $[t_1^\prime, t_2; A]\pseobj$.
    \end{proofcase}

    $\text{Case: }\begin{array}{c} s\text{ variable} \end{array}$
    \begin{proofcase}
        By cases on $s \betared t$, $t$ must be a variable.
        Thus, $t\pseobj$.
    \end{proofcase}
\end{proof}

\begin{lemma}
    \label{lem:2:pseobj_red_f}
    If $s\pseobj$, $|s| \betaconv |b|$, and $s \betastar t$ then $|t| \betaconv |b|$
\end{lemma}
\begin{proof}
    By induction on $s \betastar t$.
    The reflexivity case is trivial.
    The transitivity case is obtained from Lemma~\ref{lem:2:pseobj_red_f_step}, Lemma~\ref{lem:2:pseobj_preservation_step}, and applying the IH.
\end{proof}

\begin{lemma}
    \label{lem:2:pseobj_preservation}
    If $s\pseobj$ and $s \betastar t$ then $t\pseobj$
\end{lemma}
\begin{proof}
    By induction on $s \betastar t$.
    The reflexivity case is trivial.
    The transitivity case is obtained from Lemma~\ref{lem:2:pseobj_preservation_step} and applying the IH.
\end{proof}

\begin{lemma}
    \label{lem:2:pseobj_red_b}
    If $s\pseobj$, $|t| \betaconv |b|$, and $s \betastar t$ then $|s| \betaconv |b|$
\end{lemma}
\begin{proof}
    By induction on $s \betastar t$.
    Consequence of Lemma~\ref{lem:2:erase_pseobj_red} and Lemma~\ref{lem:2:pseobj_preservation}.
\end{proof}

\begin{lemma}
    \label{lem:2:conv_red_f}
    If $s\pseobj$, $s \equiv b$, and $s \betastar t$ then $t \equiv b$
\end{lemma}
\begin{proof}
    Note that $\exists\ z_1, z_2$ such that $s \betastar z_1$, $b \betastar z_2$, and $|z_1| \betaconv |z_2|$.
    By confluence $\exists\ z_1^\prime$ such that $z_1 \betastar z_1^\prime$ and $t \betastar z_1^\prime$.
    Then, by Lemma~\ref{lem:2:pseobj_preservation} $z_1\pseobj$.
    Finally, by Lemma~\ref{lem:2:pseobj_red_f} $|z_1^\prime| \betaconv |z_2|$.
    Therefore, $t \equiv b$.
\end{proof}

Unlike with convertibility of reduction, obtaining transitivity of conversion requires the additional assumption that the inner syntax is a pseudo-object.
Indeed, the incorporation of erasure into the definition requires this extra structure, because otherwise reductions on pairs would not agree.
For example, pick $a = [x, y; T].1$, $b = [x, y; T]$, and $c = [y, x; T].2$.
Notice that $|a| = |b|$ but $|b| \neq |c|$, however, $c \betastar x$.
There is an inconsistency because $b$ is not a pseudo-object, it is not the case that $|x| \betaconv |y|$.

\begin{lemma}
    If $b\pseobj$, $a \equiv b$, and $b \equiv c$ then $a \equiv c$
    \label{thm:2:conv_trans}
\end{lemma}
\begin{proof}
    Note that $\exists\ u_1, u_2$ such that $a \betastar u_1$, $b \betastar u_2$, and $|u_1| \betaconv |u_2|$.
    Additionally, $\exists\ v_1, v_2$ such that $b \betastar v_1$, $c \betastar v_2$, and $|v_1| \betaconv |v_2|$.
    By confluence, $\exists\ z$ such that $u_2 \betastar z$ and $v_1 \betastar z$.
    Then, by Lemma~\ref{lem:2:pseobj_preservation} $u_2\pseobj$ and $v_1\pseobj$.
    Next, by Lemma~\ref{lem:2:pseobj_red_f} $|u_1| \betaconv |z|$ and $|z| \betaconv |v_2|$.
    Thus, $|u_1| \betaconv |v_2|$ by Lemma~\ref{lem:2:beta_conv_equivalence} and $a \equiv c$.
\end{proof}

Knowing that $|s| \betaconv |t|$ if and only if $s \equiv t$ is critical for maintaining the spirit of Cedille.
While $\ced$ is its own system the purpose is to refine the design of Cedille without losing its essential features.
A critical feature of Cedille is that convertibility is done with the untyped $\lambda$-calculus (i.e. erased terms) not with annotated terms.
Having Theorem~\ref{thm:2:beta_iff_conv} means that whenever conversion is checked between terms it is safe to instead check reduction conversion of objects.
Not only does this maintain the spirit of Cedille, but it also enables optimizations in type checking.
Indeed, arbitrarily expensive sequences of reductions could potentially be erased when checking $|s| \betaconv |t|$ instead of $s \equiv t$.

\begin{theorem}
    Suppose $s\pseobj$ and $t\pseobj$, then $|s| \betaconv |t|$ iff $s \equiv t$
    \label{thm:2:beta_iff_conv}
\end{theorem}
\begin{proof}
    Case $(\Rightarrow)$:
    Suppose $|s| \betaconv |t|$.
    By definition $s \betastar s$ and $t \betastar t$.
    Thus, $s \equiv t$.
    Case $(\Leftarrow)$:
    Suppose $s \equiv t$, then $\exists\ z_1, z_2$ such that $s \betastar z_1$, $t \betastar z_2$, and $|z_1| \betaconv |z_2|$.
    By two applications of Lemma~\ref{lem:2:pseobj_red_b}: $|s| \betaconv |t|$.
\end{proof}

\begin{corollary}
    For $s\pseobj$ and $t\pseobj$ the relation $s \equiv t$ is an equivalence.
\end{corollary}

Finally, a useful lemma about the interaction of substitution with conversion is obtained from the effort of pseudo-objects.

\begin{lemma}
    If $s, t, a, b\pseobj$, $s \equiv t$, and $a \equiv b$ then $[x := s]a \equiv [x := t]b$
    \label{lem:2:conv_subst}
\end{lemma}
\begin{proof}
    By Lemma~\ref{thm:2:beta_iff_conv}: $|s| \betaconv |t|$ and $|a| \betaconv |b|$.
    Then, by Lemma~\ref{lem:2:betaconv_erased_subst}: $|[x := s]a| \betaconv |[x := t]b|$.
    Finally, by Lemma~\ref{thm:2:beta_iff_conv} again, $[x := s]a \equiv [x := t]b$.
\end{proof}

