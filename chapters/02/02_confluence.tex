\section{Confluence}


\Rule{\ParBinderReduction}
    {
        \der{\D{1}}{t_1 \parred t_1^\prime} \\
        \der{\D{2}}{t_2 \parred t_2^\prime}
    }
    {\mathfrak{b}(\kappa, x : t_1, t_2) \parred \mathfrak{b}(\kappa, x : t_1^\prime, t_2^\prime) }
    {ParBind}

\Rule{\ParConstructorReduction}
    {
        \der{\D{i}}{t_i \parred t_i^\prime \quad \forall\ i \in \{1, \ldots, \mathfrak{a}(\kappa) \}}
    }
    {\mathfrak{c}(\kappa, t_1, \ldots, t_i, \ldots, t_{\mathfrak{a}(\kappa)}) \parred \mathfrak{c}(\kappa, t_1^\prime, \ldots, t_i^\prime, \ldots, t^\prime_{\mathfrak{a}(\kappa)}) }
    {ParCtor}

\Rule{\ParBetaReduction}
    {
        \der{\D{1}}{t_1 \parred t_1^\prime} \\
        \der{\D{2}}{t_2 \parred t_2^\prime} \\
        \der{\D{3}}{t_3 \parred t_3^\prime}
    }
    { \app{(\abs{\lambda_m}{x}{t_1}{t_2})}{m}{t_3} \parred [x := t_3^\prime]t_2^\prime }
    {ParBeta}

\Rule{\ParFstReduction}
    {
        \der{\D{1}}{t_1 \parred t_1^\prime} \\
        \der{\D{2}}{t_2 \parred t_2^\prime} \\
        \der{\D{3}}{t_3 \parred t_3^\prime}
    }
    { [t_1, t_2; t_3].1 \parred t_1^\prime }
    {ParFst}

\Rule{\ParSndReduction}
    {
        \der{\D{1}}{t_1 \parred t_1^\prime} \\
        \der{\D{2}}{t_2 \parred t_2^\prime} \\
        \der{\D{3}}{t_3 \parred t_3^\prime}
    }
    { [t_1, t_2; t_3].2 \parred t_2^\prime }
    {ParSnd}

\Rule{\ParPrmReduction}
    {
        \der{\D{1}}{t_1 \parred t_1^\prime} \\
        \der{\D{2}}{t_2 \parred t_2^\prime} \\
        \der{\D{3}}{t_3 \parred t_3^\prime} \\
        \der{\D{3}}{t_4 \parred t_4^\prime} \\
        \der{\D{3}}{t_5 \parred t_5^\prime}
    }
    { \vartheta(\text{refl}(t_1; t_2), t_3, t_4; t_5) \parred \text{refl}(t_3^\prime; t_5^\prime) }
    {ParPrm}

\Rule{\ParSubstReduction}
    {
        \der{\D{1}}{t_1 \parred t_1^\prime} \\
        \der{\D{2}}{t_2 \parred t_2^\prime} \\
        \der{\D{2}}{t_3 \parred t_3^\prime} \\
        \der{\D{2}}{t_4 \parred t_4^\prime} \\
        \der{\D{2}}{t_5 \parred t_5^\prime} \\
        \der{\D{2}}{t_6 \parred t_6^\prime} \\
        \der{\D{2}}{t_7 \parred t_7^\prime}
    }
    { \app{\psi(\text{refl}(t_1; t_2), t_3, t_4; t_5, t_6)}{\omega}{t_7} \parred t_7^\prime}
    {ParSubst}

\Rule{\ParVarReduction}
    {\color{white}{\_}}
    { x_K \parred x_K}
    {ParVar}

\begin{figure}
    \centering
    $$\ParVarReduction$$
    $$\ParConstructorReduction$$
    $$\ParBinderReduction$$
    $$\ParBetaReduction$$
    $$\ParSubstReduction$$
    $$\ParFstReduction$$
    $$\ParSndReduction$$
    $$\ParPrmReduction$$
    \caption{Parallel reduction rules for arbitrary syntax.}
    \label{fig:par-reduction}
\end{figure}


The proof of confluence follows the PLFA book \cite{plfa22.08}.
This strategy involves the common technique of defining a parallel reduction variant of the one-step reduction described in Figure~\ref{fig:reduction}.
Parallel reduction allows reduction steps to occur in any subexpression, but reductions that generate new redexes cannot be reduced in a single step.
Figure~\ref{fig:par-reduction} presents the inductive definition of parallel reduction.
In fact, it is possible to compute the resulting syntax after all possible redexes are contracted by a single parallel reduction step.
This is the \textit{reduction completion} (written $\parp{t}$).
The definition of reduction completion is shown in Figure~\ref{fig:par-triangle}.
Reduction completion enables the derivation of a triangle property for parallel reduction of which confluence for parallel reduction is a consequence.
Confluence for multistep reduction is an immediate consequence of confluence for parallel reduction and logical equivalence between parallel reduction and multistep reduction.



\begin{figure}
    \centering
    \begin{align*}
        \parp{\app{(\abs{\lambda_m}{x}{t_1}{t_2})}{m}{t_3}} &= [x := \parp{t_3}]\parp{t_2} \\
        \parp{\app{\psi(\text{refl}(t_1; t_2), t_3, t_4; t_5, t_6)}{\omega}{t_7}} &= \parp{t_7} \\
        \parp{[t_1, t_2; t_3].1} &= \parp{t_1} \\
        \parp{[t_1, t_2; t_3].2} &= \parp{t_2} \\
        \parp{\vartheta(\text{refl}(t_1; t_2), t_3, t_4; t_5)} &= \text{refl}(\parp{t_3}; \parp{t_5}) \\
        \parp{\mathfrak{c}(\kappa, t_1, \ldots, t_{\mathfrak{a}(\kappa)})} &= \mathfrak{c}(\kappa, \parp{t_1}, \ldots, \parp{t_{\mathfrak{a}(\kappa)}}) \\
        \parp{\mathfrak{b}(\kappa, (x : t_1), t_2)} &= \mathfrak{b}(\kappa, (x : \parp{t_1}), \parp{t_2}) \\
        \parp{x_K} &= x_K
    \end{align*}
    \caption{
        Definition of a reduction completion function $\parp{-}$ for parallel reduction.
        Note that this function is defined by pattern matching, applying cases from top to bottom.
        Thus, the cases at the very bottom are catch-all for when the prior cases are not applicable.
    }
    \label{fig:par-triangle}
\end{figure}


\begin{lemma}
    For any $t$, $t \parred t$
    \label{lem:a:par_refl}
\end{lemma}
\begin{proof}
    Straightforward by induction on $t$.
\end{proof}

\begin{lemma}
    If $s \betared t$ then $s \parred t$
    \label{lem:a:beta_implies_par_step}
\end{lemma}
\begin{proof}
    By induction on $s \betared t$.
    The projection and promotion cases are similar to the substitution and beta case and thus omitted.
    The second structural binder reduction case is omitted.

    $\text{Case: }\begin{array}{c} \app{(\abs{\lambda_m}{x}{A}{b})}{m}{t} \betared [x := t]b \end{array}$
    \begin{proofcase}
        By Lemma~\ref{lem:a:par_refl}: $t \parred t$ and $b \parred b$.
        Applying the \textsc{ParBeta} rule concludes the case.
    \end{proofcase}

    $\text{Case: }\begin{array}{c} \app{\psi(\text{refl}(z; Z), u, v; A, P)}{\omega}{b} \betared b \end{array}$
    \begin{proofcase}
        Using Lemma~\ref{lem:a:par_refl}: $z \parred z$, $Z \parred Z$, $u \parred u$, $v \parred v$, $A \parred A$, $P \parred P$, and $b \parred b$.
        Applying the \textsc{ParSubst} rule concludes the case.
    \end{proofcase}

    $\text{Case: }\begin{array}{c} \ConstructorReduction[*] \end{array}$
    \begin{proofcase}
        By the IH applied to $\D{1}$: $t_i \parred t_i^\prime$.
        Note that there is only one subderivation.
        For all $j \neq i$ $t_j \parred t_j$ by Lemma~\ref{lem:a:par_refl}.
        Using the \textsc{ParCtor} rule concludes the case.
    \end{proofcase}

    $\text{Case: }\begin{array}{c} \BinderOneReduction[*] \end{array}$
    \begin{proofcase}
        Applying the IH to $\D{1}$ yields $t_1 \parred t_1^\prime$.
        By Lemma~\ref{lem:a:par_refl}: $t_2 \parred t_2$.
        Using the \textsc{ParBind} rule concludes the case.
    \end{proofcase}
\end{proof}

\begin{lemma}
    If $s \betastar t$ then $s \parstar t$
    \label{lem:a:beta_implies_par}
\end{lemma}
\begin{proof}
    By induction on $s \betastar t$ applying Lemma~\ref{lem:a:beta_implies_par_step} in the inductive case.
\end{proof}

\begin{lemma}
    If $s \parred t$ then $s \betastar t$
    \label{lem:a:par_implies_beta_step}
\end{lemma}
\begin{proof}
    By induction on $s \parred t$.
    The projection, promotion, and substitution cases are similar to the beta case with the only difference being applying the associated rule.

    $\text{Case: }\begin{array}{c} \ParVarReduction[*] \end{array}$
    \begin{proofcase}
        By reflexivity of reduction.
    \end{proofcase}

    $\text{Case: }\begin{array}{c} \ParConstructorReduction[*] \end{array}$
    \begin{proofcase}
        By the IH applied to each $\D{i}$: $t_i \betastar t_i^\prime$ for all $i$.
        Applying Lemma~\ref{lem:2:beta_par} concludes the case.
    \end{proofcase}

    $\text{Case: }\begin{array}{c} \ParBinderReduction[*] \end{array}$
    \begin{proofcase}
        As the previous case, the IH yields $t_1 \betastar t_1$ and $t_2 \betastar t_2^\prime$.
        Again using Lemma~\ref{lem:2:beta_par} concludes the case.
    \end{proofcase}

    $\text{Case: }\begin{array}{c} \ParBetaReduction[*] \end{array}$
    \begin{proofcase}
        Applying the IH to all available derivations and using Lemma~\ref{lem:2:beta_par} gives $\app{(\abs{\lambda_m}{x}{t_1}{t_2})}{m}{t_3} \betastar \app{(\abs{\lambda_m}{x}{t_1^\prime}{t_2^\prime})}{m}{t_3^\prime}$.
        Applying the beta rule of reduction with transitivity concludes the case.
    \end{proofcase}
\end{proof}

\begin{lemma}
    If $s \parstar t$ then $s \betastar t$
    \label{lem:a:par_implies_beta}
\end{lemma}
\begin{proof}
    By induction on $s \parstar t$ applying Lemma~\ref{lem:a:par_implies_beta_step} in the inductive case.
\end{proof}

\begin{lemma}
    If $s \parred s^\prime$ and $t \parred t^\prime$ then $[x := s]t \parred [x := s^\prime]t^\prime$
    \label{lem:a:par_subst}
\end{lemma}
\begin{proof}
    By induction on $t \parred t^\prime$.
    The second projection case is omitted because it is the same as the first projection case.

    $\text{Case: }\begin{array}{c} \ParVarReduction[*] \end{array}$
    \begin{proofcase}
        Rename to $y$.
        If $x = y$ then $s \parred s^\prime$ which is a premise.
        If $x \neq y$ then no substitution is performed and $y_K \parred y_K$.
    \end{proofcase}

    $\text{Case: }\begin{array}{c} \ParConstructorReduction[*] \end{array}$
    \begin{proofcase}
        Applying the IH to $\D{i}$ yields $[x := s]t_i \parred [x := s^\prime]t_i^\prime$ for all $i$.
        Unfolding substitution for $\mathfrak{c}$ and applying the \textsc{ParCtor} rule concludes the case.
    \end{proofcase}

    $\text{Case: }\begin{array}{c} \ParBinderReduction[*] \end{array}$
    \begin{proofcase}
        As above the IH gives $[x := s]t_i \parred [x := s^\prime]t_i^\prime$ for $i = 1$ and $i = 2$.
        Unfolding substitution for $\mathfrak{b}$ and applying the \textsc{ParBind} rule concludes.
    \end{proofcase}

    $\text{Case: }\begin{array}{c} \ParBetaReduction[*] \end{array}$
    \begin{proofcase}
        By the IH: $[x := s]t_i \parred [x := s^\prime]t_i^\prime$ for $i = 1, 2, 3$.
        The \textsc{ParBeta} rule gives the following:
        $[x := s]\app{(\abs{\lambda_m}{y}{t_1}{t_2})}{m}{t_3} = \app{(\abs{\lambda_m}{y}{[x := s]t_1}{[x := s]t_2})}{m}{[x := s]t_3} \parred [y := t_3^\prime][x := s^\prime]t_2^\prime$.
        Note that $y$ is bound and thus not a free variable in $s^\prime$ and, moreover, by implicit renaming $x \neq y$.
        Thus, by Lemma~\ref{lem:2:subst_commute} $[y := t_3^\prime][x := s^\prime]t_2^\prime = [x := s^\prime][y := t_3^\prime]t_2^\prime$.
    \end{proofcase}

    $\text{Case: }\begin{array}{c} \ParSubstReduction[*] \end{array}$
    \begin{proofcase}
        By the IH: $[x := s]t_i \parred [x := s^\prime]t_i^\prime$ for $i = 1, 2$.
        The \textsc{ParSubst} rule gives:
        $[x := s](\app{\psi(\text{refl}(t_1; t_2), t_3, t_4; t_5, t_6)}{\omega}{t_7}) = \app{\psi(\text{refl}([x := s]t_1; [x := s]t_2), [x := s]t_3, [x := s]t_4; [x := s]t_5, [x := s]t_6)}{\omega}{[x := s]t_7} \parred [x := s^\prime]t_7^\prime$.
    \end{proofcase}

    $\text{Case: }\begin{array}{c} \ParFstReduction[*] \end{array}$
    \begin{proofcase}
        By the IH: $[x := s]t_i \parred [x := s^\prime]t_i^\prime$ for $i = 1, 2, 3$.
        The \textsc{ParFst} rule gives:
        $[x := s][t_1, t_2; t_3].1 = [[x := s]t_1, [x := s]t_2; [x := s]t_3].1 \parred [x := s^\prime]t_1^\prime$.
    \end{proofcase}

    $\text{Case: }\begin{array}{c} \ParPrmReduction[*] \end{array}$
    \begin{proofcase}
        By the IH: $[x := s]t_i \parred [x := s^\prime]t_i^\prime$ for $i = 1, 2, 3$.
        The \textsc{ParFst} rule gives:
        $[x := s]\vartheta(\text{refl}(t_1; t_2), t_3, t_4; t_5) = \vartheta(\text{refl}([x := s]t_1; [x := s]t_2), [x := s]t_3, [x := s]t_4; [x := s]t_5)  \parred \text{refl}([x := s^\prime]t_3^\prime; [x := s^\prime]t_5^\prime) = [x := s^\prime]\text{refl}(t_3^\prime; t_5^\prime)$.
    \end{proofcase}
\end{proof}

\noindent \begin{minipage}{.8\textwidth}
    The triangle property of parallel reduction is used to complete the set of possible contractible redexes.
    Thus, if syntax $s \parred t$ where $t$ is only partially reduced then both $s$ and $t$ may be completed to $\parp{s}$.
    To the right the situation is visually depicted.
    Note that the triangle property is ``half'' of the diamond property.
    Indeed, if $s \parred t^\prime$ then $t^\prime \parred \parp{s}$.
    Thus, as a consequence of the triangle property, parallel reduction trivially has the diamond property.
\end{minipage}\hfill
\begin{minipage}{.2\textwidth}
    \[\begin{tikzcd}
        s \\
        & t \\
        {\parp{s}}
        \arrow[Rightarrow, scaling nfold=3, from=1-1, to=3-1]
        \arrow[Rightarrow, scaling nfold=3, from=1-1, to=2-2]
        \arrow[Rightarrow, scaling nfold=3, from=2-2, to=3-1]
    \end{tikzcd}\]
\end{minipage}

\begin{lemma}[Parallel Triangle]
    If $s \parred t$ then $t \parred \parp{s}$
    \label{lem:a:par_triangle}
\end{lemma}
\begin{proof}
    By induction on $s \parred t$.
    The second projection case is omitted.

    $\text{Case: }\begin{array}{c} \ParVarReduction[*] \end{array}$
    \begin{proofcase}
        Have $\parp{x_K} = x_K$.
        Thus, this case is trivial.
    \end{proofcase}

    $\text{Case: }\begin{array}{c} \ParConstructorReduction[*] \end{array}$
    \begin{proofcase}
        By the IH applied to $\D{i}$: $t_i^\prime \parred \parp{t_i}$ for all $i$.
        Proceed by cases of $\parp{\mathfrak{c}(\kappa, t_1, \ldots t_{\mathfrak{a}(\kappa)})}$.
        The second projection case is omitted because it is the same as the first projection case.

        $\text{Case: }\begin{array}{c} \parp{\app{(\abs{\lambda_m}{x}{t_1}{t_2})}{m}{t_3}} = [x := \parp{t_3}]\parp{t_2} \end{array}$
        \begin{proofcase}
            Note that $\mathfrak{c}(\kappa, t_1^\prime, \ldots t^\prime_{\mathfrak{a}(\kappa)}) = \app{(\abs{\lambda_m}{x}{t_1^\prime}{t_2^\prime})}{m}{t_3^\prime}$.
            Using the \textsc{ParBeta} rule yields $\app{(\abs{\lambda_m}{x}{t_1^\prime}{t_2^\prime})}{m}{t_3^\prime} \parred [x := \parp{t_3}]\parp{t_2}$.
        \end{proofcase}

        $\text{Case: }\begin{array}{c} \parp{\psi(\app{\text{refl}(t_1; t_2), t_3, t_4; t_5, t_6)}{\omega}{t_7}} = \parp{t_7} \end{array}$
        \begin{proofcase}
            Note that $\mathfrak{c}(\kappa, t_1^\prime, \ldots t^\prime_{\mathfrak{a}(\kappa)}) = \psi(\app{\text{refl}(t_1^\prime; t_2^\prime), t_3^\prime, t_4^\prime; t_5^\prime, t_6^\prime)}{\omega}{t_7^\prime}$.
            Using the \textsc{ParSubst} rule yields $\psi(\app{\text{refl}(t_1^\prime; t_2^\prime), t_3^\prime, t_4^\prime; t_5^\prime, t_6^\prime)}{\omega}{t_7^\prime} \parred \parp{t_7}$.
        \end{proofcase}

        $\text{Case: }\begin{array}{c} \parp{[t_1, t_2; t_3].1} = \parp{t_1} \end{array}$
        \begin{proofcase}
            Note that $\mathfrak{c}(\kappa, t_1^\prime, \ldots t^\prime_{\mathfrak{a}(\kappa)}) = [t_1^\prime, t_2^\prime; t_3^\prime].1$.
            Using the \textsc{ParFst} rule yields $[t_1^\prime, t_2^\prime; t_3^\prime].1 \parred \parp{t_1}$.
        \end{proofcase}

        $\text{Case: }\begin{array}{c} \parp{\vartheta(\text{refl}(t_1; t_2), t_3, t_4; t_5)} = \prefl(\parp{t_3}; \parp{t_5}) \end{array}$
        \begin{proofcase}
            Note that $\mathfrak{c}(\kappa, t_1^\prime, \ldots t^\prime_{\mathfrak{a}(\kappa)}) = \vartheta(\text{refl}(t_1^\prime; t_2^\prime), t_3^\prime, t_4^\prime; t_5^\prime)$.
            Using the \textsc{ParPrmFst} rule yields $\vartheta(\text{refl}(t_1^\prime; t_2^\prime), t_3^\prime, t_4^\prime; t_5^\prime) \parred \prefl(\parp{t_3}; \parp{t_5})$.
        \end{proofcase}

        $\text{Case: }\begin{array}{c} \parp{\mathfrak{c}(\kappa, t_1, \ldots t_{\mathfrak{a}(\kappa)})} = \mathfrak{c}(\kappa, \parp{t_1}, \ldots \parp{t_{\mathfrak{a}(\kappa)}}) \end{array}$
        \begin{proofcase}
            Using the \textsc{ParCtor} rule concludes the case.
        \end{proofcase}
    \end{proofcase}

    $\text{Case: }\begin{array}{c} \ParBinderReduction[*] \end{array}$
    \begin{proofcase}
        Note that $\parp{\mathfrak{b}(\kappa, (x : t_1), t_2)} = \mathfrak{b}(\kappa, (x : \parp{t_1}), \parp{t_2})$.
        By the IH applied to $\D{i}$: $t_i^\prime \parred \parp{t_i}$ for $i = 1, 2$.
        Thus, by the \textsc{ParBind} rule $\mathfrak{b}(\kappa, (x : t_1^\prime), t_2^\prime) \parred \mathfrak{b}(\kappa, (x : \parp{t_1}), \parp{t_2})$.
    \end{proofcase}

    $\text{Case: }\begin{array}{c} \ParBetaReduction[*] \end{array}$
    \begin{proofcase}
        Note that $\parp{\app{(\abs{\lambda_m}{x}{t_1}{t_2})}{m}{t_3}} = [x := \parp{t_3}]\parp{t_2}$.
        By the IH applied to $\D{i}$: $t_i^\prime \parred \parp{t_i}$ for $i = 1, 2, 3$.
        Thus, by Lemma~\ref{lem:a:par_subst} $[x := t_3^\prime]t_2^\prime \parred [x := \parp{t_3}]\parp{t_2}$.
    \end{proofcase}

    $\text{Case: }\begin{array}{c} \ParSubstReduction[*] \end{array}$
    \begin{proofcase}
        Note that $\parp{\app{\psi(\text{refl}(t_1; t_2), t_3, t_4; t_5, t_6)}{\omega}{t_7}} = \parp{t_7}$.
        By the IH applied to $\D{i}$: $t_i^\prime \parred \parp{t_i}$ for $i = 1$ through $i = 7$.
        Applying the \textsc{ParBind} rule yields $t_7^\prime \parred \parp{t_7}$.
    \end{proofcase}

    $\text{Case: }\begin{array}{c} \ParFstReduction[*] \end{array}$
    \begin{proofcase}
        Note that $\parp{[t_1, t_2; t_3].1} = \parp{t_1}$.
        By the IH applied to $\D{i}$: $t_i^\prime \parred \parp{t_i}$ for $i = 1, 2, 3$.
        Thus, $t_1^\prime \parred \parp{t_1}$.
    \end{proofcase}

    $\text{Case: }\begin{array}{c} \ParPrmReduction[*] \end{array}$
    \begin{proofcase}
        Note that $\parp{\vartheta_1(\text{refl}(t_1; t_2), t_3, t_4; t_5)} \parred \prefl(\parp{t_3}; \parp{t_5})$.
        By the IH applied to $\D{i}$: $t_i^\prime \parred \parp{t_i}$ for $i = 1$ through $i = 5$.
        Thus, $\prefl(t_3^\prime;t_5^\prime) \parred \prefl(\parp{t_3};\parp{t_5})$ by the \textsc{ParCtor} rule and Lemma~\ref{lem:a:par_refl}.
    \end{proofcase}
\end{proof}

\begin{lemma}[Parallel Strip]
    \label{lem:a:par_strip}
    If $s \parred t_1$ and $s \parstar t_2$ then $\exists\ t$ such that $t_1 \parstar t$ and $t_2 \parred t$
\end{lemma}
\begin{proof}
    By induction on $s \parstar t_2$, pick $t = t_1$ for the reflexivity case.
    Consider the transitivity case, $\exists\ z_1$ such that $s \parred z_1$ and $z_1 \parstar t_2$.
    Applying Lemma~\ref{lem:a:par_triangle} to $s \parred z_1$ yields $z_1 \parred \parp{s}$.
    By the IH with $z_1 \parred \parp{s}$: $\exists\ z_2$ such that $\parp{s} \parstar z_2$ and $t_2 \parred z_2$.
    Using Lemma~\ref{lem:a:par_triangle} again on $s \parred t_1$ yields $t_1 \parred \parp{s}$.
    Now by transitivity $t_1 \parstar z_2$.
\end{proof}

\begin{lemma}[Parallel Confluence]
    \label{lem:a:par_confluence}
    If $s \parstar t_1$ and $s \parstar t_2$ then $\exists\ t$ such that $t_1 \parstar t$ and $t_2 \parstar t$
\end{lemma}
\begin{proof}
    By induction on $s \parstar t_1$, pick $t = t_2$ for the reflexivity case.
    Consider the transitivity case, $\exists\ z_1$ such that $s \parred z_1$ and $z_1 \parstar t_1$.
    By Lemma~\ref{lem:a:par_strip} applied with $s \parred z_1$ and $s \parstar t_2$ yields $\exists\ z_2$ such that $z_1 \parstar z_2$ and $t_2 \parred z_2$.
    Using the IH with $z_1 \parred z_2$ gives $\exists\ z_3$ such that $t_1 \parstar z_3$ and $z_2 \parstar z_3$.
    By transitivity $t_2 \parstar z_3$.
\end{proof}

\begin{lemma}[Confluence]
    \label{lem:2:confluence}
    If $s \betastar t_1$ and $s \betastar t_2$ then $\exists\ t$ such that $t_1 \betastar t$ and $t_2 \betastar t$
\end{lemma}
\begin{proof}
    By Lemma~\ref{lem:a:beta_implies_par} applied twice: $s \parstar t_1$ and $s \parstar t_2$.
    Now by parallel confluence (Lemma~\ref{lem:a:par_confluence}) $\exists\ t$ such that $t_1 \parstar t$ and $t_2 \parstar t$.
    Finally, two applications of Lemma~\ref{lem:a:par_implies_beta} conclude the proof.
\end{proof}

As with F$^\omega$ the important consequence of confluence is that convertibility of reduction is an equivalence relation.
However, this is \textit{not} the conversion relation that will be used in the inference judgment.
Thus, while important, it is still only a stepping stone to showing judgmental conversion is transitive.

\begin{theorem}
    \label{lem:2:beta_conv_equivalence}
    For any $s$ and $t$ the relation $s \betaconv t$ is an equivalence.
\end{theorem}
\begin{proof}
    Reflexivity is immediate because $s \betastar s$.
    Symmetry is also immediate because if $s \betaconv t$ then $\exists\ z$ such that $s \betastar z$ and $t \betastar z$, but logical conjunction is commutative.
    Transitivity is a consequence of confluence, see Theorem~\ref{thm:1:trans}.
\end{proof}

Additionally, there is a final useful fact about convertibility of reduction that is occasionally used throughout the rest of this work.
That is, like reduction, conversion of subexpressions yields conversion of the entire term.

\begin{lemma}
    \label{lem:2:conv_congr}
    If $t_i \betaconv t_i^\prime$ for any $i$ then,
    \begin{enumerate}
        \item $\mathfrak{b}(\kappa, (x : t_1), t_2) \betaconv \mathfrak{b}(\kappa, (x : t_1^\prime), t_2^\prime)$
        \item $\mathfrak{c}(\kappa, t_1,\ldots,t_{\mathfrak{a}(\kappa)}) \betaconv \mathfrak{c}(\kappa, t_1^\prime,\ldots,t_{\mathfrak{a}(\kappa)}^\prime)$
    \end{enumerate}
\end{lemma}
\begin{proof}
    By Lemma~\ref{lem:2:beta_par} applied on both sides.
\end{proof}
