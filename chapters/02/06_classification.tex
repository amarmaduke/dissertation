\section{Classification}


\begin{figure}
    \centering
    \begin{minipage}{0.5\textwidth}
        \begin{align*}
            \ptrunc{\!\!\pterm} &= x_\star \\
            \ptrunc{\!\!\ptype} &= x_\square
        \end{align*}
    \end{minipage}%
    \begin{minipage}{0.5\textwidth}
        \begin{align*}
            \ptrunc{\!\!\pkind} &= \star \\
            \ptrunc{\text{undefined}} &= \diamond
        \end{align*}
    \end{minipage}
    \caption{Domain and codomains for function types. The variable $K$ is either $\star$ or $\kind$.}
\end{figure}

\begin{figure}
    \centering
    \begin{align*}
        \mathcal{C}(x_\square) &=\!\! \ptype \\
        \mathcal{C}(x_\star) &=\!\! \pterm \\
        \mathcal{C}(\star) &=\!\! \pkind \\
        \mathcal{C}(\diamond) &=\!\! \ptype \\
        \mathcal{C}(\abs{\lambda_\tau}{x}{A}{t}) &=\!\! \ptype &\text{if }& (A\pkind \text{ or } A\ptype) \text{ and } t\ptype \\
        \mathcal{C}(\abs{\lambda_0}{x}{A}{t}) &=\!\! \pterm &\text{if }& (A\pkind \text{ or } A\ptype) \text{ and } t\pterm \\
        \mathcal{C}(\abs{\lambda_\omega}{x}{A}{t}) &=\!\! \pterm &\text{if }& A\ptype \text{ and } t\pterm \\
        \mathcal{C}((x : A) \to_\tau B) &=\!\! \pkind &\text{if }& (A\pkind \text{ or } A\ptype) \text{ and } B\pkind \\
        \mathcal{C}((x : A) \to_0 B) &=\!\! \ptype &\text{if }& (A\pkind \text{ or } A\ptype) \text{ and } B\ptype \\
        \mathcal{C}((x : A) \to_\omega B) &=\!\! \ptype &\text{if }& A\ptype \text{ and } B\ptype \\
        \mathcal{C}(\app{(\abs{\lambda_\tau}{x}{A}{t})}{\tau}{a}) &=\!\! \ptype &\text{if }& (A\pkind \text{ and } a\ptype) \text{ or } (A\ptype \text{ and } a\pterm) \\
            &&&\text{and } t\ptype \text{ and } [x := \ptrunc{\mathcal{C}(a)}]t\ptype \\
        \mathcal{C}(\app{f}{\tau}{a}) &=\!\! \ptype &\text{if }& (a\ptype \text{ or } a\pterm) \text{ and } f\ptype \\
        \mathcal{C}(\app{(\abs{\lambda_0}{x}{A}{t})}{0}{a}) &=\!\! \pterm &\text{if }& (A\pkind \text{ and } a\ptype) \text{ or } (A\ptype \text{ and } a\pterm) \\
            &&&\text{and } t\pterm \text{ and } [x := \ptrunc{\mathcal{C}(a)}]t\pterm \\
        \mathcal{C}(\app{f}{0}{a}) &=\!\! \pterm &\text{if }& (a\ptype \text{ or } a\pterm) \text{ and } f\pterm \\
        \mathcal{C}(\app{(\abs{\lambda_\omega}{x}{A}{t})}{\omega}{a}) &=\!\! \pterm &\text{if }& A\ptype \text{ and } a, t\pterm \text{ and } [x := \ptrunc{\mathcal{C}(a)}]t\pterm \\
        \mathcal{C}(\app{f}{\omega}{a}) &=\!\! \pterm &\text{if }& a\pterm \text{ and } f\pterm
    \end{align*}
    \caption{Domain and codomains for function types. The variable $K$ is either $\star$ or $\kind$.}
\end{figure}

\begin{figure}
    \centering
    \begin{align*}
        \mathcal{C}((x : A) \cap B) &=\!\! \ptype &\text{if }& A\ptype \text{ and } B\ptype \\
        \mathcal{C}([t_1, t_2; A]) &=\!\! \pterm &\text{if }& t_1, t_2\pterm \text{ and } A\ptype \\
        \mathcal{C}(t.1) &=\!\! \pterm &\text{if }& t\pterm \\
        \mathcal{C}(t.2) &=\!\! \pterm &\text{if }& t\pterm \\
        \mathcal{C}(a =_A b) &=\!\! \ptype &\text{if }& a, b\pterm \text{ and } A\ptype \\
        \mathcal{C}(\prefl(t; A)) &=\!\! \pterm &\text{if }& t\pterm \text{ and } A\ptype \\
        \mathcal{C}(\vartheta_i(e, a, b; T)) &=\!\! \pterm &\text{if }& e, a, b\pterm \text{ and } T\ptype \\
        \mathcal{C}(\psi(e, a, b; A, P)) &=\!\! \pterm &\text{if }& e, a, b\pterm \text{ and } A, P\ptype \\
        \mathcal{C}(\varphi(f, e; A, T)) &=\!\! \pterm &\text{if }& f, e\pterm \text{ and } A, T\ptype \\
        \mathcal{C}(\delta(e)) &=\!\! \pterm &\text{if }& e\pterm \\
        \mathcal{C}(t) &= \text{undefined} &&\text{otherwise}
    \end{align*}
    \caption{Domain and codomains for function types. The variable $K$ is either $\star$ or $\kind$.}
\end{figure}



Classification is a critical property of a system like CC with unified syntax.
It allows for the syntax to instead be stratified into levels which enable an intrinsic presentation of the system.
For $\ced$ there are only two universes like the original CC, thus the stratification places terms into three separate levels.
A term is either a \textit{kind} (thus $A = \kind$), a \textit{type} (thus $\Gamma \vdash A : \kind$), or a \textit{term} (thus $\Gamma \vdash A : \star$).
Determining the appropriate level for syntax is also decidable with a classification function defined in Figure~\ref{fig:2:classify}.
This function is crafted to preserve classification after both reduction and erasure.
Note that because the function is defined on raw syntax it is possible that there is no valid level.
In these cases the syntax is given the classification \textit{undefined}.

\begin{definition}
    \textcolor{white}{\_}
    \begin{enumerate}
        \item $t\pterm$ iff $\mathcal{C}(t) =\!\!\pterm$
        \item $t\ptype$ iff $\mathcal{C}(t) =\!\!\ptype$
        \item $t\pkind$ iff $\mathcal{C}(t) =\!\!\pkind$
    \end{enumerate}
\end{definition}

The condition $[x := \ptrunc{\mathcal{C}(a)}]t\ptype$ and others like it in the definition of $\mathcal{C}(-)$ are necessary.
Take for example $\abs{\lambda_\tau}{x}{\star}{x_\star}$.
This is not well-typed and hence not a proof, but it also should not be a kind, type, or term because it will prevent preservation of classification during reduction.
If $a\pterm$ then the application will correctly produce a term, but if $a\ptype$ then an application will reduce to a type.

\begin{lemma}
    The definition of $\mathcal{C}(-)$ is terminating
\end{lemma}
\begin{proof}
    The definition is structural except application cases.
    In particular, application cases require evaluating $\mathcal{C}([x := \ptrunc{\mathcal{C}(a)}]t)$ for some subexpressions $a$ and $t$.
    Note that computing $\mathcal{C}(-)$ on subxpressions is terminating, and $\ptrunc{-}$ is a constant function.
    Thus, a measure of size can be constructed such that the size of $\ptrunc{\mathcal{C}(a)}$ is always zero for any $a$.
    Substitution of syntactic forms of zero size do not change the size of the resulting term, therefore $\mathcal{C}([x := \ptrunc{\mathcal{C}(a)}]t)$ is a terminating invocation.
\end{proof}

\begin{lemma}
    \label{lem:2:classify_erase}
    If $\mathcal{C}(t)$ is defined then $\mathcal{C}(t) = \mathcal{C}(|t|)$
\end{lemma}
\begin{proof}
    By induction on $t$.
    Type-like syntax is homomorphic and thus the equation holds by the IH.
    Term-like syntax eliminates most of the extra structure leaving behind only another term-like syntax.
    A few cases are presented to illuminate both situations.

    $\text{Case: }t = a =_A b$
    \begin{proofcase}
        Have $|a =_A b| = |a| =_{|A|} |b|$, and because $\mathcal{C}(a =_A b)$ is defined it must be the case that $a, b\pterm$ and $A\ptype$.
        Applying the IH gives $\mathcal{C}(a) = \mathcal{C}(|a|)$, $\mathcal{C}(b) = \mathcal{C}(|b|)$, and $\mathcal{C}(A) = \mathcal{C}(|A|)$.
        Thus, $|a| =_{|A|} |b|\ptype$.
    \end{proofcase}

    $\text{Case: }t = \app{(\abs{\lambda_0}{x}{A}{t})}{0}{a}$
    \begin{proofcase}
        Have $|\app{(\abs{\lambda_0}{x}{A}{t})}{0}{a}| = |t|$ and $t\pterm$.
        Thus, by the IH $|t|\pterm$.
    \end{proofcase}
    
    $\text{Case: }t = \prefl(t; A)$
    \begin{proofcase}
        Have $|\prefl(t; A)| = \abs{\lambda}{x}{\diamond}{x_\star}$, and by computation $\abs{\lambda_\omega}{x}{\diamond}{x_\star}\pterm$.
    \end{proofcase}
\end{proof}

\begin{lemma}
    \label{lem:2:classify_subst}
    If $\mathcal{C}(t)$ and $\mathcal{C}(b)$ are defined then $$\mathcal{C}([x := t]b) = \mathcal{C}([x := \ptrunc{\mathcal{C}(t)}]b)$$
\end{lemma}
\begin{proof}
    If $\mathcal{C}(t)$ is defined then clearly $\mathcal{C}(t) = \mathcal{C}(\ptrunc{\mathcal{C}(t)})$ by definition.
    The lemma is then shown by induction on $b$.
\end{proof}

\begin{lemma}
    \label{lem:2:classify_preservation_step}
    If $\mathcal{C}(s)$ is defined and $s \betared t$ then $\mathcal{C}(s) = \mathcal{C}(t)$
\end{lemma}
\begin{proof}
    By induction on $s \betared t$, note that $\mathcal{C}(-)$ is structural making the inductive cases trivial.
    The first projection case is similar to the second projection case and thus omitted.

    $\text{Case: }\begin{array}{c} \app{(\abs{\lambda_m}{x}{A}{b})}{m}{t} \betared [x := t]b \end{array}$
    \begin{proofcase}
        Suppose w.l.o.g. that $m = \tau$, then $(\app{(\abs{\lambda_\tau}{x}{A}{b})}{\tau}{t})\ptype$.
        Note that $t\ptype$ or $t\pterm$ by unraveling the previous definition.
        Now $[x := \ptrunc{\mathcal{C}(t)}]b\ptype$.
        By Lemma~\ref{lem:2:classify_subst} and the above observation: $[x := t]b\ptype$.
    \end{proofcase}

    $\text{Case: }\begin{array}{c} [t_1, t_2; A].2 \betared t_2 \end{array}$
    \begin{proofcase}
        Have $[t_1, t_2; A]\pterm$ and by deconstructing the definition $t_2\pterm$.
    \end{proofcase}

    $\text{Case: }\begin{array}{c} \app{\psi(\text{refl}(z; Z), a, b; A, P)}{\omega}{t} \betared t \end{array}$
    \begin{proofcase}
        Have $(\app{\psi(\text{refl}(z; Z), a, b; A, P)}{\omega}{t})\pterm$ and by deconstruction the definition $t\pterm$.
    \end{proofcase}

    $\text{Case: }\begin{array}{c} \vartheta(\text{refl}(z; Z), a, b; T) \betared \text{refl}(a; T) \end{array}$
    \begin{proofcase}
        Have $\vartheta(\text{refl}(z; Z), a, b; T)\pterm$ and by deconstruction the definition $a\pterm$ and $T\ptype$.
        Thus, $\text{refl}(a; T)\pterm$.
    \end{proofcase}
\end{proof}

\begin{lemma}
    \label{lem:2:classify_preservation}
    If $\mathcal{C}(s)$ is defined and $s \betastar t$ then $\mathcal{C}(s) = \mathcal{C}(t)$
\end{lemma}
\begin{proof}
    By induction on $s \betastar t$ and Lemma~\ref{lem:2:classify_preservation_step}.
\end{proof}

\begin{theorem}[Soundness of $\mathcal{C}(-)$]
    \label{lem:2:classify_soundness}
    \textcolor{white}{\_}
    \begin{enumerate}
        \item If $\Gamma \vdash t : A$ and $A = \kind$ then $t\pkind$
        \item If $\Gamma \vdash t : A$ and $\Gamma \vdash A : \kind$ then $t\ptype$
        \item If $\Gamma \vdash t : A$ and $\Gamma \vdash A : \star$ then $t\pterm$
    \end{enumerate}
\end{theorem}
\begin{proof}
    By induction on $\Gamma \vdash t : A$.
    The \textsc{Fst} amd \textsc{PrmFst} rules are omitted.

    $\text{Case: }\begin{array}{c} \AxiomRule[*] \end{array}$
    \begin{proofcase}
        Have $\star\pkind$ and $A = \kind$, hence trivial.
    \end{proofcase}

    $\text{Case: }\begin{array}{c} \VarRule[*] \end{array}$
    \begin{proofcase}
        If $K = \kind$ then $x_\square\ptype$ and $\Gamma \vdash A : \kind$.
        Otherwise, $K = \star$ and then $x_\star\pterm$ with $\Gamma \vdash A : \star$.
    \end{proofcase}

    $\text{Case: }\begin{array}{c} \PiRule[*] \end{array}$
    \begin{proofcase}
        Suppose w.l.o.g. that $m = \tau$, now by the IH applied to $\D{1}$: $A\pkind$ or $A\ptype$.
        Applying the IH to $\D{2}$ gives $B\pkind$.
        Thus, $(x : A) \to_\tau B\pkind$.
    \end{proofcase}

    $\text{Case: }\begin{array}{c} \LambdaRule[*] \end{array}$
    \begin{proofcase}
        Suppose w.l.o.g. that $m = \tau$.
        Applying the IH to $\D{1}$ gives $A\pkind$ or $A\ptype$.
        Note by $\D{2}$ that $\Gamma, x_\tau : A \vdash B : \kind$.
        Thus, applying the IH to $\D{2}$ yields $t\ptype$.
        Hence, $\abs{\lambda_\tau}{x}{A}{t}\ptype$.
    \end{proofcase}

    $\text{Case: }\begin{array}{c} \AppRule[*] \end{array}$
    \begin{proofcase}
        Suppose w.l.o.g. that $m = \tau$.
        By Lemma~\ref{lem:2:classification} and inversion with $\D{1}$: $\Gamma \vdash (x : A) \to_\tau B : \kind$.
        Deconstructing this judgment yields $\Gamma \vdash A : K$.
        Applying the IH to $\D{2}$ gives $a\ptype$ or $a\pterm$.
        Applying the IH to $\D{1}$ yields $f\ptype$.
        If $f$ is not an abstraction then the proof is done, thus suppose $f = \abs{\lambda}{x}{A}{t}$.
        Have $A\pkind$ or $A\ptype$, but note that $\Gamma \vdash A : K$ thus the classification of $a$ and $A$ must agree.
        Moreover, $t\pterm$.
        Suppose w.l.o.g. that $a\ptype$ then $\ptrunc{\mathcal{C}(a)} = x_\square$.
        However, this means that $\Gamma \vdash A : \kind$ and that $\Gamma, x_\tau : A \vdash t : B$.
        Thus, the occurrences of $x$ in $t$ must be annotated as $x_\square$ otherwise the \textsc{Var} rule for $x$ would fail.
        Hence, $[x := x_\square]t = t$.
    \end{proofcase}

    $\text{Case: }\begin{array}{c} \IntersectionRule[*] \end{array}$
    \begin{proofcase}
        Applying the IH to $\D{1}$ and $\D{2}$ gives $A, B\ptype$.
        Hence, $(x : A) \cap B\ptype$.
    \end{proofcase}

    $\text{Case: }\begin{array}{c} \PairRule[*] \end{array}$
    \begin{proofcase}
        Deconstructing $\D{1}$ gives $\Gamma \vdash A : \star$ and $\Gamma, x : A \vdash B : \star$.
        Lemma~\ref{lem:2:subst} gives $\Gamma \vdash [x := t]B : \star$.
        Using the IH on $\D{1}$, $\D{2}$, and $\D{3}$ yields $(x : A) \cap B\ptype$ and $t, s\pterm$.
        Thus, $[t, s; (x : A) \cap B]\pterm$.
    \end{proofcase}

    $\text{Case: }\begin{array}{c} \SecondRule[*] \end{array}$
    \begin{proofcase}
        By Lemma~\ref{lem:2:classification} and inversion on $\D{1}$: $\Gamma \vdash (x : A) \cap B : \star$.
        Using the IH on $\D{1}$ gives $t\pterm$.
        Hence, $t.2\pterm$.
    \end{proofcase}

    $\text{Case: }\begin{array}{c} \EqualityRule[*] \end{array}$
    \begin{proofcase}
        Applying the IH to $\D{1}$, $\D{2}$, and $\D{3}$ yields $A\ptype$ and $a, b\pterm$.
        Hence, $a =_A b\ptype$.
    \end{proofcase}

    $\text{Case: }\begin{array}{c} \ReflRule[*] \end{array}$
    \begin{proofcase}
        Applying the IH to $\D{1}$ and $\D{2}$ gives $A\ptype$ and $t\pterm$.
        Hence, $\prefl(t; A)\pterm$.
    \end{proofcase}

    \begin{minipage}{.8\textwidth}$\text{Case: }\begin{array}{c} \SubstRule[*] \end{array}$\end{minipage}
    \begin{proofcase}
        Lemma~\ref{lem:2:classification} and inversion on $\D{4}$ gives $\Gamma \vdash a =_A b : \star$.
        Likewise, $\Gamma \vdash (y : A) \to_\tau (p : a =_A y_\star) \to_\tau \star : \kind$.
        Applying the IH to all subderivations yields $A, P\ptype$ and $a, b, e\pterm$.
        Hence, $\psi(e, a, b; A, P)\pterm$.
    \end{proofcase}

    $\text{Case: }\begin{array}{c} \PromoteRule[*] \end{array}$
    \begin{proofcase}
        By Lemma~\ref{lem:2:classification}, inversion and the IH used with $\D{4}$: $e\pterm$.
        The IH applied to $\D{1}$, $\D{2}$ and $\D{3}$ yields $a, b\pterm$ and $(x : A) \cap B\ptype$.
    \end{proofcase}

    $\text{Case: }\begin{array}{c} \CastRule[*] \end{array}$
    \begin{proofcase}
        By Lemma~\ref{lem:2:classification}: $\Gamma \vdash (x : A) \cap B : K$.
        However, $K = \kind$ is impossible by inversion.
        Using the IH and inversion applied to the sub-derivations yields: $a, b, e\pterm$.
        Thus, $\varphi(a, b, e)\pterm$.
    \end{proofcase}

    $\text{Case: }\begin{array}{c} \SeparationRule[*] \end{array}$
    \begin{proofcase}
        Lemma~\ref{lem:2:classification}, inversion, and the IH applied to $\D{1}$ gives $e\pterm$.
        Hence, $\delta(e)\pterm$.
    \end{proofcase}

    $\text{Case: }\begin{array}{c} \ConvRule[*] \end{array}$
    \begin{proofcase}
        Lemma~\ref{lem:2:classification}, inversion, $\D{1}$ and $\D{3}$ yield $\Gamma \vdash B : K$.
        Suppose w.l.o.g. that $K = \star$.
        Applying the IH to $\D{2}$ gives $t\pterm$.
    \end{proofcase}
\end{proof}
