\section{Inference Judgment}
\label{sec:2:judgments}


The inference judgment, presented in Figure~\ref{fig:2:typ1}; Figure~\ref{fig:2:typ2}; and Figure~\ref{fig:2:typ3}, delineate what syntax are \textit{proofs}.
As stated previously, the erasure of a proof is an \textit{object}.
Thus, for $\Gamma \vdash t : A$, $t$ is a proof and $|t|$ it's object.
The judgment follows a standard PTS style, but the rules are carefully chosen so that an inference algorithm is possible.
Judgments of the form $\Gamma \vdash t : A$ should be read $t$ infers $A$ in $\Gamma$.

$\AxiomRule$ The axiom rule is the same as with F$^\omega$.
The constant $\star$ should be interpreted as a universe of types, and the constant $\kind$ as a universe of kinds.
Thus, the axiom rule states that the universe of types \textit{is} a kind in any context.

$\VarRule$ The variable rule requires that a variable at a certain type is inside the context.
Note that variables are annotated with a mode.
Modes take three forms: free ($\omega$); erased ($0$); or type ($\tau$).
The type mode is used for proofs that exist inside the type universe; the free mode for proofs that belong to some type; and the erased mode for proofs that belong to some type but whose bound variable is not computationally relevant in the associated object.
Variables are annotated with modes primarily to enable reconstruction of the appropriate binders.


\begin{figure}
    \centering
    \begin{minipage}{0.5\textwidth}
        \begin{align*}
            \text{dom}_\Pi(\omega, K) &= \star \\
            \text{dom}_\Pi(\tau, K) &= K \\
            \text{dom}_\Pi(0, K) &= K
        \end{align*}
    \end{minipage}%
    \begin{minipage}{0.5\textwidth}
        \begin{align*}
            \text{codom}_\Pi(\omega) &= \star \\
            \text{codom}_\Pi(\tau) &= \kind \\
            \text{codom}_\Pi(0) &= \star
        \end{align*}
    \end{minipage}
    \caption{Domain and codomains for function types. The metavariable $K$ is either $\star$ or $\kind$.}
\end{figure}


\begin{figure}
    \centering
    \begin{minipage}{0.5\textwidth}
        $$\AxiomRule$$
    \end{minipage}%
    \begin{minipage}{0.5\textwidth}
        $$\VarRule$$
    \end{minipage}%
    $$\ConvRule$$
    $$\PiRule$$
    $$\LambdaRule$$
    $$\AppRule$$
    \caption{
        Inference rules for function types, including erased functions.
    }
    \label{fig:2:typ1}
\end{figure}


$\PiRule$ The function type formation rule is similar to the rule for CC, but the domain and codomain are restricted.
Instead of being part of either a type or kind universe, the respective universes are restricted by the associated mode.
If the mode is $\tau$ then the domain can be either a type or a kind, but the codomain must be a kind.
If the mode is $\omega$ then the domain and codomain both must be types.
Otherwise, the mode is $0$ and the domain may be either a type or kind, but the codomain must be a type.
Note that this means polymorphic functions of data are not allowed to use their type argument computational in the object of a proof.

$\LambdaRule$ The function formation rule is again similar to the rule for CC.
Unlike the standard PTS CC rule, the codomain of the inferred function type is again restricted to $\pcodom(m)$.
Additionally, if the mode is erased then it must be explicitly shown that the bound variable does not appear in the associated object.
Note that this is exactly the requirement imposed by pseudo-objects.

$\AppRule$ The application rule is not surprising, the only notable feature is that the mode of the function type and the application must match.

$\IntersectionRule$ The intersection type formation rule is similar to the function type formation rule, but the terms are all restricted to be types.
Thus, there are no intersections of kinds in the core Cedille2 system.

$\PairRule$ The pair formation rule is standard for formation of dependent pairs.
A third type annotation argument is required in order to make the formula inferable from the proof.
Otherwise, the annotation is required to be itself a type, the first component to match the first type, and the second component to match the second type with its free variable substituted with the first component.
Additionally, the first and second component must be convertible.
This restriction is what makes this a proof of an intersection, as opposed to merely a pair.
Note that by Theorem~\ref{thm:2:beta_iff_conv} this condition is equivalent to $|t| \betaconv |s|$ which is the restriction imposed by pseudo-objects.

$\SecondRule$ The first and second projection rules are unsurprising.
Both rules model projection from a pair as expected.

$\EqualityRule$ The equality type formation rule requires that the type annotation is a type and that the left and right-hand sides infer that type.
Note that a typed equality like this is standard from the perspective of modern type theory but significantly different from the \textit{untyped} equality of Cedille.
Indeed, the equality rules are the area of significant deviation from the original Cedille design.


\begin{figure}
    \centering
    $$\IntersectionRule$$
    $$\PairRule$$
    \begin{minipage}{0.5\textwidth}
        $$\FirstRule$$
    \end{minipage}%
    \begin{minipage}{0.5\textwidth}
        $$\SecondRule$$
    \end{minipage}
    \caption{
        Inference rules for intersection types.
    }
\end{figure}


$\ReflRule$ The reflexivity rule is the only value for equality types.
It is the standard inductive formulation of the equality type.

$\SubstRule$ The substitution rule is a dependent variation of the Leibniz's Law.
It is a variation of Martin-L\"{o}f's J rule introduced by Pfenning and Paulin-Mohring \cite{pfenning1990_subst}.
Notice that the only critical difference between this rule and a standard variation of Leibniz's Law is that the predicate may depend on the equality proof as well.

$\PromoteFstRule$ The first and second promotion rules enable equational reasoning about intersections.
Indeed, because intersections are not inductive it is difficult to reason about them without some auxiliary rule.
The first promotion rule in particular states that two elements of an intersection are equal if their first projections are equal.
Second projection promotion is very similar except it equates two elements of an intersection if their second projections are equal.

$\CastRule$ The cast rule asserts that a new function $\varphi(f, e)$ exists at the associate type if there is another function, $f$, that is extensionally the identity in an erased context.
The requirement on $e$ is of particular interest to understand.
In general, it states only that $f$ is extensionally the identity function.
However, the additional restriction that $FV(|e|)$ is empty means that variables from the context may not appear in the object of $e$.
Thus, computationally, $e$ cannot become stuck at the fault of the context.
Of course, the input to $e$ may still be a variable and thus prevent reduction.

$\SeparationRule$ The separation rule states only that the equational theory is not degenerate, i.e. that there are at least two distinct proofs.


\begin{figure}
    \centering
    \begin{minipage}{0.5\textwidth}
        $$\EqualityRule$$
    \end{minipage}%
    \begin{minipage}{0.5\textwidth}
        $$\ReflRule$$
    \end{minipage}%
    $$\SubstRule$$
    $$\PromoteRule$$
    $$\CastRule$$
    $$\SeparationRule$$
    \caption{
        Inference rules for equality types where
        $\text{cBool} := (X : \star) \to_0 (x : X_\square) \to_\omega (y : X_\square) \to_\omega X_\square$;
        $\text{ctt} := \abs{\lambda_0}{X}{\star}{
            \abs{\lambda_\omega}{x}{X_\square}{
                \abs{\lambda_\omega}{y}{X_\square}{x_\star}
            }
        }$;
        and
        $\text{cff} := \abs{\lambda_0}{X}{\star}{
            \abs{\lambda_\omega}{x}{X_\square}{
                \abs{\lambda_\omega}{y}{X_\square}{y_\star}
            }
        }$.
    }
    \label{fig:2:typ3}
\end{figure}


The context of a judgment is said to be \textit{well-formed}, written $\vdash \Gamma$, if all variables in $\Gamma$ are distinct and for every $\Gamma, x : A, \Delta$ it is the case that $\Gamma \vdash A : K$.
In other words, all types in a context must be proofs in the associated context prefix.
This condition is not automatically met by the judgment, but there are no proofs of interest where this condition fails that will be considered.
Thus, whenever $\Gamma \vdash t : A$ it is assumed that $\vdash \Gamma$.

An important observation is that proofs and their types are a richer form of pseudo-objects.
Thus, conversion is an equivalence relation for proofs and their types.
Other basic lemmas of importance are the admissibility of a weakening rule, and a substitution rule.

\begin{lemma}
    If $\Gamma \vdash t : A$ then $t\pseobj$
    \label{lem:2:infer_implies_pseobj}
\end{lemma}
\begin{proof}
    Straightforward by induction.
    The only interesting case is the pair case, but it is discharged by Theorem~\ref{thm:2:beta_iff_conv}.
\end{proof}

\begin{lemma}
    If $\Gamma \vdash t : A$ then $A\pseobj$
    \label{lem:2:infer_implies_pseobj_type}
\end{lemma}
\begin{proof}
    By induction.
    The \textsc{Ax}, \textsc{Pi}, \textsc{Int} and \textsc{Eq} rules are trivial.
    Rules \textsc{Lam}, \textsc{Pair}, and \textsc{Conv} rules are immediate by applying Lemma~\ref{lem:2:infer_implies_pseobj} to a sub-derivation.
    The \textsc{Cast} rule is immediate by applying the IH to a sub-derivation.
    The \textsc{Fst} and \textsc{App} rules are omitted because it is similar to the \textsc{Snd} rule.
    Likewise, the \textsc{PrmFst} and \textsc{Refl} rules are omitted because it is similar to the \textsc{PrmSnd} rule.

    $\text{Case: }\begin{array}{c} \VarRule[*] \end{array}$
    \begin{proofcase}
        Note that it is assumed that $\vdash \Gamma$.
        Thus, there is some prefix of $\Gamma$, call it $\Delta$, such that $\Delta \vdash A : K$.
        By Lemma~\ref{lem:2:infer_implies_pseobj}: $A\pseobj$.
    \end{proofcase}

    $\text{Case: }\begin{array}{c} \SecondRule[*] \end{array}$
    \begin{proofcase}
        By the IH applied to $\D{1}$: $B\pseobj$.
        Using Lemma~\ref{lem:2:infer_implies_pseobj} gives $t\pseobj$ and thus $t.1\pseobj$.
        Now by Lemma~\ref{lem:2:pseobj_subst}: $[x := t.1]B\pseobj$.
    \end{proofcase}

    $\text{Case: }\begin{array}{c} \SubstRule[*] \end{array}$
    \begin{proofcase}
        By Lemma~\ref{lem:2:infer_implies_pseobj}: $P, e\pseobj$.
        Applying the IH to $\D{1}$ gives $A, a, b\pseobj$.
        Now building up the subexpressions using pseudo-object rules concludes the proof.
    \end{proofcase}

    $\text{Case: }\begin{array}{c} \PromoteSndRule[*] \end{array}$
    \begin{proofcase}
        Applying the IH to $\D{1}$ gives that $(x : A) \cap B\pseobj$.
        Now, by Lemma~\ref{lem:2:infer_implies_pseobj}: $a, b\pseobj$.
        Using the pseudo-object rule for equality concludes the case.
    \end{proofcase}

    $\text{Case: }\begin{array}{c} \SeparationRule[*] \end{array}$
    \begin{proofcase}
        Immediate by a short sequence of \pseobj rules.
    \end{proofcase}
\end{proof}

\begin{lemma}[Weakening]
    If $\Gamma, \Delta \vdash t : A$ and $\Gamma \vdash B : K$ then $\Gamma, x_m : B, \Delta \vdash t : A$ for $x$ fresh
    \label{lem:2:weakening}
\end{lemma}
\begin{proof}
    By induction.
    Most cases are a direct consequence of applying the IH to sub-derivations and applying the associated rule.
    Note that it is assumed that $\vdash \Gamma, \Delta$, and thus $\vdash \Gamma$.
    Now, because $B$ is a proof it is obvious that $\vdash \Gamma, x_m : B, \Delta$.

    $\text{Case: }\begin{array}{c} \AxiomRule[*] \end{array}$
    \begin{proofcase}
        Trivial by axiom rule.
    \end{proofcase}

    $\text{Case: }\begin{array}{c} \VarRule[*] \end{array}$
    \begin{proofcase}
        If $(x_m : A) \in \Gamma, \Delta$ then $(x_m : A) \in \Gamma, y : B, \Delta$.
    \end{proofcase}

    $\text{Case: }\begin{array}{c} \PiRule[*] \end{array}$
    \begin{proofcase}
        The IH applied to $\D{1}$ and $\D{2}$ and the pi-rule concludes the case.
    \end{proofcase}
\end{proof}

\begin{lemma}[Substitution]
    Suppose $\Gamma \vdash b : B$.
    \begin{enumerate}
        \item If $\Gamma, x : B, \Delta \vdash t : A$ then $\Gamma, [x := b]\Delta \vdash [x := b]t : [x := b]A$
        \item If $\vdash \Gamma, x : B, \Delta$ then $\vdash \Gamma, [x := b]\Delta$
    \end{enumerate}
    \label{lem:2:substitution}
\end{lemma}
\begin{proof}
    By mutual recursion.
    The \textsc{Ax} rule is trivial and omitted.
    The rules \textsc{Lam} and \textsc{Int} are very similar to the \textsc{Pi} rule.
    The rules \textsc{Fst}, \textsc{Eq}, \textsc{Refl}, \textsc{Subst}, \textsc{PrmFst}, \textsc{PrmSnd}, \textsc{Cast} and \textsc{Sep} rules are proven by applying \textit{1.} to sub-derivations and using the associated rule.
    Rule \textsc{Snd} is very similar to \textsc{App} and thus omitted.
    Likewise, \textsc{Conv} is very similar to \textsc{Pair} and thus omitted.
    Note that the context cannot be empty.

    $\text{Case: }\vdash \Gamma, x : B, \Delta^\prime, y : A$
    \begin{proofcase}
        Note that $\Delta^\prime$ is a smaller context, thus by \textit{2.} $\vdash [x := b]\Delta^\prime$.
        Moreover, it is the case that $\Gamma, x : B, \Delta^\prime \vdash A : K$.
        Now, using \textit{1.} with the previous derivation gives $\Gamma, [x := b]\Delta^\prime \vdash [x := b]A : K$.
        Thus, the context remains well-formed.
    \end{proofcase}

    $\text{Case: }\begin{array}{c} \VarRule[*] \end{array}$
    \begin{proofcase}
        Rename to $y$.
        If $y \neq x$ then suppose wlog that $(y : A) \in \Delta$.
        Then $y : [x := b]A \in [x := b]\Delta$.
        Thus, $\Gamma, [x := b]\Delta \vdash y : [x := b]A$.
        Suppose $y = x$, then $[x := b]y = b$.
        Note that $[x := b]B = B$, because $\vdash \Gamma, x : B, \Delta$ forces $x \notin FV(B)$.
        Moreover, $A = B$ because $y = x$.
        Thus, $\Gamma, [x := b]\Delta \vdash [x := b]y : [x := b]A$.
    \end{proofcase}

    $\text{Case: }\begin{array}{c} \PiRule[*] \end{array}$
    \begin{proofcase}
        Applying \textit{1.} to the sub-derivations gives:
        \begin{enumerate}
            \item[$\D{1}$.] $\Gamma, [x := b]\Delta \vdash [x := b]A : \pdom(m, K)$
            \item[$\D{2}$.] $\Gamma, [x := b]\Delta, y_m : [x := b]A \vdash [x := b]B : \pcodom(m)$
        \end{enumerate}
        Thus, $\Gamma, [x := b]\Delta \vdash (y : [x := b]A) \to_m [x := b]B : \pcodom(m)$.
    \end{proofcase}

    $\text{Case: }\begin{array}{c} \AppRule[*] \end{array}$
    \begin{proofcase}
        Applying \textit{1.} to $\D{1}$ and $\D{2}$ gives $\Gamma, [x := b]\Delta \vdash [x := b]f : (y : [x := b]A) \to_m [x := b]B$ and $\Gamma, [x := b]\Delta, y_m : [x := b]A \vdash [x := b]a : [x := b]A$.
        By the \textsc{App} rule $\Gamma, [x := b]\Delta \vdash \app{[x := b]f}{m}{[x := b]a} : [y := a][x := b]B$.
        Note that $y$ is fresh to $\Gamma$, thus $y \notin FV(b)$.
        By Lemma~\ref{lem:2:subst_commute} $[y := a][x := b]B = [x := b][y := a]B$.
    \end{proofcase}

    $\text{Case: }\begin{array}{c} \PairRule[*] \end{array}$
    \begin{proofcase}
        Applying \textit{1.} to the sub-derivations gives:
        \begin{enumerate}
            \item[$\D{1}$.] $\Gamma, [x := b]\Delta \vdash (y : [x := b]A) \cap [x := b]B : \star$
            \item[$\D{2}$.] $\Gamma, [x := b]\Delta \vdash [x := b]t : [x := b]A$
            \item[$\D{3}$.] $\Gamma, [x := b]\Delta \vdash [x := b]s : [x := b][y := t]B$
        \end{enumerate}
        Note that $y$ is locally-bound and thus $y \notin FV(\Gamma)$, thus by Lemma~\ref{lem:2:subst_commute} $$[x := b][y := t]B = [y := [x := b]t][x := b]B$$
        Now by Lemma~\ref{lem:2:conv_subst}: $[x := b]t \equiv [x := b]s$.
        Thus, by the \textsc{Pair} rule $\Gamma, [x := b]\Delta \vdash [[x := b]t, [x := b]s] : (y : [x := b]A) \cap [x := b]B$.
    \end{proofcase}
\end{proof}

