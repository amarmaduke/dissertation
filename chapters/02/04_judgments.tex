\section{Inference Judgment}
\label{sec:2:judgments}


The inference judgment, presented in Figure~\ref{fig:2:typ1}; Figure~\ref{fig:2:typ2}; and Figure~\ref{fig:2:typ3}, delineate what syntax are \textit{proofs}.
As stated previously, the erasure of a proof is an \textit{object}.
Thus, for $\Gamma \vdash t : A$, $t$ is a proof and $|t|$ its object.
The judgment follows a standard PTS style, but the rules are carefully chosen so that an inference algorithm is possible.
Judgments of the form $\Gamma \vdash t : A$ should be read $t$ infers $A$ in $\Gamma$.

$\AxiomRule$ The axiom rule is the same as with F$^\omega$.
The constant $\star$ should be interpreted as a universe of types, and the constant $\kind$ as a universe of kinds.
Thus, the axiom rule states that the universe of types \textit{is} a kind in any context.

$\VarRule$ The variable rule requires that a variable at a certain type is inside the context.
Note that variables are annotated with a mode.
Modes take three forms: free ($\omega$); erased ($0$); or type ($\tau$).
The type mode is used for proofs that exist inside the type universe; the free mode for proofs that belong to some type; and the erased mode for proofs that belong to some type but whose bound variable is not computationally relevant in the associated object.
Variables are annotated with modes primarily to enable reconstruction of the appropriate binders.


\begin{figure}
    \centering
    \begin{minipage}{0.5\textwidth}
        \begin{align*}
            \text{dom}_\Pi(\omega, K) &= \star \\
            \text{dom}_\Pi(\tau, K) &= K \\
            \text{dom}_\Pi(0, K) &= K
        \end{align*}
    \end{minipage}%
    \begin{minipage}{0.5\textwidth}
        \begin{align*}
            \text{codom}_\Pi(\omega) &= \star \\
            \text{codom}_\Pi(\tau) &= \kind \\
            \text{codom}_\Pi(0) &= \star
        \end{align*}
    \end{minipage}
    \caption{Domain and codomains for function types. The metavariable $K$ is either $\star$ or $\kind$.}
\end{figure}


\begin{figure}
    \centering
    \begin{minipage}{0.5\textwidth}
        $$\AxiomRule$$
        $$\HeadInferenceRule$$
        $$\ContextEmptyRule$$
    \end{minipage}%
    \begin{minipage}{0.5\textwidth}
        $$\VarRule$$
        $$\CheckRule$$
        $$\ContextAppendRule$$
    \end{minipage}%
    $$\PiRule$$
    $$\LambdaRule$$
    $$\AppRule$$
    \caption{
        Inference rules for function types, including erased functions.
        The variable $K$ is either $\star$ or $\kind$.
    }
\end{figure}


$\PiRule$ The function type formation rule is similar to the rule for CC, but the domain and codomain are restricted.
Instead of being part of either a type or kind universe, the respective universes are restricted by the associated mode.
If the mode is $\tau$ then the domain can be either a type or a kind, but the codomain must be a kind.
If the mode is $\omega$ then the domain and codomain both must be types.
Otherwise, the mode is $0$ and the domain may be either a type or kind, but the codomain must be a type.
Note that this means polymorphic functions of data are not allowed to use their type argument computationally in the object of a proof.

$\LambdaRule$ The function formation rule is again similar to the rule for CC.
Unlike the standard PTS CC rule, the codomain of the inferred function type is again restricted to $\pcodom(m)$.
Additionally, if the mode is erased then it must be explicitly shown that the bound variable does not appear in the associated object.
Note that this is exactly the requirement imposed by pseudo-objects.

$\AppRule$ The application rule is not surprising, the only notable feature is that the mode of the function type and the application must match.

$\IntersectionRule$ The intersection type formation rule is similar to the function type formation rule, but the terms are all restricted to be types.
Thus, there are no intersections of kinds in the core Cedille2 system.

$\PairRule$ The pair formation rule is standard for formation of dependent pairs.
A third type annotation argument is required in order to make the formula inferable from the proof.
Otherwise, the annotation is required to be itself a type, the first component to match the first type, and the second component to match the second type with its free variable substituted with the first component.
Additionally, the first and second component must be convertible.
This restriction is what makes this a proof of an intersection, as opposed to merely a pair.
Note that by Theorem~\ref{thm:2:beta_iff_conv} this condition is equivalent to $|t| \betaconv |s|$ which is the restriction imposed by pseudo-objects.

$\SecondRule$ The first and second projection rules are unsurprising.
Both rules model projection from a pair as expected.

$\EqualityRule$ The equality type formation rule requires that the type annotation is a type and that the left and right-hand sides infer that type.
Note that a typed equality like this is standard from the perspective of modern type theory but significantly different from the \textit{untyped} equality of Cedille.
Indeed, the equality rules are the area of significant deviation from the original Cedille design.


\begin{figure}
    \centering
    \begin{minipage}{0.5\textwidth}
        $$\IntersectionRule$$
        $$\FirstRule$$
    \end{minipage}%
    \begin{minipage}{0.5\textwidth}
        $$\PairRule$$
        $$\SecondRule$$
    \end{minipage}
    \caption{
        Inference rules for intersection types.
    }
\end{figure}


$\ReflRule$ The reflexivity rule is the only value for equality types.
It is the standard inductive formulation of the equality type.

$\SubstRule$ The substitution rule is a dependent variation of the Leibniz's Law.
It is also a variation of Martin-L\"{o}f's J rule introduced by Pfenning and Paulin-Mohring \cite{pfenning1990_subst}.
Notice that the only critical difference between this rule and a standard variation of Leibniz's Law is that the predicate may depend on the equality proof as well.

$\PromoteRule$ The promotion rule enables equational reasoning about intersections.
Indeed, because intersections are not inductive it is difficult to reason about them without some auxiliary rule.
It states that two elements of an intersection are equal if their first projections are equal.
Note that this rule is about dependent intersections.
A non-dependent version involving the second projection is internally derivable in the system.

$\CastRule$ The cast rule is a typed version of the original cast rule of Cedille.
Note that this means this rule enables non-termination.
In Chapter~\ref{chap:5} it is shown that this rule is the only source of non-termination and a precise condition for when it may be used in a terminating way is devised.
The cast rule is critical to the spirit of Cedille.
Thus, simply dropping it to obtain a strongly normalizing system is not really a sufficient choice as it throws too much away in its loss.
An observant reader may notice that there are some redundant type annotations in the rule.
These annotations are present merely to simplify models developed in Chapter~\ref{chap:3}.
A more concise version of the system with fewer redundancies is developed in Chapter~\ref{chap:6}.

$\SeparationRule$ The separation rule states only that the equational theory is not degenerate, i.e. that there are at least two distinct objects.


\begin{figure}
    \centering
    \begin{minipage}{0.5\textwidth}
        $$\EqualityRule$$
    \end{minipage}%
    \begin{minipage}{0.5\textwidth}
        $$\ReflRule$$
    \end{minipage}%
    $$\SubstRule$$
    $$\PromoteRule$$
    $$\CastRule$$
    $$\SeparationRule$$
    \caption{
        Inference rules for equality types where
        $\text{cBool} := (X : \star) \to_0 (x : X_\square) \to_\omega (y : X_\square) \to_\omega X_\square$;
        $\text{ctt} := \abs{\lambda_0}{X}{\star}{
            \abs{\lambda_\omega}{x}{X_\square}{
                \abs{\lambda_\omega}{y}{X_\square}{x_\star}
            }
        }$;
        and
        $\text{cff} := \abs{\lambda_0}{X}{\star}{
            \abs{\lambda_\omega}{x}{X_\square}{
                \abs{\lambda_\omega}{y}{X_\square}{y_\star}
            }
        }$.
    }
    \label{fig:2:typ3}
\end{figure}


The first critical observation to be made is that the syntax participating in an inference judgment are pseudo-objects.
Thus, proofs and their types enjoy transitivity of conversion.
Next three standard lemmas are proved about the type system: weakening, substitution, and a sort-hierarchy classification.

\begin{lemma}
    If $\Gamma \vdash t : A$ then $t\pseobj$
    \label{lem:2:infer_implies_pseobj}
\end{lemma}
\begin{proof}
    Straightforward by induction.
    The only interesting case is the pair case, but it is discharged by Theorem~\ref{thm:2:beta_iff_conv}.
\end{proof}

\begin{lemma}
    If $\Gamma \vdash t : A$ then $A\pseobj$
    \label{lem:2:infer_implies_pseobj_type}
\end{lemma}
\begin{proof}
    By induction.
    The \textsc{Ax}, \textsc{Pi}, \textsc{Int} and \textsc{Eq} rules are trivial.
    Rules \textsc{Lam}, \textsc{Pair}, \textsc{Cast}, and \textsc{Conv} rules are immediate by applying Lemma~\ref{lem:2:infer_implies_pseobj} to a sub-derivation.
    The \textsc{Fst} and \textsc{App} rules are omitted because it is similar to the \textsc{Snd} rule.
    Likewise, the \textsc{Refl} rule is omitted because it is similar to the \textsc{Prm} rule.

    $\text{Case: }\begin{array}{c} \VarRule[*] \end{array}$
    \begin{proofcase}
        By Lemma~\ref{lem:2:infer_implies_pseobj} applied to $\D{2}$: $A\pseobj$.
    \end{proofcase}

    $\text{Case: }\begin{array}{c} \SecondRule[*] \end{array}$
    \begin{proofcase}
        By the IH applied to $\D{1}$: $B\pseobj$.
        Using Lemma~\ref{lem:2:infer_implies_pseobj} gives $t\pseobj$ and thus $t.1\pseobj$.
        Now by Lemma~\ref{lem:2:pseobj_subst}: $[x := t.1]B\pseobj$.
    \end{proofcase}

    $\text{Case: }\begin{array}{c} \SubstRule[*] \end{array}$
    \begin{proofcase}
        By Lemma~\ref{lem:2:infer_implies_pseobj}: $P, e\pseobj$.
        Applying the IH to $\D{1}$ gives $A, a, b\pseobj$.
        Now building up the subexpressions using pseudo-object rules concludes the proof.
    \end{proofcase}

    $\text{Case: }\begin{array}{c} \PromoteRule[*] \end{array}$
    \begin{proofcase}
        Applying the IH to $\D{1}$ gives that $(x : A) \cap B\pseobj$.
        Now, by Lemma~\ref{lem:2:infer_implies_pseobj}: $a, b\pseobj$.
        Using the pseudo-object rule for equality concludes the case.
    \end{proofcase}

    $\text{Case: }\begin{array}{c} \SeparationRule[*] \end{array}$
    \begin{proofcase}
        Immediate by a short sequence of pseudo-object rules.
    \end{proofcase}
\end{proof}

\begin{lemma}[Weakening]
    \label{lem:2:weakening}
    If $\Gamma; \Delta \vdash t : A$ and $\Gamma \vdash B : K$ then $\Gamma; y_m : B; \Delta \vdash t : A$ for $y$ fresh
\end{lemma}
\begin{proof}
    By induction.
    Most cases are a direct consequence of applying the IH to sub-derivations and applying the associated rule.

    $\text{Case: }\begin{array}{c} \AxiomRule[*] \end{array}$
    \begin{proofcase}
        Trivial by axiom rule.
    \end{proofcase}

    $\text{Case: }\begin{array}{c} \VarRule[*] \end{array}$
    \begin{proofcase}
        Note that $y$ is fresh thus $x \neq y$.
        If $y$ is placed after $x$ then the case is trivial because $\Gamma_2$ is only constrained to carry fresh variables.
        Thus, suppose $y$ is placed before $x$.
        Let $\Gamma_1 = \Delta_1; \Delta_2$.
        Applying the IH to $\D{2}$ gives $\Delta_1; y_m : B; \Delta_2 \vdash A : K$.
        The \textsc{Var} rule concludes.
    \end{proofcase}

    $\text{Case: }\begin{array}{c} \PiRule[*] \end{array}$
    \begin{proofcase}
        The IH applied to $\D{1}$ and $\D{2}$ and the pi-rule concludes the case.
    \end{proofcase}
\end{proof}

\begin{lemma}[Substitution]
    \label{lem:2:subst}
    Suppose $\Gamma \vdash b : B$.
    If $\Gamma, y : B, \Delta \vdash t : A$ then $\Gamma, [y := b]\Delta \vdash [y := b]t : [y := b]A$
\end{lemma}
\begin{proof}
    By induction on $\Gamma, y : B, \Delta \vdash t : A$.
    The \textsc{Ax} rule is trivial and omitted.
    The rules \textsc{Lam} and \textsc{Int} are very similar to the \textsc{Pi} rule.
    The rules \textsc{Fst}, \textsc{Eq}, \textsc{Refl}, \textsc{Subst}, \textsc{Prm}, \textsc{Cast} and \textsc{Sep} rules are proven by applying \textit{1.} to sub-derivations and using the associated rule.
    Rule \textsc{Snd} is very similar to \textsc{App} and thus omitted.
    Likewise, \textsc{Conv} is very similar to \textsc{Pair} and thus omitted.

    $\text{Case: }\begin{array}{c} \VarRule[*] \end{array}$
    \begin{proofcase}
        Suppose wlog that $y \in \Gamma_1$.
        Let $\Gamma_1 = \Delta_1; y : B; \Delta_2$.
        Applying the IH to $\D{1}$ gives $\Delta_1; [y := b]\Delta_2 \vdash [y := b]A : K$.
        Note that $x \notin FV(\Delta_1; [y := b]\Delta_2; [y := b]\Gamma_2)$.
        Thus by the \textsc{Var} rule: $\Delta_1; [y := b]\Delta_2; x_m : [y := b]A; [y := b]\Gamma_2 \vdash x_K : [y := b]A$.
    \end{proofcase}

    $\text{Case: }\begin{array}{c} \PiRule[*] \end{array}$
    \begin{proofcase}
        Applying \textit{1.} to the sub-derivations gives:
        \begin{enumerate}
            \item[$\D{1}$.] $\Gamma, [y := b]\Delta \vdash [y := b]A : \pdom(m, K)$
            \item[$\D{2}$.] $\Gamma, [y := b]\Delta, x_m : [y := b]A \vdash [y := b]B : \pcodom(m)$
        \end{enumerate}
        Thus, $\Gamma, [y := b]\Delta \vdash (x : [y := b]A) \to_m [y := b]B : \pcodom(m)$.
    \end{proofcase}

    $\text{Case: }\begin{array}{c} \AppRule[*] \end{array}$
    \begin{proofcase}
        Applying \textit{1.} to $\D{1}$ and $\D{2}$ gives:
        \begin{enumerate}
            \item[$\D{1}$.] $\Gamma, [y := b]\Delta \vdash [y := b]f : (x : [y := b]A) \to_m [y := b]B$
            \item[$\D{2}$.] $\Gamma, [y := b]\Delta, x_m : [y := b]A \vdash [y := b]a : [y := b]A$
        \end{enumerate}
        By the \textsc{App} rule $\Gamma, [y := b]\Delta \vdash \app{[y := b]f}{m}{[y := b]a} : [x := a][y := b]B$.
        Note that $x$ is fresh to $\Gamma$, thus $x \notin FV(b)$.
        By Lemma~\ref{lem:2:subst_commute} $[x := a][y := b]B = [y := b][x := a]B$.
    \end{proofcase}

    $\text{Case: }\begin{array}{c} \PairRule[*] \end{array}$
    \begin{proofcase}
        Applying \textit{1.} to the sub-derivations gives:
        \begin{enumerate}
            \item[$\D{1}$.] $\Gamma, [y := b]\Delta \vdash (x : [y := b]A) \cap [y := b]B : \star$
            \item[$\D{2}$.] $\Gamma, [y := b]\Delta \vdash [y := b]t : [y := b]A$
            \item[$\D{3}$.] $\Gamma, [y := b]\Delta \vdash [y := b]s : [y := b][x := t]B$
        \end{enumerate}
        Note that $x$ is locally-bound and thus $x \notin FV(\Gamma)$, thus by Lemma~\ref{lem:2:subst_commute} $$[y := b][x := t]B = [x := [y := b]t][y := b]B$$
        Now by Lemma~\ref{lem:2:conv_subst}: $[y := b]t \equiv [y := b]s$.
        Thus, by the \textsc{Pair} rule $\Gamma, [y := b]\Delta \vdash [[y := b]t, [y := b]s] : (x : [y := b]A) \cap [y := b]B$.
    \end{proofcase}
\end{proof}

\begin{lemma}
    \label{lem:2:classification}
    If $\Gamma \vdash t : A$ then $A = \kind$ or $\Gamma \vdash A : K$
\end{lemma}
\begin{proof}
    By induction.
    The \textsc{Ax}, \textsc{Pi}, \textsc{Lam}, \textsc{Int}, \textsc{Pair}, \textsc{Eq}, \textsc{Cast}, and \textsc{Conv} rules are trivial.
    The \textsc{Fst} rule is omitted because it is similar to \textsc{Snd} rule.
    Likewise, the \textsc{Refl} rule is omitted because it is similar to the \textsc{Prm} rule.

    $\text{Case: }\begin{array}{c} \VarRule[*] \end{array}$
    \begin{proofcase}
        Immediate by $\D{2}$ and weakening.
    \end{proofcase}

    $\text{Case: }\begin{array}{c} \AppRule[*] \end{array}$
    \begin{proofcase}
        Applying the IH to $\D{1}$ gives $\Gamma \vdash (x : A) \to_m B : K$.
        Now $\Gamma, x : A \vdash B : K$.
        Using the substitution lemma gives $\Gamma \vdash [x := a]B : K$.
    \end{proofcase}

    $\text{Case: }\begin{array}{c} \SecondRule[*] \end{array}$
    \begin{proofcase}
        By the IH applied to $\D{1}$ gives $\Gamma \vdash (x : A )\cap B : K$.
        Thus, $\Gamma, x : A \vdash B : K$.
        Applying the substitution lemma gives $\Gamma \vdash [x := t.1]B : K$.
    \end{proofcase}

    $\text{Case: }\begin{array}{c} \SubstRule[*] \end{array}$
    \begin{proofcase}
        By the \textsc{Refl} rule: $\Gamma \vdash \prefl(a;A) : a =_A a$.
        Now by the \textsc{App} rule $\Gamma \vdash \apptwo{P}{\tau}{a}{\tau}{\prefl(a;A)} : \star$ and $\Gamma \vdash \apptwo{P}{\tau}{b}{\tau}{e} : \star$.
        Using weakening gives $\Gamma, x : \apptwo{P}{\tau}{a}{\tau}{\prefl(a;A)} \vdash \apptwo{P}{\tau}{b}{\tau}{e} : \star$.
        Now the \textsc{Pi} rule concludes the case.
    \end{proofcase}

    $\text{Case: }\begin{array}{c} \PromoteRule[*] \end{array}$
    \begin{proofcase}
        Immediate by applying the \textsc{Eq} rule.
    \end{proofcase}

    $\text{Case: }\begin{array}{c} \SeparationRule[*] \end{array}$
    \begin{proofcase}
        Have $\Gamma \vdash (X : \star) \to_\omega X : \star$ via short sequence of rules.
    \end{proofcase}
\end{proof}

The context of a judgment is, for the moment, unrestrained.
Indeed, a variable may bind a type represented by arbitrary syntax and as long as that variable is never used in the body of the term there is no issue.
To remove these considerations contexts should instead be well-formed:
\begin{definition}
    A context $\Gamma$ is \textbf{well-formed} (written $\vdash \Gamma$) iff for every possible splitting $\Gamma = \Gamma_1, x : A, \Gamma_2$ the variable $x \notin FV(\Gamma_1;\Gamma_2)$ and $\Gamma_1 \vdash A : K$ for some $K$
\end{definition}
It is not difficult to see that an inference judgment with a well-formed context is obtained from any general inference judgment.
Moving forward it will be assumed that the context is well-formed because an equivalent proof is always obtainable under this assumption and the non-well-founded proofs will not add any value.
\begin{lemma}
    If $\Gamma \vdash t : A$ then $\exists\ \Delta$ such that $\Delta \vdash t : A$ and $\vdash \Delta$
\end{lemma}
\begin{proof}
    By Lemma~\ref{lem:2:classification}: $\Gamma \vdash A : K$.
    Now, the set of free variable $S = FV(t) \cup FV(A)$ determines $\Delta$.
    Moreover, every occurrence of $x \in S$ in either $t$ or $A$ must be via a \textsc{Var} rule, hence the associated type is a proof.
    Delete any variables not found in $S$ from $\Gamma$ to form $\Delta$.
\end{proof}
