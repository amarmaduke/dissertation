\section{Syntax and Reduction}


\begin{figure}
    \centering
    \begin{align*}
        t &::= x\ |\ \mathfrak{b}(\kappa_1, x : t_1, t_2)\ |\ \mathfrak{c}(\kappa_2, t_1, \ldots, t_{\mathfrak{a}(\kappa_2)}) \\
        \kappa_1 &::= \lambda_m\ |\ \Pi_m\ |\ \cap \\
        \kappa_2 &::=\diamond\ |\ \star\ |\ \kind\ | \bullet_m |\ \text{pair}\ |\ \text{proj}_1\ |\ \text{proj}_2\ |\ \text{eq}\ |\ \text{refl}\ |\ \psi\ |\ \vartheta_1\ |\ \vartheta_2\ |\ \delta\ |\ \varphi  \\
        m &::= \omega\ |\ 0\ |\ \tau \\
        &\mathfrak{a}(\diamond) = \mathfrak{a}(\star) = \mathfrak{a}(\kind) = 0 \\
        &\mathfrak{a}(\text{proj}_1) = \mathfrak{a}(\text{proj}_2) = \mathfrak{a}(\text{refl}) = \mathfrak{a}(\delta) = 1 \\
        &\mathfrak{a}(\bullet_m) = \mathfrak{a}(\psi) = \mathfrak{a}(\varphi) = 2 \\
        &\mathfrak{a}(\text{pair}) = \mathfrak{a}(\text{eq}) = \mathfrak{a}(\vartheta_1) = \mathfrak{a}(\vartheta_2) = 3
    \end{align*}
    \vspace{-.4in}
    \begin{minipage}{0.5\textwidth}
        \begin{align*}
            \diamond &:= \mathfrak{c}(\diamond) \\
            \star &:= \mathfrak{c}(\star) \\
            \kind &:= \mathfrak{c}(\kind) \\
            \abs{\lambda_m}{x}{t_1}{t_2} &:= \mathfrak{b}(\lambda_m, x : t_1, t_2) \\
            (x : t_1) \to_m t_2 &:= \mathfrak{b}(\Pi_m, x : t_1, t_2) \\
            (x : t_1) \cap t_2 &:= \mathfrak{b}(\cap, x : t_1, t_2) \\
            t_1 \bullet_m t_2 &:= \mathfrak{c}(\bullet_m, t_1, t_2) \\
            \varphi(t_1, t_2) &:= \mathfrak{c}(\varphi, t_1, t_2) \\
            \psi(t_1, t_2) &:= \mathfrak{c}(\psi, t_1, t_2)
        \end{align*}
    \end{minipage}%
    \begin{minipage}{0.5\textwidth}
        \begin{align*}
            [t_1, t_2; t_3] &:= \mathfrak{c}(\text{pair}, t_1, t_2, t_3) \\
            t.1 &:= \mathfrak{c}(\text{proj}_1, t) \\
            t.2 &:= \mathfrak{c}(\text{proj}_2, t) \\
            t_1 =_{t_2} t_3 &:= \mathfrak{c}(\text{eq}, t_1, t_2, t_3) \\
            \text{refl}(t) &:= \mathfrak{c}(\text{refl}, t) \\
            \vartheta_1(t_1, t_2, t_3) &:= \mathfrak{c}(\vartheta_1, t_1, t_2, t_3) \\
            \vartheta_2(t_1, t_2, t_3) &:= \mathfrak{c}(\vartheta_2, t_1, t_2, t_3) \\
            \delta(t) &:= \mathfrak{c}(\delta, t)
        \end{align*}
    \end{minipage}
    \vspace{-.05in}
    %$$J(t_1, t_2, t_3, t_4, t_5, t_6) := \mathfrak{c}(J, t_1, t_2, t_3, t_4, t_5, t_6)$$
    \caption{Generic syntax, there are three constructors, variables, a generic binder, and a generic non-binder. Each are parameterized with a constant tag to specialize to a particular syntactic consruct. The non-binder constructor has a vector of subterms determined by an arity function computed on tags. Standard syntactic constructors are defined in terms of the generic forms.}
    \label{fig:syntax}
\end{figure}


\begin{figure}
    \centering
    \begin{minipage}{0.5\textwidth}
        $$\BinderReductionOne$$
    \end{minipage}%
    \begin{minipage}{0.5\textwidth}
        $$\BinderReductionTwo$$
    \end{minipage}%
    $$\NonbinderReduction$$
    \begin{align*}
        \app{(\abs{\lambda_m}{x}{A}{b})}{m}{t} &\betared [x := t]b \\
        [t_1, t_2, A].1 &\betared t_1 \\
        [t_1, t_2, A].2 &\betared t_2 \\
        J(A, P, x, y, \text{refl}(z), w) &\betared \app{w}{0}{z} \\
        \vartheta(\text{refl}(t.1)) &\betared \text{refl}(t) \\
        \vartheta(\text{refl}(t.2)) &\betared \text{refl}(t) \\
    \end{align*}
    \caption{Reduction rules for arbitrary syntax.}
\end{figure}




Syntax for the system is defined generically as before.
See Figure~\ref{fig:syntax} for a complete description.
The intended meaning of the syntax is as follows:
\begin{enumerate}
    \item tags $\lambda_m$, $\Pi_m$ and $\bullet_m$ (application) represent the function fragment of syntax parameterized by three separates \textit{modes}, $\omega$ (free), $0$ (erased), and $\tau$ (type-level);
    \item tags $\cap$, pair, proj$_1$, and proj$_2$ represent dependent intersections (i.e. dependent pairs);
    \item tags eq, refl, $\psi$ (substitution), $\vartheta$ (promotion), $\delta$ (separation), and $\varphi$ (cast) represent equality.
\end{enumerate}
At the moment raw syntax has no essential meaning beyond its intended one.
Nevertheless, a basic fact about substitution on syntax is provable.

\begin{lemma}
    If $x \neq y$ and $y \notin FV(a)$ then \begin{tightcenter} $[x := a][y := b]t = [y := [x := a]b][x := a]t$ \end{tightcenter}
    \label{lem:2:subst_commute}
\end{lemma}
\begin{proof}
    By induction on $t$.
    If $t$ is a binder or a constructor, then substitution unfolds and the IH applied to subterms concludes the case.
    Suppose $t$ is a variable, $z$.
    If $z = x$, then $z \neq y$ and $t = a$ on both sides because $y \notin FV(a)$.
    If $z = y$, then $z \neq x$ and $t = [x := a]b$ on both sides.
    If $z \neq x$ and $z \neq y$, then $t = z$ on both sides.
\end{proof}

Computational meaning is added via reduction rules described in Figure~\ref{fig:reduction}.
The new reductions model projection of pairs (e.g. $[t_1, t_2, t_3].1 \betared t_1$), promotion of equalities (e.g. $\vartheta(\text{refl}(z; Z), a, b; A) \betared \text{refl}(a; A)$) and an elimination form for equality.
Note that conversion is different from a traditional PTS.
Convertibility with respect to reduction is written: $t \betaconv s$.
A detailed discussion of conversion in $\ced$ is delayed until Section~\ref{sec:2:erasure}.

Before more important facts about reduction can be discussed it is important to observe the interaction between reduction and substitution.
First, note that multistep reduction (i.e. the reflexive-transitive closure of the reduction relation) is congruent with respect to syntax.
Second, substitution is shown to commute with multistep reduction through a series of lemmas.

\begin{lemma}
    If $t_i \betastar t_i^\prime$ for any $i$ then,
    \begin{enumerate}
        \item $\mathfrak{b}(\kappa, (x : t_1), t_2) \betastar \mathfrak{b}(\kappa, (x : t_1^\prime), t_2^\prime)$
        \item $\mathfrak{c}(\kappa, t_1,\ldots,t_{\mathfrak{a}(\kappa)}) \betastar \mathfrak{c}(\kappa, t_1^\prime,\ldots,t_{\mathfrak{a}(\kappa)}^\prime)$
    \end{enumerate}
    \label{lem:2:beta_par}
\end{lemma}
\begin{proof}
    Pick any $i$ and apply the reductions to the associate subterm.
    A straightforward induction on $t_i \betastar t_i^\prime$ demonstrates that the reductions apply only to the associated subterm.
    Repeat until all $i$ reductions are applied.
\end{proof}

\begin{lemma}
    If $a \betared b$ then $[x := t]a \betared [x := t]b$
    \label{lem:2:betared_subst}
\end{lemma}
\begin{proof}
    By induction on $a \betared b$.
    Second projection is the same as first projection case and omitted.

    $\text{Case: }\begin{array}{c} \app{(\abs{\lambda_m}{x}{A}{b})}{m}{t} \betared [x := t]b \end{array}$
    \begin{proofcase}
        $[x := s](\app{(\abs{\lambda_m}{y}{A}{b})}{m}{t}) = \app{(\abs{\lambda_m}{x}{[x := s]A}{[x := s]b})}{m}{[x := s]t} \betared [y := [x := s]t][x := s]b = [x := s][y := t]b$ \\
        Note that the final equality holds by Lemma~\ref{lem:2:subst_commute}.
    \end{proofcase}

    $\text{Case: }\begin{array}{c} [t_1, t_2; A].1 \betared t_1 \end{array}$
    \begin{proofcase}
        $[x := t][t_1, t_2, A].1 = [[x := t]t_1, [x := t]t_2, [x := t]A].1 \betared [x := t]t_1$
    \end{proofcase}

    $\text{Case: }\begin{array}{c} \app{\psi(\text{refl}(z; Z), u, v; A, P)}{\omega}{b} \betared b \end{array}$
    \begin{proofcase}
        $[x := t]\app{\psi(\text{refl}(z; Z), u, v; A, P)}{\omega}{b} = \app{\psi(\text{refl}([x := t]z; [x := t]Z), [x := t]u, [x := t]v; [x := t]A, [x := t]P)}{\omega}{[x := t]b} \betared [x := t]b$
    \end{proofcase}

    $\text{Case: }\begin{array}{c} \vartheta(\text{refl}(z; Z), u, v; A) \betared \text{refl}(u; A) \end{array}$
    \begin{proofcase}
        $[x := t]\vartheta(\text{refl}(z; Z), u, v; A) = \vartheta(\text{refl}([x := t]z; [x := t]Z), [x := t]u, [x := t]v; [x := t]A) \betared \text{refl}([x := t]u; [x := t]A) = [x := t]\text{refl}(u; A)$
    \end{proofcase}

    $\text{Case: }\begin{array}{c} \ConstructorReduction[*] \end{array}$
    \begin{proofcase}
        By the IH, $[x := t]t_i \betared [x := t]t_i^\prime$.
        Note that $$[x := t]\mathfrak{c}(\kappa, t_1, \ldots, t_{\mathfrak{a}}(\kappa)) = \mathfrak{c}(\kappa, [x := t]t_1, \ldots, [x := t]t_{\mathfrak{a}}(\kappa))$$
        Applying the constructor reduction rule and reversing the previous equality concludes the case.
    \end{proofcase}

    $\text{Case: }\begin{array}{c} \BinderOneReduction[*] \end{array}$
    \begin{proofcase}
        By the IH, $[x := t]t_1 \betared [x := t]t_1^\prime$.
        Note that $$[x := t]\mathfrak{b}(\kappa, (y : t_1), t_2) = \mathfrak{b}(\kappa, (y : [x := t]t_1), [x := t]t_2)$$
        Applying the first binder reduction rule and reversing the previous equality concludes the case.
    \end{proofcase}
\end{proof}

\begin{lemma}
    If $a \betastar b$ then $[x := t]a \betastar [x := t]b$
    \label{lem:2:betastar_subst_weak1}
\end{lemma}
\begin{proof}
    By induction on $a \betastar b$.
    The reflexivity case is trivial.

    $\text{Case: }\begin{array}{c} \MultiStep[*] \end{array}$
    \begin{proofcase}
        Let $z = t^\prime$.
        By the IH applied to $\D{2}$: $[x := t]z \betastar [x := t]b$.
        By Lemma~\ref{lem:2:betared_subst} applied to $\D{1}$: $[x := t]a \betared [x := t]z$.
        Applying the transitivity rule yields $[x := t]a \betastar [x := t]b$.
    \end{proofcase}
\end{proof}

\begin{lemma}
    If $s \betared t$ then $[x := s]a \betastar [x := t]a$
    \label{lem:2:betastar_subst_weak2}
\end{lemma}
\begin{proof}
    By induction on $a$.
    The $\mathfrak{c}$ case is omitted because it is similar to the $\mathfrak{b}$ case.

    $\text{Case: }\begin{array}{c} x \end{array}$
    \begin{proofcase}
        Rename $y$.
        Suppose $x = y$, then $[x := s]y = s \betared t = [x := t]y$.
        Thus, $[x := s]y \betastar [x := t]y$.
        Suppose $x \neq y$, then $[x := s]y = y \betastar y = [x := t]y$.
    \end{proofcase}

    $\text{Case: }\begin{array}{c} \mathfrak{b}(\kappa_1, x : t_1, t_2) \end{array}$
    \begin{proofcase}
        By the IH $[x := s]t_1 \betastar [x := t]t_1$ and $[x := s]t_2 \betastar [x := t]t_2$.
        Lemma~\ref{lem:2:beta_par} concludes the case.
    \end{proofcase}
\end{proof}

\begin{lemma}
    If $s \betastar t$ and $a \betastar b$ then $[x := s]a \betastar [x := t]b$
    \label{lem:2:betastar_subst}
\end{lemma}
\begin{proof}
    By induction on $s \betastar t$.
    The reflexivity case is Lemma~\ref{lem:2:betastar_subst_weak1}.

    $\text{Case: }\begin{array}{c} \MultiStep[*] \end{array}$
    \begin{proofcase}
        Let $z = t^\prime$.
        By the IH applied to $\D{2}$: $[x := z]a \betastar [x := t]b$.
        Lemma~\ref{lem:2:betastar_subst_weak2} yields $[x := s]a \betastar [x := z]a$.
        Transitivity concludes with $[x := s]a \betastar [x := t]b$.
    \end{proofcase}
\end{proof}

Lemma~\ref{lem:2:betastar_subst} is the only fact about the interaction of substitution and reduction that is needed moving forward.
A straightforward consequence is a similar lemma about substitution commuting with reduction conversion.

\begin{lemma}
    If $s \betaconv t$ and $a \betaconv b$ then $[x := s]a \betaconv [x := t]b$
    \label{lem:2:betaconv_subst}
\end{lemma}
\begin{proof}
    By definition $\exists\ z_1, z_2$ such that $t \betastar z_1$, $s \betastar z_1$, $a \betastar z_2$, and $b \betastar z_2$.
    Applying Lemma~\ref{lem:2:betastar_subst} twice yields $[x := s]a \betastar [x := z_1]z_2$ and $[x := t]b \betastar [x := z_1]z_2$.
\end{proof}

Transitivity, as before with F$^\omega$, is a consequence of confluence.
Confluence is not an obvious property to obtain and can also be an involved property to prove.
For example, a natural variant for the $\vartheta$ reduction rule is $\vartheta(\text{refl}(t.1)) \betared \text{refl}(t)$, but this breaks confluence.
To see why, consider $\vartheta(\text{refl}([x, y; T].1))$.
One choice leads to $\vartheta(\text{refl}(x))$, and the other leads to $\text{refl}(x)$.
However, these terms are not joinable, hence confluence fails.
