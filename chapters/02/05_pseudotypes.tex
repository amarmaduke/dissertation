\section{Pseudo-kinds and Pseudo-types}

\begin{lemma}
    \textcolor{white}{\_}
    \begin{enumerate}
        \item If $\Gamma \vdash t \infr A$ then $t \neq \kind$ and $\kind$ is not a subexpression of $t$
        \item If $\Gamma \vdash t \cinfr A$ then $t \neq \kind$ and $\kind$ is not a subexpression of $t$
        \item If $\Gamma \vdash t \chck A$ then $t \neq \kind$ and $\kind$ is not a subexpression of $t$
    \end{enumerate}
    \label{lem:2:kind_not_proof}
\end{lemma}
\begin{proof}
    By mutual induction.
    The well-formed context cases are omitted because they are not used in the induction.
    Other omitted cases are immediate by the IH.

    $\text{Case: }\begin{array}{c} \AxiomRule[*] \end{array}$
    \begin{proofcase}
        Have $\star \neq \kind$, hence trivial.
    \end{proofcase}

    $\text{Case: }\begin{array}{c} \VarRule[*] \end{array}$
    \begin{proofcase}
        Have $x \neq \kind$, hence trivial.
    \end{proofcase}

    $\text{Case: }\begin{array}{c} \PiRule[*] \end{array}$
    \begin{proofcase}
        By the IH $A = \neq \kind$, $B \neq \kind$, and $\kind$ is not a subexpression of $A$ or $B$.
        Thus, $\kind$ is not a subexpression of $(x : A) \to_m B$.
    \end{proofcase}
\end{proof}

\begin{lemma}
    \textcolor{white}{\_}
    \begin{enumerate}
        \item If $\Gamma \vdash t \infr A$ then $A = \kind$ or $\kind$ is not a subexpression of $A$
        \item If $\Gamma \vdash t \cinfr A$ then $A = \kind$ or $\kind$ is not a subexpression of $A$
        \item If $\Gamma \vdash t \chck A$ then $A = \kind$ or $\kind$ is not a subexpression of $A$
    \end{enumerate}
    \label{lem:2:kind_not_type}
\end{lemma}
\begin{proof}
    By mutual induction.
    The well-formed context cases are omitted because they are not used in the induction.

    $\text{Case: }\begin{array}{c} \AxiomRule[*] \end{array}$
    \begin{proofcase}
        Have $A = \kind$, hence trivial.
    \end{proofcase}

    $\text{Case: }\begin{array}{c} \VarRule[*] \end{array}$
    \begin{proofcase}
        By Lemma~\ref{lem:2:ctx_get}: $\Gamma \vdash A \cinfr K$.
        Thus, by Lemma~\ref{lem:2:kind_not_proof} $\kind$ is not a subexpression of $A$.
    \end{proofcase}

    $\text{Case: }\begin{array}{c} \PiRule[*] \end{array}$
    \begin{proofcase}
        If $m = \tau$ then $\pcodom(m) = \kind$.
        Otherwise, $\pcodom(m) = \star$ and $\star \neq \kind$.
    \end{proofcase}

    $\text{Case: }\begin{array}{c} \LambdaRule[*] \end{array}$
    \begin{proofcase}
        By Lemma~\ref{lem:2:kind_not_proof} applied to $\D{1}$ and $\D{3}$: $\kind$ is not a subexpression of $A$ or $B$.
        Thus, $\kind$ is not a subexpression of $(x : A) \to_m B$.
    \end{proofcase}

    $\text{Case: }\begin{array}{c} \AppRule[*] \end{array}$
    \begin{proofcase}
        Applying the IH to $\D{1}$ yields that $\kind$ is not a subexpression of $A$ or $B$.
        By Lemma~\ref{lem:2:kind_not_proof} $a \neq \kind$ and $\kind$ is not a subexpression of $a$.
        Thus, $\kind$ is not a subexpression of $[x := a]B$.
    \end{proofcase}

    $\text{Case: }\begin{array}{c} \IntersectionRule[*] \end{array}$
    \begin{proofcase}
        Have $\star \neq \kind$, hence trivial.
    \end{proofcase}

    $\text{Case: }\begin{array}{c} \PairRule[*] \end{array}$
    \begin{proofcase}
        By Lemma~\ref{lem:2:kind_not_proof} applied to $\D{1}$.
    \end{proofcase}

    $\text{Case: }\begin{array}{c} \FirstRule[*] \end{array}$
    \begin{proofcase}
        By the IH applied to $\D{1}$.
    \end{proofcase}

    $\text{Case: }\begin{array}{c} \SecondRule[*] \end{array}$
    \begin{proofcase}
        By the IH applied to $\D{1}$: $\kind$ is not a subexpression of $B$.
        Using Lemma~\ref{lem:2:kind_not_proof} with $\D{1}$ gives that $\kind$ is not a subexpression of $t$.
        Thus, $\kind$ is not a subexpression of $[x := t.1]B$.
    \end{proofcase}

    $\text{Case: }\begin{array}{c} \EqualityRule[*] \end{array}$
    \begin{proofcase}
        Have $\star \neq \kind$, hence trivial.
    \end{proofcase}

    $\text{Case: }\begin{array}{c} \ReflRule[*] \end{array}$
    \begin{proofcase}
        By Lemma~\ref{lem:2:kind_not_proof} applied to $\D{1}$ and $\D{2}$: $t \neq \kind$, $A \neq \kind$, and $\kind$ is not a subexpression of $t$ or $A$.
        Thus, $\kind$ is not a subexpression of $t =_A t$.
    \end{proofcase}

    $\text{Case: }\begin{array}{c} \SubstRule[*] \end{array}$
    \begin{proofcase}
        Applying the IH to $\D{1}$ gives that $\kind$ is not a subexpression of $a$, $b$, or $A$.
        By Lemma~\ref{lem:2:kind_not_proof} applied to $\D{2}$: $P \neq \kind$ and $\kind$ is not a subexpression of $P$.
        Thus, $\kind$ is not a subexpression of $\app{P}{\tau}{a} \to_\tau \app{P}{\tau}{b}$.
    \end{proofcase}

    $\text{Case: }\begin{array}{c} \PromoteFstRule[*] \end{array}$
    \begin{proofcase}
        By the IH on $\D{1}$: $\kind$ is not a subexpression of $A$ or $B$.
        Using Lemma~\ref{lem:2:kind_not_proof} with $\D{1}$ and $\D{2}$ yields that $a \neq \kind$, $b \neq \kind$ and $\kind$ is not a subexpression of $a$ or $b$.
        Thus, $\kind$ is not a subexpression of $a =_{(x : A) \cap B} b$.
    \end{proofcase}

    $\text{Case: }\begin{array}{c} \PromoteSndRule[*] \end{array}$
    \begin{proofcase}
        Similar to previous case.
    \end{proofcase}

    $\text{Case: }\begin{array}{c} \CastRule[*] \end{array}$
    \begin{proofcase}
        Immediate by the IH applied to $\D{1}$.
    \end{proofcase}

    $\text{Case: }\begin{array}{c} \SeparationRule[*] \end{array}$
    \begin{proofcase}
        Immediate by inspection.
    \end{proofcase}

    $\text{Case: }\begin{array}{c} \HeadInferenceRule[*] \end{array}$
    \begin{proofcase}
        By the IH applied to $\D{1}$: $A = \kind$ or $\kind$ is not a subexpression of $A$.
        If $A = \kind$ then inversion on $\kind \betastar B$ forces $B = \kind$.
        If $A \neq \kind$ and $\kind$ is not a subexpression of $A$ then $\kind$ is not a subexpression of $B$ because reduction only reduces the set of singleton subexpressions.
        That is, the leafs of $A$ must be only variables and $\star$ and reduction cannot change this.
    \end{proofcase}

    $\text{Case: }\begin{array}{c} \CheckRule[*] \end{array}$
    \begin{proofcase}
        By Lemma~\ref{lem:2:kind_not_proof} applied to $\D{2}$.
    \end{proofcase}
\end{proof}

\newcommand{\substar}{\mathfrak{sub}_\star}

\begin{align*}
    \substar(\kind) &= \top \\
    \substar(\star) &= \bot \\
    \substar(x) &= \top \\
    \substar((x : \star) \to_m \star) &= \top \\
    \substar((x : \star) \to_m B) &= \substar(B) \\
    \substar((x : A) \to_m \star) &= \substar(A) \\
    \substar(\abs{\lambda_m}{x}{\star}{t}) &= \substar(t) \\
    \substar(\mathfrak{b}(\kappa, (x : t_1), t_2)) &= \substar(t_1) \wedge \substar(t_2) \\
    \substar(\mathfrak{c}(\kappa, t_1, \ldots, t_{\mathfrak{a}(\kappa)})) &= \substar(t_1) \wedge \cdots \wedge \substar(t_{\mathfrak{a}(\kappa)})
\end{align*}

\begin{lemma}
    \textcolor{white}{\_}
    \begin{enumerate}
        \item If $\Gamma \vdash t \infr A$ then $t = \star$ or $\substar(t)$
        \item If $\Gamma \vdash t \cinfr A$ then $t = \star$ or $\substar(t)$
        \item If $\Gamma \vdash t \chck A$ then $t = \star$ or $\substar(t)$
    \end{enumerate}
    \label{lem:2:star_proof_valid}
\end{lemma}
\begin{proof}
    By mutual induction.
    Well-formed context cases are omitted because they are not used in the induction.

    $\text{Case: }\begin{array}{c} \AxiomRule[*] \end{array}$
    \begin{proofcase}
        Have $t = \star$, hence trivial.
    \end{proofcase}

    $\text{Case: }\begin{array}{c} \VarRule[*] \end{array}$
    \begin{proofcase}
        Have $\substar(x) = \top$, hence trivial.
    \end{proofcase}

    $\text{Case: }\begin{array}{c} \PiRule[*] \end{array}$
    \begin{proofcase}
        Applying the IH to both cases gives $A = \star$ or $\substar(A)$ and $B = \star$ or $\substar(B)$.
        In all four configurations $\substar((x : A) \to_m B)$ holds.
    \end{proofcase}

    $\text{Case: }\begin{array}{c} \LambdaRule[*] \end{array}$
    \begin{proofcase}
        By the IH applied to $\D{2}$: $t = \star$ or $\substar(t)$.
        If $t = \star$ then by inversion on $\D{2}$: $B = \kind$.
        However, this is impossible by Lemma~\ref{lem:2:kind_not_proof} applied to $\D{3}$, thus $t \neq \star$.
        Applying the IH to $\D{1}$ gives $A = \star$ or $\substar(A)$.
        In either case $\substar(\abs{\lambda_m}{x}{A}{t})$ holds.
    \end{proofcase}

    $\text{Case: }\begin{array}{c} \AppRule[*] \end{array}$
    \begin{proofcase}
        Applying the IH to $\D{2}$ gives $a = \star$ or $\substar(a)$.
        If $a = \star$ then by inversion on $\D{2}$ $\exists\ T$ such that $\Gamma \vdash a \infr T$ and $A \equiv \kind$.
        Thus, $|T| \betastar \kind$, but this is only possible if $\kind$ is a subexpression of $T$.
        Now by Lemma~\ref{lem:2:kind_not_type} used on $\D{1}$ this situation is impossible, thus $a \neq \star$.
        Using the IH on $\D{1}$ gives $f = \star$ or $\substar(f)$.
        Similarly, by inversion $\exists\ T_2$ such that $\Gamma \vdash f \infr T$ and $(x : A) \to_m B \equiv \kind$, but this is only possible if $\kind$ is a subexpression.
    \end{proofcase}

    $\text{Case: }\begin{array}{c} \IntersectionRule[*] \end{array}$
    \begin{proofcase}
        Applying the IH to both $\D{1}$ and $\D{2}$ gives $A = \star$ or $\substar(A)$ and $B = \star$ or $\substar(B)$.
        However, if $A = \kind$ or $B = \kind$ then by inversion $\kind \equiv \star$ which is impossible.
        Thus, $\substar((x : A) \cap B)$ holds.
    \end{proofcase}

    $\text{Case: }\begin{array}{c} \PairRule[*] \end{array}$
    \begin{proofcase}
        Similar to previous cases, apply the IH and notice that by inversion if $t = \star$ or $s = \star$ then $\kind$ must be a subexpression of $A$ and $B$ respectively.
        Thus, $\substar([t, s; (x : A) \cap B])$ holds.
    \end{proofcase}

    $\text{Case: }\begin{array}{c} \FirstRule[*] \end{array}$
    \begin{proofcase}
        Similar to previous cases.
    \end{proofcase}

    $\text{Case: }\begin{array}{c} \SecondRule[*] \end{array}$
    \begin{proofcase}
        Similar to previous cases.
    \end{proofcase}

    $\text{Case: }\begin{array}{c} \EqualityRule[*] \end{array}$
    \begin{proofcase}
        Similar to previous cases.
    \end{proofcase}

    $\text{Case: }\begin{array}{c} \ReflRule[*] \end{array}$
    \begin{proofcase}
        Similar to previous cases.
    \end{proofcase}

    $\text{Case: }\begin{array}{c} \SubstRule[*] \end{array}$
    \begin{proofcase}
        Similar to previous cases.
    \end{proofcase}

    $\text{Case: }\begin{array}{c} \PromoteFstRule[*] \end{array}$
    \begin{proofcase}
        Similar to previous cases.
    \end{proofcase}

    $\text{Case: }\begin{array}{c} \PromoteSndRule[*] \end{array}$
    \begin{proofcase}
        Similar to previous cases.
    \end{proofcase}

    $\text{Case: }\begin{array}{c} \CastRule[*] \end{array}$
    \begin{proofcase}
        Similar to previous cases.
    \end{proofcase}

    $\text{Case: }\begin{array}{c} \SeparationRule[*] \end{array}$
    \begin{proofcase}
        Similar to previous cases.
    \end{proofcase}

    $\text{Case: }\begin{array}{c} \HeadInferenceRule[*] \end{array}$
    \begin{proofcase}
        Immediate by the IH applied to $\D{1}$.
    \end{proofcase}

    $\text{Case: }\begin{array}{c} \CheckRule[*] \end{array}$
    \begin{proofcase}
        Immediate by the IH applied to $\D{1}$.
    \end{proofcase}
\end{proof}

\begin{lemma}
    If $\substar(t)$ and $\substar(b)$ then $\substar([x := t]b)$
    \label{lem:2:substar_subst}
\end{lemma}
\begin{proof}
    Straightforward by induction on $b$.
\end{proof}

\begin{lemma}
    If $\substar(s)$ and $s \betared t$ then $\substar(t)$
    \label{lem:2:substar_beta_step}
\end{lemma}
\begin{proof}
    By induction on $s \betared t$.
    All axiom cases are trivial except $\beta$-reduction.
    The inductive case for constructors is immediate by the IH except $\kappa = \star$.
    Note that $s \neq \star$ because $\neg \substar(\star)$.
    The remaining cases are shown as follows:

    $\text{Case: }\begin{array}{c} \BinderOneReduction[*] \end{array}$
    \begin{proofcase}
        Suppose $\kappa = \Pi_m$.
        Note that $t_1 = \star$ is impossible by inversion on $\star \betared t_1^\prime$.
        By the IH applied to $\D{1}$: $\substar(t_1^\prime)$.
        Thus, $\substar((x : t_1) \to_m t_2)$ holds.
        If $\kappa = \lambda_m$ then the proof is the same as for $\Pi_m$.
        Otherwise, the subexpressions cannot be $\star$ and the IH concludes the remaining cases.
    \end{proofcase}

    $\text{Case: }\begin{array}{c} \BinderOneReduction[*] \end{array}$
    \begin{proofcase}
        Suppose $\kappa = \Pi_m$
        Note that $t_2 = \star$ is impossible by inversion on $\star \betared t_2^\prime$.
        By the IH applied to $\D{1}$: $\substar(t_1^\prime)$.
        Thus, $\substar((x : t_1) \to_m t_2)$ holds.
        In all other cases applying the IH is sufficient.
    \end{proofcase}

    $\text{Case: }\begin{array}{c} \app{(\abs{\lambda_m}{x}{A}{b})}{m}{t} \betared [x := t]b \end{array}$
    \begin{proofcase}
        Note that $\substar(t)$ and $\substar(b)$ by computation.
        Thus, $\substar([x := t]b)$ holds by Lemma~\ref{lem:2:substar_subst}.
    \end{proofcase}
\end{proof}

\begin{lemma}
    If $\substar(s)$ and $s \betastar t$ then $\substar(t)$
    \label{lem:2:substar_beta}
\end{lemma}
\begin{proof}
    By induction on $s \betastar t$.
    The reflexivity case is trivial.
    Suppose $s \betared z$ and $z \betastar t$.
    By Lemma~\ref{lem:2:substar_beta_step} $\substar(z)$ holds.
    Now by the IH $\substar(t)$ holds.
\end{proof}

\begin{lemma}
    \textcolor{white}{\_}
    \begin{enumerate}
        \item If $\Gamma \vdash t \infr A$ then $A = \star$ or $\substar(A)$
        \item If $\Gamma \vdash t \cinfr A$ then $A = \star$ or $\substar(A)$
        \item If $\Gamma \vdash t \chck A$ then $A = \star$ or $\substar(A)$
    \end{enumerate}
    \label{lem:2:star_type_valid}
\end{lemma}
\begin{proof}
    By mutual induction.
    Well-formed context cases are omitted because they are not used in the induction.

    $\text{Case: }\begin{array}{c} \AxiomRule[*] \end{array}$
    \begin{proofcase}
        Have $\substar(\kind)$, hence trivial.
    \end{proofcase}

    $\text{Case: }\begin{array}{c} \VarRule[*] \end{array}$
    \begin{proofcase}
        By Lemma~\ref{lem:2:ctx_get}: $\Gamma \vdash A \cinfr K$.
        Now using Lemma~\ref{lem:2:star_proof_valid}, $A = \star$ or $\substar(A)$ as required.
    \end{proofcase}

    $\text{Case: }\begin{array}{c} \PiRule[*] \end{array}$
    \begin{proofcase}
        Either $\pcodom(m)$ is $\kind$ or $\star$, in either case the goal is reached.
    \end{proofcase}

    $\text{Case: }\begin{array}{c} \LambdaRule[*] \end{array}$
    \begin{proofcase}
        Applying the IH to $\D{1}$ and $\D{3}$ gives $A = \star$ or $\substar(A)$ and $B = \star$ or $\substar(B)$.
        In all four configurations $\substar((x : A) \to_m B)$ holds.
    \end{proofcase}

    $\text{Case: }\begin{array}{c} \AppRule[*] \end{array}$
    \begin{proofcase}
        Applying the IH to $\D{1}$ gives $\substar((x : A) \to_m B)$ which means $B = \star$ or $\substar(B)$.
        If $B = \star$ then $[x := a]B = \star$.
        Suppose $\substar(B)$.
        Applying Lemma~\ref{lem:2:star_proof_valid} to $\D{2}$ yields $a = \star$ or $\substar(a)$.
        If $a = \star$ then by inversion $\kind \equiv A$ which is impossible by Lemma~\ref{lem:2:kind_not_type}.
        Suppose $\substar(a)$.
        Finally, by Lemma~\ref{lem:2:substar_subst}: $\substar([x := a]B)$ holds.
    \end{proofcase}

    $\text{Case: }\begin{array}{c} \IntersectionRule[*] \end{array}$
    \begin{proofcase}
        Have $A = \star$, hence trivial.
    \end{proofcase}

    $\text{Case: }\begin{array}{c} \PairRule[*] \end{array}$
    \begin{proofcase}
        Immediate by Lemma~\ref{lem:2:star_proof_valid} applied to $\D{1}$.
    \end{proofcase}

    $\text{Case: }\begin{array}{c} \FirstRule[*] \end{array}$
    \begin{proofcase}
        By the IH applied to $\D{1}$: $\substar((x : A) \cap B)$.
        Thus, $\substar(A)$.
    \end{proofcase}

    $\text{Case: }\begin{array}{c} \SecondRule[*] \end{array}$
    \begin{proofcase}
        As with the previous case, $\substar(B)$.
        Using Lemma~\ref{lem:2:star_proof_valid} with $\D{1}$: $\substar(t)$ holds, and thus $\substar(t.1)$ holds.
        Finally, by Lemma~\ref{lem:2:substar_subst}: $\substar([x := t.1]B)$ holds.
    \end{proofcase}

    $\text{Case: }\begin{array}{c} \EqualityRule[*] \end{array}$
    \begin{proofcase}
        Have $A = \star$, hence trivial.
    \end{proofcase}

    $\text{Case: }\begin{array}{c} \ReflRule[*] \end{array}$
    \begin{proofcase}
        Applying the IH to $\D{1}$ gives $A = \kind$ or $\substar(A)$.
        If $A = \kind$ then $\star \equiv \kind$ which is impossible.
        Applying Lemma~\ref{lem:2:star_proof_valid} to $\D{1}$ gives $\substar(t)$ and note that $t \neq \star$ for the same reasons as other cases.
    \end{proofcase}

    $\text{Case: }\begin{array}{c} \SubstRule[*] \end{array}$
    \begin{proofcase}
        By the IH applied to $\D{1}$: $\substar(a =_A b)$.
        Computing gives that $\substar(a)$ and $\substar(b)$.
        Using Lemma~\ref{lem:2:star_proof_valid} with $\D{2}$ yields $\substar(P)$, note that $P \neq \star$ for similar reasons as previous cases.
        Now by computation $\substar(\app{P}{\tau}{a})$ and $\substar(\app{P}{\tau}{a})$.
        Finally, by computation $\substar(\app{P}{\tau}{a} \to_\tau \app{P}{\tau}{b})$ holds.
    \end{proofcase}

    $\text{Case: }\begin{array}{c} \PromoteFstRule[*] \end{array}$
    \begin{proofcase}
        Immediate by the IH applied to $\D{1}$.
    \end{proofcase}

    $\text{Case: }\begin{array}{c} \PromoteSndRule[*] \end{array}$
    \begin{proofcase}
        Same as previous case.
    \end{proofcase}

    $\text{Case: }\begin{array}{c} \CastRule[*] \end{array}$
    \begin{proofcase}
        Immediate by the IH applied to $\D{1}$.
    \end{proofcase}

    $\text{Case: }\begin{array}{c} \SeparationRule[*] \end{array}$
    \begin{proofcase}
        Immediate by computation.
    \end{proofcase}

    $\text{Case: }\begin{array}{c} \HeadInferenceRule[*] \end{array}$
    \begin{proofcase}
        Applying the IH to $\D{1}$ gives $A = \star$ or $\substar(A)$.
        If $A = \star$ then $\star \betastar B$ which forces $B = \star$.
        Otherwise, $\substar(B)$ by Lemma~\ref{lem:2:substar_beta}.
    \end{proofcase}

    $\text{Case: }\begin{array}{c} \CheckRule[*] \end{array}$
    \begin{proofcase}
        By Lemma~\ref{lem:2:star_proof_valid}: $\substar(B)$ holds.
    \end{proofcase}
\end{proof}

\begin{lemma}
    If $\substar(t)$ then $\substar(|t|)$
    \label{lem:2:substar_erased}
\end{lemma}
\begin{proof}
    By induction on $t$.
    Note that $t \neq \star$ because $\neg \substar(\star)$.
    Suppose $t$ is a constructor.
    If $\kappa = \text{eq}$, $\bullet_omega$ or $\bullet_\tau$ then erasure is a homomorphism and thus the individual components satisfy the condition by computation.
    If $\kappa = \text{refl}$ or $\kappa = \varphi$ then $|t| = \absu{\lambda_\omega}{x}{x}$ and $\substar(\absu{\lambda_\omega}{x}{x})$ holds.
    Otherwise, the erasure computes to a subexpression and thus by computation the condition holds.
    \\ \\
    Suppose $t$ is a binder.
    If $\kappa = \lambda_0$ or $\lambda_\omega$ then $|t| = |b|$ where $b$ is the body of the $\lambda$-expression, but now $\substar(b)$ by computation.
    If $\kappa = \lambda_\tau$ then either $A = \kind$ or $\substar(A)$ where $A$ is the annotation of $\lambda$-expression.
    However, in either case $\substar(\abs{\lambda_\tau}{x}{A}{b})$.
    A similar argument works for when $\kappa = \Pi_m$.
    Note that if $\kappa = \cap$ then erasure is a homomorphism.
    \\ \\
    Suppose that $t$ is a variable.
    Then $|x| = x$ and $\substar(|x|)$ holds.
\end{proof}

\begin{lemma}
    If $\substar(t)$ then $\neg (t \betared \star)$
    \label{lem:2:substar_not_beta_star_step}
\end{lemma}
\begin{proof}
    Suppose $t \betared \star$ and proceed by induction.
    Note that the inductive cases are impossible by inversion.
    The first projection case is omitted because they are very similar to the second projection case.
    The substitution, first promotion projection, and second promotion projection cases is omitted by inversion because the result of reduction cannot be equal to $\star$.

    $\text{Case: }\begin{array}{c} \app{(\abs{\lambda_m}{x}{A}{b})}{m}{t} \betared [x := t]b \end{array}$
    \begin{proofcase}
        Note that $[x := t]b = \star$, thus $b = \star$ or $b = x$ and $t = \star$.
        In either case $\neg \substar(\app{(\abs{\lambda_m}{x}{A}{b})}{m}{t})$, but this is assumed.
        Hence, contradiction.
    \end{proofcase}
    
    $\text{Case: }\begin{array}{c} [t_1, t_2; A].2 \betared t_2 \end{array}$
    \begin{proofcase}
        Note that $t_2 = \star$, but then $\neg \substar([t_1, \star; A].2)$.
        Hence, contradiction.
    \end{proofcase}
\end{proof}

\begin{lemma}
    If $\substar(t)$ then $\neg (t \betastar \star)$
    \label{lem:2:substar_not_beta_star}
\end{lemma}
\begin{proof}
    Suppose $t \betastar \star$ and proceed by induction.
    Now $t \betared z$ and $z \betastar \star$.
    By Lemma~\ref{lem:2:substar_beta_step} $\substar(z)$.
    Now the IH applied to $z \betastar \star$ produces a contradiction.
\end{proof}

\begin{lemma}
    If $\Gamma \vdash A \chck K$ then $\Gamma \vdash A \infr K$ where $K = \kind$ or $K = \star$
    \label{lem:2:check_k_is_infer_k}
\end{lemma}
\begin{proof}
    Deconstructing the premise gives $\exists\ T$ such that $\Gamma \vdash A \infr T$, $\Gamma \vdash K \cinfr K^\prime$ and $K \equiv T$.
    Suppose $K = \kind$, but this is impossible by Lemma~\ref{lem:2:kind_not_proof}.
    Suppose $K = \star$.
    Using Lemma~\ref{lem:2:star_type_valid}: $T = \star$ or $\substar(T)$.
    Note that $\star \equiv T$ implies that $|T| \betastar \star$.
    Suppose $\substar(T)$, but then $\substar(|T|)$ by Lemma~\ref{lem:2:substar_erased}.
    Thus, this case is impossible by Lemma~\ref{lem:2:substar_not_beta_star}.
    Now $T = \star$ and $\Gamma \vdash A \infr \star$.
\end{proof}
