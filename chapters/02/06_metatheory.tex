\section{Classification and Preservation}

Classification is a critical property of a system like CC with unified syntax.
It allows for the syntax to instead be stratified into levels which would enable an intrinsic presentation of the system.
For the core theory of Cedille2 there are only two universes like the original CC, thus the stratification places terms into three separate levels.
A term is either a \textit{kind} (thus $A = \kind$), a \textit{type-constructor} (thus $\Gamma \vdash A : \kind$), or a \textit{proof-term} (thus $\Gamma \vdash A : \star$).
Note that if $A = \star$ then a term is called simply a \textit{type}.

\begin{theorem}[Classification]
    If $\Gamma \vdash t : A$ then $A = \kind$ or $\Gamma \vdash A : K$
    \label{lem:2:classification}
\end{theorem}
\begin{proof}
    By induction.
    The \textsc{Ax}, \textsc{Pi}, \textsc{Lam}, \textsc{Int}, \textsc{Pair}, \textsc{Eq}, and \textsc{Conv} rules are trivial.
    The \textsc{Fst} and \textsc{PrmFst} rules are omitted because they are very similar to \textsc{Snd} and \textsc{PrmSnd} respectively.

    $\text{Case: }\begin{array}{c} \VarRule[*] \end{array}$
    \begin{proofcase}
        Because $x_m : A \in \Gamma$ then $\Gamma = \Delta_1, x_m : A, \Delta_2$.
        By $\vdash \Gamma$: $\Delta_1 \vdash A : K$.
        Now using weakening $\Gamma \vdash A : K$.
    \end{proofcase}

    $\text{Case: }\begin{array}{c} \AppRule[*] \end{array}$
    \begin{proofcase}
        Applying the IH to $\D{1}$ gives $\Gamma \vdash (x : A) \to_m B : K$.
        Now $\Gamma, x : A \vdash B : K$.
        Using the substitution lemma gives $\Gamma \vdash [x := a]B : K$.
    \end{proofcase}

    $\text{Case: }\begin{array}{c} \SecondRule[*] \end{array}$
    \begin{proofcase}
        By the IH applied to $\D{1}$ gives $\Gamma \vdash (x : A )\cap B : K$.
        Thus, $\Gamma, x : A \vdash B : K$.
        Applying the substitution lemma gives $\Gamma \vdash [x := t.1]B : K$.
    \end{proofcase}

    $\text{Case: }\begin{array}{c} \ReflRule[*] \end{array}$
    \begin{proofcase}
        Immediate by applying the \textsc{Eq} rule.
    \end{proofcase}

    $\text{Case: }\begin{array}{c} \SubstRule[*] \end{array}$
    \begin{proofcase}
        Applying the IH to $\D{1}$ gives $\Gamma \vdash a =_A b : K$.
        By inversion this gives $\Gamma \vdash A : \star$, $\Gamma \vdash a : A$, and $\Gamma \vdash b : A$.
        Now by the \textsc{App} rule $\Gamma \vdash \app{P}{\tau}{a} : \star$ and $\Gamma \vdash \app{P}{\tau}{b} : \star$.
        Using weakening gives $\Gamma, x : \app{P}{\tau}{a} \vdash \app{P}{\tau}{b} : \star$.
        Now the \textsc{Pi} rule concludes the case.
    \end{proofcase}

    $\text{Case: }\begin{array}{c} \PromoteSndRule[*] \end{array}$
    \begin{proofcase}
        By the IH applied to $\D{1}$: $\Gamma \vdash (x : A) \cap B : K$.
        Note that $K$ must be $\star$.
        Now applying the \textsc{Eq} rule gives $\Gamma \vdash a =_{(x : A) \cap B} b : \star$.
    \end{proofcase}

    $\text{Case: }\begin{array}{c} \CastRule[*] \end{array}$
    \begin{proofcase}
        Immediate by the IH applied to $\D{1}$.
    \end{proofcase}

    $\text{Case: }\begin{array}{c} \SeparationRule[*] \end{array}$
    \begin{proofcase}
        Have $\Gamma \vdash (X : \star) \to_\omega X : \star$ via short sequence of rules.
    \end{proofcase}
\end{proof}

Preservation is the second important property of the system.
It states that the status of a term (i.e. both its classification and status as a well-founded proof) do not change with respect to reduction.
Note that $\Gamma \betared \Gamma^\prime$ if there exists exactly one $(x_m : A) \in \Gamma$ such that $A \betared A^\prime$.

\begin{lemma}
    \textcolor{white}{\_}
    \begin{enumerate}
        \item If $\Gamma \vdash t : A$ and $t \betared t^\prime$ then $\Gamma \vdash t^\prime : A$
        \item If $\Gamma \vdash t : A$ and $\Gamma \betared \Gamma^\prime$ then $\Gamma^\prime \vdash t : A$
        \item If $\vdash \Gamma$ and $\Gamma \betared \Gamma^\prime$ then $\vdash \Gamma^\prime$
    \end{enumerate}
    \label{lem:2:preservation_no_type_step}
\end{lemma}
\begin{proof}
    By mutual recursion.

    \noindent \textbf{1.} Pattern-matching on $\Gamma \vdash t : A$.
    The \textsc{Ax} and \textsc{Var} cases are impossible by inversion on $t \betared t^\prime$.
    \textsc{Int} is very similar to \textsc{Pi}, \textsc{Fst} is very similar to \textsc{Snd}, and \textsc{PrmFst} is very similar to \textsc{PrmSnd}.
    The \text{Refl}, \textsc{Sep}, and \textsc{Conv} rules are trivial.

    $\text{Case: }\begin{array}{c} \PiRule[*] \end{array}$
    \begin{proofcase}
        Suppose $A \betared A^\prime$.
        Applying \textit{1} to $\D{1}$ gives $\Gamma \vdash A^\prime : \pdom(m, K)$.
        Note that $\Gamma, x_m : A \betared \Gamma, x_m : A^\prime$.
        Thus, using \textit{2} with $\D{2}$ gives $\Gamma, x_m : A^\prime \vdash B : \pcodom(m)$.
        Using the \textsc{Pi} rule concludes the case.
        \\ \\
        Suppose $B \betared B^\prime$.
        Applying \textit{1} to $\D{2}$ gives $\Gamma, x_m : A \vdash B^\prime : \pcodom(m)$.
        The \textsc{Pi} rule concludes the case.
    \end{proofcase}

    $\text{Case: }\begin{array}{c} \LambdaRule[*] \end{array}$
    \begin{proofcase}
        Suppose $A \betared A^\prime$.
        Then $(x : A) \to_m B \betared (x : A^\prime) \to_m B$.
        Now, using \textit{1} with $\D{1}$ gives $\Gamma \vdash (x : A^\prime) \to_m B : \pcodom(m)$.
        Note that $\Gamma, x_m : A \betared \Gamma, x_m : A^\prime$.
        Using \textit{2} with $\D{2}$ yields $\Gamma, x_m : A^\prime \vdash t : B$.
        Applying the \textsc{Lam} rule concludes the case.
        \\ \\
        Suppose $t \betared t^\prime$.
        Using \textit{1} with $\D{2}$ gives $\Gamma, x_m : A \vdash t^\prime : B$.
        Note that reduction does not introduce free variables, thus $x \notin FV(|t^\prime|)$ if $m = 0$.
        The \textsc{Lam} rule concludes.
    \end{proofcase}

    $\text{Case: }\begin{array}{c} \AppRule[*] \end{array}$
    \begin{proofcase}
        Suppose $f \betared f^\prime$.
        Applying \textit{1} with $\D{1}$ gives $\Gamma \vdash f^\prime : (x : A) \to_m B$.
        The \textsc{App} rule concludes.
        \\ \\
        Suppose $a \betared a^\prime$.
        Using \textit{1} with $\D{2}$ gives $\Gamma \vdash a^\prime : A$.
        Again, the \textsc{App} rule concludes the case.
        \\ \\
        Suppose $f = \abs{\lambda_m}{x}{C}{t}$ and $\app{f}{m}{a} \betared [x := a]t$.
        There must exist $C$ and $D$ such that $\Gamma \vdash C : \pdom(m, K)$ and $\Gamma, x_m : C \vdash t : D$ with $A \equiv C$ and $B \equiv D$.
        By classification (Lemma~\ref{lem:2:classification}) and the \textsc{Conv} rule, $\Gamma \vdash a : C$.
        Now using the substitution lemma (Lemma~\ref{lem:2:substitution}) $\Gamma \vdash [x := a]t : [x := a]D$.
        Using Lemma~\ref{lem:2:conv_subst} gives $[x := a]B \equiv [x := a]D$.
        Classification and \textsc{Conv} again yields $\Gamma \vdash [x := a]t : [x := a]B$.
    \end{proofcase}

    $\text{Case: }\begin{array}{c} \PairRule[*] \end{array}$
    \begin{proofcase}
        Suppose $t \betared t^\prime$.
        Applying \textit{1} to $\D{2}$ gives $\Gamma \vdash t^\prime : A$.
        Note that $[x := t]B \equiv [x := t^\prime]B$ by Lemma~\ref{lem:2:conv_subst}.
        Moreover, deconstructing $\D{1}$ yields $\Gamma, x_\tau : A \vdash B : \star$.
        By the substitution lemma $\Gamma \vdash [x := t^\prime]B : \star$.
        Thus, by the \textsc{Conv} rule $\Gamma \vdash s : [x := t^\prime]B$.
        Finally, Lemma~\ref{lem:2:conv_red_f} gives $t^\prime \equiv s$ from $\D{4}$.
        The \textsc{Pair} rule concludes the case.
        \\ \\
        Suppose $s \betared s^\prime$.
        By \textit{1} applied to $\D{3}$: $\Gamma \vdash s^\prime : [x := t]B$.
        Using Lemma~\ref{lem:2:conv_subst} with $\D{4}$ yields $t \equiv s^\prime$.
        The \textsc{Pair} rule concludes.
        \\ \\
        Suppose $A \betared A^\prime$.
        Then $(x : A) \cap B \betared (x : A^\prime) \cap B$.
        Applying this reduction to \textit{1} with $\D{1}$ gives $\Gamma \vdash (x : A^\prime) \cap B : \star$.
        Deconstructing this yields $\Gamma \vdash A^\prime : \star$.
        Now by the \textsc{Conv} rule $\Gamma \vdash t : A^\prime$.
        Using the \textsc{Pair} rule concludes.
        \\ \\
        Suppose $B \betared B^\prime$.
        Then $(x : A) \cap B \betared (x : A^\prime) \cap B$.
        Applying this reduction to \textit{1} with $\D{1}$ gives $\Gamma \vdash (x : A) \cap B^\prime : \star$.
        Deconstructing this yields $\Gamma, x_m : A^\prime \vdash B^\prime : \star$.
        Note that $B \betared B^\prime$ implies that $B \equiv B^\prime$.
        Moreover, using Lemma~\ref{lem:2:conv_subst} gives $[x := t]B \equiv [x := t]B^\prime$.
        The substitution lemma gives $\Gamma \vdash [x := t]B^\prime : \star$.
        Now the \textsc{Conv} rule yields $\Gamma \vdash s [x := t]B^\prime$.
        The \textsc{Pair} rule concludes the case.
    \end{proofcase}

    $\text{Case: }\begin{array}{c} \SecondRule[*] \end{array}$
    \begin{proofcase}
        Suppose $t \betared t^\prime$.
        Then applying \textit{1} to $\D{1}$ gives $\Gamma \vdash t^\prime : (x : A) \cap B$.
        Applying the \textsc{Snd} rule concludes the case.
        \\ \\
        Suppose $t = [t_1, t_2, t_3]$ and $t.2 \betared t_2$.
        Then we have $\Gamma \vdash [t_1, t_2, t_3] : (x : A) \cap B$.
        Deconstructing this rule yields $\Gamma \vdash t_1 : A$, $\Gamma, x_\tau : A \vdash B : \star$, and $\Gamma \vdash t_2 : [x := t_1]B$.
        By the substitution lemma $\Gamma \vdash [x := t.1]B : \star$.
        Note that $t.1 \betared t_1$ thus $t.1 \equiv t_1$.
        Now using Lemma~\ref{lem:2:conv_subst} gives $[x := t.1]B \equiv [x := t_1]B$.
        Thus, by the \textsc{Conv} rule $\Gamma \vdash t_2 : [x := t.1]B$.
    \end{proofcase}

    $\text{Case: }\begin{array}{c} \EqualityRule[*] \end{array}$
    \begin{proofcase}
        Suppose $a \betared a^\prime$.
        Applying \textit{1} to $\D{2}$ gives $\Gamma \vdash a^\prime : A$.
        The \textsc{Eq} rule concludes.
        \\ \\
        Suppose $b \betared b^\prime$.
        Applying \textit{1} to $\D{3}$ gives $\Gamma \vdash b^\prime : A$.
        The \textsc{Eq} rule concludes.
        \\ \\
        Suppose $A \betared A^\prime$.
        Applying \textit{1} to $\D{1}$ gives $\Gamma \vdash A^\prime : \star$.
        Note that $A \equiv A^\prime$.
        Thus, by the \textsc{Conv} rule applied twice: $\Gamma \vdash a : A^\prime$ and $\Gamma \vdash b : A^\prime$.
        Using the \textsc{Eq} rule concludes the case.
    \end{proofcase}

    $\text{Case: }\begin{array}{c} \SubstRule[*] \end{array}$
    \begin{proofcase}
        Suppose $e \betared e^\prime$.
        Then by \textit{1} applied to $\D{1}$: $\Gamma \vdash e^\prime : a =_A b$.
        The \textsc{Subst} rule concludes the case.
        \\ \\
        Suppose $P \betared P^\prime$.
        By \textit{1} applied to $\D{2}$: $\Gamma \vdash P : (y : A) \to_\tau (p : a =_A y) \to\tau \star$.
        The \textsc{Subst} rule concludes the case.
        \\ \\
        Suppose $e = \text{refl}(u)$ and $\psi(e, P) \betared \abs{\lambda_\omega}{x}{\apptwo{P}{\tau}{u}{\tau}{\text{refl}(u)}}{x}$.
        Now $\Gamma \vdash \text{refl}(u) : a =_A b$ which forces $u \equiv a$ and $u \equiv b$.
        Thus, $\apptwo{P}{\tau}{u}{\tau}{\text{refl}(u)} \equiv \apptwo{P}{\tau}{a}{\tau}{\text{refl}(a)}$ and $\apptwo{P}{\tau}{u}{\tau}{\text{refl}(u)} \equiv \apptwo{P}{\tau}{b}{\tau}{e}$.
        Which gives $\apptwo{P}{\tau}{u}{\tau}{\text{refl}(u)} \to_\omega \apptwo{P}{\tau}{u}{\tau}{\text{refl}(u)} \equiv \apptwo{P}{\tau}{a}{\tau}{\text{refl}(a)} \to_\omega \apptwo{P}{\tau}{b}{\tau}{e}$.
        Note that $\Gamma \vdash \apptwo{P}{\tau}{a}{\tau}{\text{refl}(a)} \to_\omega \apptwo{P}{\tau}{b}{\tau}{e} : K$ by classification.
        Therefore, using the \textsc{Conv} rule gives $\Gamma \vdash \abs{\lambda_\omega}{x}{\apptwo{P}{\tau}{u}{\tau}{\text{refl}(u)}}{x} : \apptwo{P}{\tau}{a}{\tau}{\text{refl}(a)} \to_\omega \apptwo{P}{\tau}{b}{\tau}{e}$.
    \end{proofcase}

    $\text{Case: }\begin{array}{c} \PromoteSndRule[*] \end{array}$
    \begin{proofcase}
        Suppose $e \betared e^\prime$.
        Applying \textit{1} to $\D{3}$ gives $\Gamma \vdash e^\prime : a.2 =_{[x := a.1]B} b.2$.
        Using the \textsc{PrmSnd} rule concludes the case.
        \\ \\
        Suppose $a \betared a^\prime$.
        Using \textit{1} with $\D{1}$ gives $\Gamma \vdash a^\prime : (x : A) \cap B$.
        Deconstructing $\D{3}$ gives $\Gamma \vdash b.2 : [x := a.1]B$.
        Note by Lemma~\ref{lem:2:conv_subst} that $[x := a.1]B \equiv [x := a^{\prime}.1]B$.
        Classification used with $\D{1}$ and deconstructing yields $\Gamma, x_\tau : A \vdash B : \star$.
        Using \textsc{Fst} gives $\Gamma \vdash a^{\prime}.1 : A$.
        Now by the substitution lemma $\Gamma \vdash [x := a^{\prime}.1]B : \star$.
        Finally, the \textsc{Conv} rule yields $\Gamma \vdash b.2 : [x := a^{\prime}.1]B$.
        By the \textsc{Snd} rule: $\Gamma \vdash a^{\prime}.2 : [x := a^{\prime}.1]B$.
        Piecing it all together with the \textsc{Eq} rule gives $\Gamma \vdash e : a^{\prime}.2 =_{[x := a^{\prime}.1]B} b.2$.
        Finally, by the \textsc{PrmSnd} rule $\Gamma \vdash \vartheta_2(e, a^\prime, b) : a^\prime =_{(x : A) \cap B} b$.
        \\ \\
        Suppose $b \betared b^\prime$.
        Applying \textit{1} to $\D{2}$ gives $\Gamma \vdash b^\prime : (x : A) \cap B$.
        Deconstructing $\D{3}$ gives $\Gamma \vdash b.2 : [x := a.1]B$.
        Applying \textit{1} again with this derivation yields $\Gamma \vdash b^{\prime}.2 : [x := a.1]B$.
        Recombining using the \textsc{Eq} rule gives $\Gamma \vdash e : a.2 =_{[x := a.1]B} b^{\prime}.2$.
        Thus, by the \textsc{prmSnd} rule $\Gamma \vdash \vartheta_2(e, a, b^\prime) : a =_{(x : A) \cap B} b^\prime$.
        \\ \\
        Suppose $e = \text{refl}(u)$ and $\vartheta_2(e, a, b) \betared \text{refl(a)}$.
        Now it must be the case that $u \equiv a.2$ and $u \equiv b.2$.
        Thus, $a \equiv b$.
        By classification applied to $\D{1}$: $\Gamma \vdash (x : A) \cap B : \star$.
        Using the \textsc{Eq} rule gives $\Gamma \vdash a =_{(x : A) \cap B} b : \star$.
        Note that $\Gamma \vdash \text{refl}(a) : a =_{(x : A) \cap B} a$, but $a =_{(x : A) \cap B} a \equiv a =_{(x : A) \cap B} b$.
        Thus, by the \textsc{Conv} rule $\Gamma \vdash \text{refl}(a) : a =_{(x : A) \cap B} b$.
    \end{proofcase}

    $\text{Case: }\begin{array}{c} \CastRule[*] \end{array}$
    \begin{proofcase}
        Suppose $f \betared f^\prime$.
        Applying \textit{1} to $\D{1}$ and using the \textsc{Cast} rule concludes this case.
        \\ \\
        Suppose $e \betared e^\prime$.
        Note that reduction does not introduce free variables, thus $FV(|e^\prime|)$ remains empty.
        Now, using \textit{1} with $\D{2}$ and using the \textsc{Cast} rule concludes.
    \end{proofcase}

    \noindent \textbf{2.} Pattern-matching on $\Gamma \vdash t : A$.
    Note that except \textsc{Ax} and \textsc{Var} all the other cases are immediate by applying \textit{2} to all sub-derivations and using the associated rule.

    $\text{Case: }\begin{array}{c} \AxiomRule[*] \end{array}$
    \begin{proofcase}
        Immediate by the \textsc{Ax} rule, the context does not matter.
    \end{proofcase}

    $\text{Case: }\begin{array}{c} \VarRule[*] \end{array}$
    \begin{proofcase}
        Partition $\Gamma$ in the following way: $\Delta_1, x_m : A, \Delta_2$.
        Note that $\vdash \Gamma$ is assumed, thus $\Delta_1 \vdash A : K$.
        Suppose $\Delta_1 \betared \Delta_1^\prime$.
        By \textit{2}: $\Delta_1^\prime \vdash A : K$ and $(x_m : A) \in \Gamma^\prime$.
        The \textsc{Var} rule concludes.
        \\ \\
        Suppose $A \betared A^\prime$.
        Then by \textit{2} it is the case that $\Delta_1 \vdash A^\prime : K$.
        Thus, $(x_m : A^\prime) : \Gamma^\prime$.
        By the \textsc{Var} rule $\Gamma^\prime \vdash x : A^\prime$.
        However, note that $A \equiv A^\prime$, thus by the \textsc{Conv} rule $\Gamma^\prime \vdash x : A$.
        \\ \\
        Suppose $\Delta_2 \betared \Delta_2^\prime$.
        Then immediately $(x_m : A) \in \Gamma^\prime$ and by the \textsc{Var} rule $\Gamma^\prime \vdash x : A$.
    \end{proofcase}

    \noindent \textbf{3.} Pattern-matching on $\Gamma$.
    If $\Gamma$ is empty then $\varepsilon \betared \Gamma^\prime$ forces $\Gamma^\prime = \varepsilon$ and $\vdash \varepsilon$.
    Thus, let $\Gamma = \Delta, x_m : A$.
    \\ \\
    Suppose $\Delta, x_m : A \betared \Delta^\prime, x_m : A$
    Then by \textit{3} applied to $\Delta$: $\vdash \Delta^\prime$.
    Now, because $\vdash \Delta, x_m : A$ it is the case that $\Delta \vdash A : K$.
    Using \textit{2} gives $\Delta^\prime \vdash A : K$.
    Therefore, $\vdash \Delta^\prime, x_m : A$.
    \\ \\
    Suppose $\Delta, x_m : A \betared \Delta, x_m : A^\prime$.
    Again $\vdash \Delta, x_m : A$ gives $\Delta \vdash A : K$.
    Using \textit{1} gives $\Delta \vdash A^\prime : K$.
    Thus, $\vdash \Delta, x_m : A^\prime$.
\end{proof}

\begin{lemma}
    \textcolor{white}{\_}
    \begin{enumerate}
        \item If $\Gamma \vdash t : A$ and $t \betastar t^\prime$ then $\Gamma \vdash t^\prime : A$
        \item If $\Gamma \vdash t : A$ and $\Gamma \betastar \Gamma^\prime$ then $\Gamma^\prime \vdash t : A$
        \item If $\vdash \Gamma$ and $\Gamma \betastar \Gamma^\prime$ then $\vdash \Gamma^\prime$
    \end{enumerate}
    \label{lem:2:preservation_no_type}
\end{lemma}
\begin{proof}
    For each property the proof proceeds by induction on multistep reduction using Lemma~\ref{lem:2:preservation_no_type_step} and the IH in the inductive case.
\end{proof}

\begin{lemma}
    If $\Gamma \vdash t : A$ and $A \betastar A^\prime$ then $\Gamma \vdash t : A^\prime$
    \label{lem:2:preservation_type}
\end{lemma}
\begin{proof}
    By classification: $\Gamma \vdash A : K$.
    Using Lemma~\ref{lem:2:preservation_no_type} gives $\Gamma \vdash A^\prime : K$.
    Note that $A \equiv A^\prime$.
    Thus, by the \textsc{Conv} rule $\Gamma \vdash t : A^\prime$.
\end{proof}

\begin{theorem}[Preservation]
    If $\Gamma \vdash t : A$, $\Gamma \betastar \Gamma^\prime$, $t \betastar t^\prime$, and $A \betastar A^\prime$ then $\Gamma^\prime \vdash t^\prime : A^\prime$
    \label{lem:2:preservation}
\end{theorem}
\begin{proof}
    Consequence of Lemma~\ref{lem:2:preservation_no_type} and Lemma~\ref{lem:2:preservation_type}.
\end{proof}