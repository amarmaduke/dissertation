

\begin{figure}
    \centering
    \begin{minipage}{0.5\textwidth}
        \begin{align*}
            \ptrunc{\!\!\pterm} &= x_\star \\
            \ptrunc{\!\!\ptype} &= x_\square
        \end{align*}
    \end{minipage}%
    \begin{minipage}{0.5\textwidth}
        \begin{align*}
            \ptrunc{\!\!\pkind} &= \star \\
            \ptrunc{\text{undefined}} &= \diamond
        \end{align*}
    \end{minipage}
    \caption{Domain and codomains for function types. The variable $K$ is either $\star$ or $\kind$.}
\end{figure}

\begin{figure}
    \centering
    \begin{align*}
        \mathcal{C}(x_\square) &=\!\! \ptype \\
        \mathcal{C}(x_\star) &=\!\! \pterm \\
        \mathcal{C}(\star) &=\!\! \pkind \\
        \mathcal{C}(\diamond) &=\!\! \ptype \\
        \mathcal{C}(\abs{\lambda_\tau}{x}{A}{t}) &=\!\! \ptype &\text{if }& (A\pkind \text{ or } A\ptype) \text{ and } t\ptype \\
        \mathcal{C}(\abs{\lambda_0}{x}{A}{t}) &=\!\! \pterm &\text{if }& (A\pkind \text{ or } A\ptype) \text{ and } t\pterm \\
        \mathcal{C}(\abs{\lambda_\omega}{x}{A}{t}) &=\!\! \pterm &\text{if }& A\ptype \text{ and } t\pterm \\
        \mathcal{C}((x : A) \to_\tau B) &=\!\! \pkind &\text{if }& (A\pkind \text{ or } A\ptype) \text{ and } B\pkind \\
        \mathcal{C}((x : A) \to_0 B) &=\!\! \ptype &\text{if }& (A\pkind \text{ or } A\ptype) \text{ and } B\ptype \\
        \mathcal{C}((x : A) \to_\omega B) &=\!\! \ptype &\text{if }& A\ptype \text{ and } B\ptype \\
        \mathcal{C}(\app{(\abs{\lambda_\tau}{x}{A}{t})}{\tau}{a}) &=\!\! \ptype &\text{if }& (A\pkind \text{ and } a\ptype) \text{ or } (A\ptype \text{ and } a\pterm) \\
            &&&\text{and } t\ptype \text{ and } [x := \ptrunc{\mathcal{C}(a)}]t\ptype \\
        \mathcal{C}(\app{f}{\tau}{a}) &=\!\! \ptype &\text{if }& (a\ptype \text{ or } a\pterm) \text{ and } f\ptype \\
        \mathcal{C}(\app{(\abs{\lambda_0}{x}{A}{t})}{0}{a}) &=\!\! \pterm &\text{if }& (A\pkind \text{ and } a\ptype) \text{ or } (A\ptype \text{ and } a\pterm) \\
            &&&\text{and } t\pterm \text{ and } [x := \ptrunc{\mathcal{C}(a)}]t\pterm \\
        \mathcal{C}(\app{f}{0}{a}) &=\!\! \pterm &\text{if }& (a\ptype \text{ or } a\pterm) \text{ and } f\pterm \\
        \mathcal{C}(\app{(\abs{\lambda_\omega}{x}{A}{t})}{\omega}{a}) &=\!\! \pterm &\text{if }& A\ptype \text{ and } a, t\pterm \text{ and } [x := \ptrunc{\mathcal{C}(a)}]t\pterm \\
        \mathcal{C}(\app{f}{\omega}{a}) &=\!\! \pterm &\text{if }& a\pterm \text{ and } f\pterm
    \end{align*}
    \caption{Domain and codomains for function types. The variable $K$ is either $\star$ or $\kind$.}
\end{figure}

\begin{figure}
    \centering
    \begin{align*}
        \mathcal{C}((x : A) \cap B) &=\!\! \ptype &\text{if }& A\ptype \text{ and } B\ptype \\
        \mathcal{C}([t_1, t_2; A]) &=\!\! \pterm &\text{if }& t_1, t_2\pterm \text{ and } A\ptype \\
        \mathcal{C}(t.1) &=\!\! \pterm &\text{if }& t\pterm \\
        \mathcal{C}(t.2) &=\!\! \pterm &\text{if }& t\pterm \\
        \mathcal{C}(a =_A b) &=\!\! \ptype &\text{if }& a, b\pterm \text{ and } A\ptype \\
        \mathcal{C}(\prefl(t; A)) &=\!\! \pterm &\text{if }& t\pterm \text{ and } A\ptype \\
        \mathcal{C}(\vartheta_i(e, a, b; T)) &=\!\! \pterm &\text{if }& e, a, b\pterm \text{ and } T\ptype \\
        \mathcal{C}(\psi(e, a, b; A, P)) &=\!\! \pterm &\text{if }& e, a, b\pterm \text{ and } A, P\ptype \\
        \mathcal{C}(\varphi(f, e; A, T)) &=\!\! \pterm &\text{if }& f, e\pterm \text{ and } A, T\ptype \\
        \mathcal{C}(\delta(e)) &=\!\! \pterm &\text{if }& e\pterm \\
        \mathcal{C}(t) &= \text{undefined} &&\text{otherwise}
    \end{align*}
    \caption{Domain and codomains for function types. The variable $K$ is either $\star$ or $\kind$.}
\end{figure}



\begin{theorem}[Soundness of $\mathcal{C}(-)$]
    \textcolor{white}{\_}
    \begin{enumerate}
        \item If $\Gamma \vdash t : A$ and $A = \kind$ then $\mathcal{C}(t, \Gamma) = \pkind$
        \item If $\Gamma \vdash t : A$ and $\Gamma \vdash A : \kind$ then $\mathcal{C}(t, \Gamma) = \ptype$
        \item If $\Gamma \vdash t : A$ and $\Gamma \vdash A : \star$ then $\mathcal{C}(t, \Gamma) = \pterm$
    \end{enumerate}
\end{theorem}
\begin{proof}
    By induction on $\Gamma \vdash t : A$.
    The \textsc{Eq} case is very similar to the \text{Int} case.
    The \textsc{PrmFst} and \text{PrmFst} cases are very similar to \textsc{Refl}.
    \textsc{Fst} rule is very similar to the \textsc{Snd} rule.
    Likewise, \textsc{Cast} is very similar to \textsc{Subst}.

    $\text{Case: }\begin{array}{c} \AxiomRule[*] \end{array}$
    \begin{proofcase}
        Have $A = \kind$ and $\mathcal{C}(\star, \Gamma) = \pkind$.
    \end{proofcase}

    $\text{Case: }\begin{array}{c} \VarRule[*] \end{array}$
    \begin{proofcase}
        Note that $\Gamma \vdash A : K$ because $\vdash \Gamma$ is assumed and $(x_m : A) \in \Gamma$.
        If $K = \kind$ then by the IH $\mathcal{C}(A, \Gamma) = \pkind$ and thus $\mathcal{C}(x, \Gamma) = \ptype$.
        If $K = \star$ then by the IH $\mathcal{C}(A, \Gamma) = \ptype$ and thus $\mathcal{C}(x, \Gamma) = \pterm$.
    \end{proofcase}

    $\text{Case: }\begin{array}{c} \PiRule[*] \end{array}$
    \begin{proofcase}
        Suppose $m = \tau$ then $\pcodom(m) = \kind$ and $\mathcal{C}((x : A) \to_\tau B, \Gamma) = \pkind$.
        Otherwise, $m = \omega$ or $m = 0$ then $\pcodom(m) = \star$ and $\mathcal{C}((x : A) \to_m B, \Gamma) = \ptype$.
    \end{proofcase}

    $\text{Case: }\begin{array}{c} \LambdaRule[*] \end{array}$
    \begin{proofcase}
        Suppose $m = \tau$ then $\pcodom(m) = \kind$, $\Gamma \vdash (x : A) \to_\tau B : \kind$ and $\mathcal{C}(\abs{\lambda_\tau}{x}{A}{t}, \Gamma) = \ptype$.
        Otherwise, $m = \omega$ or $m = 0$ then $\pcodom(m) = \star$, $\Gamma \vdash (x : A) \to_\tau B : \star$ and $\mathcal{C}(\abs{\lambda_\tau}{x}{A}{t}, \Gamma) = \pterm$.
    \end{proofcase}

    $\text{Case: }\begin{array}{c} \AppRule[*] \end{array}$
    \begin{proofcase}
        By classification applied to $\D{1}$: $\Gamma \vdash (x : A) \to_m B : K$.
        Suppose $K = \kind$ then $m = \tau$.
        Deconstruction and the substitution lemma gives $\Gamma \vdash [x := a]B : \kind$ and $\mathcal{C}(\app{f}{\tau}{a}, \Gamma) = \ptype$.
        Otherwise, $K = \star$ and then $m = \omega$ or $m = 0$.
        Again, deconstructing and substitution lemma yields $\Gamma \vdash [x := a]B : \star$ and $\mathcal{C}(\app{f}{m}{a}, \Gamma) = \pterm$.
    \end{proofcase}

    $\text{Case: }\begin{array}{c} \IntersectionRule[*] \end{array}$
    \begin{proofcase}
        Have $\Gamma \vdash \star : \kind$ and $\mathcal{C}((x : A) \cap B, \Gamma) = \ptype$.
    \end{proofcase}

    $\text{Case: }\begin{array}{c} \PairRule[*] \end{array}$
    \begin{proofcase}
        Have $\mathcal{C}([t, s; (x : A) \cap B], \Gamma) = \pterm$ and $\D{1}$ concludes.
    \end{proofcase}

    $\text{Case: }\begin{array}{c} \SecondRule[*] \end{array}$
    \begin{proofcase}
        By classification with $\D{1}$: $\Gamma \vdash (x : A) \cap B : \star$.
        Note the other possibilities are impossible by inversion.
        Deconstructing and using the substitution lemma gives $\Gamma \vdash [x := t.1]B : \star$.
        Finally, $\mathcal{C}(t.2, \Gamma) = \pterm$.
    \end{proofcase}

    $\text{Case: }\begin{array}{c} \ReflRule[*] \end{array}$
    \begin{proofcase}
        By classification an \text{eq} must be a type and by computation $\mathcal{C}(\text{refl}(t), \Gamma) = \pterm$.
    \end{proofcase}

    $\text{Case: }\begin{array}{c} \SubstRule[*] \end{array}$
    \begin{proofcase}
        By classification a $\Pi_\omega$ must be a type and by computation $\mathcal{C}(\psi(e, P), \Gamma) = \pterm$.
    \end{proofcase}

    $\text{Case: }\begin{array}{c} \SeparationRule[*] \end{array}$
    \begin{proofcase}
        By a short sequence of rules $\Gamma \vdash (X : \star) \to_0 X : \star$.
        Finally, $\mathcal{C}(\delta(e), \Gamma) = \pterm$.
    \end{proofcase}

    $\text{Case: }\begin{array}{c} \ConvRule[*] \end{array}$
    \begin{proofcase}
        Classification, $\D{1}$ and $\D{3}$ give $\Gamma \vdash B : K$.
        Applying the IH to $\D{2}$ concludes the case.
    \end{proofcase}
\end{proof}
