\section{Classification}


\begin{theorem}[Classification]
    If $\Gamma \vdash t \infr A$ then one (and only one) of the following statements holds:
    \begin{enumerate}
        \item $A$ is $\kind$ (i.e. $t$ is a kind)
        \item $\Gamma \vdash A \cinfr \kind$ (i.e. $t$ is a $\Gamma$-constructor)
        \item $\Gamma \vdash A \cinfr \star$ (i.e. $t$ is a $\Gamma$-term)
    \end{enumerate}
\end{theorem}
\begin{proof}
    By induction on the proof derivation of $t$ with motives
    \begin{itemize}
        \item {
            $\Gamma \vdash t \infr A$, $\Gamma \vdash t \cinfr A$, and $\Gamma \vdash t \chck A$ motives is
            $$(A = \kind) \vee (\Gamma \vdash A \cinfr \kind) \vee (\Gamma \vdash A \cinfr \star)$$
        }
        \item {
            $\vdash \Gamma$ motive is $\top$
        }
    \end{itemize}

    $\text{Case: }\begin{array}{c} \AxiomRule[*] \end{array}$
    \begin{proofcase}
        $A = \kind$, trivial.
    \end{proofcase}

    $\text{Case: }\begin{array}{c} \VarRule[*] \end{array}$
    \begin{proofcase}
        Obtained from Lemma~\ref{lem:2:ctx_get}.
    \end{proofcase}

    $\text{Case: }\begin{array}{c} \PiRule[*] \end{array}$
    \begin{proofcase}
        If $\pcodom(m) = \kind$ then trivial.
        Otherwise, $\pcodom(m) = \star$ and the \textsc{Axiom} rule with Lemma~\ref{lem:2:ctx_wf} applied to $\D{1}$ conclude the case.
    \end{proofcase}

    $\text{Case: }\begin{array}{c} \LambdaRule[*] \end{array}$
    \begin{proofcase}
        Apply the \textsc{Pi} rule with $\D{1}$ and $\D{3}$.
    \end{proofcase}

    $\text{Case: }\begin{array}{c} \AppRule[*] \end{array}$
    \begin{proofcase}
        By the IH applied to $\D{1}$ it is the case that $\exists\ K$ such that $\Gamma \vdash (x : A) \to_m B \cinfr K$.
        Note that it cannot be the case that it is equal $\kind$ by inversion on syntax.
        However, by inversion on the internal inference rule $\Gamma \vdash (x : A) \to_m B \infr \pcodom(m)$.
        By the \textsc{Pi} rule $\Gamma, x : A \vdash B \cinfr \pcodom(m)$.
        Applying Lemma~\ref{lem:2:subst_check} yields $\Gamma \vdash [x := a]B \chck \pcodom(m)$.
        Thus, by Lemma~\ref{lem:2:check_k_to_cinfer_k}: $\Gamma \vdash [x := a]B \cinfr \pcodom(m)$.
    \end{proofcase}

    $\text{Case: }\begin{array}{c} \IntersectionRule[*] \end{array}$
    \begin{proofcase}
        Immediate by the \textsc{Axiom} rule and Lemma~\ref{lem:2:ctx_get}.
    \end{proofcase}

    $\text{Case: }\begin{array}{c} \PairRule[*] \end{array}$
    \begin{proofcase}
        Apply the \textsc{Int} rule with $\D{2}$ and $\D{3}$.
    \end{proofcase}

    $\text{Case: }\begin{array}{c} \FirstRule[*] \end{array}$
    \begin{proofcase}
        Apply the IH to $\D{1}$.
        Inversion on syntax gives $\exists\ K$ such that $(x : A) \cap B \cinfr K$.
        Inversion on the internal inference rule yields $(x : A) \cap B \infr \star$.
        Thus, $K = \star$.
        Deconstructing the previous rule concludes the proof with $\Gamma \vdash A \cinfr \star$.
    \end{proofcase}

    $\text{Case: }\begin{array}{c} \SecondRule[*] \end{array}$
    \begin{proofcase}
        Exactly as the previous case $(x : A) \cap B \infr \star$ by the IH and two inversions.
        Deconstructing the previous rule gives $\Gamma, x : A \vdash B \cinfr \star$.
        By the \textsc{Fst} rule, $\Gamma \vdash t.1 \infr A$.
        Then, by Lemma~\ref{lem:2:subst_infer} it is the case that $\Gamma \vdash [x := t.1]B \cinfr \star$.
    \end{proofcase}

    $\text{Case: }\begin{array}{c} \EqualityRule[*] \end{array}$
    \begin{proofcase}
        Immediate by the \textsc{Axiom} rule and Lemma~\ref{lem:2:ctx_wf}.
    \end{proofcase}

    $\text{Case: }\begin{array}{c} \ReflRule[*] \end{array}$
    \begin{proofcase}
        Note that $\Gamma \vdash t \chck A$, thus by the \textsc{Eq} rule $\Gamma \vdash t =_A t \infr \star$.
    \end{proofcase}

    $\text{Case: }\begin{array}{c} \SubstRule[*] \end{array}$
    \begin{proofcase}
        Apply the IH to $\D{1}$.
        By inversion on syntax $\exists\ K$ such that $\Gamma \vdash a =_A b \cinfr K$.
        By inversion on the internal inference rule, $\Gamma \vdash a =_A b \infr \star$, thus $K = \star$.
        Deconstructing the previous rule yields $\Gamma \vdash A \cinfr \star$, $\Gamma \vdash a \chck A$, and $\Gamma \vdash b \chck A$.
        By Lemma~\ref{lem:2:check_app} $\Gamma \vdash \app{P}{\tau}{a} \chck \star$ and $\Gamma \vdash \app{P}{\tau}{b} \chck \star$.
        However, this means that $\Gamma \vdash \app{P}{\tau}{a} \cinfr \star$ and $\Gamma \vdash \app{P}{\tau}{b} \cinfr \star$ by Lemma~\ref{lem:2:check_k_to_cinfer_k}.
        Using weakening, $\Gamma, x : \app{P}{\tau}{a} \vdash \app{P}{\tau}{b} \cinfr \star$.
        Applying the \textsc{Pi} rule concludes the proof.
    \end{proofcase}

    $\text{Case: }\begin{array}{c} \PromoteFstRule[*] \end{array}$
    \begin{proofcase}
        Exactly as the \textsc{Fst} and \textsc{Snd} cases, $\Gamma \vdash (x : A) \cap B \cinfr \star$ by the IH and two inversions.
        By $\D{2}$, $\Gamma \vdash a \chck (x : A) \cap B$.
        Applying the \textsc{Eq} rule yields $\Gamma \vdash a =_{(x : A) \cap B} b \infr \star$.
    \end{proofcase}

    $\text{Case: }\begin{array}{c} \CastRule[*] \end{array}$
    \begin{proofcase}
        By the IH on $\D{1}$.
    \end{proofcase}

    $\text{Case: }\begin{array}{c} \SeparationRule[*] \end{array}$
    \begin{proofcase}
        It is clear that $\Gamma \vdash (X : \star) \to_0 X \infr \star$ by applying a short sequence of inference rules.
    \end{proofcase}

    $\text{Case: }\begin{array}{c} \HeadInferenceRule[*] \end{array}$
    \begin{proofcase}
        Apply the IH on $\D{1}$.
        By inversion on $\D{2}$, $A \neq \kind$.
        Thus, $\exists\ K$ such that $\Gamma \vdash A \cinfr K$.
        By preservation $\Gamma \vdash B \cinfr K$.
    \end{proofcase}

    $\text{Case: }\begin{array}{c} \CheckRule[*] \end{array}$
    \begin{proofcase}
        Immediate by $\D{2}$.
    \end{proofcase}

    $\text{Case: }\begin{array}{c} \ContextEmptyRule[*] \end{array}$
    \begin{proofcase}
        Trivial.
    \end{proofcase}

    $\text{Case: }\begin{array}{c} \ContextAppendRule[*] \end{array}$
    \begin{proofcase}
        Trivial.
    \end{proofcase}
\end{proof}
