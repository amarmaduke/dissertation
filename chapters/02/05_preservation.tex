\section{Preservation}

Preservation states that the status of a term (i.e. both its classification and status as a well-founded proof) do not change with respect to reduction.
Note that Cedille only enjoys a semantic version of preservation and not a syntactic version presented below.
While this may not matter from the perspective of the semantics it does indicate that syntax is better behaved in $\ced$.
The proof follows by induction on the typing derivation and case analysis on the associated reduction.

\begin{definition}
    $\Gamma \betared \Gamma^\prime$ iff there exists a unique $(x_m : A) \in \Gamma$ such that $A \betared A^\prime$
\end{definition}

\begin{lemma}
    \label{lem:2:preservation_no_type_step}
    \textcolor{white}{\_}
    \begin{enumerate}
        \item If $\Gamma \vdash t : A$ and $t \betared t^\prime$ then $\Gamma \vdash t^\prime : A$
        \item If $\Gamma \vdash t : A$ and $\Gamma \betared \Gamma^\prime$ then $\Gamma^\prime \vdash t : A$
        \item If $\vdash \Gamma$ and $\Gamma \betared \Gamma^\prime$ then $\vdash \Gamma^\prime$
    \end{enumerate}
\end{lemma}
\begin{proof}
    By mutual recursion.

    \noindent \textbf{1.} Pattern-matching on $\Gamma \vdash t : A$.
    The \textsc{Ax} and \textsc{Var} cases are impossible by inversion on $t \betared t^\prime$.
    \textsc{Int} is very similar to \textsc{Pi} and \textsc{Fst} is very similar to \textsc{Snd}.
    The \text{Refl}, \textsc{Sep}, and \textsc{Conv} rules are trivial.

    $\text{Case: }\begin{array}{c} \PiRule[*] \end{array}$
    \begin{proofcase}
        Suppose $A \betared A^\prime$.
        Applying \textit{1} to $\D{1}$ gives $\Gamma \vdash A^\prime : \pdom(m, K)$.
        Note that $\Gamma, x_m : A \betared \Gamma, x_m : A^\prime$.
        Thus, using \textit{2} with $\D{2}$ gives $\Gamma, x_m : A^\prime \vdash B : \pcodom(m)$.
        Using the \textsc{Pi} rule concludes the case.
        \\ \\
        Suppose $B \betared B^\prime$.
        Applying \textit{1} to $\D{2}$ gives $\Gamma, x_m : A \vdash B^\prime : \pcodom(m)$.
        The \textsc{Pi} rule concludes the case.
    \end{proofcase}

    $\text{Case: }\begin{array}{c} \LambdaRule[*] \end{array}$
    \begin{proofcase}
        Suppose $A \betared A^\prime$.
        Then $(x : A) \to_m B \betared (x : A^\prime) \to_m B$.
        Now, using \textit{1} with $\D{1}$ gives $\Gamma \vdash (x : A^\prime) \to_m B : \pcodom(m)$.
        Note that $\Gamma, x_m : A \betared \Gamma, x_m : A^\prime$.
        Using \textit{2} with $\D{2}$ yields $\Gamma, x_m : A^\prime \vdash t : B$.
        Applying the \textsc{Lam} rule concludes the case.
        \\ \\
        Suppose $t \betared t^\prime$.
        Using \textit{1} with $\D{2}$ gives $\Gamma, x_m : A \vdash t^\prime : B$.
        Note that reduction does not introduce free variables, thus $x \notin FV(|t^\prime|)$ if $m = 0$.
        The \textsc{Lam} rule concludes.
    \end{proofcase}

    $\text{Case: }\begin{array}{c} \AppRule[*] \end{array}$
    \begin{proofcase}
        Suppose $f \betared f^\prime$.
        Applying \textit{1} with $\D{1}$ gives $\Gamma \vdash f^\prime : (x : A) \to_m B$.
        The \textsc{App} rule concludes.
        \\ \\
        Suppose $a \betared a^\prime$.
        Using \textit{1} with $\D{2}$ gives $\Gamma \vdash a^\prime : A$.
        Again, the \textsc{App} rule concludes the case.
        \\ \\
        Suppose $f = \abs{\lambda_m}{x}{C}{t}$ and $\app{f}{m}{a} \betared [x := a]t$.
        There must exist $C$ and $D$ such that $\Gamma \vdash C : \pdom(m, K)$ and $\Gamma, x_m : C \vdash t : D$ with $A \equiv C$ and $B \equiv D$.
        By Lemma~\ref{lem:2:classification} and the \textsc{Conv} rule, $\Gamma \vdash a : C$.
        Now using the substitution lemma (Lemma~\ref{lem:2:subst}) $\Gamma \vdash [x := a]t : [x := a]D$.
        Using Lemma~\ref{lem:2:conv_subst} gives $[x := a]B \equiv [x := a]D$.
        Lemma~\ref{lem:2:classification} and \textsc{Conv} again yields $\Gamma \vdash [x := a]t : [x := a]B$.
        \\ \\
        Suppose $f = \psi(\prefl(z;Z), u, v; U, P)$ with $m = \omega$ and $\app{f}{\omega}{a} \betared a$.
        By inversion on $\D{1}$: $A = \apptwo{P}{\tau}{u}{\tau}{\prefl(u;U)}$ and $[x := a]B = \apptwo{P}{\tau}{v}{\tau}{\prefl(z;Z)}$.
        However, inversion also yields $\Gamma \vdash \prefl(z;Z) : u =_U v$ thus $z \equiv u$, $z \equiv v$, and $Z \equiv U$.
        Thus, $\apptwo{P}{\tau}{u}{\tau}{\prefl(u;U)} \equiv \apptwo{P}{\tau}{v}{\tau}{\prefl(z;Z)}$.
        The \textsc{Conv} rule concludes the case.
    \end{proofcase}

    $\text{Case: }\begin{array}{c} \PairRule[*] \end{array}$
    \begin{proofcase}
        Suppose $t \betared t^\prime$.
        Applying \textit{1} to $\D{2}$ gives $\Gamma \vdash t^\prime : A$.
        Note that $[x := t]B \equiv [x := t^\prime]B$ by Lemma~\ref{lem:2:conv_subst}.
        Moreover, deconstructing $\D{1}$ yields $\Gamma, x_\tau : A \vdash B : \star$.
        By the substitution lemma $\Gamma \vdash [x := t^\prime]B : \star$.
        Thus, by the \textsc{Conv} rule $\Gamma \vdash s : [x := t^\prime]B$.
        Finally, Lemma~\ref{lem:2:conv_red_f} gives $t^\prime \equiv s$ from $\D{4}$.
        The \textsc{Pair} rule concludes the case.
        \\ \\
        Suppose $s \betared s^\prime$.
        By \textit{1} applied to $\D{3}$: $\Gamma \vdash s^\prime : [x := t]B$.
        Using Lemma~\ref{lem:2:conv_subst} with $\D{4}$ yields $t \equiv s^\prime$.
        The \textsc{Pair} rule concludes.
        \\ \\
        Suppose $A \betared A^\prime$.
        Then $(x : A) \cap B \betared (x : A^\prime) \cap B$.
        Applying this reduction to \textit{1} with $\D{1}$ gives $\Gamma \vdash (x : A^\prime) \cap B : \star$.
        Deconstructing this yields $\Gamma \vdash A^\prime : \star$.
        Now by the \textsc{Conv} rule $\Gamma \vdash t : A^\prime$.
        Using the \textsc{Pair} rule concludes.
        \\ \\
        Suppose $B \betared B^\prime$.
        Then $(x : A) \cap B \betared (x : A^\prime) \cap B$.
        Applying this reduction to \textit{1} with $\D{1}$ gives $\Gamma \vdash (x : A) \cap B^\prime : \star$.
        Deconstructing this yields $\Gamma, x_m : A^\prime \vdash B^\prime : \star$.
        Note that $B \betared B^\prime$ implies that $B \equiv B^\prime$.
        Moreover, using Lemma~\ref{lem:2:conv_subst} gives $[x := t]B \equiv [x := t]B^\prime$.
        The substitution lemma gives $\Gamma \vdash [x := t]B^\prime : \star$.
        Now the \textsc{Conv} rule yields $\Gamma \vdash s [x := t]B^\prime$.
        The \textsc{Pair} rule concludes the case.
    \end{proofcase}

    $\text{Case: }\begin{array}{c} \SecondRule[*] \end{array}$
    \begin{proofcase}
        Suppose $t \betared t^\prime$.
        Then applying \textit{1} to $\D{1}$ gives $\Gamma \vdash t^\prime : (x : A) \cap B$.
        Applying the \textsc{Snd} rule concludes the case.
        \\ \\
        Suppose $t = [t_1, t_2, t_3]$ and $t.2 \betared t_2$.
        Then we have $\Gamma \vdash [t_1, t_2, t_3] : (x : A) \cap B$.
        Deconstructing this rule yields $\Gamma \vdash t_1 : A$, $\Gamma, x_\tau : A \vdash B : \star$, and $\Gamma \vdash t_2 : [x := t_1]B$.
        By the substitution lemma $\Gamma \vdash [x := t.1]B : \star$.
        Note that $t.1 \betared t_1$ thus $t.1 \equiv t_1$.
        Now using Lemma~\ref{lem:2:conv_subst} gives $[x := t.1]B \equiv [x := t_1]B$.
        Thus, by the \textsc{Conv} rule $\Gamma \vdash t_2 : [x := t.1]B$.
    \end{proofcase}

    $\text{Case: }\begin{array}{c} \EqualityRule[*] \end{array}$
    \begin{proofcase}
        Suppose $a \betared a^\prime$.
        Applying \textit{1} to $\D{2}$ gives $\Gamma \vdash a^\prime : A$.
        The \textsc{Eq} rule concludes.
        \\ \\
        Suppose $b \betared b^\prime$.
        Applying \textit{1} to $\D{3}$ gives $\Gamma \vdash b^\prime : A$.
        The \textsc{Eq} rule concludes.
        \\ \\
        Suppose $A \betared A^\prime$.
        Applying \textit{1} to $\D{1}$ gives $\Gamma \vdash A^\prime : \star$.
        Note that $A \equiv A^\prime$.
        Thus, by the \textsc{Conv} rule applied twice: $\Gamma \vdash a : A^\prime$ and $\Gamma \vdash b : A^\prime$.
        Using the \textsc{Eq} rule concludes the case.
    \end{proofcase}

    \begin{minipage}{.8\textwidth}$\text{Case: }\begin{array}{c} \SubstRule[*] \end{array}$\end{minipage}
    \begin{proofcase}
        Suppose $A \betared A^\prime$.
        Then $a =_A b \equiv a =_A^\prime b$ and $(y : A) \to_tau (p : a =_A y_\star) \to_\tau \star \equiv (y : A) \to_tau (p : a =_A y_\star) \to_\tau$.
        Thus, by the \textsc{Conv} rule: $\Gamma \vdash a : A^\prime$, $\Gamma \vdash b : A^\prime$, $\Gamma \vdash e : a =_{A^\prime} b$, and $\Gamma \vdash P : (y : A^\prime) \to_tau (p : a =_{A^\prime} y_\star) \to_\tau$.
        Applying \textit{1} to $\D{1}$ gives: $\Gamma \vdash A^\prime : \star$.
        The \textsc{Subst} rule concludes the case.
        \\ \\
        Suppose $a \betared a^\prime$.
        Then $a =_A b \equiv a^\prime _A b$ and $(y : A) \to_tau (p : a =_A y_\star) \to_\tau \star \equiv (y : A) \to_tau (p : a^\prime =_A y_\star) \to_\tau$.
        Thus, by the \textsc{Conv} rule: $\Gamma \vdash e : a^\prime =_A b$ and $\Gamma \vdash P : (y : A) \to_tau (p : a^\prime =_A y_\star) \to_\tau$.
        Applying \textit{1} to $\D{2}$ gives: $\Gamma \vdash a^\prime : A$.
        The \textsc{Subst} rule concludes the case.
        \\ \\
        Suppose $b \betared b^\prime$.
        Then $a =_A b \equiv a =_A b^\prime$ and by the \textsc{Conv} rule $\Gamma \vdash b^\prime : A$.
        Applying \textit{1} to $\D{3}$ gives: $\Gamma \vdash b^\prime : B$.
        The \textsc{Subst} rule concludes the case.
        \\ \\
        Suppose $e \betared e^\prime$.
        Then by \textit{1} applied to $\D{1}$: $\Gamma \vdash e^\prime : a =_A b$.
        The \textsc{Subst} rule concludes the case.
        \\ \\
        Suppose $P \betared P^\prime$.
        By \textit{1} applied to $\D{2}$: $\Gamma \vdash P : (y : A) \to_\tau (p : a =_A y) \to\tau \star$.
        The \textsc{Subst} rule concludes the case.
    \end{proofcase}

    $\text{Case: }\begin{array}{c} \PromoteRule[*] \end{array}$
    \begin{proofcase}
        Suppose $e \betared e^\prime$.
        Then by \textit{1} applied to $\D{4}$: $\Gamma \vdash e^\prime a.1 =_A b.1$ and the \textsc{Prm} rule concludes.
        \\ \\
        Suppose $a \betared a^\prime$.
        Then $a.1 =_A b.1 \equiv a^\prime.1 =_A b.1$ and the \textsc{Conv} rule yields $\Gamma \vdash e : a^\prime.1 =_A b.1$.
        Applying \textit{1} to $\D{2}$ gives $\Gamma \vdash a^\prime : (x : A) \cap B$.
        The \textsc{Prm} rule concludes.
        \\ \\
        Suppose $b \betared b^\prime$.
        Then $a.1 =_A b.1 \equiv a.1 =_A b^\prime.1$ and the \textsc{Conv} rule yields $\Gamma \vdash e : a.1 =_A b^\prime.1$.
        Applying \textit{1} to $\D{3}$ gives $\Gamma \vdash b^\prime : (x : A) \cap B$.
        The \textsc{Prm} rule concludes.
        \\ \\
        Suppose w.l.o.g. that $B \betared B^\prime$, the case when $A \betared A^\prime$ is similar.
        Then $(x : A) \cap B \equiv (x : A) \cap B^\prime$ and the \textsc{Conv} rule yields $\Gamma \vdash a : (x : A) \cap B^\prime$ and $\Gamma \vdash b : (x : A) \cap B^\prime$.
        Applying \textit{1} to $\D{1}$ yields $\Gamma \vdash (x : A) \cap B^\prime : \star$.
        The \textsc{Prm} rule concludes.
        \\ \\
        Suppose $e = \prefl(z;Z)$ and $\vartheta(e, a, b; (x:A) \cap B) \betared \prefl(a; (x : A) \cap B)$.
        By inversion $\Gamma \vdash \prefl(z;Z) : a.1 =_A b.1$, hence $z \equiv a.1$, $z \equiv b.1$.
        Thus, $a \equiv b$ and $\Gamma \vdash \prefl(a; (x : A) \cap B) : a =_{(x :A) \cap B} b$.
    \end{proofcase}

    $\text{Case: }\begin{array}{c} \CastRule[*] \end{array}$
    \begin{proofcase}
        Suppose $a \betared a^\prime$.
        Then $a =_A b.1 \equiv a^\prime =_A b.1$ and by \textsc{Conv} rule $\Gamma \vdash e : a^\prime =_A b.1$.
        Applying \textit{1} to $\D{1}$ yields $\Gamma \vdash a^\prime : A$.
        The \textsc{Cast} rule concludes.
        \\ \\
        Suppose $b \betared b^\prime$.
        Then $a =_A b.1 \equiv a =_A b^\prime.1$ and by \textsc{Conv} rule $\Gamma \vdash e : a =_A b^\prime.1$.
        Applying \textit{1} to $\D{2}$ yields $\Gamma \vdash b^\prime : (x : A) \cap B$.
        The \textsc{Cast} rule concludes.
        \\ \\
        Suppose $e \betared e^\prime$.
        Applying \textit{1} to $\D{3}$ yields $\Gamma \vdash e^\prime : a =_A b.1$
        The \textsc{Cast} rule concludes.
    \end{proofcase}

    \noindent \textbf{2.} Pattern-matching on $\Gamma \vdash t : A$.
    Note that except \textsc{Ax} and \textsc{Var} all the other cases are immediate by applying \textit{2} to all sub-derivations and using the associated rule.

    $\text{Case: }\begin{array}{c} \AxiomRule[*] \end{array}$
    \begin{proofcase}
        Immediate by the \textsc{Ax} rule, the context does not matter.
    \end{proofcase}

    $\text{Case: }\begin{array}{c} \VarRule[*] \end{array}$
    \begin{proofcase}
        Suppose $\Gamma_1 \betared \Gamma_1^\prime$.
        Reduction does not produce free variables, thus $x \notin FV(\Gamma^\prime_1;\Gamma_2)$.
        Applying \textit{1} to $\D{2}$ yields $\Gamma^\prime_1 \vdash A : K$.
        The \textsc{Var} rule concludes.
        \\ \\
        Suppose $\Gamma_2 \betared \Gamma_2^\prime$.
        As before $x \notin FV(\Gamma_1;\Gamma_2^\prime)$.
        The \textsc{Var} rule concludes.
        \\ \\
        Suppose $A \betared A^\prime$.
        Applying \textit{1} to $\D{2}$ gives $\Gamma_1 \vdash A^\prime : K$.
        The \textsc{Var} rule concludes.
    \end{proofcase}

    \noindent \textbf{3.} Pattern-matching on $\Gamma$.
    If $\Gamma$ is empty then $\varepsilon \betared \Gamma^\prime$ forces $\Gamma^\prime = \varepsilon$ and $\vdash \varepsilon$.
    Thus, let $\Gamma = \Delta; x_m : A$.
    \\ \\
    Suppose $\Delta; x_m : A \betared \Delta^\prime; x_m : A$.
    Then by \textit{3} applied to $\Delta$: $\vdash \Delta^\prime$.
    Now, because $\vdash \Delta; x_m : A$ it is the case that $\Delta \vdash A : K$.
    Using \textit{2} gives $\Delta^\prime \vdash A : K$.
    Therefore, $\vdash \Delta^\prime; x_m : A$.
    \\ \\
    Suppose $\Delta; x_m : A \betared \Delta; x_m : A^\prime$.
    Again $\vdash \Delta; x_m : A$ gives $\Delta \vdash A : K$.
    Using \textit{1} gives $\Delta \vdash A^\prime : K$.
    Thus, $\vdash \Delta; x_m : A^\prime$.
\end{proof}

\begin{lemma}
    \label{lem:2:preservation_no_type}
    \textcolor{white}{\_}
    \begin{enumerate}
        \item If $\Gamma \vdash t : A$ and $t \betastar t^\prime$ then $\Gamma \vdash t^\prime : A$
        \item If $\Gamma \vdash t : A$ and $\Gamma \betastar \Gamma^\prime$ then $\Gamma^\prime \vdash t : A$
        \item If $\vdash \Gamma$ and $\Gamma \betastar \Gamma^\prime$ then $\vdash \Gamma^\prime$
    \end{enumerate}
\end{lemma}
\begin{proof}
    For each property the proof proceeds by induction on multistep reduction using Lemma~\ref{lem:2:preservation_no_type_step} and the IH in the inductive case.
\end{proof}

\begin{lemma}
    \label{lem:2:preservation_type}
    If $\Gamma \vdash t : A$ and $A \betastar A^\prime$ then $\Gamma \vdash t : A^\prime$
\end{lemma}
\begin{proof}
    By classification: $\Gamma \vdash A : K$.
    Using Lemma~\ref{lem:2:preservation_no_type} gives $\Gamma \vdash A^\prime : K$.
    Note that $A \equiv A^\prime$.
    Thus, by the \textsc{Conv} rule $\Gamma \vdash t : A^\prime$.
\end{proof}

\begin{theorem}[Preservation]
    \label{lem:2:preservation}
    If $\Gamma \vdash t : A$, $\Gamma \betastar \Gamma^\prime$, $t \betastar t^\prime$, and $A \betastar A^\prime$ then $\Gamma^\prime \vdash t^\prime : A^\prime$
\end{theorem}
\begin{proof}
    Consequence of Lemma~\ref{lem:2:preservation_no_type} and Lemma~\ref{lem:2:preservation_type}.
\end{proof}
