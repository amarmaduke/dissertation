\chapter{Introduction to \texorpdfstring{System F$^\omega$}{System F Omega}}

Type Theory as a discipline is a difficult subject to thoroughly introduce because it in essence captures a wide variety of programming languages (if not all programming languages currently defined).
To trim the fat this thesis will focus on a particular type theory, System F$^\omega$, and the various bits of machinery that are required to describe it.
However, even with this focus there are different equivalent methods of presenting the theory: Pure Type Systems, Martin-L\"{o}f style presentations, Bidirectional systems, etc.
An extrinsic bidirectional presentation will be used with only summary remarks, if any, for the other styles.
This introduction is far from complete and is instead focused on providing the reader with enough background to understand the later chapters.

