\chapter{Proof Normalization}
\label{chap:3}

\Rule{\PiRuleFpOne}
    {
        \der{\D{1}}{\Gamma \vdash A : \kind}
        \\ \der{\D{2}}{\Gamma, x : A \vdash B : \kind}
    }
    {\Gamma \vdash (x : A) \to B : \kind}
    {Pi1}

\Rule{\PiRuleFpTwo}
    {
        \der{\D{1}}{\Gamma \vdash A : K}
        \\ \der{\D{2}}{\Gamma, x : A \vdash B : \star}
    }
    {\Gamma \vdash (x : A) \to B : \star}
    {Pi2}

\Rule{\LambdaRuleFp}
    {
        \der{\D{1}}{\Gamma \vdash (x:A) \to B : K}
        \\ \der{\D{2}}{\Gamma, x : A \vdash t : B}
    }
    {\Gamma \vdash \abs{\lambda}{x}{A}{t} : (x : A) \to B}
    {Lam}

\Rule{\AppRuleFp}
    {
        \der{\D{1}}{\Gamma \vdash f : (x : A) \to B} \\
        \der{\D{2}}{\Gamma \vdash a : A}
    }
    {\Gamma \vdash f\ a : [x := a]B}
    {App}

\Rule{\AxiomRuleFp}
    {\textcolor{white}{\_}}
    {\Gamma \vdash \star : \kind}
    {Axiom}

\Rule{\VarRuleFp}
    {
       \der{\D{1}}{(x : A) \in \Gamma}
    }
    {\Gamma \vdash x : A}
    {Var}

\Rule{\ConvRuleFp}
    {
        \der{\D{1}}{\Gamma \vdash A : K} \\
        \der{\D{2}}{\Gamma \vdash t : B} \\
        \der{\D{3}}{A \betaconv B}
    }
    {\Gamma \vdash t : A}
    {Conv}

\Rule{\ProductRuleFp}
    {
        \der{\D{1}}{\Gamma \vdash A : \star} \\
        \der{\D{2}}{\Gamma \vdash B : \star}
    }
    {\Gamma \vdash A \times B : \star}
    {Int}

\Rule{\PairRuleFp}
    {
        \der{\D{1}}{\Gamma \vdash A \times B : \star} \\
        \der{\D{2}}{\Gamma \vdash t : A} \\
        \der{\D{3}}{\Gamma \vdash s : B}
    }
    {\Gamma \vdash (t, s) : A \times B }
    {Pair}

\Rule{\FirstRuleFp}
    {\der{\D{1}}{\Gamma \vdash t : A \times B}}
    {\Gamma \vdash t.1 : A}
    {Fst}

\Rule{\SecondRuleFp}
    {\der{\D{1}}{\Gamma \vdash t : A \times B}}
    {\Gamma \vdash t.2 : B}
    {Snd}



There are several techniques for showing strong normalization of a PTS, including saturated sets \cite{geuvers1994_sn_satset}, model theory \cite{terlouw1995_sn}, realizability \cite{ong1993}, etc.
Geuvers and Nederhof describe a technique that models CC inside F$^\omega$ where term dependencies are all erased at the type level \cite{geuvers1991_sn_tof}.
In this chapter the technique of Geuvers and Nederhof will be adapted to show strong normalization of proof reduction.
Proof normalization ends up being a rather weak property, because it does not entail consistency.

\section{Model Description}

Figure~\ref{fig:syntax_fp} describes the syntax of System F$^\omega$ augmented with pairs.
The reduction relation for this system is presented in Figure~\ref{fig:reduction_fp} and the inference judgment in Figure~\ref{fig:typing_fp}.
System F$^\omega$ augmented with pairs is only slightly different from the original PTS description of F$^\omega$.
Moreover, it is a subsystem of the Calculus of Inductive Constructions and thus enjoys various metatheoretic properties such as substitution and weakening lemmas, preservation, strong normalization, and consistency.


\begin{figure}
    \centering
    \begin{align*}
        t &::= x\ |\ \mathfrak{b}(\kappa_1, x : t_1, t_2)\ |\ \mathfrak{c}(\kappa_2, t_1, \ldots, t_{\mathfrak{a}(\kappa_2)}) \\
        \kappa_1 &::= \lambda\ |\ \Pi \\
        \kappa_2 &::=\star\ |\ \kind\ |\ \text{app}\ |\ \text{prod}\ |\ \text{pair}\ |\ \text{fst}\ |\ \text{snd} \\
        &\mathfrak{a}(\star) = \mathfrak{a}(\kind) = 0 \\
        &\mathfrak{a}(\text{fst}) = \mathfrak{a}(\text{snd}) = 1 \\
        &\mathfrak{a}(\text{app}) = \mathfrak{a}(\text{prod}) = \mathfrak{a}(\text{pair}) = 2
    \end{align*}
    \vspace{-.4in}
    \begin{minipage}{0.5\textwidth}
        \begin{align*}
            \star &:= \mathfrak{c}(\star) \\
            \kind &:= \mathfrak{c}(\kind) \\
            \abs{\lambda}{x}{t_1}{t_2} &:= \mathfrak{b}(\lambda, x : t_1, t_2) \\
            (x : t_1) \to t_2 &:= \mathfrak{b}(\Pi, x : t_1, t_2) \\
            t_1\ t_2 &:= \mathfrak{c}(\text{app}, t_1, t_2)
        \end{align*}
    \end{minipage}%
    \begin{minipage}{0.5\textwidth}
        \begin{align*}
            t_1 \times t_2 &= \mathfrak{c}(\text{prod}, t_1, t_2) \\
            (t_1, t_2) &= \mathfrak{c}(\text{pair}, t_1, t_2) \\
            t.1 &= \mathfrak{c}(\text{fst}, t) \\
            t.2 &= \mathfrak{c}(\text{snd}, t)
        \end{align*}
    \end{minipage}
    \caption{Syntax for System F$^\omega$ with pairs.}
    \label{fig:syntax_fp}
\end{figure}


\begin{figure}
    \centering
    \begin{minipage}{0.5\textwidth}
        $$\BinderOneReduction$$
    \end{minipage}%
    \begin{minipage}{0.5\textwidth}
        $$\BinderTwoReduction$$
    \end{minipage}%
    $$\ConstructorReduction$$
    \begin{align*}
        (\abs{\lambda}{x}{A}{b})\ t &\betared [x := t]b \\
        [t_1, t_2].1 &\betared t_1 \\
        [t_1, t_2].2 &\betared t_2
    \end{align*}
    \caption{Reduction rules for System F$^\omega$ with pairs.}
    \label{fig:reduction_fp}
\end{figure}




\begin{figure}
    \centering
    \begin{minipage}{0.5\textwidth}
        $$\AxiomRuleFp$$
        $$\VarRuleFp$$
        $$\PiRuleFpOne$$
        $$\LambdaRuleFp$$
        $$\FirstRuleFp$$
        $$\SecondRuleFp$$
    \end{minipage}%
    \begin{minipage}{0.5\textwidth}
        $$\ConvRuleFp$$
        $$\PiRuleFpTwo$$
        $$\AppRuleFp$$
        $$\ProductRuleFp$$
        $$\PairRuleFp$$
    \end{minipage}%
    \caption{
        Typing rules for System F$^\omega$ with pairs.
    }
    \label{fig:typing_fp}
\end{figure}


The model follows all the same principles for the CC fragment of $\ced$.
For example, consider the \textsc{Lam} rule.
$$\LambdaRule$$
The goal is to find three semantic functions: one for kinds ($V(-)$); one for types ($\sema{-}$); and one for terms ($[-]$), such that:
    \begin{enumerate}
        \item $\sema{\Gamma} \vdash_\omega \sema{(x : A) \to_m B} : V(\pcodom(m))$
        \item $\sema{\Gamma; x_m : A} \vdash_\omega [t] : \sema{B}$
        \item $\sema{\Gamma} \vdash_\omega [\abs{\lambda_m}{x}{A}{t}] : \sema{(x : A) \to_m B}$
    \end{enumerate}
In order for this to work, term dependencies must all be dropped in function types.
Moreover, kinds are squished, such that $V(\kind) = V(\star) = \star$.
Thus, the judgment $\sema{\Gamma} \vdash_\omega \sema{(x : A) \to_m B} : V(\pcodom(m))$ must form an F$^\omega$ type.
The kind and type semantics is allowed to throw away terms and reductions because it only serves the purpose to maintain a well-typed output.
Instead, it is the term semantics that must take care to preserve all possible reductions such that strong normalization is a consequence of the model.

For dependent intersections, the type semantics is the obvious one: $\sema{(x : A) \cap B} = \sema{A} \times \sema{B}$.
Note that because $A$ is a type, it must be the case that $x \notin FV(\sema{B})$ otherwise the resulting type is not well-formed in F$^\omega$.
This is true already for function types, thus this extension needs no special treatment.
For equality the situation is special, the approach taken is to interpret all equalities as the type of the identity function: $\sema{a =_A b} = \fpId$.
There does not appear to be a more sensible choice, as the dependencies $a$ and $b$ must be dropped.


\begin{figure}
    \centering
    \begin{align*}
        V(\kind) &= \star \\
        V(\star) &= \star \\
        V((x : A) \to_m B) &= V(A) \to V(B) &\text{if} \text{ $A$ kind} \\
        V((x : A) \to_m B) &= V(B) &\text{otherwise}
    \end{align*}
    \begin{align*}
        \sema{\kind} &= 0 \\
        \sema{\star} &= 0 \\
        \sema{x_\square} &= x \\
        \sema{(x : A) \to_m B} &= (x : V(A)) \to \sema{A} \to \sema{B} &\text{if}& \text{ $A$ kind} \\
        \sema{(x : A) \to_m B} &= (x : \sema{A}) \to \sema{B} &\text{if}& \text{ $A$ type} \\
        \sema{\abs{\lambda_\tau}{x}{A}{t}} &= \abs{\lambda}{x}{V(A)}{\sema{t}} &\text{if}& \text{ $A$ kind} \\
        \sema{\abs{\lambda_\tau}{x}{A}{t}} &= \sema{t} &\text{if}& \text{ $A$ type} \\
        \sema{\app{f}{\tau}{a}} &= \sema{f}\ \sema{a} &\text{if}& \text{ $a$ type} \\
        \sema{\app{f}{\tau}{a}} &= \sema{f} &\text{if}& \text{ $a$ term} \\
        \sema{(x : A) \cap B} &= \sema{A} \times \sema{B} \\
        \sema{a =_A b} &= \fpId
    \end{align*}
    \begin{align*}
        \sema{x_m : A} &= x : V(A), w_x : \sema{A} &\text{if}& \text{ $A$ kind} \\
        \sema{x_m : A} &= x : \sema{A} &\text{if}& \text{ $A$ type} \\
        \sema{\varepsilon} &= 0 : \star, \bot : (X : \star) \to X \\
        \sema{\Gamma, x_m : A} &= \sema{\Gamma}, \sema{x_m : A}
    \end{align*}
    \caption{
        Model for kinds and types, not that type dependencies are dropped.
        Define $\fpId := (X : \star) \to X \to X$.
    }
    \label{fig:3:model_types}
\end{figure}

\begin{figure}
    \centering
    \begin{align*}
        c^B &= \bot\ B \hspace{0.5in}\text{if $B$ type} \\
        c^\star &= 0 \\
        c^{(x : A) \to B} &= \abs{\lambda}{x}{A}{c^B}
    \end{align*}
    \begin{align*}
        [*] &= c^0 \\
        [ x_{\square} ] &= w_x \\
        [ x_\star ] &= x \\
        [(x : A) \to_m B] &= c^{0 \to 0 \to 0}\ [A]\ ([x := c^{V(A)}][w_x := c^{\sema{A}}][B]) &\text{if}& \text{ $A$ kind} \\
        [(x : A) \to_m B] &= c^{0 \to 0 \to 0}\ [A]\ ([x := c^{\sema{A}}][B]) &\text{if}& \text{ $A$ type} \\
        [\abs{\lambda_m}{x}{A}{t}] &= (\abs{\lambda}{y}{0}{
            \abs{\lambda}{x}{V(A)}{
                \abs{\lambda}{w_x}{\sema{A}}{[t]}
            }
        })\ [A] &\text{if}& \text{ $A$ kind} \\
        [\abs{\lambda_m}{x}{A}{t}] &= (\abs{\lambda}{y}{0}{
            \abs{\lambda}{x}{\sema{A}}{[t]}
        })\ [A] &\text{if}& \text{ $A$ type} \\
        [\app{f}{m}{a}] &= [f]\ \sema{a}\ [a] &\text{if}& \text{ $a$ type} \\
        [\app{f}{m}{a}] &= [f]\ [a] &\text{if}& \text{ $a$ term} \\
    \end{align*}
    \begin{align*}
        [(x : A) \cap B] &= c^{0 \to 0 \to 0}\ [A]\ ([x := c^{\sema{A}}][B]) \\
        [[t_1, t_2; A]] &= (\abs{\lambda}{y}{0}{([t_1], [t_2])})\ [A] \\
        [t.1] &= [t].1 \\
        [t.2] &= [t].2 \\
        [a =_A b] &= c^{0 \to \sema{A} \to \sema{A} \to 0}\ [A]\ [a]\ [b] \\
        [\text{refl}(t; A)] &= (\abs{\lambda}{y_1}{0}{
                \abs{\lambda}{y_2}{\sema{A}}{\fpid}
            })\ [A]\ [t] \\
        [\psi(e, a, b; A, P)] &= (\abs{\lambda}{y_1}{0}{
                \abs{\lambda}{y_2\ y_3}{\sema{A}}{
                    \abs{\lambda}{y_2}{\sema{A} \to \fpId \to 0}{
                        [e]\ \sema{P}
                    }
                }
            })\ [A]\ [a]\ [b]\ [P] \\
        [\vartheta(e, a, b; T)] &= (\abs{\lambda}{y_1}{\sema{T}}{
            \abs{\lambda}{y_2}{0} {
                \abs{\lambda}{y_3}{\sema{T}}{
                    [e]
                }
            }
        })\ [b]\ [T]\ [a] \\
        [\varphi(a, b, e)] &= (\abs{\lambda}{y}{\fpId}{
            ([a], [b].2)
        })\ [e] \\
        [\delta(e)] &= (\abs{\lambda}{y}{\fpId}{
            \bot
        })\ [e]
    \end{align*}
    \caption{
        Model for terms, note that critically every subexpression is represented in the model to make sure no reductions are potentially lost.
        The definition of $c$ is used to construct a canonical element for any kind or type.
        Define $\fpid := \abs{\lambda}{X}{\star}{
            \abs{\lambda}{x}{X}{x}
        }$.
    }
    \label{fig:3:model_terms}
\end{figure}


The model interpretation for contexts always introduces two fresh variables, $0 : \star$ which is a canonical type, and $\bot : (X : \star) \to X$ which is used to construct canonical inhabitants for a type.
Note that including $\bot$ prevents this model from entailing consistency for the source system.
Regardless, F$^\omega$ is strongly normalizing in all contexts, thus the addition of $\bot$ does not prevent the model from serving its current purpose.
Before exploring more in-depth examples of the model the reader is invited to skim the semantic functions in Figure~\ref{fig:3:model_types} and Figure~\ref{fig:3:model_terms}.

Consider the following examples to garner intuition for the semantic model:
\begin{enumerate}
    \item {
        Given $\varepsilon \vdash_{\ced} \abs{\lambda_0}{X}{\star}{
            \abs{\lambda_\omega}{x}{X_\square}{x_\star}
        } : (X : \star) \to_0 X_\square \to_\omega X_\square$
        then $$\sema{\varepsilon} = 0 : \star;\ \bot : (X : \star) \to X$$
        $$[\abs{\lambda_0}{X}{\star}{
            \abs{\lambda_\omega}{x}{X_\square}{x_\star}
        }] = (\abs{\lambda}{y}{0}{
            \abs{\lambda}{X}{\star}{
                \abs{\lambda}{w_X}{0}{
                    (\abs{\lambda}{y}{0}
                        \abs{\lambda}{x}{X}{x}
                    )\ w_X
                }
            }
        })\ c^0$$
        $$\sema{(X : \star) \to_0 X_\square \to_\omega X_\square}
            = (X : \star) \to 0 \to X \to X$$
    }
    \item {
        Given $\Gamma \vdash_{\ced} t : T$ where $\Gamma = A : \star;\ B : \star;\ a : A_\square;\ f : A_\square \to_\omega (x : A_\square) \cap B_\square$, $t = [(\app{f_\star}{\omega}{a_\star}).1, (\app{f_\star}{\omega}{a_\star}).2; (x : A_\square) \cap B_\square]$, and $T = (x : A_\square) \cap B_\square$ then
        \begin{align*}
            &\sema{\Gamma = A : \star;\ B : \star;\ a : A_\square;\ f : A_\square \to_\omega (x : A_\square) \cap B_\square} =\\
            &\quad 0 : \star;\ \bot : (X : \star) \to X;\ A : \star;\ w_A : 0;\ B : \star;\ w_B : 0; \\
            &\quad a : A;\ f : A \to A \times B
        \end{align*}
            $$[[(\app{f_\star}{\omega}{a_\star}).1, (\app{f_\star}{\omega}{a_\star}).2; (x : A_\square) \cap B_\square]] = (\abs{\lambda}{y}{0}{
                ((f\ a).1, (f\ a).2)
            })\ (c^{0 \to 0 \to 0}\ w_A\ w_B)$$
        $$\sema{(x : A_\square) \cap B_\square} = A \times B$$
    }
\end{enumerate}

Notice that from the perspective of the type semantics ($\sema{-}$) that term dependencies in predicates must be dropped, but that they are preserved in the term semantics ($[-]$).
Thus, extra layers of abstraction are added when interpreting function arguments that are kinds.

\section{Model Soundness}

With the model defined the next step is to prove it is sound.
The process begins by showing the interpretation of kinds ($V(-)$) is sound.
This is not particularly difficult as the kind interpretation is quite simple.
After, lemmas about substitution and conversion follow without difficulty.

\begin{theorem}[Soundness of $V$]
    \label{lem:3:soudness_of_v}
    If $\Gamma \vdash_{\ced} t : \kind$ then $\Delta \vdash_\omega V(t) : \kind$ for any $\Delta$
\end{theorem}
\begin{proof}
    By induction on $\Gamma \vdash_{\ced} t : \kind$.
    The cases: \textsc{Lam}, \textsc{App}, \textsc{Int}, \textsc{Pair}, \textsc{Fst}, \textsc{Snd}, \textsc{Eq}, \textsc{Refl}, \textsc{Subst}, \textsc{Prm}, \textsc{Cast}, \textsc{Sep}, and \textsc{Conv} are impossible by inversion.

    $\text{Case: }\begin{array}{c} \AxiomRule[*] \end{array}$
    \begin{proofcase}
        Trivial by the \textsc{Ax} rule.
    \end{proofcase}

    $\text{Case: }\begin{array}{c} \VarRule[*] \end{array}$
    \begin{proofcase}
        By $\D{2}$: $\Gamma \vdash \kind : K$ which is impossible.
    \end{proofcase}

    $\text{Case: }\begin{array}{c} \PiRule[*] \end{array}$
    \begin{proofcase}
        Suppose $A$ is a kind, then $\pdom(m, K) = \kind$ and $V((x : A) \to_m B) = V(A) \to V(B)$.
        Applying the IH to $\D{1}$ and $\D{2}$ gives $\Delta_1 \vdash_\omega V(A) : \kind$ and $\Delta_2 \vdash_\omega V(B) : \kind$.
        However, note that there are no variables in any well-defined $V(t)$ which $V(A)$ and $V(B)$ are.
        Thus, $\Delta \vdash_\omega V(A) : \kind$ and $\Delta, x : V(A) \vdash_\omega V(B) : \kind$ by properties of $F^\omega$.
        Now by the \textsc{Pi1} rule $\Delta \vdash_\omega V(A) \to V(B) : \kind$ as required.
        \\ \\
        Suppose $A$ is a type, then $\pdom(m, K) = \star$ and $V((x : A) \to_m B) = V(B)$.
        By the IH applied to $\D{2}$: $\Delta \vdash_\omega V(B) : \kind$.
    \end{proofcase}
\end{proof}

\begin{lemma}
    \label{lem:3:conv_of_v}
    If $\Gamma_1 \vdash_{\ced} A : \kind$, $\Gamma_2 \vdash_{\ced} B : \kind$, and $A \equiv B$ then $V(A) = V(B)$
\end{lemma}
\begin{proof}
    By induction on $\Gamma \vdash A : \kind$.
    Note that $A$ is either $\star$ or $(x : C) \to_\tau D$.
    Suppose $A = \star$, then because $\star \equiv B$ it must be that $B = \star$.
    Thus, $V(A) = \star = V(B)$.
    \\ \\
    Suppose $A = (x : C_1) \to_\tau D_1$, but this forces $B = (x : C_2) \to_\tau D_2$ where $C_1 \equiv C_2$ and $D_1 \equiv D_2$.
    Note that $\Gamma \vdash C_1 : K$ and $\Gamma, x : C_1 \vdash D_1 : \kind$.
    Now by the IH: $V(D_1) = V(D_2)$ (note that the contexts need not agree).
    If $C_1$ is a kind, then $V((x : C_1) \to_\tau D_1) = V(C_1) \to V(D_1)$ and by the IH $V(C_1) = V(C_2)$.
    Instead, if $C_1$ is a type then $V((x : C_1) \to_\tau D_1) = V(D_1)$, but $V(D_1) = V(D_2)$.
    Thus, $V(A) = V((x : C_1) \to_\tau D_1) = V((x : C_2) \to_\tau D_2) = V(B)$.
\end{proof}

\begin{lemma}
    \label{lem:3:v_subst}
    If $\Gamma \vdash_\omega V(t) : \kind$ then $[x := b]V(t) = V(t) = V([x := b]t)$
\end{lemma}
\begin{proof}
    By induction on $t$ and inversion on $\Gamma \vdash V(t) : \kind$.
    Note that there are only two possibilities:

    $\text{Case: } t = \star$
    \begin{proofcase}
        Have $[x := b]V(\star) = [x := b]\star = \star = V(\star) = V([x := b]\star)$.
    \end{proofcase}

    $\text{Case: }\begin{array}{c} t = (x : A) \to_m B \end{array}$
    \begin{proofcase}
        Note that $A$ must be a kind or a type because $\Gamma \vdash V(t) : \kind$.
        Suppose $A$ is a kind, then $V((x : A) \to_m B) = V(A) \to V(B)$.
        Destructing the judgment gives $\Gamma \vdash V(A) : \kind$ and $\Gamma, x : V(A) \vdash V(B) : \kind$.
        Thus, by the IH: $[x := b]V(A) = V(A) = V([x := b]A)$ and $[x := b]V(B) = V(B) = V([x := b]B)$.
        By computation, $V([x := b](x : A) \to_m B) = V((x : [x := b]A) \to_m [x := b]B) = V([x := b]A) \to V([x := b]B) = V(A) \to V(B) = V((x : A) \to_m B)$.
        Also, by computation $[x := b]V((x : A) \to_m B) = [x := b](V(A) \to V(B)) = [x := b]V(A) \to [x := b]V(B) = V(A) \to V(B) = V((x : A) \to_m B)$. 
        \\ \\
        Suppose $A$ is a type, then $V((x : A) \to_m B) = V(B)$.
        By the IH: $[x := b]V(B) = V(B) = V([x := b]B)$.
    \end{proofcase}
\end{proof}

Next is demonstrating soundness of the type semantics.
Note again that type variables cannot appear free in the result of a well-defined interpretation of types.
This is codified in the next lemma, and soundness follows from it and soundness of the model for kinds.
A standard substitution lemma is proven after.

\begin{lemma}
    \label{lem:3:var_not_in_sema_when_kind}
    Suppose $\Gamma \vdash t : A$, $x_m : B \in \Gamma$, and $B$ type, then $x \notin FV(\sema{t})$
    where $A = \kind$ or $\Gamma \vdash_{\ced} A : \kind$
\end{lemma}
\begin{proof}
    Note that the restrictions on $A$ makes sure that $\sema{-}$ is well-defined.
    The definition of $\sema{-}$ intentionally throws away any dependence on terms.
    Thus, if $x$ is a term, because $B$ is a type, the only places where $x$ may appear in $t$ have all been thrown away.
    Therefore, $x \notin FV(\sema{t})$.
\end{proof}

\begin{theorem}[Soundness of $\sema{-}$]
    \label{lem:3:soundness_of_sema}
    If $\Gamma \vdash_{\ced} t : A$ then $\sema{\Gamma} \vdash_\omega \sema{t} : V(A)$
    where $A = \kind$ or $\Gamma \vdash A : \kind$
\end{theorem}
\begin{proof}
    By induction on $\Gamma \vdash_{\ced} t : A$.
    The cases: \textsc{Pair}, \textsc{Fst}, \textsc{Snd}, \textsc{Refl}, \textsc{Subst}, \textsc{Prm}, \textsc{Cast}, and \textsc{Sep} are impossible by inversion on $A = \kind$ or $\Gamma \vdash A : \kind$.

    $\text{Case: }\begin{array}{c} \AxiomRule[*] \end{array}$
    \begin{proofcase}
        By computation $\sema{\star} = 0$ and $V(\kind) = \star$.
        Note that $0 : \star \in \sema{\Gamma}$ thus this case is concluded by the \textsc{Var} rule.
    \end{proofcase}

    $\text{Case: }\begin{array}{c} \VarRule[*] \end{array}$
    \begin{proofcase}
        Note that $A \neq \kind$ by $\D{2}$, thus $K = \kind$.
        By computation $\sema{x_\square} = x$.
        Moreover, $A\pkind$ thus $x : V(A) \in \sema{\Gamma}$.
        Thus, $\sema{\Gamma} \vdash_\omega x : V(A)$
    \end{proofcase}

    $\text{Case: }\begin{array}{c} \PiRule[*] \end{array}$
    \begin{proofcase}
        By computation $V(\pcodom(m)) = V(\pdom(m, K)) = \star$.
        Applying the IH gives:
        \begin{enumerate}
            \item[$\D{1}$.] $\sema{\Gamma} \vdash_\omega \sema{A} : \star$
            \item[$\D{2}$.] $\sema{\Gamma, x_m : A} \vdash_\omega \sema{B} : \star$ 
        \end{enumerate}
        Suppose that $A$ is a kind.
        Then $\sema{(x : A) \to_m B} = (x : V(A)) \to \sema{A} \to \sema{B}$ and $\sema{\Gamma, x_m : A} = \sema{\Gamma}, x : V(A), w_x : \sema{A}$.\
        The \textsc{Pi2} rule applied with the results of the IH gives
            \begin{tightcenter} $\sema{\Gamma}, x : V(A) \vdash_\omega \sema{A} \to \sema{B} : \star$ \end{tightcenter}
        Now by Lemma~\ref{lem:3:soudness_of_v} applied to $\D{1}$: $\sema{\Gamma} \vdash_\omega V(A) : \kind$.
        Using the \textsc{Pi1} rule gives $\sema{\Gamma} \vdash_\omega V(A) \to \sema{A} \to \sema{B} : \star$.
        \\ \\
        Suppose that $A$ is a type.
        Then $\sema{(x : A) \to_m B} = (x : \sema{A}) \to \sema{B}$ and $\sema{\Gamma, x_m : A} = \sema{\Gamma}, x : \sema{A}$.
        Thus, by the \textsc{Pi2} rule $\sema{\Gamma} \vdash \sema{A} \to \sema{B} : \star$.
    \end{proofcase}

    $\text{Case: }\begin{array}{c} \LambdaRule[*] \end{array}$
    \begin{proofcase}
        It must be the case that $\Gamma \vdash (x : A) \to_m B : \kind$.
        Thus, $m = \tau$.
        Applying the IH gives:
        \begin{enumerate}
            \item[$\D{1}$.] $\sema{\Gamma} \vdash_\omega \sema{(x : A) \to_\tau B} : \star$
            \item[$\D{2}$.] $\sema{\Gamma, x_\tau : A} \vdash_\omega \sema{t} : V(B)$ 
        \end{enumerate}
        Suppose $A$ is a kind.
        Then $\sema{(x : A) \to_\tau B} = (x : V(A)) \to \sema{A} \to \sema{B}$, $\sema{\Gamma, x_m : A} = \sema{\Gamma}, x : V(A), w_x : \sema{A}$, and $\sema{\abs{\lambda_\tau}{x}{A}{t}} = \abs{\lambda}{x}{V(A)}{\sema{t}}$.
        Note that $\sema{\Gamma} \vdash c^{\sema{A}} : \sema{A}$.
        Thus, by substitution lemma for F$^\omega$: $\sema{\Gamma}, x : V(A) \vdash_\omega [w_x := c^{\sema{A}}]\sema{t} : [w_x := c^{\sema{A}}]V(B)$.
        However, because $A$ is kind and by Lemma~\ref{lem:3:var_not_in_sema_when_kind}: $[w_x := c^{\sema{A}}]\sema{t} = \sema{t}$.
        Note also that $FV(V(B))$ is empty, thus $[w_x := c^{\sema{A}}]V(B) = V(B)$.
        Thus, $\sema{\Gamma}, x : V(A) \vdash_\omega \sema{t} : V(B)$.
        Moreover, by Theorem~\ref{lem:3:soudness_of_v} it is the case that $\sema{\Gamma} \vdash V(A) : \kind$.
        Using the \textsc{Lam} rule gives $\sema{\Gamma} \vdash_\omega \abs{\lambda}{x}{V(A)}{\sema{t}} : V(A) \to V(B)$.
        \\ \\
        Suppose $A$ is a type.
        Then $\sema{\Gamma, x_m : A} = \sema{\Gamma}, x : \sema{A}$ and $\sema{\abs{\lambda_\tau}{x}{A}{t}} = \sema{t}$.
        Note additionally that $V((x : A) \to_m B) = V(B)$.
        Note that $\sema{\Gamma} \vdash c^{\sema{A}} : \sema{A}$.
        By substitution lemma, Lemma~\ref{lem:3:var_not_in_sema_when_kind}, and as above: $\sema{\Gamma} \vdash_\omega \sema{t} : V(B)$.
    \end{proofcase}

    $\text{Case: }\begin{array}{c} \AppRule[*] \end{array}$
    \begin{proofcase}
        Note that it cannot be the case that $[x := a]B = \kind$ by inversion on $\D{1}$, thus $\Gamma \vdash [x := a]B : \kind$ which force $m = \tau$.
        Furthermore, by $\D{1}$: $\Gamma \vdash (x : A) \to_\tau B : \kind$.
        Applying the IH to $\D{1}$ thus gives $\sema{\Gamma} \vdash_\omega \sema{f} : V((x : A) \to_\tau B)$.
        \\ \\
        Suppose $A$ is a kind, then $a$ is a type.
        Thus, $V((x : A) \to_\tau B) = V(A) \to V(B)$ and $\sema{\app{f}{\tau}{a}} = \sema{f}\ \sema{a}$.
        Applying the IH to $\D{2}$ gives $\sema{\Gamma} \vdash_\omega \sema{a} : V(A)$.
        By the \textsc{App} rule: $\sema{\Gamma} \vdash \sema{f}\ \sema{a} : V(B)$.
        Now by Lemma~\ref{lem:3:v_subst}: $V(B) = V([x := a]B)$.
        \\ \\
        Suppose $A$ is a type, then $a$ is a term.
        Thus, $V((x : A) \to_\tau B) = V(B)$ and $\sema{\app{f}{\tau}{a}} = \sema{f}$.
        But, $\sema{\Gamma} \vdash_\omega \sema{f} : V(B)$ already.
        Now by Lemma~\ref{lem:3:v_subst}: $V(B) = V([x := a]B)$.
    \end{proofcase}

    $\text{Case: }\begin{array}{c} \IntersectionRule[*] \end{array}$
    \begin{proofcase}
        Applying the IH gives:
        \begin{enumerate}
            \item[$\D{1}$.] $\sema{\Gamma} \vdash_\omega \sema{A} : \star$
            \item[$\D{2}$.] $\sema{\Gamma, x_\tau : A} \vdash_\omega \sema{B} : \star$ 
        \end{enumerate}
        Note that $A$ is a type thus $\sema{\Gamma, x_\tau : A} = \sema{\Gamma}, x : \sema{A}$.
        Applying the \textsc{Lam} rule twice reduces the goal to $\sema{\Gamma}, \sema{A} : \star, \sema{B} : \star \vdash_\omega \sema{A} \times \sema{B} : \star$.
        However, the pair case is an otherwise simple F$^\omega$ type, thus a short sequence of rules concludes the case.
    \end{proofcase}

    $\text{Case: }\begin{array}{c} \EqualityRule[*] \end{array}$
    \begin{proofcase}
        By computation $\sema{a =_A b} = \fpId$ and $V(\star) = \star$.
        A short sequence of rules in F$^\omega$ yields $\sema{\Gamma} \vdash \fpId : \star$.
    \end{proofcase}

    $\text{Case: }\begin{array}{c} \ConvRule[*] \end{array}$
    \begin{proofcase}
        Note that $A \neq \kind$ by $\D{1}$, and furthermore that $K = \kind$.
        Now by classification and $\D{3}$: $\Gamma \vdash B : \kind$.
        Applying the IH to $\D{2}$ gives $\sema{\Gamma} \vdash_\omega \sema{t} : V(B)$.
        Using Lemma~\ref{lem:3:conv_of_v} with $\D{3}$ gives $V(A) = V(B)$.
        Thus, the \textsc{Conv} rule concludes the case.
    \end{proofcase}
\end{proof}

\begin{lemma}
    \label{lem:3:sema_subst}
    Suppose $\Gamma \vdash_\omega \sema{t} : T$ then $\sema{[x := b]t} = [x := \sema{b}]\sema{t}$
\end{lemma}
\begin{proof}
    By induction on $t$ and inversion on $\Gamma \vdash_\omega \sema{t} : T$.
    Thus, only the cases where $\sema{t}$ is well-defined need to be considered.

    $\text{Case: } t = \star \text{ or } t = \kind$
    \begin{proofcase}
        The situation is the same because $\sema{\star} = \sema{\kind}$.
        By computation $\sema{[x := b]\star} = \sema{\star} = 0$ and $[x := \sema{b}]\sema{\star} = [x := \sema{b}]0 = 0$.
    \end{proofcase}

    $\text{Case: } t = y_\square$
    \begin{proofcase}
        Suppose $x \neq y$, then by computation $\sema{[x := b]y_\square} = \sema{y_\square} = y$ and $[x := \sema{b}]\sema{y_\square} = [x := \sema{b}]y = y$.
        Suppose $x = y$, then $\sema{[x := b]y_\square} = \sema{b}$ and $[x := \sema{b}]\sema{y_\square} = [x := \sema{b}]y = \sema{b}$.
    \end{proofcase}

    $\text{Case: } t = (y : C) \to_m D$
    \begin{proofcase}
        Suppose $A$ is a kind.
        Then $\sema{[x := b](y : C) \to_m D} = \sema{(y : [x := b]C) \to_m ([x := b]D)} = (y : V([x := b]A)) \to \sema{[x := b]C} \to \sema{[x := b]D}$.
        By Lemma~\ref{lem:3:v_subst} and applying the IH:
        \begin{align*}
            (y &: V([x := b]A)) \to \sema{[x := b]C} \to \sema{[x := b]D} \\
            &= (y : [x := \sema{b}]V(A)) \to [x := \sema{b}]\sema{C} \to [x := \sema{b}]\sema{D} \\
            &= [x := \sema{b}]((y : V(A)) \to \sema{C} \to \sema{D}) \\
            &= [x := \sema{b}]\sema{(y : C) \to_m D}
        \end{align*}
        Suppose $A$ is a type.
        Then $\sema{[x := b](y : C) \to_m D} = \sema{(y : [x := b]C) \to_m ([x := b]D)} = (y : \sema{[x := b]C}) \to \sema{[x := b]D}$.
        Applying the IH and chasing similar computations as above concludes the case.
    \end{proofcase}

    $\text{Case: } t = \abs{\lambda_\tau}{C}{c}$
    \begin{proofcase}
        Suppose $C$ is a kind.
        Then $\sema{[x := b](\abs{\lambda_\tau}{x}{C}{c})} = \sema{\abs{\lambda_\tau}{x}{[x := b]C}{[x := b]c}} = \abs{\lambda}{x}{V([x := b]C)}{\sema{[x := b]c}}$.
        By Lemma~\ref{lem:3:v_subst} and the IH:
        \begin{align*}
            \abs{\lambda}{x}{V([x := b]C)}&{\sema{[x := b]c}} \\
            &= \abs{\lambda}{x}{[x := \sema{b}]V(C)}{[x := \sema{b}]\sema{c}} \\
            &= [x := \sema{b}](\abs{\lambda}{x}{V(C)}{\sema{c}}) \\
            &= [x := \sema{b}]\sema{\abs{\lambda}{x}{C}{c}} \\
        \end{align*}
        Suppose $C$ is a type.
        Then $\sema{[x := b](\abs{\lambda_\tau}{x}{C}{c})} = \sema{\abs{\lambda_\tau}{x}{[x := b]C}{[x := b]c}} = \sema{[x := b]c}$.
        By the IH: $\sema{[x := b]c} = [x := \sema{b}]\sema{c} = [x := \sema{b}]\sema{\abs{\lambda_\tau}{x}{C}{c}}$.
    \end{proofcase}

    $\text{Case: } t = \app{f}{\tau}{a}$
    \begin{proofcase}
        Suppose $a$ is a type.
        Then $\sema{[x := b](\app{f}{\tau}{a})} = \sema{(\app{[x := b]f}{\tau}{[x := b]a})} = \sema{[x := b]f}\ \sema{[x := b]a}$.
        Using the IH gives $\sema{[x := b]f}\ \sema{[x := b]a} = ([x := \sema{b}]\sema{f})\ ([x := \sema{b}]\sema{a}) = [x := \sema{b}](\sema{f}\ \sema{a}) = [x := \sema{b}]\sema{\app{f}{\tau}{a}}$.
        \\ \\
        Supppose $a$ is a term.
        Then $\sema{[x := b](\app{f}{\tau}{a})} = \sema{(\app{[x := b]f}{\tau}{[x := b]a})} = \sema{[x := b]f}$.
        Using the IH gives $\sema{[x := b]f} = [x := \sema{b}]\sema{f} = [x := \sema{b}]\sema{\app{f}{\tau}{a}}$.
    \end{proofcase}

    $\text{Case: } t = (y : C) \cap D$
    \begin{proofcase}
        By computation $\sema{[x := b]((y : C) \cap D)} = \sema{(y : [x := b]C) \cap [x := b]D} = \sema{[x := b]C} \times \sema{[x := b]D}$.
        Using the IH gives $\sema{[x := b]C} \times \sema{[x := b]D} = ([x := \sema{b}]\sema{C}) \times ([x := \sema{b}]\sema{D}) = [x := \sema{b}](\sema{C} \times \sema{D}) = [x := \sema{b}]\sema{(x : C) \cap D}$.
    \end{proofcase}

    $\text{Case: } t = c =_C d$
    \begin{proofcase}
        By computation $\sema{[x := b](c =_C d)} = \sema{([x := b]c) =_{[x := b]C} ([x := b]d)} = \fpId$.
        Again, by computation $[x := \sema{b}]\sema{c =_C d} = [x := \sema{b}]\fpId = \fpId$.
    \end{proofcase}
\end{proof}

Finally, soundness of the term semantics must be shown.
This is not as simple as the original argument for CC modelled in F$^\omega$ because conversion is relative to erasure.
Luckily, erasure is locally homomorphic on type-like structure and the type semantics drops all term dependencies.

\begin{lemma}
    \label{lem:3:vsema_erase}
    If $\Gamma \vdash_\omega V(t) : \kind$ then $V(t) = V(|t|)$
\end{lemma}
\begin{proof}
    By induction on $t$ and inversion on $\Gamma \vdash V(t) : \kind$.

    $\text{Case: } t = \star \text{ or } t = \kind$
    \begin{proofcase}
        By computation $V(|\kind|) = V(\kind) = V(\star) = V(|\!\star\!|)$.
    \end{proofcase}

    $\text{Case: } t = (x : A) \to_m B$
    \begin{proofcase}
        Suppose $A$ is a kind.
        By Lemma~\ref{lem:2:classify_erase}: $|A|$ kind.
        Then $V((x : A) \to_m B) = V(A) \to V(B)$.
        Note that the subexpressions are well-typed, thus by the IH $V(|A|) = V(A)$ and $V(|B|) = V(B)$.
        Now by computation $V(|(x : A) \to_m B|) = V((x : |A|) \to_m |B|) = V(|A|) \to V(|B|) = V(A) \to V(B)$.
        \\ \\
        Suppose $A$ is not a kind.
        Then $V((x : A) \to_m B) = V(B)$.
        By the IH $V(|B|) = V(B)$.
        Thus, by computation $V(|(x : A) \to_m B|) = V((x : |A|) \to_m |B|) = V(|B|) = V(B)$.
    \end{proofcase}
\end{proof}

\begin{lemma}
    \label{lem:3:sema_erase}
    If $\Gamma \vdash_\omega \sema{t} : T$ then $\sema{t} = \sema{|t|}$
\end{lemma}
\begin{proof}
    By induction on $t$ and inversion on $\Gamma \vdash \sema{t} : T$.
    Erasure is again homomorphic on all remaining syntactic forms after inversion, thus only two cases are presented.

    $\text{Case: } t = \star \text{ or } t = \kind \text{ or } t = x_\square$
    \begin{proofcase}
        In each case $|t| = t$ thus trivial.
    \end{proofcase}

    $\text{Case: } t = (x : A) \to_m B$
    \begin{proofcase}
        Have $|(x : A) \to_m B| = (x : |A|) \to_m |B|$.
        Suppose wlog that $A$ is a kind.
        Then $\sema{(x : |A|) \to_m |B|} = (x : V(|A|)) \to \sema{|A|} \to \sema{|B|}$.
        By Lemma~\ref{lem:3:vsema_erase} and the IH $(x : V(|A|)) \to \sema{|A|} \to \sema{|B|} = (x : V(A)) \to \sema{A} \to \sema{B}$.
        Likewise, $\sema{(x : A) \to_m B} = (x : V(A)) \to \sema{A} \to \sema{B}$.
    \end{proofcase}
\end{proof}

The above lemmas demonstrate that if reduction happens in the erased term it should somehow be mirrored in reduction for the well-typed terms.
For kinds this turns out to be simple equality, as any possible dependence involving reduction are always dropped the structure of $V(t)$ for any $t$ is rigid.
The type semantics is slightly more complicated, but the same intuition holds: if a reduction were to occur in a term dependency then the result of the type semantics is equal, otherwise the reduction is exactly mirrored in the model.

\begin{lemma}
    \label{lem:3:vsema_erase_red_step}
    If $\Gamma \vdash_\omega V(s) : \kind$ and $|s| \betared t$ then $V(s) = V(t)$
\end{lemma}
\begin{proof}
    By induction on $|s| \betared t$.
    Note that only binder reduction is possible by inversion on $\Gamma \vdash V(s) : \kind$.

    $\text{Case: }\begin{array}{c} \BinderOneReduction[*] \end{array}$
    \begin{proofcase}
        Inversion on $\Gamma \vdash V(s) : \kind$ forces $s = (x : A) \to_m B$.
        Note that $|A| \betared A^\prime$.
        Suppose $A$ kind, then $V((x : A) \to_m B) = V(A) \to V(B)$.
        Now by the IH $V(A) = V(A^\prime)$ and $V((x : A^\prime) \to_m |B|) = V(A^\prime) \to V(B)$ by Lemma~\ref{lem:3:vsema_erase}.
        Suppose $A$ is not a kind, then $V((x : A) \to_m B) = V(B) = V((x : A^\prime) \to_m |B|)$.
    \end{proofcase}

    $\text{Case: }\begin{array}{c} \BinderTwoReduction[*] \end{array}$
    \begin{proofcase}
        Inversion on $\Gamma \vdash V(s) : \kind$ forces $s = (x : A) \to_m B$.
        Note that $|B| \betared B^\prime$.
        Suppose $A$ kind, then $V((x : A) \to_m B) = V(A) \to V(B)$.
        Now by the IH $V(B) = V(B^\prime)$ and $V((x : |A|) \to_m B^\prime) = V(A) \to V(B^\prime)$ by Lemma~\ref{lem:3:vsema_erase}.
        Suppose $A$ is not a kind, then $V((x : A) \to_m B) = V(B) = V(B^\prime) = V((x : |A|) \to_m B^\prime)$.
    \end{proofcase}
\end{proof}

\begin{lemma}
    \label{lem:3:sema_erase_red_step}
    If $\Gamma \vdash_\omega \sema{s} : T$ and $|s| \betared t$ then $\sema{s} \betared \sema{t}$ or $\sema{s} = \sema{t}$
\end{lemma}
\begin{proof}
    By induction on $|s| \betared t$.
    Note that only $\beta$-reduction is possible, as all other possible reduction steps are erased.

    $\text{Case: }\begin{array}{c} \app{(\abs{\lambda_m}{x}{A}{b})}{m}{t} \betared [x := t]b \end{array}$
    \begin{proofcase}
        By inversion on $\Gamma \vdash \sema{s} : T$ it must be the case that $m = \tau$.
        Thus, $|s| = \app{(\abs{\lambda_\tau}{x}{|A|}{|b|})}{\tau}{|t|}$ and $|s| \betared [x := |t|]|b|$.
        By Lemma~\ref{lem:2:erase_subst}: $[x := |t|]|b| = |[x := t]b|$.
        Now, Lemma~\ref{lem:3:sema_erase} yields $\sema{|[x := t]b|} = \sema{[x := t]b}$ and $\sema{|s|} = \sema{s}$.
        Using Lemma~\ref{lem:3:sema_subst} gives $\sema{[x := t]b} = [x := \sema{t}]\sema{b}$.
        Suppose $A$ is a kind, and thus $t$ is a type.
        Then $\sema{\app{(\abs{\lambda_\tau}{x}{A}{b})}{\tau}{t}} = (\abs{\lambda}{x}{V(A)}{\sema{b}})\ \sema{t} \betared [x := \sema{t}] \sema{b}$.
        Suppose $A$ is a type, and thus $t$ is a term.
        Then $\sema{\app{(\abs{\lambda_\tau}{x}{A}{b})}{\tau}{t}} = \sema{b}$, however this also means that $\Gamma \vdash \sema{b} : T$.
        The internally bound variable $x$ is thrown away, so it cannot be the case that $\sema{b}$ is well-typed in F$^\omega$ while $x \in FV(b)$ (Note that $x$ can be renamed to be disjoint from $\Gamma$), hence $x \notin FV(b)$.
        Thus, $[x := \sema{t}] \sema{b} = \sema{b}$ and the case is concluded.
    \end{proofcase}

    $\text{Case: }\begin{array}{c} \ConstructorReduction[*] \end{array}$
    \begin{proofcase}
        By inversion on $\Gamma \vdash \sema{s} : T$ it must be the case that $\kappa$ is $*$, $\kind$, $\bullet_\tau$, or eq.
        However, the cases $*$ and $\kind$ are impossible because they do not reduce.
        Suppose $|s| = \app{|f|}{\tau}{|a|}$ and assume wlog that $|a| \betared a^\prime$.
        If $a$ is a term then $\sema{\app{|f|}{\tau}{|a|}} = \sema{|f|} = \sema{\app{|f|}{\tau}{|a^\prime|}}$ and $\sema{|f|} = \sema{f}$ by Lemma~\ref{lem:3:sema_erase}.
        Suppose $a$ is a type.
        Then, by the IH $\sema{a} \betared \sema{a^\prime}$ or $\sema{a} = \sema{a^\prime}$.
        Now $\sema{\app{|f|}{\tau}{|a|}} = \sema{|f|}\ \sema{|a|}$, but by Lemma~\ref{lem:3:sema_erase}: $\sema{|f|}\ \sema{|a|} = \sema{f}\ \sema{a}$.
        Thus, $\sema{f}\ \sema{a} \betared \sema{f}\ \sema{a^\prime}$ or $\sema{f}\ \sema{a} = \sema{f}\ \sema{a^\prime}$.
        \\ \\
        Suppose $|s| = |a| =_{|A|} |b|$.
        Note that $\sema{u =_U v} = \fpId$ for any $u, v, U$.
        Thus, $\sema{s} = \sema{|s|} = \sema{t}$.
    \end{proofcase}

    $\text{Case: }\begin{array}{c} \BinderOneReduction[*] \end{array}$
    \begin{proofcase}
        By inversion on $\Gamma \vdash \sema{s} : T$ it must be the case that $\kappa$ is $\Pi_m$, $\lambda_\tau$, or $\cap$.
        The $\cap$ and $\lambda_\tau$ cases are similar to the $\Pi_m$ case and thus omitted.
        Have $|s| = (x : |A|) \to_m |B|$ and note that $|A| \betared A^\prime$.
        Suppose wlog that $A$ kind.
        Now $\sema{(x : |A|) \to_m |B|} = (x : V(|A|)) \to \sema{|A|} \to \sema{|B|}$.
        By the IH: $\sema{A} \betared \sema{A^\prime}$ or $\sema{A} = \sema{A^\prime}$.
        Suppose wlog that $\sema{A} \betared \sema{A^\prime}$, then $(x : V(|A|)) \to \sema{|A|} \to \sema{|B|} \betared (x : V(A^\prime)) \to \sema{A^\prime} \to \sema{|B|}$ by Lemma~\ref{lem:3:vsema_erase_red_step}.
        Now $\sema{(x : A^\prime) \to_m |B|} = (x : V(A^\prime)) \to \sema{A^\prime} \to \sema{|B|}$.
    \end{proofcase}

    $\text{Case: }\begin{array}{c} \BinderTwoReduction[*] \end{array}$
    \begin{proofcase}
        By inversion on $\Gamma \vdash \sema{s} : T$ it must be the case that $\kappa$ is $\Pi_m$, $\lambda_\tau$, or $\cap$.
        The $\cap$ and $\lambda_\tau$ cases are similar to the $\Pi_m$ case and thus omitted.
        Have $|s| = (x : |A|) \to_m |B|$ and note that $|B| \betared B^\prime$.
        Suppose wlog that $A$ kind.
        Now $\sema{(x : |A|) \to_m |B|} = (x : V(|A|)) \to \sema{|A|} \to \sema{|B|}$.
        By the IH: $\sema{B} \betared \sema{B^\prime}$ or $\sema{B} = \sema{B^\prime}$.
        Suppose wlog that $\sema{B} \betared \sema{B^\prime}$, then $(x : V(|A|)) \to \sema{|A|} \to \sema{|B|} \betared (x : V(|A|)) \to \sema{|A|} \to \sema{B^\prime}$.
        Now $\sema{(x : |A|) \to_m B^\prime} = (x : V(|A|)) \to \sema{|A|} \to \sema{B^\prime}$.
    \end{proofcase}
\end{proof}

\begin{lemma}
    \label{lem:3:sema_erase_red}
    If $\Gamma \vdash_\omega \sema{s} : T$ and $|s| \betastar t$ then $\sema{s} \betastar \sema{t}$
\end{lemma}
\begin{proof}
    By induction on $|s| \betastar t$.
    The reflexivity case is trivial by Lemma~\ref{lem:3:sema_erase}.
    Suppose $|s| \betared z$ and $z \betastar t$.
    By Lemma~\ref{lem:3:sema_erase_red_step} either $\sema{s} \betared \sema{z}$ or $\sema{s} = \sema{z}$.
    If $\sema{s} \betared \sema{z}$ then by preservation $\Gamma \vdash \sema{z} : T$.
    Note that $|z| = z$ by Lemma~\ref{lem:2:erasure_idempotent} and because reduction does not introduce new syntactic forms.
    Applying the IH to $|z| \betastar t$ gives $\sema{z} \betastar \sema{t}$, thus $\sema{s} \betastar \sema{t}$.
    If $\sema{s} = \sema{z}$ then obviously $\Gamma \vdash \sema{z} : T$ and the same argument as above works.
\end{proof}

With the reduction lemmas the required lemma about conversion is provable.
Finally, soundness of the term semantics is proven by a straightforward induction on the inference judgment of $\ced$.

\begin{lemma}
    \label{lem:3:sema_conv}
    If $\Gamma \vdash_\omega \sema{A} : T$, $\Gamma \vdash_\omega \sema{B} : T$, $A, B$\pseobj, and $A \equiv B$ then $\sema{A} \betaconv \sema{B}$
\end{lemma}
\begin{proof}
    By Lemma~\ref{thm:2:beta_iff_conv} $|A| \betaconv |B|$.
    Deconstructing this gives $|A| \betastar z$ and $|B| \betastar z$.
    By Lemma~\ref{lem:3:sema_erase_red}: $\sema{A} \betastar \sema{z}$ and $\sema{B} \betastar \sema{z}$.
    Thus, $\sema{A} \betaconv \sema{B}$.
\end{proof}

\begin{lemma}
    \label{lem:3:fomega_eta}
    If $\Gamma \vdash_\omega t : T$ and $\Gamma \vdash_\omega a : A$ then $\Gamma \vdash (\abs{\lambda}{x}{A}{t})\ a : T$
\end{lemma}
\begin{proof}
    Have $\Gamma \vdash_\omega \abs{\lambda}{x}{A}{t} : A \to T$ because $x$ does not appear free in $t$.
    Thus, by the \textsc{App} rule $\Gamma \vdash (\abs{\lambda}{x}{A}{t})\ a : T$.
\end{proof}

\begin{lemma}
    \label{lem:3:canonical_element}
    If $\Gamma \vdash_\omega A : T$ and $(\bot : (X : \star) \to X) \in \Gamma$ then $\Gamma \vdash_\omega c^A : A$
\end{lemma}
\begin{proof}
    If $A$ type then the proof is trivial.
    If $A$ kind then the proof follows by induction on the depth of the function type.
\end{proof}

\begin{theorem}[Soundness of \text{[}$-$\text{]}]
    \label{lem:3:soundness}
    If $\Gamma \vdash_{\ced} t : A$ then $\sema{\Gamma} \vdash_\omega [t] : \sema{A}$
\end{theorem}
\begin{proof}
    By induction on $\Gamma \vdash_{\ced} t : A$.
    The \textsc{Fst} case is omitted because it is very similar to \textsc{Snd}.
    The cases \textsc{Ax}, \textsc{Var}, \textsc{Pi}, \textsc{Lam}, and \textsc{App} are the same as the translation from CC to F$^\omega$.

    $\text{Case: }\begin{array}{c} \IntersectionRule[*] \end{array}$
    \begin{proofcase}
        Applying the IH to subderivations:
        \begin{enumerate}
            \item[$\D{1}.$] $\sema{\Gamma} \vdash_\omega [A] : 0$
            \item[$\D{2}.$] $\sema{\Gamma, x_\tau : A} \vdash_\omega [B] : 0$
        \end{enumerate}
        Note that $\sema{\Gamma} \vdash_\omega 0 \to 0 \to 0 : \star$.
        Thus, $\sema{\Gamma} \vdash_\omega c^{0 \to 0 \to 0} : 0 \to 0 \to 0$.
        By $\D{1}$ it is the case that $A$ type, thus $\sema{\Gamma, x_\tau : A} = \sema{\Gamma}, x : \sema{A}$.
        Using Lemma~\ref{lem:3:soundness_of_sema} on $\D{1}$ gives $\sema{\Gamma} \vdash_\omega \sema{A} : \star$.
        The substitution lemma yields $\sema{\Gamma} \vdash_\omega [x := c^{\sema{A}}][B] : 0$.
        Now applying the \textsc{App} rule two times concludes the case.
    \end{proofcase}

    $\text{Case: }\begin{array}{c} \PairRule[*] \end{array}$
    \begin{proofcase}
        Applying the IH to subderivations:
        \begin{enumerate}
            \item[$\D{1}.$] $\sema{\Gamma} \vdash_\omega [(x : A) \cap B] : 0$
            \item[$\D{2}.$] $\sema{\Gamma} \vdash_\omega [t] : \sema{A}$
            \item[$\D{3}.$] $\sema{\Gamma} \vdash_\omega [s] : \sema{[x := t]B}$
        \end{enumerate}
        By Lemma~\ref{lem:3:sema_subst}: $\sema{[x := t]B} = [x := \sema{t}]\sema{B}$.
        However, $A$ is a type by $\D{1}$ and thus $x \notin FV(\sema{B})$, hence $[x := \sema{t}]\sema{B} = \sema{B}$.
        Now $\sema{\Gamma} \vdash_\omega ([t_1], [t_2]) : \sema{A} \times \sema{B}$ by the \textsc{Pair} rule.
        Applying \ref{lem:3:fomega_eta} concludes the case.
    \end{proofcase}

    $\text{Case: }\begin{array}{c} \SecondRule[*] \end{array}$
    \begin{proofcase}
        Note by $\D{1}$ that $A$ is a type, thus $x \notin FV(\sema{B})$.
        By Lemma~\ref{lem:3:sema_subst}: $\sema{[x := t.1]B} = [x := \sema{t.1}]\sema{B} = \sema{B}$.
        Applying the IH to $\D{1}$ gives $\sema{\Gamma} \vdash_\omega [t] : \sema{A} \times \sema{B}$.
        The \textsc{Snd} rule concludes the case.
    \end{proofcase}

    $\text{Case: }\begin{array}{c} \EqualityRule[*] \end{array}$
    \begin{proofcase}
        Applying the IH to subderivations:
        \begin{enumerate}
            \item[$\D{1}.$] $\sema{\Gamma} \vdash_\omega [A] : 0$
            \item[$\D{2}.$] $\sema{\Gamma} \vdash_\omega [a] : \sema{A}$
            \item[$\D{3}.$] $\sema{\Gamma} \vdash_\omega [b] : \sema{A}$
        \end{enumerate}
        Note that $\sema{\Gamma} \vdash_\omega 0 \to \sema{A} \to \sema{A} \to 0 : \star$.
        Thus, $\sema{\Gamma} \vdash_\omega c^{0 \to \sema{A} \to \sema{A} \to 0} : 0 \to \sema{A} \to \sema{A} \to 0$.
        Now applying the \textsc{App} rule three times concludes the case.
    \end{proofcase}

    $\text{Case: }\begin{array}{c} \ReflRule[*] \end{array}$
    \begin{proofcase}
        Applying the IH to subderivations:
        \begin{enumerate}
            \item[$\D{1}.$] $\sema{\Gamma} \vdash_\omega [A] : 0$
            \item[$\D{2}.$] $\sema{\Gamma} \vdash_\omega [t] : \sema{A}$
        \end{enumerate}
        Of course, $\sema{\Gamma} \vdash_\omega \fpid : \fpId$.
        Thus, applying Lemma~\ref{lem:3:fomega_eta} twice concludes the case.
    \end{proofcase}

    $\text{Case: }\begin{array}{c} \SubstRule[*] \end{array}$
    \begin{proofcase}
        Note that by classification and $\D{1}$ it is that case that $A$ type.
        Applying the IH to subderivations:
        \begin{enumerate}
            \item[$\D{1}.$] $\sema{\Gamma} \vdash_\omega [A] : 0$
            \item[$\D{2}.$] $\sema{\Gamma} \vdash_\omega [a] : \sema{A}$
            \item[$\D{3}.$] $\sema{\Gamma} \vdash_\omega [b] : \sema{A}$ 
            \item[$\D{4}.$] $\sema{\Gamma} \vdash_\omega [e] : \fpId$
            \item[$\D{5}.$] $\sema{\Gamma} \vdash_\omega [P] : \sema{A} \to \fpId \to 0$
        \end{enumerate}
        Now $\sema{\Gamma} \vdash_\omega [e]\ \sema{P} : \sema{P} \to \sema{P}$.
        Note also that $\sema{\apptwo{P}{\tau}{a}{\tau}{\text{refl}(a;A)} \to_\omega \apptwo{P}{\tau}{b}{\tau}{e}} = \sema{P} \to \sema{P}$ because $\apptwo{P}{\tau}{a}{\tau}{\text{refl}(a;A)}$ is a type by $\D{3}$ and $a, b, e, \text{refl}(a;A)$ are all terms.
        Applying Lemma~\ref{lem:3:fomega_eta} four times concludes the case.
    \end{proofcase}

    $\text{Case: }\begin{array}{c} \PromoteRule[*] \end{array}$
    \begin{proofcase}
        Applying the IH to subderivations:
        \begin{enumerate}
            \item[$\D{1}.$] $\sema{\Gamma} \vdash_\omega [(x : A) \cap B] : 0$
            \item[$\D{2}.$] $\sema{\Gamma} \vdash_\omega [a] : \sema{A} \times \sema{B}$
            \item[$\D{3}.$] $\sema{\Gamma} \vdash_\omega [b] : \sema{A} \times \sema{B}$
            \item[$\D{4}.$] $\sema{\Gamma} \vdash_\omega [e] : \fpId$
        \end{enumerate}
        Applying Lemma~\ref{lem:3:fomega_eta} three times concludes the case.
    \end{proofcase}

    $\text{Case: }\begin{array}{c} \CastRule[*] \end{array}$
    \begin{proofcase}
        Note by $\D{1}$ it is clear that $A$ is a type.
        Applying the IH to subderivations:
        \begin{enumerate}
            \item[$\D{1}.$] $\sema{\Gamma} \vdash_\omega [a] : \sema{A}$
            \item[$\D{2}.$] $\sema{\Gamma} \vdash_\omega [b] : \sema{A} \times \sema{B}$
            \item[$\D{3}.$] $\sema{\Gamma} \vdash_\omega [e] : \fpId$
        \end{enumerate}
        Note that $\sema{\Gamma} \vdash_\omega [b].2 : \sema{B}$.
        Thus, $\sema{\Gamma} \vdash_\omega ([a], [b].2) : \sema{A} \times \sema{B}$.
        Applying Lemma~\ref{lem:3:fomega_eta} concludes the case.
    \end{proofcase}

    $\text{Case: }\begin{array}{c} \SeparationRule[*] \end{array}$
    \begin{proofcase}
        By computation $[\delta(e)] = (\abs{\lambda}{x}{\mathcal{I}([e])}{\bot})\ [e]$ and $\sema{(X : \star) \to_0 X} = (X : \star) \to X$.
        Note that $\sema{\Gamma} \vdash_\omega \bot : (X : \star) \to X$ and by definition $\sema{\Gamma} \vdash_\omega [e] : \mathcal{I}([e])$.
        Thus, by Lemma~\ref{lem:3:fomega_eta}: $\sema{\Gamma} \vdash [\delta(e)] : \sema{(X : \star) \to_0 X}$.
    \end{proofcase}

    $\text{Case: }\begin{array}{c} \ConvRule[*] \end{array}$
    \begin{proofcase}
        By classification, $\D{1}$ and $\D{3}$: $\Gamma \vdash B : K$.
        Now using Theorem~\ref{lem:3:soundness_of_sema} gives $\sema{\Gamma} \vdash_\omega \sema{A} : \star$ and $\sema{\Gamma} \vdash_\omega \sema{B} : \star$.
        Note that $A,B\pseobj$ by Lemma~\ref{lem:2:infer_implies_pseobj_type} and $|A| \betaconv |B|$ by Lemma~\ref{thm:2:beta_iff_conv}.
        By Lemma~\ref{lem:3:sema_conv}: $\sema{A} \betaconv \sema{B}$.
        Applying the IH to $\D{2}$ gives $\sema{\Gamma} \vdash_\omega [t] : \sema{B}$.
        The \textsc{Conv} rule concludes the case.
    \end{proofcase}
\end{proof}

\section{Normalization}

With soundness of the model shown the normalization argument follows in the same way as for CC modelled in F$^\omega$.
That is, proof reduction in $\ced$ is bounded by reduction in F$^\omega$, and thus because F$^\omega$ is strongly normalizing it provides a maximum number of reduction steps for which any proof must normalize in $\ced$.
Note that some reduction steps are technical, especially $\vartheta$, but are not conceptually difficult.

\begin{lemma}
    \label{lem:3:canonical_subst}
    $[x := b]c^A = c^{[x := b]A}$
\end{lemma}
\begin{proof}
    Straightforward by unraveling the definition of canonical elements ($c$) and applying substitution computation rules.
\end{proof}

\begin{lemma}
    \label{lem:3:subst}
    If $\Gamma \vdash t : A$ and $(x : B) \in \Gamma$ then
    \begin{enumerate}
        \item $[[x := b]a] = [x := \sema{b}][w_x := [b]][a]$ if $B$ kind
        \item $[[x := b]a] = [x := [b]][a]$ if $B$ type
    \end{enumerate}
\end{lemma}
\begin{proof}
    By induction on $\Gamma \vdash t : A$.
    Substitution is structural and with Lemma~\ref{lem:3:sema_subst}, Lemma~\ref{lem:3:v_subst}, and Lemma~\ref{lem:3:canonical_subst} many cases are straightforward by induction.
    Thus, only the variable cases and the \textsc{Int} case are presented.

    $\text{Case: }\begin{array}{c} \VarRule[*] \end{array}$
    \begin{proofcase}
        Rename to $y$.
        Suppose $x \neq y$, then $[[x := b]y_\star] = y$, $[x := \sema{b}][w_x := [b]][y_\star] = y$, and $[x := [b]][y_\star] = y$.
        When $y_\square$ the situation is the same.
        Suppose $x = y$ and that $B$ kind.
        If $B$ is kind, then it must be the case that $y_\square$.
        Now $[[x := b]y_\square] = [b]$ and  $[x := \sema{b}][w_x := [b]][y_\square] = [x := \sema{b}][w_x := [b]]w_y = [b]$.
        Suppose instead that $B$ type, then $[[x := b]y_\star] = [b]$ and $[x := [b]][y_\star] = [x := [b]]y = [b]$.
    \end{proofcase}

    $\text{Case: }\begin{array}{c} \IntersectionRule[*] \end{array}$
    \begin{proofcase}
        Suppose wlog that $B$ is a kind.
        Then $[[x := b](y : A) \cap B] = [(y : [x := b]A) \cap [x := b]B] = c^{0 \to 0 \to 0}[[x := b]A] ([y := c^{\sema{[x := b]A}}][[x := b]B])$.
        Now by the IH, Lemma~\ref{lem:3:sema_subst}, and the fact that $w_x \notin FV(\sema{A})$ the right-hand side is equal to $c^{0 \to 0 \to 0}[x := \sema{b}][w_x := [b]][A] ([y := c^{[x := \sema{b}][w_x := [b]]\sema{A}}][x := \sema{b}][w_x := [b]][B])$.
        Consider $[x := \sema{b}][w_x := [b]][(y : A) \cap B] = [x := \sema{b}][w_x := [b]]c^{0 \to 0 \to 0}[A] ([y := c^{\sema{A}}][B])$.
        Note that $x, w_x \notin FV(0 \to 0 \to 0)$, thus by Lemma~\ref{lem:2:subst_commute}, Lemma~\ref{lem:3:canonical_subst}, and computation rules of substitution this matches the previous right-hand side.
    \end{proofcase}
\end{proof}

\begin{lemma}
    \label{lem:3:reduction_bounded}
    If $\Gamma \vdash s : T$ and $s \betared t$ then $[s] \betastar_{\neq 0} [t]$
\end{lemma}
\begin{proof}
    By induction on $s \betared t$.
    The first projection case is very similar to the second projection case.
    Note by a simple observation that $[-]$ replicates every subexpression on the left-hand side with a matching invocation of $[-]$ on the right-hand side.
    Thus, if there is a reduction inside a subexpression it will always be tracked in the corresponding $[-]$ invocation via the inductive hypothesis.
    For this reason the structural reduction cases are omitted.

    $\text{Case: }\begin{array}{c} \app{(\abs{\lambda_m}{x}{A}{b})}{m}{t} \betared [x := t]b \end{array}$
    \begin{proofcase}
        Note by $\Gamma \vdash s : T$ that $A$ is either a kind or a type.
        Suppose $A$ is a kind and note that makes $t$ a type.
        Then $[\app{(\abs{\lambda_m}{x}{A}{b})}{m}{t}] = (\abs{\lambda}{y}{0}{\abs{\lambda}{x}{V(A)}{\abs{\lambda}{w_x}{\sema{A}}{[b]}}}) \ [A]\ \sema{t}\ [t]$.
        The variable $y$ is fresh thus after one $\beta$-reduction $(\abs{\lambda}{x}{V(A)}{\abs{\lambda}{w_x}{\sema{A}}{[b]}})\ \sema{t}\ [t]$.
        Now applying two more $\beta$-reductions yields $[x := \sema{t}][w_x := [t]][b]$.
        Note that $[[x := t]b] = [x := \sema{t}][w_x := [t]][b]$ by Lemma~\ref{lem:3:subst}.
        Thus, $[s] \betastar_{=3} [t]$, i.e. $[s]$ reduces to $[t]$ in three steps.
        \\ \\
        Suppose $A$ is a type and note that makes $t$ a term.
        Then $[\app{(\abs{\lambda_m}{x}{A}{b})}{m}{t}] = (\abs{\lambda}{y}{0}{\abs{\lambda}{w_x}{\sema{A}}{[b]}}) \ [A]\ [t]$.
        The variable $y$ is fresh thus after one $\beta$-reduction $(\abs{\lambda}{w_x}{\sema{A}}{[b]})\ [t]$.
        Applying one more $\beta$-reduction yields $[x := [t]][b]$.
        Note that $[[x := t]b] = [x := [t]][b]$ by Lemma~\ref{lem:3:subst}.
        Thus, $[s] \betastar_{=2} [t]$.
    \end{proofcase}

    $\text{Case: }\begin{array}{c} [t_1, t_2; A].2 \betared t_2 \end{array}$
    \begin{proofcase}
        Have $[[t_1, t_2; A].2] = ((\abs{\lambda}{y}{0}{([t_1], [t_2])})\ [A]).2$.
        Note that the variable $y$ is fresh and thus not in $FV([t_1])$ or $FV([t_2])$.
        A second projection and one $\beta$-reduction yields $[t_2]$.
        Thus, $[s] \betastar_{=2} [t_2]$.
    \end{proofcase}
    
    $\text{Case: }\begin{array}{c} \app{\psi(\text{refl}(t; A_1), a, b; A_2, P)}{\omega}{t} \betared t \end{array}$
    \begin{proofcase}
        Note that $t$ is a term by inversion on $\Gamma \vdash s : T$.
        Have $[\app{\psi(\text{refl}(t; A_1); A_2, P)}{\omega}{t}] =
            (\abs{\lambda}{y_1}{0}{
                \abs{\lambda}{y_2\ y_3}{\sema{A}}{
                    \abs{\lambda}{y_4}{\sema{A_2} \to \fpId \to 0}{
                        [\text{refl}(t; A_1)]\ \sema{P}
                    }
                }
            })\ [A_2]\ [a]\ [b]\ [P]\ [t]$.
        Applying four $\beta$-reductions yields $[\text{refl}(t; A_1)]\ \sema{P}\ [t]$.
        Now $[\text{refl}(t; A_1)] =
            (\abs{\lambda}{y_1}{0}{
                \abs{\lambda}{y_2}{\sema{A_1}}{\fpid}
            })\ [A_1]\ [t]$.
        Applying two more $\beta$-reductions gives $\fpid\ \sema{P}\ [t]$.
        Finally, applying two remaining $\beta$-reductions yields $[t]$.
        Thus, $[s] \betastar_{=8} [t]$.
    \end{proofcase}
    
    $\text{Case: }\begin{array}{c} \vartheta(\text{refl}(t; A), a, b; T) \betared \text{refl}(a; T) \end{array}$
    \begin{proofcase}
        Have $[\vartheta(\text{refl}(t; A), a, b; T)] = (\abs{\lambda}{y_1}{\sema{T}}{
                \abs{\lambda}{y_2}{0} {
                    \abs{\lambda}{y_3}{\sema{T}}{
                        ((\abs{\lambda}{y_1}{0}{
                            \abs{\lambda}{y_2}{\sema{A}}{\fpid}
                        })\ [A]\ [t])
                    }
                }
            })\ [b]\ [T]\ [a]$.
        Note that all $y_i$ are fresh and thus not in the free variables of any subexpressions.
        Performing two $\beta$-reductions on the interior (the result of $[\text{refl}(t_1;A)]$) and the outermost $\beta$-reduction yields:
            $(\abs{\lambda}{y_2}{0} {
                \abs{\lambda}{y_3}{\sema{T}}{
                    \fpid
                }
            })\ [T]\ [a]$.
        Now $[\text{refl}(a; T)] = (\abs{\lambda}{y_2}{0} {
                \abs{\lambda}{y_3}{\sema{T}}{
                    \fpid
                }
            })\ [T]\ [a]$.
        Thus, $[s] \betastar_{=3} [t]$.
    \end{proofcase}
\end{proof}

\begin{theorem}[Proof Normalization]
    \label{lem:3:proof_normalization}
    If $\Gamma \vdash t : A$ then $t$ is strongly normalizing and there exists a unique value $t_n$ such that $t \betastar t_n$
\end{theorem}
\begin{proof}
    Using Lemma~\ref{lem:3:soundness_of_sema} gives $\sema{\Gamma} \vdash_\omega [t] : \sema{A}$.
    Note that F$^\omega$ with pairs is strongly normalizing with a unique normal form (because it is also confluent).
    Thus, \textit{all} possible reduction paths to the normal form are terminating.
    Let $\partial$ be the \textit{maximum} number of reduction steps $[t]$ could take to reach a normal form.
    Note that this value is computable by brute force search.
    Pick any sequence of reductions in $t$ bounded by $\partial$.
    If this sequence concludes in a value then $t$ is strongly normalizing, because the sequence is arbitrary.
    If $t$ is not a value then $t \betastar_{>\partial} t^\prime$, but this is impossible by Lemma~\ref{lem:3:reduction_bounded}.
    Now by confluence of reduction, all values reached from any arbitrary reduction path must be joinable at a single value.
    Thus, $t \betastar t_n$ where $t_n$ is a unique value.
\end{proof}
