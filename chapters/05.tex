\chapter{Object Normalization}
\label{chap:5}

Consistency guarantees that the logic and equational theory of $\ced$ is non-trivial.
Proof normalization guarantees that, at least, inference for kinds and types is decidable.
Neither of these properties are strong enough on their own to guarantee decidability of type checking.
To obtain decidability of type checking it must be the case that objects are normalizing.
However, the technique described below depends on both proof normalization and consistency.
While this is not the only possible way it is suggestive how difficult a property object normalization is to prove.

The technique for proving object normalization proceeds in two parts.
First, a notion of strict syntax is defined where $\varphi$ forms are eliminated.
Proofs will yield strict proofs and utilizing both proof normalization and consistency one can show that proof reduction bounds strict object reduction.
Hence, strict objects will be strongly normalizing.
Second, a notion of well-typed observational equivalence where the only observation of interest is divergence of objects is defined.
Then it is demonstrated that proofs are observational equivalent to their strict version.
Thus, because strict objects are normalizing, it must be the case that objects are normalizing.

\section{Strict Syntax}

An important observation to make about $\varphi$ forms is that they do not expand the logical theory.
Meaning, it is not possible to prove \textit{new} propositions or inhabit \textit{new} types.
Instead, the $\varphi$ form is meant primarily as a means of obtaining efficient objects for an already proven property.
To that end a strict variation of syntax is easily defined on arbitrary syntax by replacing any $\varphi$ form with its internal proof witness.
Figure~\ref{fig:strict} describes strictification of syntax.
Notice that $\mathfrak{s}(t)$ is structural on all syntax forms \textit{except} the $\varphi$ form where everything but the internal proof witness is discarded.


\begin{figure}
    \centering
    \begin{align*}
        \mathfrak{s}(x_K) &= x_K \\
        \mathfrak{s}(\mathfrak{b}(\kappa_1, x : t_1, t_2)) &= \mathfrak{b}(\kappa_1, x : \mathfrak{s}(t_1), \mathfrak{s}(t_2)) \\
        \mathfrak{s}(\mathfrak{c}(\kappa_2, t_1, \ldots, t_{\mathfrak{a}(\kappa_2)})) &= \mathfrak{c}(\kappa_2, \mathfrak{s}(t_1), \ldots, \mathfrak{s}(t_{\mathfrak{a}(\kappa_2)})) \text{ if }\kappa_2 \neq \varphi \\
        \mathfrak{s}(\varphi(f, e; A, T)) &= \mathfrak{s}(f)
    \end{align*}
    \caption{Strictification of raw syntax to remove any $\varphi$ subexpressions.}
    \label{fig:strict}
\end{figure}



An immediate consequence of this definition is that strict proofs remain proofs of the original type.
Note, however, that the type itself is not made strict.
Strictification of the type would not hold because $\varphi$ constructs inside the type may be compared via the conversion relation and strict objects are not equal to the corresponding object.
\begin{lemma}
    \label{lem:5:strict_proof}
    If $\Gamma \vdash t : A$ then $\Gamma \vdash \mathfrak{s}(t) : A$
\end{lemma}
\begin{proof}
    By induction on $\Gamma \vdash t : A$, because $\mathfrak{s}$ is structural the $\varphi$ rule is the only one of interest:

    $\text{Case: }\begin{array}{c} \CastRule[*] \end{array}$
    \begin{proofcase}
        Immediate by IH applied to $\D{3}$.
    \end{proofcase}
\end{proof}
Of course, if strict syntax reduces, then the resulting syntax must be strict.
This is an obvious consequence of the fact that reduction never introduces $\varphi$ forms.
\begin{lemma}
    \label{lem:5:strict_preservation}
    If $\mathfrak{s}(s) \betared t$ then $t = \mathfrak{s}(t)$
\end{lemma}
\begin{proof}
    Without $\varphi$ constructs $\mathfrak{s}(s) = s$ by a trivial induction.
    Reduction cannot produce $\varphi$ constructs, thus $t$ necessarily does not have $\varphi$ as a subexpression.
    Hence, $\mathfrak{s}(t) = t$.
\end{proof}
The core observation is that proof reduction in strict proofs upper-bounds reduction in strict objects.
Thus, if a strict object steps, and note that this must be a $\beta$-step, then there is some strict proof such that the original strict proof reduces to it and the erasures match.
There could be many more reductions in the strict proof because syntax forms for equality and intersections are all mostly erased.
However, none of these forms will block a $\beta$-redex because the proof is well-typed.
Note that this property hinges on both proof normalization and equational consistency.
Proof normalization is used to eliminate any extraneous redexes that would otherwise be erased.
Consistency is used to eliminate the $\delta$ case as it could theoretically generate a $\beta$-redex after erasure if the theory was not consistent.
Of course, $\varphi$ could also generate a $\beta$-redex after erasure, but this is impossible because the syntax under consideration is strict.
\begin{lemma}
    \label{lem:5:strict_erase_red_step}
    If $\Gamma \vdash s : A$ and $|\mathfrak{s}(s)| \betared t$ then $\exists\ t^\prime$ such that $\mathfrak{s}(s) \betastar_{\neq 0} t^\prime$ and $|t^\prime| = t$
\end{lemma}
\begin{proof}
    By induction on $\Gamma \vdash s : A$.
    The erasure of the \textsc{Ax}, \textsc{Var}, \textsc{Refl}, and \textsc{Cast} cases are values and thus do not reduce.
    First projection and first promotion are very similar to second projection and second promotion (respectively).
    The \textsc{Int} and \textsc{Eq} cases are structural in both erasure and strictification and are thus very similar to the \textsc{Pi} case.

    $\text{Case: }\begin{array}{c} \PiRule[*] \end{array}$
    \begin{proofcase}
        Have $|\mathfrak{s}((x : A) \to_m B)| = (x : |\mathfrak{s}(A)|) \to_m |\mathfrak{s}(B)|$.
        Suppose that $|\mathfrak{s}(A)| \betared t$.
        By the IH applied to $\D{1}$: $\exists\ t^\prime$ such that $\mathfrak{s}(A) \betastar_{\neq 0} t^\prime$ and $|t^\prime| = t$.
        Thus, $(x : \mathfrak{s}(A)) \to_m \mathfrak{s}(B) \betastar_{\neq 0} (x : t^\prime) \to_m \mathfrak{s}(B)$ and $|(x : t^\prime) \to_m \mathfrak{s}(B)| = (x : t) \to_m |\mathfrak{s}(B)|$.
        The case where a reduction happens in $|\mathfrak{s}(B)|$ is similar.
    \end{proofcase}

    $\text{Case: }\begin{array}{c} \LambdaRule[*] \end{array}$
    \begin{proofcase}
        Suppose $m = 0$.
        Have $|\mathfrak{s}(\abs{\lambda_0}{x}{A}{b}|)| = |\mathfrak{s}(b)|$ with $|\mathfrak{s}(b)| \betared t$.
        Applying the IH to $\D{2}$ concludes the case.
        \\ \\
        Suppose that $m = \omega$, note that $m = \tau$ is very similar and thus omitted.
        Have $|\mathfrak{s}(\abs{\lambda_\omega}{x}{A}{b})| = \abs{\lambda_\omega}{x}{\diamond}{|\mathfrak{s}(b)|}$ and $|\mathfrak{s}(b)| \betared t$.
        Applying the IH to $\D{2}$ yields $\exists\ t^\prime$ such that $\mathfrak{s}(b) \betastar_{\neq 0} t^\prime$ and $|t^\prime| = t$.
        Now $\abs{\lambda_\omega}{x}{\mathfrak{s}(A)}{\mathfrak{s}(b)} \betastar_{\neq 0} \abs{\lambda_\omega}{x}{\mathfrak{s}(A)}{t^\prime}$ and $|\abs{\lambda_\omega}{x}{\mathfrak{s}(A)}{t^\prime}| = \abs{\lambda_\omega}{x}{\diamond}{t}$.
    \end{proofcase}

    $\text{Case: }\begin{array}{c} \AppRule[*] \end{array}$
    \begin{proofcase}
        If $m = 0$ then the proof follows by a straightforward application of the IH to $\D{1}$.
        \\ \\
        Suppose that $m = \omega$.
        Let $|\mathfrak{s}(f)| = \abs{\lambda_\omega}{x}{\diamond}{v}$ and $\app{|\mathfrak{s}(f)|}{\omega}{|\mathfrak{s}(a)|} \betared [x := |\mathfrak{s}(a)|]v$.
        By Theorem~\ref{lem:3:proof_normalization} $\mathfrak{s}(f)$ is strongly normalizing in proof reduction.
        If $f$ contains a projection redex, promotion redex, or erased application redex then produce $f_i$ by contracting that redex.
        Continue contracting these redexes until none remain, assume $k$ such redexes are contracted, thus $\mathfrak{s}(f) \betastar f_k$.
        Note that none of these redexes affect the erasure of $f$, thus $|\mathfrak{s}(f)| = |f_k|$.
        Now $f_k$ has only four possibilities: $f_k = \abs{\lambda_\omega}{x}{A}{b}$, or $f_k = \psi(\prefl(z; Z), a, b; A, P)$, or $f_k = \delta(\prefl(t; A))$, or $f_k = \varphi(f, e; A, T)$.
        By strictification the $\varphi$ case is impossible and by Theorem~\ref{lem:4:eq_consistency} the $\delta$ case is impossible.
        \begin{itemize}
            \item {
                Suppose $f_k = \abs{\lambda_\omega}{x}{A}{b}$.
                Now $\app{f_k}{\omega}{\mathfrak{s}(a)} \betared [x := \mathfrak{s}(a)]b$ and $|[x := \mathfrak{s}(a)]b| = [x := |\mathfrak{s}(a)|]v$.
            }
            \item {
                Suppose $f_k = \psi(\prefl(z; Z), a, b; A, P)$.
                Now $\app{\psi(\prefl(z; Z), a, b; A, P)}{\omega}{\mathfrak{s}(a)} \betared \mathfrak{s}(a)$.
                Note that $|f_k| = |\mathfrak{s}(f)|$, but $|\psi(\prefl(z; Z), a, b; A, P)| = \abs{\lambda_\omega}{x}{\diamond}{x}$ and $|\mathfrak{s}(f)| = \abs{\lambda_\omega}{x}{\diamond}{v}$.
                Thus, $v = x$ and $|\mathfrak{s}(a)| = [x := |\mathfrak{s}(a)|]v$.
            }
        \end{itemize}
        \vspace{.15in}
        Suppose $m = \omega$ and $|\mathfrak{s}(f)| \betared t$.
        Note that the case where $|\mathfrak{s}(a)| \betared t$ is very similar and thus omitted.
        Applying the IH to $\D{1}$ gives $\exists\ t^\prime$ such that $\mathfrak{s}(f) \betastar_{\neq 0} t^\prime$ and $|t^\prime| = t$.
        Now $\app{\mathfrak{s}(f)}{\omega}{\mathfrak{s}(a)} \betastar_{\neq 0} \app{t^\prime}{\omega}{\mathfrak{s}(a)}$ and $|\app{t^\prime}{\omega}{\mathfrak{s}(a)}| = \app{t}{\omega}{|\mathfrak{s}(a)|}$.
        \\ \\
        Suppose $m = \tau$ then strictification and erasure are structural.
        Thus, a $\beta$-redex is tracked exactly and any structural redexes are very similar to the $m = \omega$ case.
    \end{proofcase}

    $\text{Case: }\begin{array}{c} \PairRule[*] \end{array}$
    \begin{proofcase}
        Have $|\mathfrak{s}([t_1, t_2; A])| = |\mathfrak{s}(t_1)|$ and $|\mathfrak{s}(t_1)| \betared t$.
        Applying the IH to $\D{1}$ yields $\exists\ t^\prime$ such that $\mathfrak{s}(t_1) \betastar_{\neq 0} t^\prime$ and $|t^\prime| = t$.
        Now $[\mathfrak{s}(t_1), \mathfrak{s}(t_2); \mathfrak{s}(A)] \betastar_{\neq 0} [t^\prime, \mathfrak{s}(t_2); \mathfrak{s}(A)]$ and $|[t^\prime, \mathfrak{s}(t_2); \mathfrak{s}(A)]| = t$.
    \end{proofcase}

    $\text{Case: }\begin{array}{c} \SecondRule[*] \end{array}$
    \begin{proofcase}
        Have $|\mathfrak{s}(b.2)| = |\mathfrak{s}(b)|$ and $|\mathfrak{s}(b)| \betared t$.
        Applying the IH to $\D{1}$ gives $\exists\ t^\prime$ such that $\mathfrak{s}(b) \betastar_{\neq 0} t^\prime$ and $|t^\prime| = t$.
        Now $(\mathfrak{s}(b)).2 \betastar_{\neq 0} t^\prime.2$ and $|t^\prime.2| = t$.
    \end{proofcase}

    $\text{Case: }\begin{array}{c} \SubstRule[*] \end{array}$
    \begin{proofcase}
        Have $|\mathfrak{s}(\psi(e, a, b; A, T))| = |\mathfrak{s}(e)|$ and $|\mathfrak{s}(e)| \betared t$.
        Applying the IH to $\D{4}$ yields $\exists\ t^\prime$ such that $\mathfrak{s}(e) \betastar_{\neq 0} t^\prime$ and $|t^\prime| = t$.
        Now $\psi(\mathfrak{s}(e), \mathfrak{s}(a), \mathfrak{s}(b); \mathfrak{s}(A), \mathfrak{s}(T)) \betastar_{\neq 0} \psi(t^\prime, \mathfrak{s}(a), \mathfrak{s}(b); \mathfrak{s}(A), \mathfrak{s}(T))$
            and $|\psi(t^\prime, \mathfrak{s}(a), \mathfrak{s}(b); \mathfrak{s}(A), \mathfrak{s}(T))| = t$.
    \end{proofcase}

    $\text{Case: }\begin{array}{c} \PromoteRule[*] \end{array}$
    \begin{proofcase}
        Have $|\mathfrak{s}(\vartheta_2(e, a, b; (x : A) \cap B))| = |\mathfrak{s}(e)|$ and $|\mathfrak{s}(e)| \betared t$.
        Applying the IH to $\D{5}$ gives $\exists\ t^\prime$ where $\mathfrak{s}(e) \betastar_{\neq 0} t^\prime$ and $|t^\prime| = t$.
        Now $\vartheta_2(\mathfrak{s}(e), \mathfrak{s}(a), \mathfrak{s}(b); \mathfrak{s}((x : A) \cap B)) \betastar_{\neq 0} \vartheta_2(t^\prime, \mathfrak{s}(a), \mathfrak{s}(b); \mathfrak{s}((x : A) \cap B))$
            and $|\vartheta_2(t^\prime, \mathfrak{s}(a), \mathfrak{s}(b); \mathfrak{s}((x : A) \cap B))| = t$.
    \end{proofcase}

    $\text{Case: }\begin{array}{c} \SeparationRule[*] \end{array}$
    \begin{proofcase}
        Have $|\mathfrak{s}(\delta(e))| = |\mathfrak{s}(e)|$ and $|\mathfrak{s}(e)| \betared t$.
        Applying the IH to $\D{1}$ gives $\exists\ t^\prime$ where $\mathfrak{s}(e) \betastar_{\neq 0} t^\prime$ and $|t^\prime| = t$.
        Now $\delta(\mathfrak{s}(e)) \betastar_{\neq 0} \delta(t^\prime)$ and $|\delta(t^\prime)| = t$.
    \end{proofcase}

    $\text{Case: }\begin{array}{c} \ConvRule[*] \end{array}$
    \begin{proofcase}
        Immediate by the IH applied to $\D{2}$.
    \end{proofcase}

\end{proof}

\begin{theorem}[Strict Object Normalization]
    \label{lem:5:strict_object_sn}
    If $\Gamma \vdash t : A$ then $|\mathfrak{s}(t)|$ is strongly normalizing
\end{theorem}
\begin{proof}
    Lemma~\ref{lem:5:strict_proof} gives $\Gamma \vdash \mathfrak{s}(t) : A$ and thus by Theorem~\ref{lem:3:proof_normalization}: $\mathfrak{s}(t)$ is strongly normalizing wrt proof reduction.
    Let $\partial$ be the maximum length reduction sequence $\mathfrak{s}(t)$ could take to reach the unique value.
    Suppose wlog that $|\mathfrak{s}(t)|$ contains a redex.
    Contract this redex giving $|\mathfrak{s}(t)| \betared e_1$.
    By Lemma~\ref{lem:5:strict_erase_red_step}: $\exists\ t_1$ such that $\mathfrak{s}(t) \betastar_{\neq 0} t_1$ and $|t_1| = e_1$.
    Using preservation of proof reduction: $\Gamma \vdash t_1 : A$ and moreover by Lemma~\ref{lem:5:strict_preservation} $\mathfrak{s}(t_1) = t_1$.
    Let the number of contracted redexes by the reduction $\mathfrak{s}(t) \betastar_{\neq 0} t_1$ be $k$, then there is a maximum of $\partial - k$ redexes in $t_1$.
    If redexes remain in $e_1$ than the process can be repeated because $t_1$ is a strict proof whose erasure is $e_1$.
    However, eventually the number of steps taken must run out, because $\partial$ is a finite value.
    Thus, the procedure may be repeated as many times as desired, but $e_i$, the value after $i$ iterations of this process, must eventually run out of redexes by Lemma~\ref{lem:5:strict_erase_red_step}.
    Therefore, $|\mathfrak{s}(t)|$ is strongly normalizing.
\end{proof}

Strong normalization of strict objects leads to an interesting observation.
Recall the definition of conversion: $a \equiv b$ if and only if $\exists\ u, v$ such that $a \betastar u$, $b \betastar v$ and $|u| \betaconv |v|$.
An observant reader may wonder why reduction is allowed after two candidates objects, $|u|$ and $|v|$ are obtained.
In other words, why not merely compare for equality: $|u| = |v|$.
The answer is because $\varphi$ may generate $\beta$-redexes after erasure, and it turns out that it is the only syntax form for which this is possible.
Thus, if $\varphi$ was removed from the system then conversion \textit{could} be defined using equality of objects instead of reduction convertibility of objects.
The $\varphi$ form is unique amongst all the other syntax.
It is both not responsible for any new properties, but also responsible for the only interesting reduction in objects.

Another question that the reader may have is why not represent the reduction of $\varphi$ in the proof system.
The answer is that there is no obvious way to make the reduction well-typed, thus preservation would be lost.
Indeed, the proof witness of a $\varphi$ form, $f : (a : A) \to (x : A) \cap B$, is allowed to be as complicated as required to produce a subtype, $(x : A) \cap B$, from the corresponding supertype, $A$.
However, the object for the $\varphi$ is merely the identity $\lambda$-term ($\abs{\lambda}{x}{\diamond}{x}$).
To make it possible to type this term in the proof system some notion of subtyping would have to be added directly into the rules.
It is not immediately clear how to make this move without producing a radically different system.
Yet, it does hint that the $\varphi$ rule is, in some sense, expanding a semantic subtyping relation that is later realized internally via a notion of casts.
Indeed, it may be fruitful to view the proof-object distinction as being fundamentally related to subtyping.

\section{Observational Equivalence of Objects}

Unfortunately, the structural relationship between strict objects and regular objects is not easy to exploit, if it can used at all.
Observational equivalence gives a method of relating regular objects to strict objects.
Objects being the concept of interest means that contexts need to be well-typed because an object is only the erasure of a proof.
To make contexts the inductive structure of syntax is reused with a unique fresh free variable, labelled $h$, that represents a hole.
The variable is unique meaning it occurs only once in the given syntax.
Context structure could be defined inductively, but this methodology allows reuse of erasure and substitution.
\begin{definition}
    A \textbf{context} $\gamma : (\Gamma, A) \to (\Delta, B)$ is a syntactic form with a unique free variable $h$ representing a hole such that if $\Gamma \vdash t : A$ then $\Delta \vdash [h := t]\gamma : B$.
\end{definition}

Observational equivalence is then defined to be equivalence of divergence of the associated objects substituted for $h$ in the given context.
There are several possible ways to defined observational equivalence including the choice of what counts as an observation.
For the purposes of this chapter divergence is the only observation of interest.

\begin{definition}
    The syntax $a$ and $b$ are \textbf{observationally equivalent} at $A$ in $\Gamma$ (written: $\Gamma \vdash a \approx_A b$) iff
    for any context $\gamma : (\Gamma, A) \to (\varepsilon, \pUnit)$ with unique fresh variable $h$: $|[h := a]\gamma|$ normalizes iff $|[h := b]\gamma|$ normalizes
\end{definition}

It is easy to see that observational equivalence forms an equivalence relation relative to the parameters $\Gamma$ and $A$.

\begin{lemma}
    \label{lem:5:obs_refl}
    $\Gamma \vdash a \approx_A a$
\end{lemma}
\begin{proof}
    Immediate by definition.
\end{proof}

\begin{lemma}
    \label{lem:5:obs_sym}
    If $\Gamma \vdash a \approx_A b$ then $\Gamma \vdash b \approx_A a$
\end{lemma}
\begin{proof}
    By definition the stated condition holds via an if-and-only-if.
    Hence, observational equivalence is symmetric.
\end{proof}

\begin{lemma}
    \label{lem:5:obs_trans}
    If $\Gamma \vdash a \approx_A b$ and $\Gamma \vdash b \approx_A c$ then $\Gamma \vdash a \approx_A c$
\end{lemma}
\begin{proof}
    Let $\gamma : (\Gamma, A) \to (\varepsilon, \pUnit)$ be an arbitrary context with unique fresh variable $h$.
    Suppose $|[h := b]\gamma|$ diverges, then by $\Gamma \vdash b \approx_A c$ it must be the case that $|[h := c]\gamma|$ diverges.
    By Lemma~\ref{lem:5:obs_sym}: $\Gamma \vdash b \approx_A a$ and thus as above $|[h := a]\gamma|$ diverges.
    Suppose $|[h := b]\gamma|$ normalizes, then by $\Gamma \vdash b \approx_A c$: $|[h := c]\gamma|$ normalizes.
    Likewise, using symmetry and the same reasoning: $|[h := a]\gamma|$ normalizes.
    Hence, $|[h := a]\gamma|$ normalizes if and only if $|[h := c]\gamma|$ normalizes.
\end{proof}

Moreover, because observational equivalence only tracks divergence it must be the case that if an object is observationally equivalent to a strongly normalizing object, then it is also strongly normalizing.

\begin{lemma}
    \label{lem:5:obs_implies_sn}
    If $\Gamma \vdash a : A$, $\Gamma \vdash b : A$, $|a|$ is strongly normalizing, and $\Gamma \vdash a \approx_A b$ then $|b|$ is strongly normalizing
\end{lemma}
\begin{proof}
    Immediate by definition of observational equivalence, $|b|$ must be normalizing in any possible context $\gamma : (\Gamma, A) \to (\varepsilon, \pUnit)$.
    Thus, $|b|$ must be strongly normalizing.
\end{proof}

Knowing that $\varphi$ objects behave like the identity function is critical.
Note that the condition $FV(|e|)$ being empty is necessary to ensure the below lemma holds.
If this condition where dropped than $\app{e}{\omega}{a}$ for when $a$ is a value would say nothing about the reduction behavior of $f$.
This is because $e$ could be a free variable in $\Gamma$ and thus the equational proof it is evidence for would not be ``actionable'' in some sense.
It is not clear if this condition is too strict, as one could imagine a more refined or semantic condition that allows rewriting in contexts that are in some sense ordered.
To that point, the free variable condition is communicating that the computational nature of $|e|$ cannot depend on any hypothetical evidence.
Thus, the computational nature of $|e|$ is a sort of ground truth.
Regardless, the condition is one that works and that is not too onerous as to prevent constructions of interest.

\begin{lemma}
    \label{lem:5:phi_obs_id}
    Let $T = (a : A) \to_\omega (x : A) \cap B$.
    Suppose:
    \begin{enumerate}
        \item $\Gamma \vdash T : \star$
        \item $\Gamma \vdash f : T$
        \item $\Gamma \vdash e : (a : A) \to_\omega a_\star =_A (\app{f}{\omega}{a_\star}).1$
        \item $FV(|e|)$ is empty
        \item $\Gamma \vdash a : A$
    \end{enumerate}
    Then $\Gamma \vdash a \approx_A (\app{f}{\omega}{a}).1$
\end{lemma}
\begin{proof}
    Suppose $a \betastar a^\prime$ where $FV(|a^\prime|)$ is empty then $\app{e}{\omega}{a}$ must reduce to a value because $FV(|e|)$ is empty.
    Note that theoretically the definition of $e$ could use $\varphi$, but because $e$ is erased and Lemma~\ref{lem:5:strict_proof} shows that $\varphi$ does not enable any new proofs it can be safely assumed that $e = \mathfrak{s}(e)$.
    In other words, any usage of $\varphi$ that contains a $\varphi$ in $e$ can always be transformed into a usage that does not contain $\varphi$ in the definition of $e$ without affecting either the typing relation or object reduction.
    Hence, $\app{e}{\omega}{a} \betastar \prefl(z; Z)$ where $|z|$ is in normal form, $z \equiv a^\prime$, and $z \equiv (\app{f}{\omega}{a^\prime}).1$.
    Thus, $\app{|f|}{\omega}{|a^\prime|} \betastar |z|$ and $|a^\prime| \betastar |z|$.
    Meaning both are normalizing and hence observationally equivalent.
    \\ \\
    Suppose $a$ is neutral, i.e. $FV(|a|)$ is not empty.
    In this case $\app{e}{\omega}{a}$ does not necessarily reduce to a value because $a$ is neutral.
    Let $\gamma : (\Gamma, A) \to (\varepsilon, \pUnit)$ be a context with hole variable $h$.
    It is possible that $a$ reduces to a value after substitution into $h$.
    However, this falls into the purview of the above case, thus suppose wlog that $a$ remains neutral after substitution.
    If $|[h := a]\gamma|$ diverges then clearly $|[h := \app{f}{\omega}{a}.1]\gamma|$ diverges as well.
    Thus, suppose $|[h := a]\gamma|$ normalizes.
    There are two possibilities to consider:
    \begin{enumerate}
        \item {
            Suppose $A$ has no inhabitants and is equivalent to false.
            In this scenario, $|f|$ must reduce to a term that looks like $\abs{\lambda_\omega}{x}{\diamond}{\app{y}{\omega}{\cdots}}$ for $y = x$ or $y \in \Gamma$.
            One alternative possibility is for $f$ to contain a $\varphi$ rule, but this requires equational evidence that $f$ behaves like an identity and critically that equational evidence cannot depend on $x : A$ or any other variable in $\Gamma$.
            Hence, any $\varphi$ rule used in $f$ must be via closed terms, but because $f$ produces a $(x : A) \cap B$ the body must contain a variable representing false to inhabit $A$.
            Therefore, regardless of the situation, $|f|$ must be neutral and thus $\app{|f|}{\omega}{|a|}$ must be neutral.
        }
        \item {
            Suppose $A$ has inhabitants and is not equivalent to false.
            Let $\vdash v : A$ represent an arbitrary closed value of $A$.
            Using the same above argument, $\app{e}{\omega}{v}$ would reduce to a value meaning $\app{|f|}{\omega}{|v|} \betastar |v|$.
            Thus, $f$ must reduce to some intermediary $f^\prime$ such that $FV(|f^\prime|)$ is empty.
            Note that technically $f$ may contain free variables, but they cannot be used in a computationally relevant way.
            Now $|f^\prime|$ must reduce to a term that looks like $\abs{\lambda_\omega}{x}{\diamond}{\app{x}{\omega}{\cdots}}$, but when a neutral is applied to this function it produces a neutral.
            Hence, $\app{|f|}{\omega}{|a|}$ is neutral.
        }
    \end{enumerate}
\end{proof}

The key theorem is that regular proofs are observationally equivalent to their strictification.
The only way this theorem could fail is because of $\varphi$ subexpressions.
However, Lemma~\ref{lem:5:phi_obs_id} conveys that $\varphi$ proofs are observationally equivalent to an identity function.
Indeed, the previous lemma does all the heavy lifting in making this theorem go through.

\begin{theorem}
    \label{lem:5:strict_obs_regular}
    If $\Gamma \vdash t : A$ then $\Gamma \vdash t \approx_A \mathfrak{s}(t)$
\end{theorem}
\begin{proof}
    By induction on $\Gamma \vdash t : A$.
    The \textsc{Ax} and \textsc{Var} cases are trivial because $\mathfrak{s}(\star) = \star$ and $\mathfrak{s}(x_K) = x_K$.
    Cases that cannot participate in well-typed reductions: \textsc{Int} and \textsc{Eq} are omitted in favor of presenting only the \textsc{Pi} case.
    The \textsc{Fst} case is very similiar to the \textsc{Snd} case and omitted.
    Likewise, the \textsc{PrmSnd} case is very similar to the \textsc{PrmFst} case and omitted.

    $\text{Case: }\begin{array}{c} \PiRule[*] \end{array}$
    \begin{proofcase}
        Have $\mathfrak{s}((x : A) \to_m B) = (x : \mathfrak{s}(A)) \to_m \mathfrak{s}(B)$.
        Applying the IH to $\D{1}$ and $\D{2}$ yields $\Gamma \vdash A \approx_{\pdom(m, K)} \mathfrak{s}(A)$ and $\Gamma; x_m : A \vdash B \approx_{\pcodom(m)} \mathfrak{s}(B)$.
        Thus, because no reduction in $A$ or $B$ can alter the initial structural shape of this syntax it must be the case that $\Gamma \vdash (x : A) \to_m B \approx_{\pcodom(m)} \mathfrak{s}((x : A) \to_m B)$.
    \end{proofcase}

    $\text{Case: }\begin{array}{c} \LambdaRule[*] \end{array}$
    \begin{proofcase}
        Have $\mathfrak{s}(\abs{\lambda_m}{x}{A}{t}) = \abs{\lambda_m}{x}{\mathfrak{s}(A)}{\mathfrak{s}(t)}$.
        If $m = \omega$ or $m = \tau$ then the case is similar to the \textsc{Pi} case.
        Suppose $m = 0$.
        Let $\gamma : (\Gamma, (x : A) \to_m B) \to (\varepsilon, \pUnit)$ be an arbitrary context with hole variable $h$.
        Now $|[h := \abs{\lambda_m}{x}{A}{t}]\gamma| = [h := |t|]|\gamma|$ and $|[h := \abs{\lambda_m}{x}{\mathfrak{s}(A)}{\mathfrak{s}(t)}]\gamma = [h := |\mathfrak{s}(t)|]|\gamma|$.
        By the IH applied to $\D{2}$: $\Gamma; x_m : A \vdash t \approx_B \mathfrak{s}(t)$ which means that $[h := |t|]|\gamma|$ normalizes iff $[h := |\mathfrak{s}(t)|]|\gamma|$.
        Note that while it is true the type of the contexts differ, the consequence remains the same because observational equivalence is phrased relative to erasure.
    \end{proofcase}

    $\text{Case: }\begin{array}{c} \AppRule[*] \end{array}$
    \begin{proofcase}
        Suppose $m = \omega$.
        The only way that $\app{f}{\omega}{a}$ could be non-terminating is if $f = \varphi(f^\prime, e; A, T)$.
        Thus, suppose this is the case wlog.
        Destructing $\D{1}$ and applying Lemma~\ref{lem:5:phi_obs_id} gives $\Gamma \vdash a \approx_A (\app{f^\prime}{\omega}{a}).1$.
        Applying the IH to $\D{2}$ gives $\Gamma \vdash a \approx_A \mathfrak{s}(a)$.
        Applying the IH to a subderivation of $\D{1}$ yields $\Gamma \vdash f^\prime \approx_T \mathfrak{s}(f^\prime)$.
        However, if $f^\prime$ behaves like the identity for any well-typed input it must be the case that $\mathfrak{s}(f^\prime)$ does as well.
        Hence, $\Gamma \vdash \app{f}{\omega}{a} \approx_B \app{\mathfrak{s}(f)}{\omega}{\mathfrak{s}(a)}$
        \\ \\
        Suppose $m = \tau$.
        There are no possible $\beta$-reductions involving $\varphi$, thus by proof normalization and the IH the case is concluded.
        \\ \\
        Suppose $m = 0$.
        Let $\gamma$ be a context with a hole variable $h$.
        Now $|[h := \app{f}{0}{a}]\gamma| = [h := |f|]|\gamma|$ and $|[h := \mathfrak{s}(\app{f}{0}{a})]\gamma| = [h := |\mathfrak{s}(f)|]|\gamma|$.
        Applying the IH to $\D{1}$ concludes the case.
    \end{proofcase}

    $\text{Case: }\begin{array}{c} \PairRule[*] \end{array}$
    \begin{proofcase}
        Let $\gamma$ be a context with hole variable $h$.
        Now $|[h := [t, s; T]]\gamma| = [h := |t|]|\gamma|$ and $|[h := \mathfrak{s}([t, s; T])]\gamma| = [h := |\mathfrak{s}(t)|]|\gamma|$.
        These objects are observationally equivalent by applying the IH to $\D{2}$.
    \end{proofcase}

    $\text{Case: }\begin{array}{c} \SecondRule[*] \end{array}$
    \begin{proofcase}
        Let $\gamma$ be a context with hole variable $h$.
        Now $|[h := t.2]\gamma| = [h := |t|]|\gamma|$ and $|[h := \mathfrak{s}(t.2)]\gamma| = [h := |\mathfrak{s}(t)|]|\gamma$.
        Applying the IH to $\D{1}$ concludes the case.
    \end{proofcase}

    $\text{Case: }\begin{array}{c} \ReflRule[*] \end{array}$
    \begin{proofcase}
        Note that $|\prefl(t; A)| = \abs{\lambda}{x}{\diamond}{x}$ and this erasure does not depend on $t$ or $A$.
        Thus, $|\mathfrak{s}(\prefl(t; A)) = \abs{\lambda}{x}{\diamond}{x}$, hence the case is trivial.
    \end{proofcase}

    $\text{Case: }\begin{array}{c} \SubstRule[*] \end{array}$
    \begin{proofcase}
        Let $\gamma$ be a context with a hole variable $h$.
        Now $|[h := \psi(e, a, b; A, P)]\gamma| = [h := |e|]|\gamma|$ and $|[h := \mathfrak{s}(\psi(e, a, b; A, P))]\gamma| = [h := |\mathfrak{s}(e)|]|\gamma|$.
        Applying the IH to $\D{4}$ concludes the case.
    \end{proofcase}

    $\text{Case: }\begin{array}{c} \PromoteRule[*] \end{array}$
    \begin{proofcase}
        Let $\gamma$ be a context with a hole variable $h$.
        Now $|[h := \vartheta_1(e, a, b; T)]\gamma| = [h := |e|]|\gamma|$ and $|[h := \mathfrak{s}(\vartheta_1(e, a, b; T))]\gamma| = [h := |\mathfrak{s}(e)|]|\gamma|$.
        Applying the IH to $\D{4}$ concludes the case.
    \end{proofcase}

    $\text{Case: }\begin{array}{c} \CastRule[*] \end{array}$
    \begin{proofcase}
        Let $\gamma$ be a context with a hole variable $h$.
        Now $|[h := \varphi(f, e; A, T)]\gamma| = [h := \abs{\lambda_\omega}{x}{\diamond}{x}]|\gamma|$ and $|[h := \mathfrak{s}(\varphi(f, e; A,T))]\gamma| = [h := |\mathfrak{s}(f)|]|\gamma|$.
        Applying the IH to $\D{3}$ gives that $\Gamma \vdash f \approx_T \mathfrak{s}(f)$.
        Let $\Gamma \vdash a : A$ be an arbitrary proof of $a$, then by Lemma~\ref{lem:5:phi_obs_id}: $\Gamma \vdash a \approx_A (\app{f}{\omega}{a}).1$.
        Hence, $f$ behaves like the identity for any well-typed $a$ proof and thus so does $\mathfrak{s}(f)$ because it is observationally equivalent to $f$.
    \end{proofcase}

    $\text{Case: }\begin{array}{c} \SeparationRule[*] \end{array}$
    \begin{proofcase}
        Let $\gamma$ be a context with a hole variable $h$.
        Now $|[h := \delta(e)]\gamma| = [h := |e|]|\gamma|$ and $|[h := \mathfrak{s}(\delta(e))]\gamma| = [h := |\mathfrak{s}(e)|]|\gamma|$.
        Applying the IH to $\D{1}$ concludes the case.
    \end{proofcase}

    $\text{Case: }\begin{array}{c} \ConvRule[*] \end{array}$
    \begin{proofcase}
        Applying the IH to $\D{1}$ gives $\Gamma \vdash t \approx_B \mathfrak{s}(t)$.
        Note that $\Gamma \vdash t : A$ by assumption and Lemma~\ref{lem:5:strict_proof} gives $\Gamma \vdash \mathfrak{s}(t) : A$.
        With $\D{1}$ and $\D{2}$ the following derivation is obtained: $\Gamma \vdash \mathfrak{s}(t) : B$.
        Thus, a context $\gamma : (\Gamma, A) \to (\varepsilon, \pUnit)$ is constructed by working through the intermediary context at type $B$.
    \end{proofcase}
\end{proof}

\begin{theorem}[Object Normalization]
    \label{lem:5:object_normalization}
    If $\Gamma \vdash t : A$ then $|t|$ is strongly normalizing
\end{theorem}
\begin{proof}
    By Lemma~\ref{lem:5:strict_proof}: $\Gamma \vdash \mathfrak{s}(t) : A$.
    Now Theorem~\ref{lem:5:strict_obs_regular} yields $\Gamma \vdash t \approx_A \mathfrak{s}(t)$.
    Lemma~\ref{lem:5:strict_object_sn} gives $|\mathfrak{s}(t)|$ is strongly normalizing.
    Therefore, Lemma~\ref{lem:5:obs_implies_sn} concludes.
\end{proof}