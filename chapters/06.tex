\chapter{Conclusion}

The design of Cedille followed an extrinsic (or curry-style) philosophy that placed programs as primary with types as annotations.
Under this philosophy it is only natural to consider an untyped equality because it is closer to ones semantic understanding.
However, these decisions led to equality being a source of undecidability for type checking.
In this work an alternative road is taken where $\ced$ is designed around a hybrid philosophy of intrinsic and extrinsic.
Proofs are considered primary and the design itself follows proof theoretic principles, but universal (dependent) quantification is with respect to objects.
An object is the erasure of a proof, and objects do not exist without proofs.
This distinction allowed for a description of proof reduction and, separately, object reduction.
Moreover, many metatheoretic properties are shown relative to proof reduction and conversion including: syntactic proof preservation and strong proof normalization.
Additionally, in the absence of the \textsc{Cast} rule, object reduction is normalizing.

A failure of the proof theoretic discipline and victory of the extrinsic philosophy is the $\varphi$ construct.
The \textsc{Cast} rule does prevent many desirable properties such as decidability of type checking.
However, it does not prevent logical consistency and the apparent strength it adds to the theory is substantial.
It is not clear how to derive any of the interesting encodings published in existing literature without the \textsc{Cast} rule.
Yet, the \textsc{Cast} rule is the only source of non-termination for object reduction in the system $\ced$.

It is not clear how to systematically correct $\varphi$ without destroying all of its benefits.
However, there is a question if the $\varphi$ construct is necessary for efficient inductive data.
Without the \textsc{Cast} rule it is possible to derive inductive Church encodings that are \textit{not} efficient because computing an out (e.g. a predecessor of a natural number) is proportional to the size of the data (e.g. linear time for natural numbers).
One potential fix is using Scott encodings instead which have been shown to be inductive in Cedille \cite{jenkins2021monotone}.
Moreover, it seems unlikely that all ink has dried on the topic of impredicative encodings, especially in Cedille.
Knowing which encodings do not depend on it would be useful to delineate the relative power between the system with and without $\varphi$.

Independent of the $\varphi$ rule there is an open question of whether function extensionality is an admissible axiom in $\ced$.
With a typed equality the separation rule of $\ced$ may only act on typed terms, unlike in Cedille where equality is untyped.
Therefore, separation in $\ced$ is only capable of differentiating functions by applying test inputs in the domain of a function.
While this is convincing intuition showing that this axiom is consistent requires an extensional model which cannot exist for Cedille.

The type theory $\ced$ represents a step forward to a proof theoretic version of Cedille.
With the \textsc{Cast} rule removed the system \textit{is} a proof theory according to the philosophy of Kreisel and Gentzen.
The full system, by contrast, is a proof theory relative to an oracle deciding $\varphi$-safety.
Nevertheless, $\ced$ has narrowed the gap between Cedille and existing research in modern type theory, with the hope of making its unique ideas more palatable to a wider audience.
