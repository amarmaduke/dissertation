\chapter{Conclusion}

The design of Cedille followed an extrinsic (or curry-style) philosophy that placed programs as primary with types as annotations.
Under this philosophy it is only natural to consider an untyped equality because that is the version that is closer to ones semantic understanding.
However, these decision led to equality enabling non-terminating derivations originating independently using: untyped indices; untyped witnesses of reflexivity; untyped Leibniz Law; erased $\rho$ (when type equality is present); erased $\delta$; and $\varphi$.
In this work an alternative road is taken where $\ced$ is designed around a hybrid philosophy of intrinsic and extrinsic.
Proofs are considered primary and the design itself follows proof theoretic principles, but universal (dependent) quantification is with respect to objects.
An object is the erasure of a proof, but critically it is not merely an untyped $\lambda$-calculus term.
Objects do not exist without proofs.
This distinction allowed for a description of proof reduction and, separately, object reduction.
Moreover, many metatheoretic properties are shown relative to proof reduction and conversion including: syntactic proof preservation and strong proof normalization.
Additionally, in the absence of the $\varphi$ construct, proof reduction upper-bounds object reduction.

A failure of the proof theoretic discipline and victory of the extrinsic philosophy is the $\varphi$ construct.
Unfortunately, the \textsc{Cast} rule does block many desirable properties such as decidability of type checking.
However, it does not prevent logical consistency and the apparent strength it adds to the theory is substantial.
Indeed, it is not clear how to derive any of the interesting encodings published in existing literature without the $\varphi$ construct.
It is the only source of non-termination in the system $\ced$.

This trade-off is the best that can be offered at this moment in time as it is not clear how to systematically correct $\varphi$ without destroying all of its benefits.
To that end, the remaining of this chapter will inform the reader on alternative paths taken with equality in other type theories and other pertinent related work.
Additionally, a brief section of future work is presented to communicate open problems and, in the opinion of the author, promising ideas.

\section{Related Work}




\section{Future Work}

There are three important open questions that this work does not address that should hopefully be answered in the future.
While the first two questions are concerned with the $\varphi$ rule the third notes of a new possibility obtained by making the equality typed in $\ced$.

First, there is a question if the $\varphi$ construct is necessary for efficient inductive data.
Without the $\textsc{Cast}$ rule it is possible to derive inductive Church encodings that are \textit{not} efficient because computing an out (e.g. a predecessor of a natural number) is proportional to the size of the data (e.g. linear time for natural numbers).
One potential method is using Scott encodings instead which have been shown to be inductive in Cedille \cite{jenkins2021monotone}.
Note that the method for showing this in Cedille does leverage the $\varphi$ construct.
However, it seems unlikely that all ink has dried on the topic of impredicative encodings, especially in Cedille.
While an ``escape hatch'' such as $\varphi$ would still be present in the system knowing which encodings do not depend on it would be useful to delineate the relative power between the system with and without $\varphi$.

Second, modifying the $\varphi$ rule is another avenue that requires more investigation.
While restricting the rule with internal derivations seems to be a dead end because any restriction may be satisfied by an inconsistent context there could be lightweight external additions to enable using the \textsc{Cast} rule without caveats.
Another possibility is dropping the $\varphi$ rule and replacing it with weaker variants that are still capable enough to derive existing encodings.
It is, however, not clear just how many special rules would be required.
For example, requiring a special rule for each encoding is tantamount to extending the system with that particular construct anyway.
Thus, why bother with the special rules.

Finally, now that the equality type is typed it is possible that function extensionality is a consistent axiom.
Indeed, because equality is untyped in Cedille the $\delta$ rule is powerful enough to separate two extensionally equal functions, refuting the axiom.
With a typed equality the separation rule of $\ced$ may only act on typed terms.
Therefore, separation is only capable of differentiation functions by applying test inputs.
While this is convincing intuition showing that this axiom is consistent requires an extensional model which, of course, cannot exist for Cedille.

\section{Closing Remarks}

The core system $\ced$ represents a step forward to a proof theoretic version of Cedille.
With this design the equality type is modified to remove all sources of non-termination with the lone exception of the $\varphi$ construct.
Ideally, this construct would also be corrected to disallow creation of non-terminating terms, but there is currently no known method to systematically accomplish this.
A system without $\varphi$ is capable of deriving inductive Church encodings, but many (if not all) of the other encodings published in existing research use $\varphi$ in an essential way.
Nevertheless, there is an external condition that may be checked to ensure a particular usage of $\varphi$ does not introduce non-termination.
Even with this trade-off the system $\ced$ enjoys a better behaved metatheory than Cedille while enabling derivation of the currently existing encodings.
