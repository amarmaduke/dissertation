\begin{publicAbstract}

Language is a medium of expression, both artistic and technical.
Like constrained art, a programming language consists of self-imposed technical restrictions.
These restrictions yield interesting properties and enable a precise communication of ideas.
The programming language Cedille is significantly constrained but with a small grammar and vocabulary that surpasses the expressiveness of similar languages.
This work proposes a refinement to Cedille that imposes more constraints to obtain better properties without sacrificing expressiveness.
Precisely, Cedille's notion of equality is modified and, as a result, several useful properties are proven about the refinement.
Most importantly, however, is that almost all the communicable ideas in Cedille are not lost as a consequence.

\end{publicAbstract}