\documentclass[letterpaper, 11pt]{uiowa-thesis}
\usepackage{graphicx}
\usepackage{textcomp}
\usepackage{lipsum}
\usepackage{mathpartir}
\usepackage{amsmath}
\usepackage{amssymb}
\usepackage{amsthm}
\usepackage{changepage}

\newenvironment{proofcase}{\vspace{0.3cm}\begin{adjustwidth}{2cm}{}}{\end{adjustwidth}\vspace{0.4cm}}

%% Premiliminaries
\newlength\cinfrheight
\settoheight{\cinfrheight}{\textless}
\newcommand\raisedtextbar{\raisebox{0.4ex}{\textbar}}
\newcommand{\cinfrbar}{\resizebox{!}{\cinfrheight}{\raisedtextbar}}
\newcommand{\cinfr}{\kern+3.5pt\cinfrbar\kern-3.5pt\vartriangleright}

\newcommand{\infr}{\vartriangleright}
\newcommand{\chck}{\vartriangleleft}

\newcommand{\abs}[4]{{#1}\, #2\! : \! #3.\, #4}
\newcommand{\absu}[3]{{#1}\, #2.\, #3}
\newcommand{\app}[3]{#1 \bullet_{#2} #3}
\newcommand{\apptwo}[5]{\app{\app{#1}{#2}{#3}}{#4}{#5}}
\newcommand{\appthr}[7]{\apptwo{\app{#1}{#2}{#3}}{#4}{#5}{#6}{#7}}

\newcommand{\betared}{\rightsquigarrow_\beta}
\newcommand{\betastar}{\rightsquigarrow_\beta^*}

\newcommand{\D}[1]{\mathcal{D}_{#1}}
\newcommand{\der}[2]{#2}

\newtheorem{theorem}{Theorem}
\newtheorem{lemma}{Lemma}

\addbibresource{biblio.bib}
\nocite{*}

\title{The Calculus of Set Constructions}
\author{Andrew Marmaduke}
\date{May 2024}

\usepackage{hyperref}
\hypersetup{
  colorlinks,
  linkcolor=[rgb]{0.62,0.22,0.24},
  urlcolor=[rgb]{0.16,0.20,0.36},
  citecolor=[rgb]{0.95,0.65,0.07},
  pdftitle={\thetitle},
  pdfauthor={\theauthor}
}
\degree{Doctor of Philosophy}
\concentration{Computer Science}

\ThesisSupervisor{Aaron Stump}
\CommitteeMember{Cesare Tinelli}
\CommitteeMember{J. Garrett Morris}
\CommitteeMember{Sriram Pemmaraju}
\CommitteeMember{William J. Bowman}

\begin{document}

\titlepage

\copyrightPage

\frontmatter

\iowaEpigraph{\lipsum[1]}{Some wise dude}

\begin{acknowledgments}

Thank you to my advisor, Aaron Stump, for your guidance and feedback throughout my time as a graduate student.
You enabled and nurtured my exploration into Cedille.
Thank you to everyone at the Computational Logic Center for a fun and welcoming environment, especially: Christa, Tony, Alex, and Apoorv.
In the office I have a hard time not bothering people with ideas and discussion, thank you for tolerating it.
Finally, thank you to my partner, Emily, for being here at the start of the journey and all the way to the end.
Your help through stressful times and tribulations was invaluable.
I could not imagine having done it without you.

\end{acknowledgments}


\begin{abstract}
    \lipsum[1-2]
\end{abstract}

\begin{publicAbstract}
    \lipsum[1-2]
\end{publicAbstract}

\tableofcontents

\listoffigures

\listoftables

\preface


\mainmatter

\chapter{Introduction}

%%% System F Omega specific:

\Rule{\PiRuleFOne}
    {
        \der{\D{1}}{\Gamma \vdash A \cinfr \kind}
        \\ \der{\D{2}}{\Gamma, x : A \vdash B \cinfr \kind}
    }
    {\Gamma \vdash (x : A) \to B \infr \kind}
    {Pi1}

\Rule{\PiRuleFTwo}
    {
        \der{\D{1}}{\Gamma \vdash A \cinfr K}
        \\ \der{\D{2}}{\Gamma, x : A \vdash B \cinfr \star}
    }
    {\Gamma \vdash (x : A) \to B \infr \star}
    {Pi2}

\Rule{\PiRuleCC}
    {
        \der{\D{1}}{\Gamma \vdash A \cinfr K_1}
        \\ \der{\D{2}}{\Gamma, x : A \vdash B \cinfr K_2}
    }
    {\Gamma \vdash (x : A) \to B \infr K_2}
    {Pi}

\Rule{\LambdaRuleF}
    {
        \der{\D{1}}{\Gamma \vdash (x:A) \to B \cinfr K}
        \\ \der{\D{2}}{\Gamma, x : A \vdash t \infr B}
    }
    {\Gamma \vdash \abs{\lambda}{x}{A}{t} \infr (x : A) \to B}
    {Lam}

\Rule{\AppRuleF}
    {
        \der{\D{1}}{\Gamma \vdash f \cinfr (x : A) \to B} \\
        \der{\D{2}}{\Gamma \vdash a \chck A}
    }
    {\Gamma \vdash f\ a \infr [x := a]B}
    {App}

%%% Non-specific Macros:

\Rule{\AxiomRule}
    {\der{\D{1}}{\vdash \Gamma}}
    {\Gamma \vdash \star \infr \kind}
    {Axiom}

\Rule{\VarRule}
    {
        \der{\D{1}}{\vdash \Gamma}
        \\ \der{\D{2}}{(x : A) \in \Gamma}
    }
    {\Gamma \vdash x \infr A}
    {Var}


    \Rule{\HeadInferenceRule}
    {
        \der{\D{1}}{\Gamma \vdash t \infr A} \\
        \der{\D{2}}{A \betastar B}
    }
    {\Gamma \vdash t \cinfr B}
    {RedInf}

\Rule{\CheckRule}
    {
        \der{\D{1}}{\Gamma \vdash t \infr A} \\
        \der{\D{2}}{\Gamma \vdash B \cinfr K} \\
        \der{\D{2}}{A \equiv B}
    }  
    {\Gamma \vdash t \chck B}
    {Chk}

\Rule{\ContextEmptyRule}
    {\color{white}{\_}}
    {\vdash \varepsilon}
    {CtxEm}

\Rule{\ContextAppendRule}
    {
        \der{\D{1}}{x \notin \text{FV}(\Gamma)} \\
        \der{\D{2}}{\vdash \Gamma} \\
        \der{\D{3}}{\Gamma \vdash A \cinfr K}
    }
    {\vdash \Gamma, x : A}
    {CtxApp}


\Rule{\BinderReductionOne}
    {
        \der{\D{1}}{t_1 \betared t_1^\prime}
    }
    {\mathfrak{b}(\kappa, x : t_1, t_2) \betared \mathfrak{b}(\kappa, x : t_1^\prime, t_2) }
    {}

\Rule{\BinderReductionTwo}
    {
        \der{\D{1}}{t_2 \betared t_2^\prime}
    }
    {\mathfrak{b}(\kappa, x : t_1, t_2) \betared \mathfrak{b}(\kappa, x : t_1, t_2^\prime) }
    {}

\Rule{\NonbinderReduction}
    {
        \der{\D{1}}{t_i \betared t_i^\prime} \\
        i \in {1, \ldots, \mathfrak{a}(\kappa)}
    }
    {\mathfrak{c}(\kappa, t_1, \ldots t_i, \ldots t_{\mathfrak{a}(\kappa)}) \betared \mathfrak{c}(\kappa, t_1, \ldots t_i^\prime, \ldots t_{\mathfrak{a}(\kappa)}) }
    {}

\Rule{\ContextReductionHeadRule}
    {
        \der{\D{1}}{A \betared A^\prime}
    }
    {\Gamma, x : A \betared \Gamma, x : A^\prime}
    {CtxHeadRed}

\Rule{\ContextReductionTailRule}
    {
        \der{\D{1}}{\Gamma \betared \Gamma^\prime}
    }
    {\Gamma, x : A \betared \Gamma^\prime, x : A}
    {CtxTailRed}

\Rule{\ExampleRule}
    {
        \der{\D{1}}{P_1}
        \\ \ldots
        \\ \der{\D{n}}{P_n}
    }
    {C}
    {Example}

\Rule{\MultiRefl}
    {\color{white}{\_}}
    {t\ R^*\ t}
    {Reflexive}

\Rule{\MultiStep}
    {
        \der{\D{1}}{t\ R\ t^\prime} \\
        \der{\D{2}}{t^\prime\ R^*\ t^{\prime\prime}}
    }
    {t\ R^*\ t^{\prime\prime}}
    {Transitive}


Type theory is a tool for reasoning about assertions of some domain of discourse.
When applied to programming languages, that domain is the expressible programs and their properties.
Of course, a type theory may be rich enough to express detailed properties about a program, such that it halts or returns an even number.
Therein lies a tension between what properties a type theory can faithfully (i.e. consistently) encode and the complexity of the type theory itself.
If the theory is too complex then it may be untenable to prove that the type theory is well-behaved.
Indeed, the design space of type theories is vast, likely infinite.
When incorporating features the designer must balance complexity against capability.

Modern type theory arguably began with Martin-L\"{o}f in the 1970s and 1980s when he introduced a dependent type theory with the philosophical aspirations of being an alternative foundation of mathematics \cite{lof1975,lof1984}.
Soon after in 1985, the Calculus of Constructions (CC) was introduced by Coquand \cite{coquand1985,coquand1986}.
Inductive data (e.g. natural numbers, lists, trees) was shown by Guevers to be impossible to derive in CC \cite{geuvers2001_noind}.
Nevertheless, inductive data was added as an extension by Pfenning \cite{pfenning1989} and the Calculus of Inductive Constructions (CIC) became the basis for the proof assistant Coq \cite{paulin-mohring1993}.

In the early 1990s Barendregt introduced a generalization to Pure Type Systems (PTS) and studied CC under his now famous $\lambda$-cube \cite{barendregt1990_cube,barendregt1991_pts}.
The $\lambda$-cube demonstrated how CC could be deconstructed into four essential sorts of functions.
At its base was the Simply Typed Lambda Calculus (STLC) a type theory introduced in the 1940s by Church to correct logical consistency issues in his (untyped) $\lambda$-calculus \cite{church1940_stlc}.
The STLC has only basic functions found in all programming languages.
System F, a type theory introduced by Girard \cite{girard1972,girard1989} and independently by Reynolds \cite{reynolds1974_systemf}, is obtained from STLC by adding quantification over types (i.e. polymorphic functions).
Adding a copy of STLC at the type-layer, functions from types to types, yields System F$^\omega$.
Finally, the addition of quantification over terms or functions from terms to types, completes CC.
While this is not the only path through the $\lambda$-cube to arrive at CC it is the most well-known and the most immediately relevant.

Perhaps surprisingly, all the systems of the $\lambda$-cube correspond to a logic.
In the 1970s Curry circulated his observations about the STLC corresponding to intuitionistic propositional logic \cite{howard1980}.
Reynolds and Girard's combined work demonstrated that System F corresponds to second-order intuitionistic propositional logic \cite{girard1972,reynolds1974_systemf,reynolds1983}.
Indeed, Barendregt extended the correspondence to all systems in his $\lambda$-cube noting System F$^\omega$ as corresponding to higher-order intuitionistic propositional logic and CC as corresponding to higher-order intuitionistic predicate logic \cite{barendregt1991_pts}.
Fundamentally, the Curry-Howard correspondence associates programs of a type theory with proofs of a logic, and types with formula.

Cedille is a programming language with a core type theory based on CC \cite{stump2017_cdle,stump2021_cedillecore}.
However, Cedille took an alternative road to obtaining inductive data than what was done in the 1980s.
Instead, CC was modified to add the implicit products of Miquel \cite{miquel2001}, the dependent intersections of Kopylov \cite{kopylov2003}, and an equality type over untyped terms.
The initial goal of Cedille was to find an efficient way to encode inductive data.
This was achieved in 2018 with Mendler-style lambda encodings \cite{firsov2018_mendler}. 
However, the design of Cedille sacrificed certain properties such as the decidability of type checking.
Decidability of type checking was stressed by Kreisel to Scott as necessary to reduce proof checking to type checking because a proof does not, under Kreisel's philosophy, diverge \cite{scott1970}.
This puts into contention if Cedille corresponds to a logic at all.
The primary objective of this work is to improve this state of affairs.
However, completing this journey requires a deeper introduction into the type theories of the $\lambda$-cube.

\section{System \texorpdfstring{F$^\omega$}{F Omega}}

The following description of System F$^\omega$ differs from the standard presentation in a few important ways:
\begin{enumerate}
    \item the syntax introduced is of a generic form which makes certain definitions more economical,
    \item a bidirectional PTS style is used but weakening is replaced with a well-formed context relation.
\end{enumerate}
These changes do not affect the set of proofs or formula that are derivable internally in the system.


\begin{figure}
    \centering
    \begin{align*}
        t &::= x\ |\ \mathfrak{b}(\kappa_1, x : t_1, t_2)\ |\ \mathfrak{c}(\kappa_2, t_1, \ldots, t_{\mathfrak{a}(\kappa_2)}) \\
        \kappa_1 &::= \lambda\ |\ \Pi \\
        \kappa_2 &::=\star\ |\ \kind\ |\ \text{app} \\
    \end{align*}
    \vspace{-.6in}
    \begin{minipage}{0.32\textwidth}
        \begin{align*}
            \mathfrak{a}(\star) = \mathfrak{a}(\kind) &= 0 \\
            \mathfrak{a}(\text{app}) &= 2
        \end{align*}
    \end{minipage}%
    \begin{minipage}{0.32\textwidth}
        \begin{align*}
            \abs{\lambda}{x}{t_1}{t_2} &:= \mathfrak{b}(\lambda, x : t_1, t_2) \\
            (x : t_1) \to t_2 &:= \mathfrak{b}(\Pi, x : t_1, t_2) \\
            t_1\ t_2 &:= \mathfrak{c}(\text{app}, t_1, t_2)
        \end{align*}
    \end{minipage}%
    \begin{minipage}{0.32\textwidth}
        \begin{align*}
            \star &:= \mathfrak{c}(\star) \\
            \kind &:= \mathfrak{c}(\kind)
        \end{align*}
    \end{minipage}
    \caption{Syntax for System F$^\omega$.}
    \label{fig:syntax_f}
\end{figure}


Syntax consists of three forms: variables ($x, y, z, \ldots$), binders ($\mathfrak{b}$), and constructors ($\mathfrak{c}$).
Every binder and constructor has an associated discriminate (or tag) to determine the specific syntactic form.
Constructor tags have an associated arity ($\mathfrak{a}$) which determines the number of arguments the specific constructor contains.
A particular syntactic expression will be interchangeably called a syntactic form, a term, or a subterm if it exists inside another term in context.
See Figure~\ref{fig:syntax_f} for the complete syntax of F$^\omega$.
Note that the grammar for the syntax is defined using a BNF-style \cite{floyd1961_bnf} where $t ::= f(t_1, t_2, \ldots)$ represents a recursive definition defining a category of syntax, $t$, by its allowed subterms.
For convenience a shorthand form is defined for each tag to maintain a more familiar appearance with standard syntactic definitions.
Thus, instead of writing $\mathfrak{b}(\lambda, (x : A), t)$ the more common form is used: $\abs{\lambda}{x}{A}{t}$.
Whenever the tag for a particular syntactic form is known the shorthand will always be used instead.


\begin{figure}
    \centering
    \begin{align*}
        FV(x) &= \{x\} \\
        FV(\mathfrak{b}(\kappa_1, x : t_1, t_2)) &= FV(t_1) \cup (FV(t_2) - \{x\}) \\
        FV(\mathfrak{c}(\kappa_2, t_1, \ldots, t_{\mathfrak{a}(\kappa_2)})) &= FV(t_1) \cup \cdots \cup FV(t_{\mathfrak{a}(\kappa_2)})
    \end{align*}
    \begin{align*}
        [y := t]x &= x \\
        [y := t]y &= t \\
        [y := t]\mathfrak{b}(\kappa_1, x : t_1, t_2) &= \mathfrak{b}(\kappa_1, x : [y := t]t_1, [y := t]t_2)  \\
        [y := t]\mathfrak{c}(\kappa_2, t_1, \ldots, t_{\mathfrak{a}(\kappa_2)}) &= \mathfrak{c}(\kappa_2, [y := t]t_1, \ldots, [y := t]t_{\mathfrak{a}(\kappa_2)})
    \end{align*}
    \caption{Operations on syntax for System F$^\omega$, including computing free variables and susbtitution.}
    \label{fig:ops_f}
\end{figure}


Free variables of syntax is defined by a straightforward recursion that collects variables that are not bound in a set.
Likewise, substitution is recursively defined by searching through subterms and replacing the associated free variable with the desired term.
See Figure~\ref{fig:ops_f} for the definitions of substitution and computing free variables.
However, there are issues with variable renaming that must be solved.
A syntactic form is renamed by consistently replacing bound and free variables such that there is no variable capture.
For example, the syntax $\abs{\lambda}{x}{A}{y\ x}$ cannot be renamed to $\abs{\lambda}{y}{A}{y\ y}$ because it captures the free variable $y$ with the binder $\lambda$.
More critically, variable capture changes the meaning of a term.
There are several rigorous ways to solve variable renaming including (non-exhaustively): De Bruijn indices (or levels) \cite{debruijn1972}, locally-nameless representations \cite{chargueraud2012}, nominal sets \cite{pitts2013_nominal}, locally-nameless sets \cite{pitts2023_lns}, etc.
All techniques incorporate some method of representing syntax uniquely with respect to renaming.
For this work the variable bureaucracy will be dispensed with.
It will be assumed that renaming is implicitly applied whenever necessary to maintain the meaning of a term.
For example, $\abs{\lambda}{x}{A}{y\ x} = \abs{\lambda}{z}{A}{y\ z}$ and the substitution $[x := t]\abs{\lambda}{x}{A}{y\ x}$ unfolds to $\abs{\lambda}{x}{[x := t]A}{[z := t](y\ x)}$.


\begin{figure}
    \centering
    \begin{minipage}{0.5\textwidth}
        $$\BinderOneReduction$$
    \end{minipage}%
    \begin{minipage}{0.5\textwidth}
        $$\BinderTwoReduction$$
    \end{minipage}%
    $$\ConstructorReduction$$
    \begin{align*}
        (\abs{\lambda}{x}{A}{b})\ t &\betared [x := t]b
    \end{align*}
    \caption{Reduction rules for System F$^\omega$.}
    \label{fig:reduction_f}
\end{figure}




The syntax of F$^\omega$ has a well understood notion of reduction (or dynamics, or computation) defined in Figure~\ref{fig:reduction_f}.
This is an \textit{inductive} definition of a two-argument relation on terms.
A given rule of the definition is represented by a collection of premises ($P_1, \ldots, P_n$) written above the horizontal line and a conclusion ($C$) written below the line.
An optional name for the rule (\textsc{Example}) appears to the right of the horizontal line.
An inductive definition induces a structural induction principle allowing reasoning by cases on the rules and applying the induction hypothesis on the premises.
During inductive proofs it is convenient to name the derivation of a premise ($\D{1}, \ldots, \D{n}$).
Moreover, to minimize clutter during proofs the name of the rule is removed.

$$\ExampleRule \quad\quad \ExampleRule[*]$$

Inductive definitions build a finite tree of rule applications concluding with axioms (or leaves).
Axioms are written without premises and optionally include the horizontal line.
The reduction relation for F$^\omega$ consists of three rules and one axiom.
Relations defined in this manner are always the \textit{least} relation that satisfies the definition.
In other words, any related terms must have a corresponding inductive tree witnessing the relation.

The reduction relation (or step relation) models function application anywhere in a term via its axiom, called the $\beta$-rule.
This relation is antisymmetric.
There is a \textit{source} term $s$ and a \textit{target} term $t$, $s \betared t$, where $t$ is the result of one function evaluation in $s$.
Alternatively, $s \betared t$ is read as $s$ \textit{steps} to $t$.
Note that if there is no $\lambda$-term applied to an argument (i.e. no function ready to be evaluated) for a given term $t$ then that term cannot be the source term in the reduction relation.
A term that cannot be a source is called a \textit{value}.
If there exists some sequence of terms related by reduction that end with a value, then all source terms in the sequence are \textit{normalizing}.
If \textit{all} possible sequences of related terms end with a value for a particular source term $s$, then $s$ is \textit{strongly normalizing}.
Restricting the set of terms to a normalizing subset is critical to achieve decidability of the reduction relation.


\begin{figure}
    \centering
    \begin{minipage}{0.5\textwidth}
        $$\MultiRefl$$
    \end{minipage}%
    \begin{minipage}{0.5\textwidth}
        $$\MultiStep$$
    \end{minipage}%
    \caption{Reflexive-transitive closure of a relation $R$.}
    \label{fig:multi}
\end{figure}




For any relation $-R-$, the reflexive-transitive closure ($-R^*-$) is inductively defined with two rules as shown in Figure~\ref{fig:multi}.
In the case of the step relation the reflexive-transitive closure, $s \betastar t$, is called the \textit{multistep relation}.
Additionally, when $s \betastar t$ then $s$ \textit{multisteps} to $t$.
It is easy to show that any reflexive-transitive closure is itself transitive.

\begin{lemma}
    Let $R$ be a relation on a set $A$ and let $a, b, c \in A$. If $a\ R^*\ b$ and $b\ R^*\ c$ then $a\ R^*\ c$
\end{lemma}
\begin{proof}
    By induction on $a\ R^*\ b$.

    $\text{Case: }\begin{array}{c} \MultiRefl[*] \end{array}$
    \begin{proofcase}
        It must be the case that $a = b$.
    \end{proofcase}

    $\text{Case: }\begin{array}{c} \MultiStep[*] \end{array}$
    \begin{proofcase}
        Let $z = t^\prime$, then we have $a\ R\ z$ and $z\ R^*\ b$.
        By the inductive hypothesis (IH) we have $z\ R^*\ c$ and by the transitive rule we have $a\ R^*\ c$ as desired.
    \end{proofcase}
\end{proof}

Two terms are \textit{convertible}, written $t_1 \betaconv t_2$, if $\exists\ t^\prime$ such that $t_1 \betastar t^\prime$ and $t_2 \betastar t^\prime$.
Note that this is not the only way to define convertibility in a type theory, but it is the standard method for a PTS.
Convertibility is used in the typing rules to maintain a well-typed relation after term reduction.
It may be tempting to view conversion as the reflexive-symmetric-transitive closure of the step relation, but transitivity is not an obvious property.
In fact, proving transitivity of conversion is often a significant effort, beginning with the confluence lemma.

\begin{lemma}[Confluence]
    If $s \betastar t_1$ and $s \betastar t_2$ then $\exists\ t^\prime$ such that $t_1 \betastar t^\prime$ and $t_2 \betastar t^\prime$
\end{lemma}
\begin{proof}
    See Lemma~\ref{lem:2:confluence} for a proof of confluence involving a larger reduction relation.
    Note that F$^\omega$'s step relation is a subset of this relation and thus is confluent.
\end{proof}

\begin{theorem}[Transitivity of Conversion]
    \label{thm:1:trans}
    If $a \betaconv b$ and $b \betaconv c$ then $a \betaconv c$
\end{theorem}
\begin{proof}
    By premises we know $\exists u, v$ such that $a \betastar u$, $b \betastar u$, $b \betastar v$, and $c \betastar v$.
    By confluence, $\exists z$ such that $u \betastar z$ and $v \betastar z$.
    By transitivity of multistep reduction, $a \betastar z$ and $c \betastar z$.
    Therefore, $a \betaconv c$.
\end{proof}


\begin{figure}
    \centering
    \begin{minipage}{0.5\textwidth}
        $$\HeadInferenceRule$$
        $$\ContextEmptyRule$$
        $$\AxiomRuleF$$
        $$\PiRuleFOne$$
        $$\LambdaRuleF$$
    \end{minipage}%
    \begin{minipage}{0.5\textwidth}
        $$\CheckRule$$
        $$\ContextAppendRuleF$$
        $$\VarRuleF$$
        $$\PiRuleFTwo$$
        $$\AppRuleF$$
    \end{minipage}%
    \caption{
        Typing rules for System F$^\omega$. The variable $K$ is a metavariable representing either $\star$ or $\kind$.
    }
    \label{fig:typing_f}
\end{figure}


Figure~\ref{fig:typing_f} defines the typing relation on terms for F$^\omega$.
As previously mentioned this formulation is different from standard presentations.
Four relations are defined mutually:
\begin{enumerate}
    \item $\Gamma \vdash t \infr T$, to be read as $T$ is the inferred type of the term $t$ in the context $\Gamma$ or, $t$ infers $T$ in $\Gamma$;
    \item $\Gamma \vdash t \cinfr T$, to be read as $T$ is the inferred type, possibly after some reduction, of the term $t$ in the context $\Gamma$ or, $t$ reduction-infers $T$ in $\Gamma$;
    \item $\Gamma \vdash t \chck T$, to be read as $T$ is checked against the inferred type of the term $t$ in the context $\Gamma$ or, $t$ checks against $T$ in $\Gamma$;
    \item $\vdash \Gamma$, to be read as the context $\Gamma$ is well-formed, and thus consists only of types that themselves have a type
\end{enumerate}
Note that there are two \textsc{Pi} rules that restrict the domain and codomain pairs of function types to three possibilities: $(\kind, \kind)$, $(\star, \star)$, and $(\kind, \star)$.
This is exactly what is required by the $\lambda$-cube for this definition to be F$^\omega$.
For the unfamiliar, interpreting rules can be difficult, thus exposition explaining a small selected set is provided.

$\AxiomRuleF$ The axiom rule has one premise, requiring that the context is well-formed.
It concludes that the constant term $\star$ has type $\kind$.
Intuitively, the term $\star$ should be viewed as a universe of types, or a type of types, often referred to as a \textit{kind}.
Likewise, the term $\kind$ should be viewed as a universe of kinds, or a kind of kinds.
An alternative idea would be to change the conclusion to $\Gamma \vdash \star \infr \star$.
This is called the \textit{type-in-type} rule, and it causes the type theory to be inconsistent \cite{girard1972,hurkens1995}.
Note that there is no way to determine a type for $\kind$.
It plays the role of a type only.

$\VarRuleF$ The variable rule is a context lookup.
It scans the context to determine if the variable is anywhere in context and then the associated type is what that variable infers.
This rule is what requires the typing relation to mention a context.
Whenever a type is inferred or checked it is always desired that the context is well-formed.
That is why the variable rule also requires the context to be well-formed as a premise, because it is a leaf relative to the inference relation.
Without this additional premise there could be typed terms in ill-formed contexts.

$\AppRuleF$ The application rule infers the type of the term $f$ and reduces that type until it looks like a function-type.
Once a function type is acquired it is clear that the type of the term $a$ must match the function-type's argument-type.
Thus, $a$ is checked against the type $A$.
Finally, the inferred result of the application is the codomain of the function-type $B$ with the term $a$ substituted for any free occurrences of $x$ in $B$.
This substitution is necessary because this application could be a type application to a type function.
For example, let $f = \abs{\lambda}{X}{\star}{\text{id}\ X}$ where id is the identity term.
The inferred type of $f$ is then $(X : \star) \to X \to X$.
Let $a = \mathbb{N}$ (any type constant), then $f\ \mathbb{N} \infr [X := \mathbb{N}](X \to X)$ or $f\ \mathbb{N} \infr \mathbb{N} \to \mathbb{N}$.

While this presentation of F$^\omega$ is not standard Lennon-Bertrand demonstrated that it is equivalent to the standard formulation \cite{lennon2021}.
In fact, Lennon-Bertrand showed that a similar formulation is logically equivalent for the stronger CIC.
Thus, standard metatheoretical results such as preservation and strong normalization still hold.

\begin{lemma}[Preservation of F$^\omega$]
    If $\Gamma \vdash s \chck T$ and $s \betastar t$ then $\Gamma \vdash t \chck T$
\end{lemma}
\begin{proof}
    See Theorem~\ref{lem:2:preservation} for an example proof of preservation.
    The proof for F$^\omega$ is very similar.
\end{proof}

\begin{theorem}[Strong Normalization of F$^\omega$]
    If $\Gamma \vdash t \infr T$ then $t$ and $T$ are strongly normalizing
\end{theorem}
\begin{proof}
    System F$^\omega$ is a subsystem of CC which has several proofs of strong normalization.
    See (non-exhaustively) proofs using saturated sets \cite{geuvers1994_sn_satset}, model theory \cite{terlouw1995_sn}, realizability \cite{ong1993}, etc.
\end{proof}

With strong normalization the convertibility relation is decidable, and moreover, type checking is decidable.
Let \textit{red} be a function that reduces its input until it is either $\star$, $\kind$, a binder, or in normal form.
Note that this function is defined easily by applying the outermost reduction and matching on the resulting term.
Let \textit{conv} test the convertibility of two terms.
Note that this function may be defined by reducing both terms to normal forms and comparing them for syntactic identity.
Both functions are well-defined because F$^\omega$ is strongly normalizing.
Then the functions \textit{infer}, \textit{check}, and \textit{wf} can be mutually defined by following the typing rules.
Thus, type inference and type checking are decidable for F$^\omega$.

While it is true that F$^\omega$ only has function types as primitives several other data types are internally derivable using function types.
For example, the type of natural numbers is defined:
$$\mathbb{N} = (X : \star) \to X \to (X \to X) \to X$$
Likewise, pairs and sum types are defined:
$$A \times B = (X : \star) \to (A \to B \to X) \to X$$
$$A + B = (X : \star) \to (A \to X) \to (B \to X) \to X$$
The logical constants true and false are defined:
$$\top = (X : \star) \to X \to X$$
$$\bot = (X : \star) \to X$$
Negation is defined as implying false:
$$\neg A = A \to \bot$$
These definitions are called \textit{Church encodings} and originate from Church's initial encodings of data in the $\lambda$-calculus \cite{church1932,church1933}.
Note that if there existed a term such that $\vdash t \chck \bot$ then trivially for \textit{any} type $T$: $\vdash t\ T \chck T$.
Thus, $\bot$ is both the constant false and the proposition representing the principle of explosion from logic.
Moreover, this allows a concise statement of the consistency of F$^\omega$.

\begin{theorem}[Consistency of System F$^\omega$]
    There is no term $t$ such that $\vdash t \chck \bot$
\end{theorem}
\begin{proof}
    Suppose $\vdash t \chck \bot$.
    Let $n$ be the value of $t$ after it is normalized.
    By preservation $\vdash n \chck \bot$.
    Deconstructing the checking judgment we know that $\vdash n \infr T$ and $T \betaconv \bot$, but $\bot$ is a value and values like $n$ infer types that are also values.
    Thus, $T = \bot$ and we know that $\vdash n \infr \bot$.
    By inversion on the typing rules $n = \abs{\lambda}{X}{\star}{b}$, and we have $X : \star \vdash b \infr X$.
    The term $b$ can only be $\star$, $\kind$, or $X$, but none of these options infer type $X$.
    Therefore, there does not exist a term $b$, nor a term $n$, nor a term $t$.
\end{proof}

Recall that induction principles cannot be derived internally for any encoding of data in CC \cite{geuvers2001_noind}.
This is not only cumbersome but unsatisfactory as the natural numbers are in their essence the least set satisfying induction.
Ultimately, the issue is that these encodings are too general.
They admit theoretical elements that F$^\omega$ is not flexible enough to express nor strong enough to exclude.

\section{Calculus of Constructions and Cedille}

As previously mentioned, CC is one extension away from F$^\omega$ on the $\lambda$-cube.
Indeed, the two rules \textsc{Pi1} and \textsc{Pi2} can be merged to form CC:
$$\PiRuleCC$$
where now both $K_1$ and $K_2$ are metavariables representing either $\star$ or $\kind$.
No other rules, syntax, or reductions need to be changed.
Replacing \textsc{Pi1} and \textsc{Pi2} with this new \textsc{Pi} rule is enough to obtain a complete and faithful definition of CC.

With this merger types are allowed to depend on terms.
From a logical point of view, this is a quantification over terms in formula.
Hence, CC is a predicate logic instead of a propositional one according to the Curry-Howard correspondence.
Yet, there is a question about what exactly quantification over terms means.
Surely it does not mean quantification over syntactic forms.

It means, at minimum, quantification over well-typed terms, but from a logical perspective these terms correspond to proofs.
In first order predicate logic the domain of quantification ranges over a set of \textit{individuals}.
The set of individuals represents any potential set of interest with specific individuals identified through predicates expressing their properties.
With proofs the situation is different.
A proof has meaning relative to its formula, but this meaning may not be relevant as an individual in predicate logic.
For example, the proof $2$ for a Church encoded natural number is intuitively data, but a proof that $2$ is even is intuitively not.
In CC, both are merely proofs that can be quantified over.

Cedille alters the domain of quantification from proofs to (untyped) $\lambda$-calculus terms.
Thus, for Cedille, the proof $2$ becomes the encoding of $2$ and the proof that $2$ is even can \textit{also} be the encoding of $2$.
This is achieved through a notion of \textit{erasure} which removes type information and auxiliary syntactic forms from a term.
Additionally, convertibility is modified to be convertibility of untyped $\lambda$-calculus terms.
However, erasure as it is defined in Cedille enables diverging terms in inconsistent contexts.
The result by Abel and Coquand, which applies to a wide range of type theories including Cedille, is one way to construct a diverging term \cite{abel2020_normalization}.

If terms are able to diverge, in what sense are they a proof?
What a proof is or is not is difficult to say.
As early as Aristotle there are documented forms of argument, Aristotle's syllogisms \cite{aristotle}.
More than a millennium later Euclid's \textit{Elements} is the most well-known example of a mathematical text containing what a modern audience would call proofs.
Moreover, visual renditions of \textit{Elements}, initiated by Byrne, challenge the notion of a proof being an algebraic object \cite{byrne}.
The study of proof as a mathematical object dates first to Frege \cite{frege1879} followed soon after by Peano's formalism of arithmetic \cite{peano1889} and Whitehead and Russell's \textit{Principia Mathematica} \cite{whitehead}.
For the kinds of logics discussed by the Curry-Howard correspondence, structural proof theories, the originator is Gentzen \cite{gentzen1935_i,gentzen1935_ii}.
Gentzen's natural deduction describes proofs as finite trees labelled by rules.
Note that this is, of course, a very brief history of mathematical proof.

All of these formulations may be acceptable notions of proof, but the purpose of proof from an epistemological perspective is to provide justification.
It is unsatisfactory to have a claimed proof and be unable to check that it is constructed only by the rules of the proof theory.
This is the situation with Cedille.
Although rare, there are terms where reduction diverges making it impossible to check a type.
However, it is unfair to levy this criticism against Cedille alone, as well-known type theories also lack decidability of type checking.
For example, Nuprl with its equality reflection rule \cite{allen2000}, and the proof assistant Lean with its notion of casts \cite{moura2021}.
Moreover, Lean has been incredibly successful in formalizing research mathematics including the Liquid Tensor Experiment \cite{liquid_tensor_experiment} and Tao's formalization of The Polynomial Freiman-Ruzsa Conjecture \cite{tao2024_pfr}.
Indeed, not having decidability of type checking does not necessarily prevent a tool from producing convincing arguments.
Ultimately, the definition of proof is a philosophical one with no absolute answer, but this work will follow Gentzen and Kreisel in requiring that a proof is a finite tree, labelled by rules, supporting decidable proof checking.
Under such a definition, it can be claimed that a derivation in Cedille is a proof \textit{relative} to an oracle that decides convertibility.
However, one should strive for elimination of these external oracle conditions if possible.

The major modifications that will be made to Cedille all involve its equality type.
Thus, a brief introduction and history of equality in type theories is beneficial to understanding the design space and alternative solutions.

\section{Equality}


\Rule{\LofEq}
    {
        \der{\D{1}}{\Gamma \vdash A : \star}
        \\ \der{\D{2}}{\Gamma \vdash x : A}
        \\ \der{\D{2}}{\Gamma \vdash y : A}
    }
    {\Gamma \vdash x =_A y : \star}
    {}

\Rule{\LofRefl}
    {
        \der{\D{1}}{\Gamma \vdash A : \star}
        \\ \der{\D{2}}{\Gamma \vdash x : A}
    }
    {\Gamma \vdash \prefl(x; A) : x =_A x}
    {}

\Rule{\LofJ}
    {
        \der{\D{1}}{\Gamma \vdash A : \star}
        \\ \der{\D{2}}{\Gamma \vdash x : A}
        \\ \der{\D{3}}{\Gamma \vdash y : A}
        \\ \der{\D{4}}{\Gamma \vdash e : x =_A y}
        \\ \der{\D{5}}{\Gamma \vdash P : (x:A) \to (y:A) \to (e: x =_A y) \to \star}
        \\ \der{\D{6}}{\Gamma \vdash w : (i:A) \to P\ i\ i\ \prefl(i;A)}
    }
    {\Gamma \vdash J(x, y, e, w; A, P) : P\ x\ y\ e}
    {}

\begin{figure}
    \centering
    \begin{minipage}{0.5\textwidth}
        $$\LofEq$$
    \end{minipage}%
    \begin{minipage}{0.5\textwidth}
        $$\LofRefl$$
    \end{minipage}%
    $$\LofJ$$
    \caption{
        Typing rules for the Martin-L\"{o}f intensional propositional equality type.
    }
    \label{fig:lof_eq}
\end{figure}



While a Leibniz Law may be stated in CC using a Church encoding it does not enable strong reasoning principles.
The standard definition of equality used to extend systems like CC is Martin-L\"{o}f's identity type, depicted in Figure~\ref{fig:lof_eq}.
This formulation comes with a so-called inductive eliminator: the $J$ rule.
However, the identity type does not admit function extensionality, which is formally defined below.
$$(f\ g : A \to B) \to ((x : A) \to f\ x =_B g\ x) \to f =_{A \to B} g$$
Function extensionality is a commonly presumed reasoning principle in mathematics and many successful mathematical formulations assume it as an additional axiom.
Defining systems where function extensionality is derivable instead of axiomatically postulated was (and still is) an active area of research.
Note that the Church encoded Leibniz Law is logically equivalent to the identity type.
Moreover, they are isomorphic if quantification is parametric and function extensionality holds \cite{abel2020_equality}.

Attempting to bridge the gap between intensional and extensional features Streicher proposed his Axiom K in 1993 \cite{streicher1993}.
Today this is known as Uniqueness of Identity Proofs (UIP), formally defined below.
$$(x\ y : A) \to (p\ q : x =_A y) \to p =_{x =_A y} q$$
Initially, it was believed that UIP should be a consequence of Martin-L\"{o}f's rules for the identity type because there is only one value.
However, a proof of UIP remained elusive.
In 1995 Hofmann answered this equation negatively: UIP is independent of Martin-L\"{o}f's identity type.
Hofmann accomplished this by modelling identity types in two separate ways: as equivalence relations defined inductively on type structure and as groupoids.
His models solved a long-standing open question about the nature of the identity type \cite{hofmann1995, hofmann1996}.

Propositional extensionality and quotients are two additional notions that add more mathematically intuitive reasoning principles to a system that depend on equality.
Propositional extensionality states that logical equivalence of propositions is logically equivalent to equality of propositions.
This principle is stated relative to some universe of propositions, \textsc{Prop}, and defined formally below.
$$(P\ Q : \textsc{Prop}) \to (P \leftrightarrow Q) \leftrightarrow (P =_\textsc{Prop} Q)$$
In extensions of CC this universe is $\star$, but in systems like Cedille there is no single universe of propositions.
Quotient types ($A/\!\!\sim$) are constructed from a carrier type $A$ and an equivalence relation $\sim$ such that equality for elements of the quotient respects $\sim$.
The simplest example quotient is the set of rational numbers which is the quotient of fractions (i.e. pairs of integers) and the equivalence relation $((n_1, d_1)\ (n_2, d_2) : \mathbb{Z \times Z}) \to n_1d_2 =_{\mathbb{Z}} n_2d_1$.
Additionally, other algebraic objects are constructed from quotients in Set Theory (e.g. groups, fields, modules).
Indeed, the Lean mathlib library heavily relies on quotients \cite{mathlib}.

The equivalence relation for a quotient induces a partition of the elements of that type into equivalence classes.
If a canonical representative of an equivalence class can be effectively computed then a quotient is called \textit{definable}.
The real numbers and multi-sets of unorderable elements are two examples of quotient types where a canonical representative is uncomputable \cite{li2015}.
Indeed, the axiom of choice is equivalent to the ability to pick a unique representative from equivalence classes for any arbitrary equivalence relation \cite{lof2009}.
Additionally, if a type theory supports an impredicative universe of types such as Cedille, then adding quotients such as the real numbers causes inconsistency \cite{chicli2002}.

Equality is a critical part of the design of many type theories, Cedille included.
However, the powerful extensional reasoning principles of equality are surprisingly difficult to obtain without causing other issues.
A brief history of type theories that capture some extensionality is useful in understanding what can and cannot be accomplished.

Extensional Type Theory (ETT) enjoys all the aforementioned reasoning principles.
The distinguishing feature is the addition of the equality reflection rule, allowing promotion of propositional equalities to definitional (i.e. automatic) equalities.
However, propositional equality is almost always undecidable, and thus definitional equality becomes undecidable as a consequence.
It is difficult to pin an exact year when equality reflection is first introduced, but some of Martin-L\"{o}f's early systems are extensional type theories and likewise some of the earliest proof assistant implementations were extensional type theories (e.g. Nuprl) \cite{constable1986}.
In 2016 Andrej Bauer proposed a method for designing equality reflection around effect handlers.
Thus, the undecidability of equality reflection is solved by a provided handler which resolves the proof obligation \cite{bauer2016}.
Rewriting can be added to type theory in a multitude of different ways to achieve ad hoc equality reflection.
Cockx investigated a Rewriting Type Theory in detail where rewriting allows the addition of computational axioms.
He notes that rewriting can encode extensional principles like quotients \cite{cockx2020, cockx2021}.
It has also been shown that ETT can be modelled in a type theory without equality reflection as long as the axioms UIP and function extensionality are postulated \cite{winterhalter2019}.
In fact, this result was strengthened to modelling ETT in ``weak'' theories where definitional equality is $\alpha$-equivalence and reduction is pushed into the propositional equality \cite{boulier2019}.

Observational Type Theory (OTT) was introduced in 2007 by Altenkirch and McBride.
The core idea behind OTT is to define propositional equality by recursion on type constructors.
This recursive definition grants greater flexibility in how equality is treated for individual type formers.
Thus, equality of function types may be defined to be exactly function extensionality \cite{altenkirch2007}.
In 2022, Pujet et al. introduced an improvement that resolved limitations in the original formulation of OTT.
The authors note that a critical difference from prior attempts at OTT is that propositions satisfy definitional proof-irrelevance preventing proof obligations muddying goals of judgments.
Moreover, many desirable properties are proven including: propositional extensionality, UIP, strong normalization, consistency, decidability of type checking, and quotients provided the equivalence relation is proof-irrelevant \cite{pujet2022}.

Setoid Type Theory (SeTT) is an alternative path taken by Altenkirch after work on the original OTT.
Using setoids internally to reason mathematically is a standard (and often heavily disliked) method of reasoning with extensional principles in a system that lacks them.
Altenkirch's idea is to construct a setoid model directly in an intensional theory while making the bureaucracy of working with setoids automatic.
The model is a syntactic translation which gives a way to bootstrap extensionality principles from intensional theories.
Initial models where strong enough to support function extensionality and propositional extensionality \cite{altenkirch2019}.
Later, the models where improved to internalize a universe of setoids \cite{altenkirch2021}.

In 2006, the mathematician Voevodsky began studying type theory as an alternative foundation for mathematics after expressing doubts about the correctness of results in the mathematical literature.
He proposed the Univalence Axiom (UA), shown below, as a desirable feature of type theory.
In his opinion, it accurately modelled the transport of properties between objects that mathematicians take for granted \cite{voevodsky2006}.
Upon closer inspection of UA it becomes clear that it is a generalization of propositional extensionality.
$$(A\ B : \textsc{Type}) \to (A \simeq B) \simeq (A =_\textsc{Type} B)$$

Homotopy Type Theory (HoTT) is one of the first theories devised satisfying UA.
HoTT interprets the identity type as homotopy equivalences and has been used effectively to build a foundation of mathematics while working synthetically in the field of Homotopy Theory \cite{hottbook}.
However, HoTT does not give an internal derivation of UA, thus the computational property, canonicity, is lost.
The search for computational models of UA began with a type theory in the category of simplicial sets, but this was noted to be problematic because of the classical metatheory \cite{kapulkin2012}.
Moreover, Bezem demonstrated that Voevodsky's simplicial sets model is \textit{necessarily} classical \cite{bezem2015}.
Later, Bezem devised a model using constructive metatheory with cubical sets initiating the possibility for a type theory that derives UA \cite{bezem2014}.

Cubical Type Theory (CTT) provided a computational interpretation of UA by introducing a fundamental interval pretype.
Cohen et al. devised a variation of CTT that they claim simplified semantic justifications using a De Morgan algebra for the interval \cite{cohen2016}.
Indeed, the advent of cubical sets as a model of UA introduced not one CTT, but several variants.
In 2018, Pitts et al. investigated a minimal set of axioms for modeling CTTs \cite{pitts2018}.
Two years later Cavallo et al. observed that the minimal set could be further simplified while still achieving the goals of Pitts \cite{cavallo2020}.
The work of Cavallo et al. arguably presents the essence of CTT:
\begin{enumerate}
    \item {
        the interval must be connected, which prevents a discretization and maintains an internal continuity for smooth deformations;
    }
    \item {
        the two endpoints must be distinct, which prevents collapsing the interval to a unit type and obliterating the internal structure;
    }
    \item {
        and a description of face formulas that encode a simple universe of propositions that allow distinguishing endpoints and disjunctive combination.
    }
\end{enumerate}
CTT has been effective in incorporating extensional features into type theory.
For instance, quotient types are representable with an additional truncation rule \cite{kraus2020}.
Moreover, a new variant of inductive types named \textit{higher inductive types} generalizes inductive quotients \cite{angiuli2017}.
Cubical techniques are also useful in constructing setoid type theories that support definitional UIP \cite{sterling2020}.
Finally, cubical type theory has recently been implemented as Cubical Agda which extends the development with records, coinductive types, and dependent pattern matching on higher inductive types \cite{vezzosi2021}.

Type systems satisfying UA can have some undesirable side effects from a programmatic perspective.
For instance, Hofmann noticed early on that in a groupoid model of types it can be the case that $\mathbb{N} = \mathbb{Z}$.
Altenkirch enforces this observation by noting that any construction in CTT is necessarily stable under homotopy equivalence \cite{altenkirch2016}.
Indeed, Voevodsky's early proposals contain two separate propositional equalities: one to capture isomorphism and one to capture strict equalities \cite{voevodsky2013}.

Voevodksy's idea was explored further in Two-Level Type Theory (2LTT).
The core idea behind 2LTT is to have two universes of types, an ``inner'' universe satisfying UA, and an ``outer'' universe satisfying UIP.
This setup can be viewed as an internalization of the ``inner'' theories metatheory as the ``outer'' theory.
The two theories communicate via the dependent function and dependent pair types which agree between both ``inner'' and ``outer'' systems \cite{annenkov2017}.
Capriotti expanded upon the foundational aspects of 2LTT showing a conservativity result with respect to HoTT confirming that including the ``outer'' theory does not break any internal results constructed in the ``inner'' theory \cite{capriotti2017}.
Angiuli adapted 2LTT to incorporate a strict equality and justify a computational semantics in the same spirit as the one used for Nuprl \cite{angiuli2019}.
However, Gilbert noted that if a strict equality is included in a universe of strict propositions then UA is no longer compatible hinting at limitations \cite{gilbert2019}.

Over the course of type theory history many researchers have sought theories that further enable extensional principles of reasoning.
However, function extensionality is refuted by Cedille.
Indeed, because Cedille's equality is untyped it is possible to distinguish functions with tests outside their typed domain.
Thus, there is little hope of modifying Cedille in its current state with existing techniques to obtain extensional reasoning principles.

\section{Thesis}

Cedille is a powerful type theory capable of deriving inductive data with relatively modest extension and modification to CC.
However, this capability comes at the cost of a poorly behaved metatheory.
A redesign of Cedille that focuses on maintaining a proof-theoretic view could improve this state of affairs and shorten the gap between Cedille and existing type systems.
However, an improvement must balance capability and complexity, maintaining the power of the current Cedille.
The redesign described herein treads this balance to obtain a better metatheory with minimal sacrifice of existing encodings.

\section{Contributions}

\textbf{Chapter 2}\quad defines the core theory ($\ced$), including its syntax, and typing rules.
Erasure from Cedille is rephrased as a projection from proofs to objects.
Basic metatheoretical results are proven including: confluence, preservation, and classification.
Important derivations including irrelevance of casts and construction of a view type are presented, demonstrating the core data used to construct almost all encodings is still possible.
\\ \\
\textbf{Chapter 3}\quad models $\ced$ in F$^\omega$ obtaining a strong normalization result for proof reduction.
This model is a straightforward extension of a similar model for CC.
Critically, proof normalization is not powerful enough to show consistency nor object normalization.
\\ \\
\textbf{Chapter 4}\quad models $\ced$ in CDLE obtaining consistency for $\ced$.
Although CDLE is not strongly normalizing it still possess a realizability model which justifies its logical consistency.
$\ced$ is closely related to CDLE which makes this model straightforward to define.
Additionally, a collection of counterexamples to normalization in CDLE is presented.
\\ \\
\textbf{Chapter 5}\quad proves object normalization from proof normalization and consistency for proofs that do not use the \textsc{Cast} rule.
Unfortunately, the full theory does not enjoy object normalization.
Nevertheless, a condition is formulated that indicates when the usage of a \textsc{Cast} rule does not introduce non-termination.
Thus, the full system of $\ced$ has decidable type checking \textit{relative} to an oracle that decides this condition.
\\ \\
\textbf{Chapter 6}\quad concludes with comments on future work.
An open conjecture remains that $\ced$ may be consistently extended with axiom for function extensionality.
Proving this conjecture is beyond the scope of this work as it requires a novel extensional model.

\chapter{Theory Description and Basic Metatheory}

\Rule{\AxiomRule}
    {\textcolor{white}{\_}}
    {\Gamma \vdash \star : \kind}
    {Axiom}

\Rule{\VarRule}
    {
        \der{\D{1}}{{x \notin FV(\Gamma_1; \Gamma_2)}} \\
        \der{\D{2}}{\Gamma_1 \vdash A : K}
    }
    {\Gamma_1; x_m : A; \Gamma_2 \vdash x_K : A}
    {Var}

\Rule{\WeakenRule}
    {
        \der{\D{1}}{x \notin FV(\Gamma)} \\
        \der{\D{2}}{\Gamma \vdash t : A} \\
        \der{\D{3}}{\Gamma \vdash B : K}
    }
    {\Gamma, x : B \vdash t : A}
    {Weaken}

\Rule{\ConvRule}
    {
        \der{\D{1}}{\Gamma \vdash A : K} \\
        \der{\D{2}}{\Gamma \vdash t : B} \\
        \der{\D{3}}{A \equiv B}
    }
    {\Gamma \vdash t : A}
    {Conv}

\Rule{\PiRule}
    {
        \der{\D{1}}{\Gamma \vdash A : \pdom(m, K)}
        \\ \der{\D{2}}{\Gamma; x_m : A \vdash B : \pcodom(m)}
    }
    {\Gamma \vdash (x : A) \to_m B : \pcodom(m)}
    {Pi}

\Rule{\LambdaRule}
    {
        \der{\D{1}}{\Gamma \vdash (x : A) \to_m B : \pcodom(m)}
        \\ \der{\D{2}}{\Gamma; x_m : A \vdash t : B}
        \\ \der{\D{3}}{x \notin FV(|t|)\text{ if } m = 0}
    }
    {\Gamma \vdash \abs{\lambda_m}{x}{A}{t} : (x : A) \to_m B}
    {Lam}

\Rule{\AppRule}
    {
        \der{\D{1}}{\Gamma \vdash f : (x : A) \to_m B} \\
        \der{\D{2}}{\Gamma \vdash a : A}
    }
    {\Gamma \vdash \app{f}{m}{a} : [x := a]B}
    {App}

\Rule{\IntersectionRule}
    {
        \der{\D{1}}{\Gamma \vdash A : \star} \\
        \der{\D{2}}{\Gamma; x_\tau : A \vdash B : \star}
    }
    {\Gamma \vdash (x : A) \cap B : \star}
    {Int}

\Rule{\PairRule}
    {
        \der{\D{1}}{\Gamma \vdash (x : A) \cap B : \star} \\
        \der{\D{2}}{\Gamma \vdash t : A} \\
        \der{\D{3}}{\Gamma \vdash s : [x := t]B} \\
        \der{\D{4}}{t \equiv s}
    }
    {\Gamma \vdash [t, s; (x : A) \cap B] : (x : A) \cap B }
    {Pair}

\Rule{\FirstRule}
    {\der{\D{1}}{\Gamma \vdash t : (x:A) \cap B}}
    {\Gamma \vdash t.1 : A}
    {Fst}

\Rule{\SecondRule}
    {\der{\D{1}}{\Gamma \vdash t : (x:A) \cap B}}
    {\Gamma \vdash t.2 : [x := t.1]B}
    {Snd}

\Rule{\EqualityRule}
    {
        \der{\D{1}}{\Gamma \vdash A : \star} \\
        \der{\D{2}}{\Gamma \vdash a : A} \\
        \der{\D{2}}{\Gamma \vdash b : A}
    }
    {\Gamma \vdash a =_A b : \star}
    {Eq}

\Rule{\ReflRule}
    {
        \der{\D{1}}{\Gamma \vdash A : \star} \\
        \der{\D{2}}{\Gamma \vdash t : A}
    }
    {\Gamma \vdash \text{refl}(t; A) : t =_A t}
    {Refl}

\Rule{\SubstRule}
    {
        \der{\D{1}}{\Gamma \vdash A : \star} \\
        \der{\D{2}}{\Gamma \vdash a : A} \\
        \der{\D{3}}{\Gamma \vdash b : A} \\
        \der{\D{4}}{\Gamma \vdash e : a =_A b} \\
        \der{\D{5}}{\Gamma \vdash P : (y : A) \to_\tau (p : a =_A y_\star) \to_\tau \star}
    }
    {\Gamma \vdash \psi(e, a, b; A, P) : \apptwo{P}{\tau}{a}{\tau}{\text{refl}(a; A)} \to_\omega \apptwo{P}{\tau}{b}{\tau}{e}}
    {Subst}

\Rule{\PromoteRule}
    {
        \der{\D{1}}{\Gamma \vdash (x:A) \cap B : \star} \\
        \der{\D{2}}{\Gamma \vdash a : (x:A) \cap B} \\
        \der{\D{3}}{\Gamma \vdash b : (x:A) \cap B} \\
        \der{\D{4}}{\Gamma \vdash e : a.1 =_A b.1} \\
    }
    {\Gamma \vdash \vartheta(e, a, b; (x : A) \cap B) : a =_{(x:A) \cap B} b}
    {Prm}

% \Rule{\PromoteSndRule}
%     {
%         \der{\D{1}}{\Gamma \vdash A : \star} \\
%         \der{\D{2}}{\Gamma \vdash B : \star} \\
%         \der{\D{3}}{\Gamma \vdash a : (x:A) \cap B} \\
%         \der{\D{4}}{\Gamma \vdash b : (x:A) \cap B} \\
%         \der{\D{5}}{\Gamma \vdash e : a.2 =_B b.2} \\
%     }
%     {\Gamma \vdash \vartheta_2(e, a, b; (x : A) \cap B) : a =_{(x:A) \cap B} b}
%     {PrmSnd}

% \Rule{\CastRule}
%     {
%         \der{\D{1}}{\Gamma \vdash b \cinfr (x:A) \cap B} \\
%         \der{\D{2}}{\Gamma \vdash a \chck A} \\
%         \der{\D{3}}{\Gamma \vdash e \chck a =_A b.1} \\
%         \der{\D{4}}{\text{FV}(|e|) \subseteq \text{FV}(|a|)}
%     }
%     {\Gamma \vdash \varphi(a, b, e) \infr (x:A) \cap B}
%     {Cast}

\Rule{\CastRule}
    {
        \der{\D{1}}{\Gamma \vdash a : A} \\
        \der{\D{2}}{\Gamma \vdash b : (x:A) \cap B} \\
        \der{\D{3}}{\Gamma \vdash e : a =_A b.1}
    }
    {\Gamma \vdash \varphi(a, b, e) : (x:A) \cap B}
    {Cast}

\Rule{\SeparationRule}
    {
        \der{\D{1}}{\Gamma \vdash e : \text{ctt} =_{\text{cBool}} \text{cff}}
    }
    {\Gamma \vdash \delta(e) : (X : \star) \to_0 X_\square}
    {Sep}


This chapter describes the syntax, reduction, and inference judgment of the core system for Cedille2.
Near the conclusion, this chapter also proves basic metatheoretic properties such as a weakening lemma, substitution lemma, classification, and preservation.
The presentation is a classical PTS-style with a single inference judgment.
As it stands it is not obvious how this judgment admits an inference algorithm, but this situation will be remedied in Chapter~\ref{chap:6} with an explicit algorithm.

\section{Syntax and Reduction}


\begin{figure}
    \centering
    \begin{align*}
        t &::= x\ |\ \mathfrak{b}(\kappa_1, x : t_1, t_2)\ |\ \mathfrak{c}(\kappa_2, t_1, \ldots, t_{\mathfrak{a}(\kappa_2)}) \\
        \kappa_1 &::= \lambda_m\ |\ \Pi_m\ |\ \cap \\
        \kappa_2 &::=\diamond\ |\ \star\ |\ \kind\ | \bullet_m |\ \text{pair}\ |\ \text{proj}_1\ |\ \text{proj}_2\ |\ \text{eq}\ |\ \text{refl}\ |\ \psi\ |\ \vartheta_1\ |\ \vartheta_2\ |\ \delta\ |\ \varphi  \\
        m &::= \omega\ |\ 0\ |\ \tau \\
        &\mathfrak{a}(\diamond) = \mathfrak{a}(\star) = \mathfrak{a}(\kind) = 0 \\
        &\mathfrak{a}(\text{proj}_1) = \mathfrak{a}(\text{proj}_2) = \mathfrak{a}(\text{refl}) = \mathfrak{a}(\delta) = 1 \\
        &\mathfrak{a}(\bullet_m) = \mathfrak{a}(\psi) = \mathfrak{a}(\varphi) = 2 \\
        &\mathfrak{a}(\text{pair}) = \mathfrak{a}(\text{eq}) = \mathfrak{a}(\vartheta_1) = \mathfrak{a}(\vartheta_2) = 3
    \end{align*}
    \vspace{-.4in}
    \begin{minipage}{0.5\textwidth}
        \begin{align*}
            \diamond &:= \mathfrak{c}(\diamond) \\
            \star &:= \mathfrak{c}(\star) \\
            \kind &:= \mathfrak{c}(\kind) \\
            \abs{\lambda_m}{x}{t_1}{t_2} &:= \mathfrak{b}(\lambda_m, x : t_1, t_2) \\
            (x : t_1) \to_m t_2 &:= \mathfrak{b}(\Pi_m, x : t_1, t_2) \\
            (x : t_1) \cap t_2 &:= \mathfrak{b}(\cap, x : t_1, t_2) \\
            t_1 \bullet_m t_2 &:= \mathfrak{c}(\bullet_m, t_1, t_2) \\
            \varphi(t_1, t_2) &:= \mathfrak{c}(\varphi, t_1, t_2) \\
            \psi(t_1, t_2) &:= \mathfrak{c}(\psi, t_1, t_2)
        \end{align*}
    \end{minipage}%
    \begin{minipage}{0.5\textwidth}
        \begin{align*}
            [t_1, t_2; t_3] &:= \mathfrak{c}(\text{pair}, t_1, t_2, t_3) \\
            t.1 &:= \mathfrak{c}(\text{proj}_1, t) \\
            t.2 &:= \mathfrak{c}(\text{proj}_2, t) \\
            t_1 =_{t_2} t_3 &:= \mathfrak{c}(\text{eq}, t_1, t_2, t_3) \\
            \text{refl}(t) &:= \mathfrak{c}(\text{refl}, t) \\
            \vartheta_1(t_1, t_2, t_3) &:= \mathfrak{c}(\vartheta_1, t_1, t_2, t_3) \\
            \vartheta_2(t_1, t_2, t_3) &:= \mathfrak{c}(\vartheta_2, t_1, t_2, t_3) \\
            \delta(t) &:= \mathfrak{c}(\delta, t)
        \end{align*}
    \end{minipage}
    \vspace{-.05in}
    %$$J(t_1, t_2, t_3, t_4, t_5, t_6) := \mathfrak{c}(J, t_1, t_2, t_3, t_4, t_5, t_6)$$
    \caption{Generic syntax, there are three constructors, variables, a generic binder, and a generic non-binder. Each are parameterized with a constant tag to specialize to a particular syntactic consruct. The non-binder constructor has a vector of subterms determined by an arity function computed on tags. Standard syntactic constructors are defined in terms of the generic forms.}
    \label{fig:syntax}
\end{figure}


\begin{figure}
    \centering
    \begin{minipage}{0.5\textwidth}
        $$\BinderReductionOne$$
    \end{minipage}%
    \begin{minipage}{0.5\textwidth}
        $$\BinderReductionTwo$$
    \end{minipage}%
    $$\NonbinderReduction$$
    \begin{align*}
        \app{(\abs{\lambda_m}{x}{A}{b})}{m}{t} &\betared [x := t]b \\
        [t_1, t_2, A].1 &\betared t_1 \\
        [t_1, t_2, A].2 &\betared t_2 \\
        J(A, P, x, y, \text{refl}(z), w) &\betared \app{w}{0}{z} \\
        \vartheta(\text{refl}(t.1)) &\betared \text{refl}(t) \\
        \vartheta(\text{refl}(t.2)) &\betared \text{refl}(t) \\
    \end{align*}
    \caption{Reduction rules for arbitrary syntax.}
\end{figure}




Syntax for the system is defined generically as before.
See Figure~\ref{fig:syntax} for a complete description.
The intended meaning of the syntax is as follows:
\begin{enumerate}
    \item tags $\lambda_m$, $\Pi_m$ and $\bullet_m$ (application) represent the function fragment of syntax parameterized by three separates \textit{modes}, $\omega$ (free), $0$ (erased), and $\tau$ (type-level);
    \item tags $\cap$, pair, proj$_1$, and proj$_2$ represent dependent intersections (i.e. dependent pairs);
    \item tags eq, refl, $\psi$ (substitution), $\vartheta$ (promotion), $\delta$ (separation), and $\varphi$ (cast) represent equality.
\end{enumerate}
At the moment raw syntax has no essential meaning beyond its intended one.
Nevertheless, a basic fact about substitution on syntax is provable.

\begin{lemma}
    If $x \neq y$ and $y \notin FV(a)$ then \begin{tightcenter} $[x := a][y := b]t = [y := [x := a]b][x := a]t$ \end{tightcenter}
    \label{lem:2:subst_commute}
\end{lemma}
\begin{proof}
    By induction on $t$.
    If $t$ is a binder or a constructor, then substitution unfolds and the IH applied to subterms concludes those cases.
    Suppose $t$ is a variable, $z$.
    If $z = x$, then $z \neq y$ and $t = a$ on both sides because $y \notin FV(a)$.
    If $z = y$, then $z \neq x$ and $t = [x := a]b$ on both sides.
    If $z \neq x$ and $z \neq y$, then $t = z$ on both sides.
\end{proof}

Computational meaning is added via reduction rules described in Figure~\ref{fig:reduction}.
The new reductions model projection of pairs (e.g. $[t_1, t_2, t_3].1 \betared t_1$), promotion of equalities (e.g. $\vartheta(\text{refl}(z; Z), a, b; A) \betared \text{refl}(a; A)$) and an elimination form for equality.
Note that conversion is different from a traditional PTS.
Convertibility with respect to reduction is written: $t \betaconv s$.
A detailed discussion of conversion is delayed until Section~\ref{sec:2:erasure}.

Before more important facts about reduction can be discussed it is important to observe the interaction between reduction and substitution.
First, note that multistep reduction (i.e. the reflexive-transitive closure of the reduction relation) is congruent with respect to syntax.
Second, substitution is shown to commute with multistep reduction through a series of lemmas.

\begin{lemma}
    If $t_i \betastar t_i^\prime$ for any $i$ then,
    \begin{enumerate}
        \item $\mathfrak{b}(\kappa, (x : t_1), t_2) \betastar \mathfrak{b}(\kappa, (x : t_1^\prime), t_2^\prime)$
        \item $\mathfrak{c}(\kappa, t_1,\ldots,t_{\mathfrak{a}(\kappa)}) \betastar \mathfrak{c}(\kappa, t_1^\prime,\ldots,t_{\mathfrak{a}(\kappa)}^\prime)$
    \end{enumerate}
    \label{lem:2:beta_par}
\end{lemma}
\begin{proof}
    Pick any $i$ and apply the reductions to the associate subterm.
    A straightforward induction on $t_i \betastar t_i^\prime$ demonstrates that the reductions apply only to the associated subterm.
    Repeat until all $i$ reductions are applied.
\end{proof}

\begin{lemma}
    If $a \betared b$ then $[x := t]a \betared [x := t]b$
    \label{lem:2:betared_subst}
\end{lemma}
\begin{proof}
    By induction on $a \betared b$.
    Second projection is the same as first projection case and omitted.

    $\text{Case: }\begin{array}{c} \app{(\abs{\lambda_m}{x}{A}{b})}{m}{t} \betared [x := t]b \end{array}$
    \begin{proofcase}
        $[x := s](\app{(\abs{\lambda_m}{y}{A}{b})}{m}{t}) = \app{(\abs{\lambda_m}{x}{[x := s]A}{[x := s]b})}{m}{[x := s]t} \betared [y := [x := s]t][x := s]b = [x := s][y := t]b$ \\
        Note that the final equality holds by Lemma~\ref{lem:2:subst_commute}.
    \end{proofcase}

    $\text{Case: }\begin{array}{c} [t_1, t_2; A].1 \betared t_1 \end{array}$
    \begin{proofcase}
        $[x := t][t_1, t_2, A].1 = [[x := t]t_1, [x := t]t_2, [x := t]A].1 \betared [x := t]t_1$
    \end{proofcase}

    $\text{Case: }\begin{array}{c} \app{\psi(\text{refl}(z; Z), u, v; A, P)}{\omega}{b} \betared b \end{array}$
    \begin{proofcase}
        $[x := t]\app{\psi(\text{refl}(z; Z), u, v; A, P)}{\omega}{b} = \app{\psi(\text{refl}([x := t]z; [x := t]Z), [x := t]u, [x := t]v; [x := t]A, [x := t]P)}{\omega}{[x := t]b} \betared [x := t]b$
    \end{proofcase}

    $\text{Case: }\begin{array}{c} \vartheta(\text{refl}(z; Z), u, v; A) \betared \text{refl}(u; A) \end{array}$
    \begin{proofcase}
        $[x := t]\vartheta(\text{refl}(z; Z), u, v; A) = \vartheta(\text{refl}([x := t]z; [x := t]Z), [x := t]u, [x := t]v; [x := t]A) \betared \text{refl}([x := t]u; [x := t]A) = [x := t]\text{refl}(u; A)$
    \end{proofcase}

    $\text{Case: }\begin{array}{c} \ConstructorReduction[*] \end{array}$
    \begin{proofcase}
        By the IH, $[x := t]t_i \betared [x := t]t_i^\prime$.
        Note that $$[x := t]\mathfrak{c}(\kappa, t_1, \ldots, t_{\mathfrak{a}}(\kappa)) = \mathfrak{c}(\kappa, [x := t]t_1, \ldots, [x := t]t_{\mathfrak{a}}(\kappa))$$
        Applying the constructor reduction rule and reversing the previous equality concludes the case.
    \end{proofcase}

    $\text{Case: }\begin{array}{c} \BinderOneReduction[*] \end{array}$
    \begin{proofcase}
        By the IH, $[x := t]t_1 \betared [x := t]t_1^\prime$.
        Note that $$[x := t]\mathfrak{b}(\kappa, (y : t_1), t_2) = \mathfrak{b}(\kappa, (y : [x := t]t_1), [x := t]t_2)$$
        Applying the first binder reduction rule and reversing the previous equality concludes the case.
    \end{proofcase}
\end{proof}

\begin{lemma}
    If $a \betastar b$ then $[x := t]a \betastar [x := t]b$
    \label{lem:2:betastar_subst_weak1}
\end{lemma}
\begin{proof}
    By induction on $a \betastar b$.
    The reflexivity case is trivial.

    $\text{Case: }\begin{array}{c} \MultiStep[*] \end{array}$
    \begin{proofcase}
        Let $z = t^\prime$.
        By the IH applied to $\D{2}$: $[x := t]z \betastar [x := t]b$.
        By Lemma~\ref{lem:2:betared_subst} applied to $\D{1}$: $[x := t]a \betared [x := t]z$.
        Applying the transitivity rule yields $[x := t]a \betastar [x := t]b$.
    \end{proofcase}
\end{proof}

\begin{lemma}
    If $s \betared t$ then $[x := s]a \betastar [x := t]a$
    \label{lem:2:betastar_subst_weak2}
\end{lemma}
\begin{proof}
    By induction on $a$.
    The $\mathfrak{c}$ case is omitted because it is similar to the $\mathfrak{b}$ case.

    $\text{Case: }\begin{array}{c} x \end{array}$
    \begin{proofcase}
        Rename $y$.
        Suppose $x = y$, then $[x := s]y = s \betared t = [x := t]y$.
        Thus, $[x := s]y \betastar [x := t]y$.
        Suppose $x \neq y$, then $[x := s]y = y \betastar y = [x := t]y$.
    \end{proofcase}

    $\text{Case: }\begin{array}{c} \mathfrak{b}(\kappa_1, x : t_1, t_2) \end{array}$
    \begin{proofcase}
        By the IH $[x := s]t_1 \betastar [x := t]t_1$ and $[x := s]t_2 \betastar [x := t]t_2$.
        Lemma~\ref{lem:2:beta_par} concludes the case.
    \end{proofcase}
\end{proof}

\begin{lemma}
    If $s \betastar t$ and $a \betastar b$ then $[x := s]a \betastar [x := t]b$
    \label{lem:2:betastar_subst}
\end{lemma}
\begin{proof}
    By induction on $s \betastar t$.
    The reflexivity case is Lemma~\ref{lem:2:betastar_subst_weak1}.

    $\text{Case: }\begin{array}{c} \MultiStep[*] \end{array}$
    \begin{proofcase}
        Let $z = t^\prime$.
        By the IH applied to $\D{2}$: $[x := z]a \betastar [x := t]b$.
        Lemma~\ref{lem:2:betastar_subst_weak2} yields $[x := s]a \betastar [x := z]a$.
        Transitivity concludes with $[x := s]a \betastar [x := t]b$.
    \end{proofcase}
\end{proof}

Lemma~\ref{lem:2:betastar_subst} is the only fact about the interaction of substitution and reduction that is needed moving forward.
A straightforward consequence is a similar lemma about substitution commuting with convertibility w.r.t. reduction.

\begin{lemma}
    If $s \betaconv t$ and $a \betaconv b$ then $[x := s]a \betaconv [x := t]b$
    \label{lem:2:betaconv_subst}
\end{lemma}
\begin{proof}
    By definition $\exists\ z_1, z_2$ such that $t \betastar z_1$, $s \betastar z_1$, $a \betastar z_2$, and $b \betastar z_2$.
    Applying Lemma~\ref{lem:2:betastar_subst} twice yields $[x := s]a \betastar [x := z_1]z_2$ and $[x := t]b \betastar [x := z_1]z_2$.
\end{proof}

Transitivity, as before, is a consequence of confluence.
Confluence is not an obvious property to obtain and can also be an involved property to prove.
For example, a natural variant for the $\vartheta$ reduction rule is $\vartheta(\text{refl}(t.1)) \betared \text{refl}(t)$, but this breaks confluence.
To see why, consider $\vartheta(\text{refl}([x, y; T].1))$.
One choice leads to $\vartheta(\text{refl}(x))$, and the other leads to $\text{refl}(x)$.
However, these terms are not joinable, hence confluence fails.

\section{Confluence}


\Rule{\ParBinderReduction}
    {
        \der{\D{1}}{t_1 \parred t_1^\prime} \\
        \der{\D{2}}{t_2 \parred t_2^\prime}
    }
    {\mathfrak{b}(\kappa, x : t_1, t_2) \parred \mathfrak{b}(\kappa, x : t_1^\prime, t_2^\prime) }
    {ParBind}

\Rule{\ParConstructorReduction}
    {
        \der{\D{i}}{t_i \parred t_i^\prime \quad \forall\ i \in \{1, \ldots, \mathfrak{a}(\kappa) \}}
    }
    {\mathfrak{c}(\kappa, t_1, \ldots, t_i, \ldots, t_{\mathfrak{a}(\kappa)}) \parred \mathfrak{c}(\kappa, t_1^\prime, \ldots, t_i^\prime, \ldots, t^\prime_{\mathfrak{a}(\kappa)}) }
    {ParCtor}

\Rule{\ParBetaReduction}
    {
        \der{\D{1}}{t_1 \parred t_1^\prime} \\
        \der{\D{2}}{t_2 \parred t_2^\prime} \\
        \der{\D{3}}{t_3 \parred t_3^\prime}
    }
    { \app{(\abs{\lambda_m}{x}{t_1}{t_2})}{m}{t_3} \parred [x := t_3^\prime]t_2^\prime }
    {ParBeta}

\Rule{\ParFstReduction}
    {
        \der{\D{1}}{t_1 \parred t_1^\prime} \\
        \der{\D{2}}{t_2 \parred t_2^\prime} \\
        \der{\D{3}}{t_3 \parred t_3^\prime}
    }
    { [t_1, t_2; t_3].1 \parred t_1^\prime }
    {ParFst}

\Rule{\ParSndReduction}
    {
        \der{\D{1}}{t_1 \parred t_1^\prime} \\
        \der{\D{2}}{t_2 \parred t_2^\prime} \\
        \der{\D{3}}{t_3 \parred t_3^\prime}
    }
    { [t_1, t_2; t_3].2 \parred t_2^\prime }
    {ParSnd}

\Rule{\ParPrmReduction}
    {
        \der{\D{1}}{t_1 \parred t_1^\prime} \\
        \der{\D{2}}{t_2 \parred t_2^\prime} \\
        \der{\D{3}}{t_3 \parred t_3^\prime} \\
        \der{\D{3}}{t_4 \parred t_4^\prime} \\
        \der{\D{3}}{t_5 \parred t_5^\prime}
    }
    { \vartheta(\text{refl}(t_1; t_2), t_3, t_4; t_5) \parred \text{refl}(t_3^\prime; t_5^\prime) }
    {ParPrm}

\Rule{\ParSubstReduction}
    {
        \der{\D{1}}{t_1 \parred t_1^\prime} \\
        \der{\D{2}}{t_2 \parred t_2^\prime} \\
        \der{\D{2}}{t_3 \parred t_3^\prime} \\
        \der{\D{2}}{t_4 \parred t_4^\prime} \\
        \der{\D{2}}{t_5 \parred t_5^\prime} \\
        \der{\D{2}}{t_6 \parred t_6^\prime} \\
        \der{\D{2}}{t_7 \parred t_7^\prime}
    }
    { \app{\psi(\text{refl}(t_1; t_2), t_3, t_4; t_5, t_6)}{\omega}{t_7} \parred t_7^\prime}
    {ParSubst}

\Rule{\ParVarReduction}
    {\color{white}{\_}}
    { x_K \parred x_K}
    {ParVar}

\begin{figure}
    \centering
    $$\ParVarReduction$$
    $$\ParConstructorReduction$$
    $$\ParBinderReduction$$
    $$\ParBetaReduction$$
    $$\ParSubstReduction$$
    $$\ParFstReduction$$
    $$\ParSndReduction$$
    $$\ParPrmReduction$$
    \caption{Parallel reduction rules for arbitrary syntax.}
    \label{fig:par-reduction}
\end{figure}


The proof of confluence follows the PLFA book \cite{plfa22.08}.
This strategy involves the common technique of defining a parallel reduction variant of the one-step reduction described in Figure~\ref{fig:reduction}.
Parallel reduction allows reduction steps to occur in any subexpression, but reductions that generate new redexes cannot be reduced in a single step.
Figure~\ref{fig:par-reduction} presents the inductive definition of parallel reduction.
In fact, it is possible to compute the resulting syntax after all possible redexes are contracted by a single parallel reduction step.
This is the \textit{reduction completion} (written $\parp{t}$).
The definition of reduction completion is shown in Figure~\ref{fig:par-triangle}.
Reduction completion enables the derivation of a triangle property for which confluence for parallel reduction is a consequence.
Confluence for multistep reduction is an immediate consequence of confluence for parallel reduction and logical equivalence between both reduction variants.



\begin{figure}
    \centering
    \begin{align*}
        \parp{\app{(\abs{\lambda_m}{x}{t_1}{t_2})}{m}{t_3}} &= [x := \parp{t_3}]\parp{t_2} \\
        \parp{\app{\psi(\text{refl}(t_1; t_2), t_3, t_4; t_5, t_6)}{\omega}{t_7}} &= \parp{t_7} \\
        \parp{[t_1, t_2; t_3].1} &= \parp{t_1} \\
        \parp{[t_1, t_2; t_3].2} &= \parp{t_2} \\
        \parp{\vartheta(\text{refl}(t_1; t_2), t_3, t_4; t_5)} &= \text{refl}(\parp{t_3}; \parp{t_5}) \\
        \parp{\mathfrak{c}(\kappa, t_1, \ldots, t_{\mathfrak{a}(\kappa)})} &= \mathfrak{c}(\kappa, \parp{t_1}, \ldots, \parp{t_{\mathfrak{a}(\kappa)}}) \\
        \parp{\mathfrak{b}(\kappa, (x : t_1), t_2)} &= \mathfrak{b}(\kappa, (x : \parp{t_1}), \parp{t_2}) \\
        \parp{x_K} &= x_K
    \end{align*}
    \caption{
        Definition of a reduction completion function $\parp{-}$ for parallel reduction.
        Note that this function is defined by pattern matching, applying cases from top to bottom.
        Thus, the cases at the very bottom are catch-all for when the prior cases are not applicable.
    }
    \label{fig:par-triangle}
\end{figure}


\begin{lemma}
    For any $t$, $t \parred t$
    \label{lem:a:par_refl}
\end{lemma}
\begin{proof}
    Straightforward by induction on $t$.
\end{proof}

\begin{lemma}
    If $s \betared t$ then $s \parred t$
    \label{lem:a:beta_implies_par_step}
\end{lemma}
\begin{proof}
    By induction on $s \betared t$.
    The projection and promotion cases are similar to the substitution and beta case and thus omitted.
    The second structural binder reduction case is omitted.

    $\text{Case: }\begin{array}{c} \app{(\abs{\lambda_m}{x}{A}{b})}{m}{t} \betared [x := t]b \end{array}$
    \begin{proofcase}
        By Lemma~\ref{lem:a:par_refl}: $t \parred t$ and $b \parred b$.
        Applying the \textsc{ParBeta} rule concludes the case.
    \end{proofcase}

    $\text{Case: }\begin{array}{c} \app{\psi(\text{refl}(z; Z), u, v; A, P)}{\omega}{b} \betared b \end{array}$
    \begin{proofcase}
        Using Lemma~\ref{lem:a:par_refl}: $z \parred z$, $Z \parred Z$, $u \parred u$, $v \parred v$, $A \parred A$, $P \parred P$, and $b \parred b$.
        Applying the \textsc{ParSubst} rule concludes the case.
    \end{proofcase}

    $\text{Case: }\begin{array}{c} \ConstructorReduction[*] \end{array}$
    \begin{proofcase}
        By the IH applied to $\D{1}$: $t_i \parred t_i^\prime$.
        Note that there is only one subderivation.
        For all $j \neq i$ $t_j \parred t_j$ by Lemma~\ref{lem:a:par_refl}.
        Using the \textsc{ParCtor} rule concludes the case.
    \end{proofcase}

    $\text{Case: }\begin{array}{c} \BinderOneReduction[*] \end{array}$
    \begin{proofcase}
        Applying the IH to $\D{1}$ yields $t_1 \parred t_1^\prime$.
        By Lemma~\ref{lem:a:par_refl}: $t_2 \parred t_2$.
        Using the \textsc{ParBind} rule concludes the case.
    \end{proofcase}
\end{proof}

\begin{lemma}
    If $s \betastar t$ then $s \parstar t$
    \label{lem:a:beta_implies_par}
\end{lemma}
\begin{proof}
    By induction on $s \betastar t$ applying Lemma~\ref{lem:a:beta_implies_par_step} in the inductive case.
\end{proof}

\begin{lemma}
    If $s \parred t$ then $s \betastar t$
    \label{lem:a:par_implies_beta_step}
\end{lemma}
\begin{proof}
    By induction on $s \parred t$.
    The projection, promotion, and substitution cases are similar to the beta case with the only difference being applying the associated rule.

    $\text{Case: }\begin{array}{c} \ParVarReduction[*] \end{array}$
    \begin{proofcase}
        By reflexivity of reduction.
    \end{proofcase}

    $\text{Case: }\begin{array}{c} \ParConstructorReduction[*] \end{array}$
    \begin{proofcase}
        By the IH applied to each $\D{i}$: $t_i \betastar t_i^\prime$ for all $i$.
        Applying Lemma~\ref{lem:2:beta_par} concludes the case.
    \end{proofcase}

    $\text{Case: }\begin{array}{c} \ParBinderReduction[*] \end{array}$
    \begin{proofcase}
        As the previous case, the IH yields $t_1 \betastar t_1$ and $t_2 \betastar t_2^\prime$.
        Again using Lemma~\ref{lem:2:beta_par} concludes the case.
    \end{proofcase}

    $\text{Case: }\begin{array}{c} \ParBetaReduction[*] \end{array}$
    \begin{proofcase}
        Applying the IH to all available derivations and using Lemma~\ref{lem:2:beta_par} gives $\app{(\abs{\lambda_m}{x}{t_1}{t_2})}{m}{t_3} \betastar \app{(\abs{\lambda_m}{x}{t_1^\prime}{t_2^\prime})}{m}{t_3^\prime}$.
        Applying the beta rule of reduction with transitivity concludes the case.
    \end{proofcase}
\end{proof}

\begin{lemma}
    If $s \parstar t$ then $s \betastar t$
    \label{lem:a:par_implies_beta}
\end{lemma}
\begin{proof}
    By induction on $s \parstar t$ applying Lemma~\ref{lem:a:par_implies_beta_step} in the inductive case.
\end{proof}

\begin{lemma}
    If $s \parred s^\prime$ and $t \parred t^\prime$ then $[x := s]t \parred [x := s^\prime]t^\prime$
    \label{lem:a:par_subst}
\end{lemma}
\begin{proof}
    By induction on $t \parred t^\prime$.
    The second projection case is omitted because it is the same as the first projection case.

    $\text{Case: }\begin{array}{c} \ParVarReduction[*] \end{array}$
    \begin{proofcase}
        Rename to $y$.
        If $x = y$ then $s \parred s^\prime$ which is a premise.
        If $x \neq y$ then no substitution is performed and $y_K \parred y_K$.
    \end{proofcase}

    $\text{Case: }\begin{array}{c} \ParConstructorReduction[*] \end{array}$
    \begin{proofcase}
        Applying the IH to $\D{i}$ yields $[x := s]t_i \parred [x := s^\prime]t_i^\prime$ for all $i$.
        Unfolding substitution for $\mathfrak{c}$ and applying the \textsc{ParCtor} rule concludes the case.
    \end{proofcase}

    $\text{Case: }\begin{array}{c} \ParBinderReduction[*] \end{array}$
    \begin{proofcase}
        As above the IH gives $[x := s]t_i \parred [x := s^\prime]t_i^\prime$ for $i = 1$ and $i = 2$.
        Unfolding substitution for $\mathfrak{b}$ and applying the \textsc{ParBind} rule concludes.
    \end{proofcase}

    $\text{Case: }\begin{array}{c} \ParBetaReduction[*] \end{array}$
    \begin{proofcase}
        By the IH: $[x := s]t_i \parred [x := s^\prime]t_i^\prime$ for $i = 1, 2, 3$.
        The \textsc{ParBeta} rule gives the following:
        $[x := s]\app{(\abs{\lambda_m}{y}{t_1}{t_2})}{m}{t_3} = \app{(\abs{\lambda_m}{y}{[x := s]t_1}{[x := s]t_2})}{m}{[x := s]t_3} \parred [y := t_3^\prime][x := s^\prime]t_2^\prime$.
        Note that $y$ is bound and thus not a free variable in $s^\prime$ and, moreover, by implicit renaming $x \neq y$.
        Thus, by Lemma~\ref{lem:2:subst_commute} $[y := t_3^\prime][x := s^\prime]t_2^\prime = [x := s^\prime][y := t_3^\prime]t_2^\prime$.
    \end{proofcase}

    $\text{Case: }\begin{array}{c} \ParSubstReduction[*] \end{array}$
    \begin{proofcase}
        By the IH: $[x := s]t_i \parred [x := s^\prime]t_i^\prime$ for $i = 1, 2$.
        The \textsc{ParSubst} rule gives:
        $[x := s](\app{\psi(\text{refl}(t_1; t_2), t_3, t_4; t_5, t_6)}{\omega}{t_7}) = \app{\psi(\text{refl}([x := s]t_1; [x := s]t_2), [x := s]t_3, [x := s]t_4; [x := s]t_5, [x := s]t_6)}{\omega}{[x := s]t_7} \parred [x := s^\prime]t_7^\prime$.
    \end{proofcase}

    $\text{Case: }\begin{array}{c} \ParFstReduction[*] \end{array}$
    \begin{proofcase}
        By the IH: $[x := s]t_i \parred [x := s^\prime]t_i^\prime$ for $i = 1, 2, 3$.
        The \textsc{ParFst} rule gives:
        $[x := s][t_1, t_2; t_3].1 = [[x := s]t_1, [x := s]t_2; [x := s]t_3].1 \parred [x := s^\prime]t_1^\prime$.
    \end{proofcase}

    $\text{Case: }\begin{array}{c} \ParPrmReduction[*] \end{array}$
    \begin{proofcase}
        By the IH: $[x := s]t_i \parred [x := s^\prime]t_i^\prime$ for $i = 1, 2, 3$.
        The \textsc{ParFst} rule gives:
        $[x := s]\vartheta(\text{refl}(t_1; t_2), t_3, t_4; t_5) = \vartheta(\text{refl}([x := s]t_1; [x := s]t_2), [x := s]t_3, [x := s]t_4; [x := s]t_5)  \parred \text{refl}([x := s^\prime]t_3^\prime; [x := s^\prime]t_5^\prime) = [x := s^\prime]\text{refl}(t_3^\prime; t_5^\prime)$.
    \end{proofcase}
\end{proof}

\noindent \begin{minipage}{.8\textwidth}
    The triangle property of parallel reduction is used to complete the set of possible contractible redexes.
    Thus, if syntax $s \parred t$ where $t$ is only partially reduced then both $s$ and $t$ may be completed to $\parp{s}$.
    To the right the situation is visually depicted.
    Note that the triangle property is ``half'' of the diamond property.
    Indeed, if $s \parred t^\prime$ then $t^\prime \parred \parp{s}$.
    As a consequence of the triangle property, parallel reduction trivially has the diamond property.
\end{minipage}\hfill
\begin{minipage}{.2\textwidth}
    \[\begin{tikzcd}
        s \\
        & t \\
        {\parp{s}}
        \arrow[Rightarrow, scaling nfold=3, from=1-1, to=3-1]
        \arrow[Rightarrow, scaling nfold=3, from=1-1, to=2-2]
        \arrow[Rightarrow, scaling nfold=3, from=2-2, to=3-1]
    \end{tikzcd}\]
\end{minipage}

\begin{lemma}[Parallel Triangle]
    If $s \parred t$ then $t \parred \parp{s}$
    \label{lem:a:par_triangle}
\end{lemma}
\begin{proof}
    By induction on $s \parred t$.
    The second projection case is omitted.

    $\text{Case: }\begin{array}{c} \ParVarReduction[*] \end{array}$
    \begin{proofcase}
        Have $\parp{x_K} = x_K$.
        Thus, this case is trivial.
    \end{proofcase}

    $\text{Case: }\begin{array}{c} \ParConstructorReduction[*] \end{array}$
    \begin{proofcase}
        By the IH applied to $\D{i}$: $t_i^\prime \parred \parp{t_i}$ for all $i$.
        Proceed by cases of $\parp{\mathfrak{c}(\kappa, t_1, \ldots t_{\mathfrak{a}(\kappa)})}$.
        The second projection case is omitted because it is the same as the first projection case.

        $\text{Case: }\begin{array}{c} \parp{\app{(\abs{\lambda_m}{x}{t_1}{t_2})}{m}{t_3}} = [x := \parp{t_3}]\parp{t_2} \end{array}$
        \begin{proofcase}
            Note that $\mathfrak{c}(\kappa, t_1^\prime, \ldots t^\prime_{\mathfrak{a}(\kappa)}) = \app{(\abs{\lambda_m}{x}{t_1^\prime}{t_2^\prime})}{m}{t_3^\prime}$.
            Using the \textsc{ParBeta} rule yields $\app{(\abs{\lambda_m}{x}{t_1^\prime}{t_2^\prime})}{m}{t_3^\prime} \parred [x := \parp{t_3}]\parp{t_2}$.
        \end{proofcase}

        $\text{Case: }\begin{array}{c} \parp{\psi(\app{\text{refl}(t_1; t_2), t_3, t_4; t_5, t_6)}{\omega}{t_7}} = \parp{t_7} \end{array}$
        \begin{proofcase}
            Note that $\mathfrak{c}(\kappa, t_1^\prime, \ldots t^\prime_{\mathfrak{a}(\kappa)}) = \psi(\app{\text{refl}(t_1^\prime; t_2^\prime), t_3^\prime, t_4^\prime; t_5^\prime, t_6^\prime)}{\omega}{t_7^\prime}$.
            Using the \textsc{ParSubst} rule yields $\psi(\app{\text{refl}(t_1^\prime; t_2^\prime), t_3^\prime, t_4^\prime; t_5^\prime, t_6^\prime)}{\omega}{t_7^\prime} \parred \parp{t_7}$.
        \end{proofcase}

        $\text{Case: }\begin{array}{c} \parp{[t_1, t_2; t_3].1} = \parp{t_1} \end{array}$
        \begin{proofcase}
            Note that $\mathfrak{c}(\kappa, t_1^\prime, \ldots t^\prime_{\mathfrak{a}(\kappa)}) = [t_1^\prime, t_2^\prime; t_3^\prime].1$.
            Using the \textsc{ParFst} rule yields $[t_1^\prime, t_2^\prime; t_3^\prime].1 \parred \parp{t_1}$.
        \end{proofcase}

        $\text{Case: }\begin{array}{c} \parp{\vartheta(\text{refl}(t_1; t_2), t_3, t_4; t_5)} = \prefl(\parp{t_3}; \parp{t_5}) \end{array}$
        \begin{proofcase}
            Note that $\mathfrak{c}(\kappa, t_1^\prime, \ldots t^\prime_{\mathfrak{a}(\kappa)}) = \vartheta(\text{refl}(t_1^\prime; t_2^\prime), t_3^\prime, t_4^\prime; t_5^\prime)$.
            Using the \textsc{ParPrmFst} rule yields $\vartheta(\text{refl}(t_1^\prime; t_2^\prime), t_3^\prime, t_4^\prime; t_5^\prime) \parred \prefl(\parp{t_3}; \parp{t_5})$.
        \end{proofcase}

        $\text{Case: }\begin{array}{c} \parp{\mathfrak{c}(\kappa, t_1, \ldots t_{\mathfrak{a}(\kappa)})} = \mathfrak{c}(\kappa, \parp{t_1}, \ldots \parp{t_{\mathfrak{a}(\kappa)}}) \end{array}$
        \begin{proofcase}
            Using the \textsc{ParCtor} rule concludes the case.
        \end{proofcase}
    \end{proofcase}

    $\text{Case: }\begin{array}{c} \ParBinderReduction[*] \end{array}$
    \begin{proofcase}
        Note that $\parp{\mathfrak{b}(\kappa, (x : t_1), t_2)} = \mathfrak{b}(\kappa, (x : \parp{t_1}), \parp{t_2})$.
        By the IH applied to $\D{i}$: $t_i^\prime \parred \parp{t_i}$ for $i = 1, 2$.
        Thus, by the \textsc{ParBind} rule $\mathfrak{b}(\kappa, (x : t_1^\prime), t_2^\prime) \parred \mathfrak{b}(\kappa, (x : \parp{t_1}), \parp{t_2})$.
    \end{proofcase}

    $\text{Case: }\begin{array}{c} \ParBetaReduction[*] \end{array}$
    \begin{proofcase}
        Note that $\parp{\app{(\abs{\lambda_m}{x}{t_1}{t_2})}{m}{t_3}} = [x := \parp{t_3}]\parp{t_2}$.
        By the IH applied to $\D{i}$: $t_i^\prime \parred \parp{t_i}$ for $i = 1, 2, 3$.
        Thus, by Lemma~\ref{lem:a:par_subst} $[x := t_3^\prime]t_2^\prime \parred [x := \parp{t_3}]\parp{t_2}$.
    \end{proofcase}

    $\text{Case: }\begin{array}{c} \ParSubstReduction[*] \end{array}$
    \begin{proofcase}
        Note that $\parp{\app{\psi(\text{refl}(t_1; t_2), t_3, t_4; t_5, t_6)}{\omega}{t_7}} = \parp{t_7}$.
        By the IH applied to $\D{i}$: $t_i^\prime \parred \parp{t_i}$ for $i = 1$ through $i = 7$.
        Applying the \textsc{ParBind} rule yields $t_7^\prime \parred \parp{t_7}$.
    \end{proofcase}

    $\text{Case: }\begin{array}{c} \ParFstReduction[*] \end{array}$
    \begin{proofcase}
        Note that $\parp{[t_1, t_2; t_3].1} = \parp{t_1}$.
        By the IH applied to $\D{i}$: $t_i^\prime \parred \parp{t_i}$ for $i = 1, 2, 3$.
        Thus, $t_1^\prime \parred \parp{t_1}$.
    \end{proofcase}

    $\text{Case: }\begin{array}{c} \ParPrmReduction[*] \end{array}$
    \begin{proofcase}
        Note that $\parp{\vartheta_1(\text{refl}(t_1; t_2), t_3, t_4; t_5)} \parred \prefl(\parp{t_3}; \parp{t_5})$.
        By the IH applied to $\D{i}$: $t_i^\prime \parred \parp{t_i}$ for $i = 1$ through $i = 5$.
        Thus, $\prefl(t_3^\prime;t_5^\prime) \parred \prefl(\parp{t_3};\parp{t_5})$ by the \textsc{ParCtor} rule and Lemma~\ref{lem:a:par_refl}.
    \end{proofcase}
\end{proof}

\begin{lemma}[Parallel Strip]
    \label{lem:a:par_strip}
    If $s \parred t_1$ and $s \parstar t_2$ then $\exists\ t$ such that $t_1 \parstar t$ and $t_2 \parred t$
\end{lemma}
\begin{proof}
    By induction on $s \parstar t_2$, pick $t = t_1$ for the reflexivity case.
    Consider the transitivity case, $\exists\ z_1$ such that $s \parred z_1$ and $z_1 \parstar t_2$.
    Applying Lemma~\ref{lem:a:par_triangle} to $s \parred z_1$ yields $z_1 \parred \parp{s}$.
    By the IH with $z_1 \parred \parp{s}$: $\exists\ z_2$ such that $\parp{s} \parstar z_2$ and $t_2 \parred z_2$.
    Using Lemma~\ref{lem:a:par_triangle} again on $s \parred t_1$ yields $t_1 \parred \parp{s}$.
    Now by transitivity $t_1 \parstar z_2$.
\end{proof}

\begin{lemma}[Parallel Confluence]
    \label{lem:a:par_confluence}
    If $s \parstar t_1$ and $s \parstar t_2$ then $\exists\ t$ such that $t_1 \parstar t$ and $t_2 \parstar t$
\end{lemma}
\begin{proof}
    By induction on $s \parstar t_1$, pick $t = t_2$ for the reflexivity case.
    Consider the transitivity case, $\exists\ z_1$ such that $s \parred z_1$ and $z_1 \parstar t_1$.
    By Lemma~\ref{lem:a:par_strip} applied with $s \parred z_1$ and $s \parstar t_2$ yields $\exists\ z_2$ such that $z_1 \parstar z_2$ and $t_2 \parred z_2$.
    Using the IH with $z_1 \parred z_2$ gives $\exists\ z_3$ such that $t_1 \parstar z_3$ and $z_2 \parstar z_3$.
    By transitivity $t_2 \parstar z_3$.
\end{proof}

\begin{lemma}[Confluence]
    \label{lem:2:confluence}
    If $s \betastar t_1$ and $s \betastar t_2$ then $\exists\ t$ such that $t_1 \betastar t$ and $t_2 \betastar t$
\end{lemma}
\begin{proof}
    By Lemma~\ref{lem:a:beta_implies_par} applied twice: $s \parstar t_1$ and $s \parstar t_2$.
    Now by parallel confluence (Lemma~\ref{lem:a:par_confluence}) $\exists\ t$ such that $t_1 \parstar t$ and $t_2 \parstar t$.
    Finally, two applications of Lemma~\ref{lem:a:par_implies_beta} conclude the proof.
\end{proof}

As with F$^\omega$ the important consequence of confluence is that convertibility of reduction is an equivalence relation.
However, this is \textit{not} the conversion relation that will be used in the inference judgment.
Thus, while important, it is still only a stepping stone to showing judgmental conversion is transitive.

\begin{theorem}
    \label{lem:2:beta_conv_equivalence}
    For any $s$ and $t$ the relation $s \betaconv t$ is an equivalence.
\end{theorem}
\begin{proof}
    Reflexivity is immediate because $s \betastar s$.
    Symmetry is also immediate because if $s \betaconv t$ then $\exists\ z$ such that $s \betastar z$ and $t \betastar z$, but logical conjunction is commutative.
    Transitivity is a consequence of confluence, see Theorem~\ref{thm:1:trans}.
\end{proof}

Additionally, there is a final useful fact about reduction conversion that is occasionally used throughout the rest of this work.
That is, like reduction, conversion of subexpressions yields conversion of the entire term.

\begin{lemma}
    \label{lem:2:conv_congr}
    If $t_i \betaconv t_i^\prime$ for any $i$ then,
    \begin{enumerate}
        \item $\mathfrak{b}(\kappa, (x : t_1), t_2) \betaconv \mathfrak{b}(\kappa, (x : t_1^\prime), t_2^\prime)$
        \item $\mathfrak{c}(\kappa, t_1,\ldots,t_{\mathfrak{a}(\kappa)}) \betaconv \mathfrak{c}(\kappa, t_1^\prime,\ldots,t_{\mathfrak{a}(\kappa)}^\prime)$
    \end{enumerate}
\end{lemma}
\begin{proof}
    By Lemma~\ref{lem:2:beta_par} applied on both sides.
\end{proof}

\section{Erasure and Pseudo-objects}
\label{sec:2:erasure}

Cedille has a notion of erasure of syntax that transforms a subset of syntax into the untyped $\lambda$-calculus.
This concept is generalized in $\ced$ to operate on general syntax.
While this is still called erasure, the concept no longer actually erases all type annotations.
Instead, erasure should be thought of as computing the raw syntactic forms of objects.
In Section~\ref{sec:2:judgments}, the notion of proof will be defined.
An object is the erasure of a proof.
Erasure is defined in Figure~\ref{fig:2:erasure}.
With erasure the desired conversion relation is also definable.
This definition enables equating objects in a dependent quantification instead of proofs.

\begin{definition}
    \label{def:2:conv}
    $s_1 \equiv s_2$ iff $\exists\ t_1, t_2.\ s_1 \betastar t_1, s_2 \betastar t_2, \text{ and } |t_1| \betaconv |t_2|$
\end{definition}



\begin{figure}
    \centering
    \begin{minipage}{0.5\textwidth}
        \begin{align*}
            |x_K| &= x_K \\
            |\!\star\!| &= \star \\
            |\kind| &= \kind \\
            |\abs{\lambda_0}{x}{A}{t}| &= |t| \\
            |\abs{\lambda_\omega}{x}{A}{t}| &= \abs{\lambda_\omega}{x}{\diamond}{|t|} \\
            |\abs{\lambda_\tau}{x}{A}{t}| &= \abs{\lambda_\tau}{x}{|A|}{|t|} \\
            |(x:A) \to_m B| &= (x:|A|) \to_m |B| \\
            |(x:A) \cap B| &= (x:|A|) \cap |B| \\
            |\app{f}{0}{a}| &= |f| \\
            |\app{f}{\omega}{a}| &= \app{|f|}{\omega}{|a|} \\
            |\app{f}{\tau}{a}| &= \app{|f|}{\tau}{|a|}
        \end{align*}
    \end{minipage}%
    \begin{minipage}{0.5\textwidth}
        \begin{align*}
            |\!\diamond\!| &= \diamond \\
            |[t_1, t_2; A]| &= |t_1| \\
            |t.1| &= |t| \\
            |t.2| &= |t| \\
            |x =_A y| &= |x| =_{|A|} |y| \\
            |\text{refl}(t; A)| &= \abs{\lambda_\omega}{x}{\diamond}{x_\star} \\
            %|J(A, P, x, y, e, w)| &= \app{|e|}{\omega}{|w|} \\
            |\psi(e, a, b; A, P)| &= |e| \\
            |\vartheta_1(e, a, b; T)| &= |e| \\
            |\vartheta_2(e, a, b; T)| &= |e| \\
            |\delta(e)| &= |e| \\
            |\varphi(f, e; A, T)| &= \abs{\lambda_\omega}{x}{\diamond}{x_\star}
        \end{align*}
    \end{minipage}
    \caption{Erasure of syntax, for type-like and kind-like syntax erasure is homomorphic, for term-like syntax erasure reduces to the untyped lambda calculus.}
    \label{fig:2:erasure}
\end{figure}


Note that the only purpose of the syntactic constructor $\diamond$ is to be a placeholder for erased type annotations of $\lambda_\omega$ syntactic forms.
However, for $\lambda_\tau$ variants, the annotation is \textit{not} erased.
This is an example of where erasure no longer discards all type information.
Regardless, it is faithful to the interpretation from Cedille when focused on non-type-like syntax.
Indeed, any form that is not type-like does reduce to the untyped $\lambda$-calculus.
For type-like syntax, erasure is instead locally homomorphic.
Erasure of raw syntax does not possess much structure, but it is idempotent and commutes with substitution.
Additionally, as a consequence an extension of Lemma~\ref{lem:2:betaconv_subst} is possible.

\begin{lemma}
    \label{lem:2:erasure_idempotent}
    $||t|| = |t|$
\end{lemma}
\begin{proof}
    By induction on $t$.
\end{proof}

\begin{lemma}
    \label{lem:2:erase_subst}
    $|[x := t]b| = [x := |t|]|b|$
\end{lemma}
\begin{proof}
    By induction on the size of $b$.

    $\text{Case: }\mathfrak{b}(\kappa, (x : t_1), t_2)$
    \begin{proofcase}
        If $b = \abs{\lambda_0}{y}{A}{b^\prime}$, then $|b| = |b^\prime|$ which is a smaller term.
        Then, by the IH $|[x := t]b^\prime| = [x := |t|]|b^\prime|$.
        Thus,
        \begin{align*}
            &|[x := t]\abs{\lambda_0}{y}{A}{b^\prime}| = |\abs{\lambda_0}{y}{[x := t]A}{[x := t]b^\prime}| \\
            &= |[x := t]b^\prime| = [x := |t|]|b^\prime| = [x := |t|]|\abs{\lambda_0}{y}{A}{b^\prime}|
        \end{align*}
        For the remaining tags, assume w.l.o.g., $\kappa = \cap$.
        Then $b = (y : A) \cap B$, and by the IH $|[x := t]A| = [x := |t|]|A|$ and $|[x := t]B| = [x := |t|]|B|$.
        Thus,
        \begin{align*}
            &|[x := t]((y : A) \cap B)| = |(y : [x := t]A) \cap [x := t]B| \\
            &= (y : |[x := t]A|) \cap |[x := t]B| = (y : [x := |t|]|A|) \cap [x := |t|]|B|
        \end{align*}
        And, $[x := |t|]|(y : A) \cap B| = (y : [x := |t|]|A|) \cap [x := |t|]|B|$.
        Thus, both sides are equal.
    \end{proofcase}

    $\text{Case: }\mathfrak{c}(\kappa, t_1, \ldots, t_{\mathfrak{a}(\kappa)})$
    \begin{proofcase}
        If $\kappa \in \{ \diamond, \star, \kind \}$ then the equality is trivial.
        \\ \\
        If $\kappa \in \{ \bullet_0, \text{pair}, \text{proj}_1, \text{proj}_2, \psi, \vartheta, \delta, \varphi \}$ then $|\mathfrak{c}(\kappa, t_1, \ldots)| = |t_1|$.
        Moreover, substitution commutes and both sides of the equality are equal.
        \\ \\
        If $\kappa \in \{ \text{refl} \}$ then the equality is trivial.
        \\ \\
        If $\kappa \in \{ \bullet_\omega, \bullet_\tau, \text{eq} \}$ then w.l.o.g., assume $\kappa = \text{eq}$.
        Now $|[x := t](a =_A b)| = |[x := t]a| =_{|[x := t]A|} |[x := t]b|$.
        By the IH this becomes $[x := |t|]|a| =_{[x := |t|]|A|} [x := |t|]|b|$.
        On the right-hand side, $[x := |t|]|a =_A b| = [x := |t|]|a| =_{[x := |t|]|A|} [x := |t|]|b|$.
        Thus, both sides are equal.
    \end{proofcase}

    $\text{Case: }b \text{ variable}$
    \begin{proofcase}
        Suppose $b = x_K$, then $|[x := t]x_K| = |t|$ and $[x := |t|]|x_K| = |t|$.
        Suppose $b = y_K$ where $x \neq y$, then $|[x := t]y_K| = y_K$ and $[x := |t|]|y_K| = y_K$.
        Thus, both sides are equal.
    \end{proofcase}
\end{proof}

\begin{lemma}
    \label{lem:2:betaconv_erased_subst}
    If $|s| \betaconv |t|$ and $|a| \betaconv |b|$ then $|[x := s]a| \betaconv |[x := t]b|$
\end{lemma}
\begin{proof}
    By definition $\exists\ z_1, z_2$ such that $|s| \betastar z_1$, $|t| \betastar z_1$, $|a| \betastar z_2$ and $|b| \betastar z_2$.
    By Lemma~\ref{lem:2:betastar_subst} applied twice $[x := |s|]|a| \betastar [x := |z_1|]z_2$ and $[x := |t|]|b| \betastar [x := |z_1|]z_2$.
    Finally, by Lemma~\ref{lem:2:erase_subst} $[x := |s|]|a| = |[x := s]a|$ and $[x := |t|]|b| = |[x := t]b|$.
\end{proof}

Beyond these lemmas more structure needs to be imposed on raw syntax to obtain better behavior with erasure.
In particular, the pair case and the $\lambda_0$ case are problematic.
For pairs there is an assumption that the first and second component are convertible.
This restriction is what transforms pairs into something more, an element of an intersection.
Likewise, the $\lambda_0$ binder is meant to signify that the bound variable does not appear free in the erasure of the body.
Imposing these restrictions on syntax retains the spirit of what it means to be an object.
However, because syntax is still not a proof, this restriction on syntax instead forms a set of \textit{pseudo-objects}.
The inductive definition of pseudo-objects is presented in Figure~\ref{fig:2:pseobj}.


\Rule{\BinderPseObj}
    {
        \der{\D{1}}{t_1 \pseobj} \\
        \der{\D{2}}{t_2 \pseobj} \\
        \der{\D{3}}{\kappa \neq \lambda_0}
    }
    {\mathfrak{b}(\kappa, x : t_1, t_2) \pseobj}
    {}

\Rule{\LambdaPseObj}
    {
        \der{\D{1}}{A\pseobj} \\
        \der{\D{2}}{t \pseobj} \\
        \der{\D{3}}{x \notin FV(|t|)}
    }
    {\abs{\lambda_0}{x}{A}{t} \pseobj}
    {}

\Rule{\CtorPseObj}
    {
        \der{\D{1}}{\forall\ i \in {1, \ldots, \mathfrak{a}(\kappa)}.\ t_i \pseobj} \\
        \der{\D{2}}{\kappa \neq \text{pair}}
    }
    {\mathfrak{c}(\kappa, t_1, \ldots, t_{\mathfrak{a}(\kappa)}) \pseobj}
    {}

\Rule{\PairPseObj}
    {
        \der{\D{1}}{t_1 \pseobj} \\
        \der{\D{2}}{t_2 \pseobj} \\
        \der{\D{3}}{t_3 \pseobj} \\
        \der{\D{4}}{|t_1| \betaconv |t_2|}
    }
    {[t_1, t_2; t_3] \pseobj}
    {}

\begin{figure}
    \centering
    \begin{minipage}{0.5\textwidth}
        $$\BinderPseObj$$
        $$\LambdaPseObj$$
    \end{minipage}%
    \begin{minipage}{0.5\textwidth}
        $$\CtorPseObj$$
        $$\PairPseObj$$
    \end{minipage}%
    $$x_K \pseobj$$
    \caption{Definition of Pseudo Objects.}
    \label{fig:2:pseobj}
\end{figure}


Note that the restriction for pairs is $|t_1| \betaconv |t_2|$ as opposed to $t_1 \equiv t_2$.
The distinction here is subtle, but it enables proving one of the important properties for the structure of pseudo-objects, that $|t_1| \betaconv |t_2|$ if and only if $t_1 \equiv t_2$.
To reach that goal requires a series of technical lemmas about pseudo-objects and the concepts introduced so far.
The critical property about pseudo-objects is that reduction preserves the equivalence class of a pseudo-object after erasure.

\begin{lemma}
    \label{lem:2:erase_pseobj_red}
    If $s\pseobj$ and $s \betared t$ then $|s| \betaconv |t|$
\end{lemma}
\begin{proof}
    By induction on $s\pseobj$.

    $\text{Case: }\begin{array}{c} \BinderPseObj[*] \end{array}$
    \begin{proofcase}
        By cases on $s \betared t$, applying the IH and Lemma~\ref{lem:2:conv_congr}.
    \end{proofcase}

    $\text{Case: }\begin{array}{c} \LambdaPseObj[*] \end{array}$
    \begin{proofcase}
        By cases on $s \betared t$, applying the IH and Lemma~\ref{lem:2:conv_congr}.
    \end{proofcase}

    $\text{Case: }\begin{array}{c} \CtorPseObj[*] \end{array}$
    \begin{proofcase}
        By cases on $s \betared t$.

        $\text{Case: }\begin{array}{c} \app{(\abs{\lambda_m}{x}{A}{b})}{m}{t} \betared [x := t]b \end{array}$
        \begin{proofcase}
            Note that $\abs{\lambda_m}{x}{A}{b}\pseobj$.
            If $m = 0$ then $x \notin FV(b)$ and $|[x := t]b| = |b|$.
            Thus, $|\app{(\abs{\lambda_0}{x}{A}{b})}{0}{t}| = |\abs{\lambda_0}{x}{A}{b}| = |b|$.
            If $m = \omega$, then $|\app{(\abs{\lambda_\omega}{x}{A}{b})}{\omega}{t}| = \app{(\abs{\lambda_\omega}{x}{\diamond}{|b|})}{\omega}{|t|}$.
            By definition of reduction $\app{(\abs{\lambda_\omega}{x}{\diamond}{|b|})}{\omega}{|t|} \betaconv [x := |t|]|b|$.
            Finally, by Lemma~\ref{lem:2:erase_subst} the goal is obtained.
            The case of $m = \tau$ is almost exactly the same.
        \end{proofcase}

        $\text{Case: }\begin{array}{c} [t_1, t_2; A].1 \betared t_1 \end{array}$
        \begin{proofcase}
            $|[t_1, t_2; A].1| = |[t_1, t_2; A]| = |t_1|$
        \end{proofcase}

        $\text{Case: }\begin{array}{c} [t_1, t_2; A].2 \betared t_2 \end{array}$
        \begin{proofcase}
            Observe that $|[t_1, t_2; A].2| = |t_1|$ and $[t_1, t_2; A]\pseobj$. \\
            Thus, $|s| = |t_1| \betaconv |t_2|$.
        \end{proofcase}

        $\text{Case: }\begin{array}{c} \app{\psi(\text{refl}(z; Z), a, b; A, P)}{\omega}{t} \betared t \end{array}$
        \begin{proofcase}
            $|\app{\psi(\text{refl}(z; Z), a, b; A, P)}{\omega}{t}| = \app{|\text{refl}(z;Z)|}{\omega}{|t|} \betaconv |t|$
        \end{proofcase}

        $\text{Case: }\begin{array}{c} \vartheta(\text{refl}(z; Z), a, b; A) \betared \text{refl}(a; A) \end{array}$
        \begin{proofcase}
            $|\vartheta(\text{refl}(z; Z), a, b; A)| = |\text{refl}(z; Z)| = \abs{\lambda_\omega}{x}{\diamond}{x} = |\text{refl}(a; A)|$
        \end{proofcase}

        $\text{Case: }\begin{array}{c} \ConstructorReduction[*] \end{array}$
        \begin{proofcase}
            By the IH, $|t_i| \betaconv |t_i^\prime|$. The goal is achieved by Lemma~\ref{lem:2:conv_congr}
        \end{proofcase}

    \end{proofcase}

    $\text{Case: }\begin{array}{c} \PairPseObj[*] \end{array}$
    \begin{proofcase}
        By cases on $s \betared t$, applying the IH and Lemma~\ref{lem:2:conv_congr}.
    \end{proofcase}

    $\text{Case: }\begin{array}{c} s\text{ variable} \end{array}$
    \begin{proofcase}
        By cases on $s \betared t$, $t$ must be a variable.
        Thus, $|s| = |t|$.
    \end{proofcase}
\end{proof}

\begin{lemma}
    \label{lem:2:pseobj_red_f_step}
    If $s\pseobj$, $|s| \betaconv |b|$, and $s \betared t$ then $|t| \betaconv |b|$
\end{lemma}
\begin{proof}
    By Lemma~\ref{lem:2:erase_pseobj_red} $|s| \betaconv |t|$ and by Theorem~\ref{lem:2:beta_conv_equivalence} $|t| \betaconv |b|$.
\end{proof}

Pseudo-objects also preserve substitution.
Note that the pseudo-object predicate imposes a quotient on raw syntax.
With this intuition, the purpose of pseudo-objects is to filter syntactic forms that break the intended meaning of conversion.

\begin{lemma}
    \label{lem:2:pseobj_subst}
    If $b\pseobj$ and $t\pseobj$ then $[x := t]b\pseobj$
\end{lemma}
\begin{proof}
    By induction on $b\pseobj$. The $\lambda_0$ and pair cases are no different from the respective $\mathfrak{b}$ and $\mathfrak{c}$ cases.

    $\text{Case: }\begin{array}{c} \BinderPseObj[*] \end{array}$
    \begin{proofcase}
        By the IH $[x := t]t_1\pseobj$ and $[x := t]t_2\pseobj$.
        Thus, $\mathfrak{b}(\kappa, (y : [x := t]t_1), [x := t]t_2)\pseobj$.
    \end{proofcase}

    $\text{Case: }\begin{array}{c} \CtorPseObj[*] \end{array}$
    \begin{proofcase}
        By the IH $[x := t]t_i\pseobj$. \\
        Thus, $\mathfrak{c}(\kappa, [x := t]t_1, \ldots [x := t]t_{\mathfrak{a}(\kappa)})\pseobj$.
    \end{proofcase}

    $\text{Case: }\begin{array}{c} s\text{ variable} \end{array}$
    \begin{proofcase}
        If $s = x_K$ then $[x := t]x_K = t$, and $t\pseobj$.
        Otherwise, $s = y_K$ with $y$ a variable and $y_K\pseobj$.
    \end{proofcase}
\end{proof}

\begin{lemma}
    \label{lem:2:pseobj_preservation_step}
    If $s\pseobj$ and $s \betared t$ then $t\pseobj$
\end{lemma}
\begin{proof}
    By induction on $s\pseobj$.

    $\text{Case: }\begin{array}{c} \BinderPseObj[*] \end{array}$
    \begin{proofcase}
        By cases on $s \betared t$. Suppose w.l.o.g., that $t_2 \betared t_2^\prime$.
        Observe that $t_2\pseobj$ because it is a subterm of $s$.
        Then by the IH $t_2^\prime\pseobj$.
        Thus, $\mathfrak{b}(\kappa, x : t_1, t_2^\prime) \pseobj$.
    \end{proofcase}

    $\text{Case: }\begin{array}{c} \LambdaPseObj[*] \end{array}$
    \begin{proofcase}
        By cases on $s \betared t$. Suppose w.l.o.g., that $t \betared t^\prime$.
        Note that if $x \notin FV(|t|)$ then $x \notin FV(|t^\prime|)$, reduction only reduces the amount of free variables.
        Observe that $t \pseobj$.
        Then by the IH $t^\prime\pseobj$.
        Thus, $\abs{\lambda_0}{x}{A}{t^\prime}\pseobj$.
    \end{proofcase}

    $\text{Case: }\begin{array}{c} \CtorPseObj[*] \end{array}$
    \begin{proofcase}
        By cases on $s \betared t$.
        The first and second projection cases are very similar to the substitution case.

        $\text{Case: }\begin{array}{c} \app{(\abs{\lambda_m}{x}{A}{b})}{m}{t} \betared [x := t]b \end{array}$
        \begin{proofcase}
            Observe that $b\pseobj$ and $t\pseobj$ because both are subterms of $s$.
            By Lemma~\ref{lem:2:pseobj_subst} $[x := t]b\pseobj$.
        \end{proofcase}

        $\text{Case: }\begin{array}{c} \app{\psi(\text{refl}(z;Z), a, b; A, P)}{\omega}{t} \betared t \end{array}$
        \begin{proofcase}
            Immediate by the IH: $t\pseobj$.
        \end{proofcase}

        $\text{Case: }\begin{array}{c} \vartheta_1(\text{refl}(z; Z), a, b; A) \betared \text{refl}(a; A) \end{array}$
        \begin{proofcase}
            Observe that $a\pseobj$ and $A\pseobj$.
            By application of constructor rule $\text{refl}(a; A)\pseobj$.
        \end{proofcase}

        $\text{Case: }\begin{array}{c} \ConstructorReduction[*] \end{array}$
        \begin{proofcase}
            By the IH $t_i^\prime\pseobj$.
            By application of the constructor rule the goal is obtained.
        \end{proofcase}
    \end{proofcase}

    $\text{Case: }\begin{array}{c} \PairPseObj[*] \end{array}$
    \begin{proofcase}
        By cases on $s \betared t$.
        Suppose w.l.o.g., $t_1 \betared t_1^\prime$.
        Note that $t_1\pseobj$ because it is a subterm of $s$.
        By the IH $t_1^\prime\pseobj$.
        By Lemma~\ref{lem:2:pseobj_red_f_step} $|t_1^\prime| \betaconv |t_2|$.
        Thus, $[t_1^\prime, t_2; A]\pseobj$.
    \end{proofcase}

    $\text{Case: }\begin{array}{c} s\text{ variable} \end{array}$
    \begin{proofcase}
        By cases on $s \betared t$, $t$ must be a variable.
        Thus, $t\pseobj$.
    \end{proofcase}
\end{proof}

\begin{lemma}
    \label{lem:2:pseobj_red_f}
    If $s\pseobj$, $|s| \betaconv |b|$, and $s \betastar t$ then $|t| \betaconv |b|$
\end{lemma}
\begin{proof}
    By induction on $s \betastar t$.
    The reflexivity case is trivial.
    The transitivity case is obtained from Lemma~\ref{lem:2:pseobj_red_f_step}, Lemma~\ref{lem:2:pseobj_preservation_step}, and applying the IH.
\end{proof}

\begin{lemma}
    \label{lem:2:pseobj_preservation}
    If $s\pseobj$ and $s \betastar t$ then $t\pseobj$
\end{lemma}
\begin{proof}
    By induction on $s \betastar t$.
    The reflexivity case is trivial.
    The transitivity case is obtained from Lemma~\ref{lem:2:pseobj_preservation_step} and applying the IH.
\end{proof}

\begin{lemma}
    \label{lem:2:pseobj_red_b}
    If $s\pseobj$, $|t| \betaconv |b|$, and $s \betastar t$ then $|s| \betaconv |b|$
\end{lemma}
\begin{proof}
    By induction on $s \betastar t$.
    Consequence of Lemma~\ref{lem:2:erase_pseobj_red} and Lemma~\ref{lem:2:pseobj_preservation}.
\end{proof}

\begin{lemma}
    \label{lem:2:conv_red_f}
    If $s\pseobj$, $s \equiv b$, and $s \betastar t$ then $t \equiv b$
\end{lemma}
\begin{proof}
    Note that $\exists\ z_1, z_2$ such that $s \betastar z_1$, $b \betastar z_2$, and $|z_1| \betaconv |z_2|$.
    By confluence $\exists\ z_1^\prime$ such that $z_1 \betastar z_1^\prime$ and $t \betastar z_1^\prime$.
    Then, by Lemma~\ref{lem:2:pseobj_preservation} $z_1\pseobj$.
    Finally, by Lemma~\ref{lem:2:pseobj_red_f} $|z_1^\prime| \betaconv |z_2|$.
    Therefore, $t \equiv b$.
\end{proof}

Unlike with convertibility of reduction, obtaining transitivity of conversion requires the additional assumption that the inner syntax is a pseudo-object.
The incorporation of erasure into the definition requires this extra structure, because otherwise reductions on pairs would not agree.
For example, pick $a = [x, y; T].1$, $b = [x, y; T]$, and $c = [y, x; T].2$.
Notice that $|a| = |b|$ but $|b| \neq |c|$, however, $c \betastar x$.
There is an inconsistency because $b$ is not a pseudo-object, it is not the case that $|x| \betaconv |y|$.

\begin{lemma}
    If $b\pseobj$, $a \equiv b$, and $b \equiv c$ then $a \equiv c$
    \label{thm:2:conv_trans}
\end{lemma}
\begin{proof}
    Note that $\exists\ u_1, u_2$ such that $a \betastar u_1$, $b \betastar u_2$, and $|u_1| \betaconv |u_2|$.
    Additionally, $\exists\ v_1, v_2$ such that $b \betastar v_1$, $c \betastar v_2$, and $|v_1| \betaconv |v_2|$.
    By confluence, $\exists\ z$ such that $u_2 \betastar z$ and $v_1 \betastar z$.
    Then, by Lemma~\ref{lem:2:pseobj_preservation} $u_2\pseobj$ and $v_1\pseobj$.
    Next, by Lemma~\ref{lem:2:pseobj_red_f} $|u_1| \betaconv |z|$ and $|z| \betaconv |v_2|$.
    Thus, $|u_1| \betaconv |v_2|$ by Lemma~\ref{lem:2:beta_conv_equivalence} and $a \equiv c$.
\end{proof}

Knowing that $|s| \betaconv |t|$ if and only if $s \equiv t$ is critical for maintaining the spirit of Cedille.
The purpose of the system $\ced$ is to refine the design of Cedille without losing its essential features.
A critical feature of Cedille is that convertibility is done with the untyped $\lambda$-calculus (i.e., erased terms) not with annotated terms.
Theorem~\ref{thm:2:beta_iff_conv} means that whenever conversion is checked between terms it is safe to instead check reduction conversion of objects.
Not only does this maintain the spirit of Cedille, but it also enables optimizations in type checking.
Indeed, arbitrarily expensive sequences of reductions could potentially be erased when checking $|s| \betaconv |t|$ instead of $s \equiv t$.

\begin{theorem}
    Suppose $s\pseobj$ and $t\pseobj$, then $|s| \betaconv |t|$ iff $s \equiv t$
    \label{thm:2:beta_iff_conv}
\end{theorem}
\begin{proof}
    Case $(\Rightarrow)$:
    Suppose $|s| \betaconv |t|$.
    By definition $s \betastar s$ and $t \betastar t$.
    Thus, $s \equiv t$.
    Case $(\Leftarrow)$:
    Suppose $s \equiv t$, then $\exists\ z_1, z_2$ such that $s \betastar z_1$, $t \betastar z_2$, and $|z_1| \betaconv |z_2|$.
    By two applications of Lemma~\ref{lem:2:pseobj_red_b}: $|s| \betaconv |t|$.
\end{proof}

\begin{corollary}
    For $s\pseobj$ and $t\pseobj$ the relation $s \equiv t$ is an equivalence.
\end{corollary}

Finally, a useful lemma about the interaction of substitution with conversion.

\begin{lemma}
    If $s, t, a, b\pseobj$, $s \equiv t$, and $a \equiv b$ then $[x := s]a \equiv [x := t]b$
    \label{lem:2:conv_subst}
\end{lemma}
\begin{proof}
    By Lemma~\ref{thm:2:beta_iff_conv}: $|s| \betaconv |t|$ and $|a| \betaconv |b|$.
    Then, by Lemma~\ref{lem:2:betaconv_erased_subst}: $|[x := s]a| \betaconv |[x := t]b|$.
    Finally, by Lemma~\ref{thm:2:beta_iff_conv} again, $[x := s]a \equiv [x := t]b$.
\end{proof}


\section{Inference Judgment}
\label{sec:2:judgments}


The inference judgment, presented in Figure~\ref{fig:2:typ1}; Figure~\ref{fig:2:typ2}; and Figure~\ref{fig:2:typ3}, delineate what syntax are \textit{proofs}.
As stated previously, the erasure of a proof is an \textit{object}.
Thus, for $\Gamma \vdash t : A$, $t$ is a proof and $|t|$ its object.
The judgment follows a standard PTS style, but the rules are carefully chosen so that an inference algorithm is possible.
Judgments of the form $\Gamma \vdash t : A$ should be read $t$ infers $A$ in $\Gamma$.

$\AxiomRule$ The axiom rule is the same as with F$^\omega$.
The constant $\star$ should be interpreted as a universe of types, and the constant $\kind$ as a universe of kinds.
Thus, the axiom rule states that the universe of types \textit{is} a kind in any context.

$\VarRule$ The variable rule requires that a variable at a certain type is inside the context.
Note that variables are annotated with a mode.
Modes take three forms: free ($\omega$); erased ($0$); or type ($\tau$).
The type mode is used for proofs that exist inside the type universe; the free mode for proofs that belong to some type; and the erased mode for proofs that belong to some type but whose bound variable is not computationally relevant in the associated object.
Variables are annotated with modes primarily to enable reconstruction of the appropriate binders.


\begin{figure}
    \centering
    \begin{minipage}{0.5\textwidth}
        \begin{align*}
            \text{dom}_\Pi(\omega, K) &= \star \\
            \text{dom}_\Pi(\tau, K) &= K \\
            \text{dom}_\Pi(0, K) &= K
        \end{align*}
    \end{minipage}%
    \begin{minipage}{0.5\textwidth}
        \begin{align*}
            \text{codom}_\Pi(\omega) &= \star \\
            \text{codom}_\Pi(\tau) &= \kind \\
            \text{codom}_\Pi(0) &= \star
        \end{align*}
    \end{minipage}
    \caption{Domain and codomains for function types. The metavariable $K$ is either $\star$ or $\kind$.}
\end{figure}


\begin{figure}
    \centering
    \begin{minipage}{0.5\textwidth}
        $$\AxiomRule$$
        $$\HeadInferenceRule$$
        $$\ContextEmptyRule$$
    \end{minipage}%
    \begin{minipage}{0.5\textwidth}
        $$\VarRule$$
        $$\CheckRule$$
        $$\ContextAppendRule$$
    \end{minipage}%
    $$\PiRule$$
    $$\LambdaRule$$
    $$\AppRule$$
    \caption{
        Inference rules for function types, including erased functions.
        The variable $K$ is either $\star$ or $\kind$.
    }
\end{figure}


$\PiRule$ The function type formation rule is similar to the rule for CC, but the domain and codomain are restricted.
Instead of being part of either a type or kind universe, the respective universes are restricted by the associated mode.
If the mode is $\tau$ then the domain can be either a type or a kind, but the codomain must be a kind.
If the mode is $\omega$ then the domain and codomain both must be types.
Otherwise, the mode is $0$ and the domain may be either a type or kind, but the codomain must be a type.
Note that this means polymorphic functions of data are not allowed to use their type argument computationally in the object of a proof.

$\LambdaRule$ The function formation rule is again similar to the rule for CC.
Unlike the standard PTS CC rule, the codomain of the inferred function type is again restricted to $\pcodom(m)$.
Additionally, if the mode is erased then it must be explicitly shown that the bound variable does not appear in the associated object.
Note that this is exactly the requirement imposed by pseudo-objects.

$\AppRule$ The application rule is not surprising, the only notable feature is that the mode of the function type and the application must match.

$\IntersectionRule$ The intersection type formation rule is similar to the function type formation rule, but the terms are all restricted to be types.
Thus, there are no intersections of kinds in the core Cedille2 system.

$\PairRule$ The pair formation rule is standard for formation of dependent pairs.
A third type annotation argument is required in order to make the formula inferable from the proof.
Otherwise, the annotation is required to be itself a type, the first component to match the first type, and the second component to match the second type with its free variable substituted with the first component.
Additionally, the first and second component must be convertible.
This restriction is what makes this a proof of an intersection, as opposed to merely a pair.
Note that by Theorem~\ref{thm:2:beta_iff_conv} this condition is equivalent to $|t| \betaconv |s|$ which is the restriction imposed by pseudo-objects.

$\SecondRule$ The first and second projection rules are unsurprising.
Both rules model projection from a pair as expected.

$\EqualityRule$ The equality type formation rule requires that the type annotation is a type and that the left and right-hand sides infer that type.
Note that a typed equality like this is standard from the perspective of modern type theory but significantly different from the \textit{untyped} equality of Cedille.
Indeed, the equality rules are the area of significant deviation from the original Cedille design.


\begin{figure}
    \centering
    \begin{minipage}{0.5\textwidth}
        $$\IntersectionRule$$
        $$\FirstRule$$
    \end{minipage}%
    \begin{minipage}{0.5\textwidth}
        $$\PairRule$$
        $$\SecondRule$$
    \end{minipage}
    \caption{
        Inference rules for intersection types.
    }
\end{figure}


$\ReflRule$ The reflexivity rule is the only value for equality types.
It is the standard inductive formulation of the equality type.

$\SubstRule$ The substitution rule is a dependent variation of the Leibniz's Law.
It is also a variation of Martin-L\"{o}f's J rule introduced by Pfenning and Paulin-Mohring \cite{pfenning1990_subst}.
Notice that the only critical difference between this rule and a standard variation of Leibniz's Law is that the predicate may depend on the equality proof as well.

$\PromoteRule$ The promotion rule enables equational reasoning about intersections.
Indeed, because intersections are not inductive it is difficult to reason about them without some auxiliary rule.
It states that two elements of an intersection are equal if their first projections are equal.
Note that this rule is about dependent intersections.
A non-dependent version involving the second projection is internally derivable in the system.

$\CastRule$ The cast rule is a typed version of the original cast rule of Cedille.
Note that this means this rule enables non-termination.
In Chapter~\ref{chap:5} it is shown that this rule is the only source of non-termination and a precise condition for when it may be used in a terminating way is devised.
The cast rule is critical to the spirit of Cedille.
Thus, simply dropping it to obtain a strongly normalizing system is not really a sufficient choice as it throws too much away in its loss.
An observant reader may notice that there are some redundant type annotations in the rule.
These annotations are present merely to simplify models developed in Chapter~\ref{chap:3}.
A more concise version of the system with fewer redundancies is developed in Chapter~\ref{chap:6}.

$\SeparationRule$ The separation rule states only that the equational theory is not degenerate, i.e. that there are at least two distinct objects.


\begin{figure}
    \centering
    \begin{minipage}{0.5\textwidth}
        $$\EqualityRule$$
    \end{minipage}%
    \begin{minipage}{0.5\textwidth}
        $$\ReflRule$$
    \end{minipage}%
    $$\SubstRule$$
    $$\PromoteRule$$
    $$\CastRule$$
    $$\SeparationRule$$
    \caption{
        Inference rules for equality types where
        $\text{cBool} := (X : \star) \to_0 (x : X_\square) \to_\omega (y : X_\square) \to_\omega X_\square$;
        $\text{ctt} := \abs{\lambda_0}{X}{\star}{
            \abs{\lambda_\omega}{x}{X_\square}{
                \abs{\lambda_\omega}{y}{X_\square}{x_\star}
            }
        }$;
        and
        $\text{cff} := \abs{\lambda_0}{X}{\star}{
            \abs{\lambda_\omega}{x}{X_\square}{
                \abs{\lambda_\omega}{y}{X_\square}{y_\star}
            }
        }$.
    }
    \label{fig:2:typ3}
\end{figure}


The first critical observation to be made is that the syntax participating in an inference judgment are pseudo-objects.
Thus, proofs and their types enjoy transitivity of conversion.
Next three standard lemmas are proved about the type system: weakening, substitution, and a sort-hierarchy classification.

\begin{lemma}
    If $\Gamma \vdash t : A$ then $t\pseobj$
    \label{lem:2:infer_implies_pseobj}
\end{lemma}
\begin{proof}
    Straightforward by induction.
    The only interesting case is the pair case, but it is discharged by Theorem~\ref{thm:2:beta_iff_conv}.
\end{proof}

\begin{lemma}
    If $\Gamma \vdash t : A$ then $A\pseobj$
    \label{lem:2:infer_implies_pseobj_type}
\end{lemma}
\begin{proof}
    By induction.
    The \textsc{Ax}, \textsc{Pi}, \textsc{Int} and \textsc{Eq} rules are trivial.
    Rules \textsc{Lam}, \textsc{Pair}, \textsc{Cast}, and \textsc{Conv} rules are immediate by applying Lemma~\ref{lem:2:infer_implies_pseobj} to a sub-derivation.
    The \textsc{Fst} and \textsc{App} rules are omitted because it is similar to the \textsc{Snd} rule.
    Likewise, the \textsc{Refl} rule is omitted because it is similar to the \textsc{Prm} rule.

    $\text{Case: }\begin{array}{c} \VarRule[*] \end{array}$
    \begin{proofcase}
        By Lemma~\ref{lem:2:infer_implies_pseobj} applied to $\D{2}$: $A\pseobj$.
    \end{proofcase}

    $\text{Case: }\begin{array}{c} \SecondRule[*] \end{array}$
    \begin{proofcase}
        By the IH applied to $\D{1}$: $B\pseobj$.
        Using Lemma~\ref{lem:2:infer_implies_pseobj} gives $t\pseobj$ and thus $t.1\pseobj$.
        Now by Lemma~\ref{lem:2:pseobj_subst}: $[x := t.1]B\pseobj$.
    \end{proofcase}

    $\text{Case: }\begin{array}{c} \SubstRule[*] \end{array}$
    \begin{proofcase}
        By Lemma~\ref{lem:2:infer_implies_pseobj}: $P, e\pseobj$.
        Applying the IH to $\D{1}$ gives $A, a, b\pseobj$.
        Now building up the subexpressions using pseudo-object rules concludes the proof.
    \end{proofcase}

    $\text{Case: }\begin{array}{c} \PromoteRule[*] \end{array}$
    \begin{proofcase}
        Applying the IH to $\D{1}$ gives that $(x : A) \cap B\pseobj$.
        Now, by Lemma~\ref{lem:2:infer_implies_pseobj}: $a, b\pseobj$.
        Using the pseudo-object rule for equality concludes the case.
    \end{proofcase}

    $\text{Case: }\begin{array}{c} \SeparationRule[*] \end{array}$
    \begin{proofcase}
        Immediate by a short sequence of pseudo-object rules.
    \end{proofcase}
\end{proof}

\begin{lemma}[Weakening]
    \label{lem:2:weakening}
    If $\Gamma; \Delta \vdash t : A$ and $\Gamma \vdash B : K$ then $\Gamma; y_m : B; \Delta \vdash t : A$ for $y$ fresh
\end{lemma}
\begin{proof}
    By induction.
    Most cases are a direct consequence of applying the IH to sub-derivations and applying the associated rule.

    $\text{Case: }\begin{array}{c} \AxiomRule[*] \end{array}$
    \begin{proofcase}
        Trivial by axiom rule.
    \end{proofcase}

    $\text{Case: }\begin{array}{c} \VarRule[*] \end{array}$
    \begin{proofcase}
        Note that $y$ is fresh thus $x \neq y$.
        If $y$ is placed after $x$ then the case is trivial because $\Gamma_2$ is only constrained to carry fresh variables.
        Thus, suppose $y$ is placed before $x$.
        Let $\Gamma_1 = \Delta_1; \Delta_2$.
        Applying the IH to $\D{2}$ gives $\Delta_1; y_m : B; \Delta_2 \vdash A : K$.
        The \textsc{Var} rule concludes.
    \end{proofcase}

    $\text{Case: }\begin{array}{c} \PiRule[*] \end{array}$
    \begin{proofcase}
        The IH applied to $\D{1}$ and $\D{2}$ and the pi-rule concludes the case.
    \end{proofcase}
\end{proof}

\begin{lemma}[Substitution]
    \label{lem:2:subst}
    Suppose $\Gamma \vdash b : B$.
    If $\Gamma, y : B, \Delta \vdash t : A$ then $\Gamma, [y := b]\Delta \vdash [y := b]t : [y := b]A$
\end{lemma}
\begin{proof}
    By induction on $\Gamma, y : B, \Delta \vdash t : A$.
    The \textsc{Ax} rule is trivial and omitted.
    The rules \textsc{Lam} and \textsc{Int} are very similar to the \textsc{Pi} rule.
    The rules \textsc{Fst}, \textsc{Eq}, \textsc{Refl}, \textsc{Subst}, \textsc{Prm}, \textsc{Cast} and \textsc{Sep} rules are proven by applying \textit{1.} to sub-derivations and using the associated rule.
    Rule \textsc{Snd} is very similar to \textsc{App} and thus omitted.
    Likewise, \textsc{Conv} is very similar to \textsc{Pair} and thus omitted.

    $\text{Case: }\begin{array}{c} \VarRule[*] \end{array}$
    \begin{proofcase}
        Suppose wlog that $y \in \Gamma_1$.
        Let $\Gamma_1 = \Delta_1; y : B; \Delta_2$.
        Applying the IH to $\D{1}$ gives $\Delta_1; [y := b]\Delta_2 \vdash [y := b]A : K$.
        Note that $x \notin FV(\Delta_1; [y := b]\Delta_2; [y := b]\Gamma_2)$.
        Thus by the \textsc{Var} rule: $\Delta_1; [y := b]\Delta_2; x_m : [y := b]A; [y := b]\Gamma_2 \vdash x_K : [y := b]A$.
    \end{proofcase}

    $\text{Case: }\begin{array}{c} \PiRule[*] \end{array}$
    \begin{proofcase}
        Applying \textit{1.} to the sub-derivations gives:
        \begin{enumerate}
            \item[$\D{1}$.] $\Gamma, [y := b]\Delta \vdash [y := b]A : \pdom(m, K)$
            \item[$\D{2}$.] $\Gamma, [y := b]\Delta, x_m : [y := b]A \vdash [y := b]B : \pcodom(m)$
        \end{enumerate}
        Thus, $\Gamma, [y := b]\Delta \vdash (x : [y := b]A) \to_m [y := b]B : \pcodom(m)$.
    \end{proofcase}

    $\text{Case: }\begin{array}{c} \AppRule[*] \end{array}$
    \begin{proofcase}
        Applying \textit{1.} to $\D{1}$ and $\D{2}$ gives:
        \begin{enumerate}
            \item[$\D{1}$.] $\Gamma, [y := b]\Delta \vdash [y := b]f : (x : [y := b]A) \to_m [y := b]B$
            \item[$\D{2}$.] $\Gamma, [y := b]\Delta, x_m : [y := b]A \vdash [y := b]a : [y := b]A$
        \end{enumerate}
        By the \textsc{App} rule $\Gamma, [y := b]\Delta \vdash \app{[y := b]f}{m}{[y := b]a} : [x := a][y := b]B$.
        Note that $x$ is fresh to $\Gamma$, thus $x \notin FV(b)$.
        By Lemma~\ref{lem:2:subst_commute} $[x := a][y := b]B = [y := b][x := a]B$.
    \end{proofcase}

    $\text{Case: }\begin{array}{c} \PairRule[*] \end{array}$
    \begin{proofcase}
        Applying \textit{1.} to the sub-derivations gives:
        \begin{enumerate}
            \item[$\D{1}$.] $\Gamma, [y := b]\Delta \vdash (x : [y := b]A) \cap [y := b]B : \star$
            \item[$\D{2}$.] $\Gamma, [y := b]\Delta \vdash [y := b]t : [y := b]A$
            \item[$\D{3}$.] $\Gamma, [y := b]\Delta \vdash [y := b]s : [y := b][x := t]B$
        \end{enumerate}
        Note that $x$ is locally-bound and thus $x \notin FV(\Gamma)$, thus by Lemma~\ref{lem:2:subst_commute} $$[y := b][x := t]B = [x := [y := b]t][y := b]B$$
        Now by Lemma~\ref{lem:2:conv_subst}: $[y := b]t \equiv [y := b]s$.
        Thus, by the \textsc{Pair} rule $\Gamma, [y := b]\Delta \vdash [[y := b]t, [y := b]s] : (x : [y := b]A) \cap [y := b]B$.
    \end{proofcase}
\end{proof}

\begin{lemma}
    \label{lem:2:classification}
    If $\Gamma \vdash t : A$ then $A = \kind$ or $\Gamma \vdash A : K$
\end{lemma}
\begin{proof}
    By induction.
    The \textsc{Ax}, \textsc{Pi}, \textsc{Lam}, \textsc{Int}, \textsc{Pair}, \textsc{Eq}, \textsc{Cast}, and \textsc{Conv} rules are trivial.
    The \textsc{Fst} rule is omitted because it is similar to \textsc{Snd} rule.
    Likewise, the \textsc{Refl} rule is omitted because it is similar to the \textsc{Prm} rule.

    $\text{Case: }\begin{array}{c} \VarRule[*] \end{array}$
    \begin{proofcase}
        Immediate by $\D{2}$ and weakening.
    \end{proofcase}

    $\text{Case: }\begin{array}{c} \AppRule[*] \end{array}$
    \begin{proofcase}
        Applying the IH to $\D{1}$ gives $\Gamma \vdash (x : A) \to_m B : K$.
        Now $\Gamma, x : A \vdash B : K$.
        Using the substitution lemma gives $\Gamma \vdash [x := a]B : K$.
    \end{proofcase}

    $\text{Case: }\begin{array}{c} \SecondRule[*] \end{array}$
    \begin{proofcase}
        By the IH applied to $\D{1}$ gives $\Gamma \vdash (x : A )\cap B : K$.
        Thus, $\Gamma, x : A \vdash B : K$.
        Applying the substitution lemma gives $\Gamma \vdash [x := t.1]B : K$.
    \end{proofcase}

    $\text{Case: }\begin{array}{c} \SubstRule[*] \end{array}$
    \begin{proofcase}
        By the \textsc{Refl} rule: $\Gamma \vdash \prefl(a;A) : a =_A a$.
        Now by the \textsc{App} rule $\Gamma \vdash \apptwo{P}{\tau}{a}{\tau}{\prefl(a;A)} : \star$ and $\Gamma \vdash \apptwo{P}{\tau}{b}{\tau}{e} : \star$.
        Using weakening gives $\Gamma, x : \apptwo{P}{\tau}{a}{\tau}{\prefl(a;A)} \vdash \apptwo{P}{\tau}{b}{\tau}{e} : \star$.
        Now the \textsc{Pi} rule concludes the case.
    \end{proofcase}

    $\text{Case: }\begin{array}{c} \PromoteRule[*] \end{array}$
    \begin{proofcase}
        Immediate by applying the \textsc{Eq} rule.
    \end{proofcase}

    $\text{Case: }\begin{array}{c} \SeparationRule[*] \end{array}$
    \begin{proofcase}
        Have $\Gamma \vdash (X : \star) \to_\omega X : \star$ via short sequence of rules.
    \end{proofcase}
\end{proof}

The context of a judgment is, for the moment, unrestrained.
Indeed, a variable may bind a type represented by arbitrary syntax and as long as that variable is never used in the body of the term there is no issue.
To remove these considerations contexts should instead be well-formed:
\begin{definition}
    A context $\Gamma$ is \textbf{well-formed} (written $\vdash \Gamma$) iff for every possible splitting $\Gamma = \Gamma_1, x : A, \Gamma_2$ the variable $x \notin FV(\Gamma_1;\Gamma_2)$ and $\Gamma_1 \vdash A : K$ for some $K$
\end{definition}
It is not difficult to see that an inference judgment with a well-formed context is obtained from any general inference judgment.
Moving forward it will be assumed that the context is well-formed because an equivalent proof is always obtainable under this assumption and the non-well-founded proofs will not add any value.
\begin{lemma}
    If $\Gamma \vdash t : A$ then $\exists\ \Delta$ such that $\Delta \vdash t : A$ and $\vdash \Delta$
\end{lemma}
\begin{proof}
    By Lemma~\ref{lem:2:classification}: $\Gamma \vdash A : K$.
    Now, the set of free variable $S = FV(t) \cup FV(A)$ determines $\Delta$.
    Moreover, every occurrence of $x \in S$ in either $t$ or $A$ must be via a \textsc{Var} rule, hence the associated type is a proof.
    Delete any variables not found in $S$ from $\Gamma$ to form $\Delta$.
\end{proof}

\section{Preservation}

Preservation states that the status of a term (i.e. both its classification and status as a well-founded proof) do not change with respect to reduction.
Note that Cedille only enjoys a semantic version of preservation and not a syntactic version presented below.
While this may not matter from the perspective of the semantics it does indicate that syntax is better behaved.
The proof follows by induction on the typing derivation and case analysis on the associated reduction.

\begin{definition}
    $\Gamma \betared \Gamma^\prime$ iff there exists a unique $(x_m : A) \in \Gamma$ such that $A \betared A^\prime$
\end{definition}

\begin{lemma}
    \label{lem:2:preservation_no_type_step}
    \textcolor{white}{\_}
    \begin{enumerate}
        \item If $\Gamma \vdash t : A$ and $t \betared t^\prime$ then $\Gamma \vdash t^\prime : A$
        \item If $\Gamma \vdash t : A$ and $\Gamma \betared \Gamma^\prime$ then $\Gamma^\prime \vdash t : A$
        \item If $\vdash \Gamma$ and $\Gamma \betared \Gamma^\prime$ then $\vdash \Gamma^\prime$
    \end{enumerate}
\end{lemma}
\begin{proof}
    By mutual recursion.

    \noindent \textbf{1.} Pattern-matching on $\Gamma \vdash t : A$.
    The \textsc{Ax} and \textsc{Var} cases are impossible by inversion on $t \betared t^\prime$.
    \textsc{Int} is very similar to \textsc{Pi}, \textsc{Fst} is very similar to \textsc{Snd}.
    The \text{Refl}, \textsc{Sep}, and \textsc{Conv} rules are trivial.

    $\text{Case: }\begin{array}{c} \PiRule[*] \end{array}$
    \begin{proofcase}
        Suppose $A \betared A^\prime$.
        Applying \textit{1} to $\D{1}$ gives $\Gamma \vdash A^\prime : \pdom(m, K)$.
        Note that $\Gamma, x_m : A \betared \Gamma, x_m : A^\prime$.
        Thus, using \textit{2} with $\D{2}$ gives $\Gamma, x_m : A^\prime \vdash B : \pcodom(m)$.
        Using the \textsc{Pi} rule concludes the case.
        \\ \\
        Suppose $B \betared B^\prime$.
        Applying \textit{1} to $\D{2}$ gives $\Gamma, x_m : A \vdash B^\prime : \pcodom(m)$.
        The \textsc{Pi} rule concludes the case.
    \end{proofcase}

    $\text{Case: }\begin{array}{c} \LambdaRule[*] \end{array}$
    \begin{proofcase}
        Suppose $A \betared A^\prime$.
        Then $(x : A) \to_m B \betared (x : A^\prime) \to_m B$.
        Now, using \textit{1} with $\D{1}$ gives $\Gamma \vdash (x : A^\prime) \to_m B : \pcodom(m)$.
        Note that $\Gamma, x_m : A \betared \Gamma, x_m : A^\prime$.
        Using \textit{2} with $\D{2}$ yields $\Gamma, x_m : A^\prime \vdash t : B$.
        Applying the \textsc{Lam} rule concludes the case.
        \\ \\
        Suppose $t \betared t^\prime$.
        Using \textit{1} with $\D{2}$ gives $\Gamma, x_m : A \vdash t^\prime : B$.
        Note that reduction does not introduce free variables, thus $x \notin FV(|t^\prime|)$ if $m = 0$.
        The \textsc{Lam} rule concludes.
    \end{proofcase}

    $\text{Case: }\begin{array}{c} \AppRule[*] \end{array}$
    \begin{proofcase}
        Suppose $f \betared f^\prime$.
        Applying \textit{1} with $\D{1}$ gives $\Gamma \vdash f^\prime : (x : A) \to_m B$.
        The \textsc{App} rule concludes.
        \\ \\
        Suppose $a \betared a^\prime$.
        Using \textit{1} with $\D{2}$ gives $\Gamma \vdash a^\prime : A$.
        Again, the \textsc{App} rule concludes the case.
        \\ \\
        Suppose $f = \abs{\lambda_m}{x}{C}{t}$ and $\app{f}{m}{a} \betared [x := a]t$.
        There must exist $C$ and $D$ such that $\Gamma \vdash C : \pdom(m, K)$ and $\Gamma, x_m : C \vdash t : D$ with $A \equiv C$ and $B \equiv D$.
        By classification (Lemma~\ref{lem:2:classification}) and the \textsc{Conv} rule, $\Gamma \vdash a : C$.
        Now using the substitution lemma (Lemma~\ref{lem:2:subst}) $\Gamma \vdash [x := a]t : [x := a]D$.
        Using Lemma~\ref{lem:2:conv_subst} gives $[x := a]B \equiv [x := a]D$.
        Classification and \textsc{Conv} again yields $\Gamma \vdash [x := a]t : [x := a]B$.
        \\ \\
        Suppose $f = \psi(\prefl(z;Z), u, v; U, P)$ with $m = \omega$ and $\app{f}{\omega}{a} \betared a$.
        By inversion on $\D{1}$: $A = \apptwo{P}{\tau}{u}{\tau}{\prefl(u;U)}$ and $[x := a]B = \apptwo{P}{\tau}{v}{\tau}{\prefl(z;Z)}$.
        However, inversion also yields $\Gamma \vdash \prefl(z;Z) : u =_U v$ thus $z \equiv u$, $z \equiv v$, and $Z \equiv U$.
        Thus, $\apptwo{P}{\tau}{u}{\tau}{\prefl(u;U)} \equiv \apptwo{P}{\tau}{v}{\tau}{\prefl(z;Z)}$.
        The \textsc{Conv} rule concludes the case.
    \end{proofcase}

    $\text{Case: }\begin{array}{c} \PairRule[*] \end{array}$
    \begin{proofcase}
        Suppose $t \betared t^\prime$.
        Applying \textit{1} to $\D{2}$ gives $\Gamma \vdash t^\prime : A$.
        Note that $[x := t]B \equiv [x := t^\prime]B$ by Lemma~\ref{lem:2:conv_subst}.
        Moreover, deconstructing $\D{1}$ yields $\Gamma, x_\tau : A \vdash B : \star$.
        By the substitution lemma $\Gamma \vdash [x := t^\prime]B : \star$.
        Thus, by the \textsc{Conv} rule $\Gamma \vdash s : [x := t^\prime]B$.
        Finally, Lemma~\ref{lem:2:conv_red_f} gives $t^\prime \equiv s$ from $\D{4}$.
        The \textsc{Pair} rule concludes the case.
        \\ \\
        Suppose $s \betared s^\prime$.
        By \textit{1} applied to $\D{3}$: $\Gamma \vdash s^\prime : [x := t]B$.
        Using Lemma~\ref{lem:2:conv_subst} with $\D{4}$ yields $t \equiv s^\prime$.
        The \textsc{Pair} rule concludes.
        \\ \\
        Suppose $A \betared A^\prime$.
        Then $(x : A) \cap B \betared (x : A^\prime) \cap B$.
        Applying this reduction to \textit{1} with $\D{1}$ gives $\Gamma \vdash (x : A^\prime) \cap B : \star$.
        Deconstructing this yields $\Gamma \vdash A^\prime : \star$.
        Now by the \textsc{Conv} rule $\Gamma \vdash t : A^\prime$.
        Using the \textsc{Pair} rule concludes.
        \\ \\
        Suppose $B \betared B^\prime$.
        Then $(x : A) \cap B \betared (x : A^\prime) \cap B$.
        Applying this reduction to \textit{1} with $\D{1}$ gives $\Gamma \vdash (x : A) \cap B^\prime : \star$.
        Deconstructing this yields $\Gamma, x_m : A^\prime \vdash B^\prime : \star$.
        Note that $B \betared B^\prime$ implies that $B \equiv B^\prime$.
        Moreover, using Lemma~\ref{lem:2:conv_subst} gives $[x := t]B \equiv [x := t]B^\prime$.
        The substitution lemma gives $\Gamma \vdash [x := t]B^\prime : \star$.
        Now the \textsc{Conv} rule yields $\Gamma \vdash s [x := t]B^\prime$.
        The \textsc{Pair} rule concludes the case.
    \end{proofcase}

    $\text{Case: }\begin{array}{c} \SecondRule[*] \end{array}$
    \begin{proofcase}
        Suppose $t \betared t^\prime$.
        Then applying \textit{1} to $\D{1}$ gives $\Gamma \vdash t^\prime : (x : A) \cap B$.
        Applying the \textsc{Snd} rule concludes the case.
        \\ \\
        Suppose $t = [t_1, t_2, t_3]$ and $t.2 \betared t_2$.
        Then we have $\Gamma \vdash [t_1, t_2, t_3] : (x : A) \cap B$.
        Deconstructing this rule yields $\Gamma \vdash t_1 : A$, $\Gamma, x_\tau : A \vdash B : \star$, and $\Gamma \vdash t_2 : [x := t_1]B$.
        By the substitution lemma $\Gamma \vdash [x := t.1]B : \star$.
        Note that $t.1 \betared t_1$ thus $t.1 \equiv t_1$.
        Now using Lemma~\ref{lem:2:conv_subst} gives $[x := t.1]B \equiv [x := t_1]B$.
        Thus, by the \textsc{Conv} rule $\Gamma \vdash t_2 : [x := t.1]B$.
    \end{proofcase}

    $\text{Case: }\begin{array}{c} \EqualityRule[*] \end{array}$
    \begin{proofcase}
        Suppose $a \betared a^\prime$.
        Applying \textit{1} to $\D{2}$ gives $\Gamma \vdash a^\prime : A$.
        The \textsc{Eq} rule concludes.
        \\ \\
        Suppose $b \betared b^\prime$.
        Applying \textit{1} to $\D{3}$ gives $\Gamma \vdash b^\prime : A$.
        The \textsc{Eq} rule concludes.
        \\ \\
        Suppose $A \betared A^\prime$.
        Applying \textit{1} to $\D{1}$ gives $\Gamma \vdash A^\prime : \star$.
        Note that $A \equiv A^\prime$.
        Thus, by the \textsc{Conv} rule applied twice: $\Gamma \vdash a : A^\prime$ and $\Gamma \vdash b : A^\prime$.
        Using the \textsc{Eq} rule concludes the case.
    \end{proofcase}

    $\text{Case: }\begin{array}{c} \SubstRule[*] \end{array}$
    \begin{proofcase}
        Suppose $A \betared A^\prime$.
        Then $a =_A b \equiv a =_A^\prime b$ and $(y : A) \to_tau (p : a =_A y_\star) \to_\tau \star \equiv (y : A) \to_tau (p : a =_A y_\star) \to_\tau$.
        Thus, by the \textsc{Conv} rule: $\Gamma \vdash a : A^\prime$, $\Gamma \vdash b : A^\prime$, $\Gamma \vdash e : a =_{A^\prime} b$, and $\Gamma \vdash P : (y : A^\prime) \to_tau (p : a =_{A^\prime} y_\star) \to_\tau$.
        Applying \textit{1} to $\D{1}$ gives: $\Gamma \vdash A^\prime : \star$.
        The \textsc{Subst} rule concludes the case.
        \\ \\
        Suppose $a \betared a^\prime$.
        Then $a =_A b \equiv a^\prime _A b$ and $(y : A) \to_tau (p : a =_A y_\star) \to_\tau \star \equiv (y : A) \to_tau (p : a^\prime =_A y_\star) \to_\tau$.
        Thus, by the \textsc{Conv} rule: $\Gamma \vdash e : a^\prime =_A b$ and $\Gamma \vdash P : (y : A) \to_tau (p : a^\prime =_A y_\star) \to_\tau$.
        Applying \textit{1} to $\D{2}$ gives: $\Gamma \vdash a^\prime : A$.
        The \textsc{Subst} rule concludes the case.
        \\ \\
        Suppose $b \betared b^\prime$.
        Then $a =_A b \equiv a =_A b^\prime$ and by the \textsc{Conv} rule $\Gamma \vdash b^\prime : A$.
        Applying \textit{1} to $\D{3}$ gives: $\Gamma \vdash b^\prime : B$.
        The \textsc{Subst} rule concludes the case.
        \\ \\
        Suppose $e \betared e^\prime$.
        Then by \textit{1} applied to $\D{1}$: $\Gamma \vdash e^\prime : a =_A b$.
        The \textsc{Subst} rule concludes the case.
        \\ \\
        Suppose $P \betared P^\prime$.
        By \textit{1} applied to $\D{2}$: $\Gamma \vdash P : (y : A) \to_\tau (p : a =_A y) \to\tau \star$.
        The \textsc{Subst} rule concludes the case.
    \end{proofcase}

    $\text{Case: }\begin{array}{c} \PromoteRule[*] \end{array}$
    \begin{proofcase}
        Suppose $e \betared e^\prime$.
        Then by \textit{1} applied to $\D{4}$: $\Gamma \vdash e^\prime a.1 =_A b.1$ and the \textsc{Prm} rule concludes.
        \\ \\
        Suppose $a \betared a^\prime$.
        Then $a.1 =_A b.1 \equiv a^\prime.1 =_A b.1$ and the \textsc{Conv} rule yields $\Gamma \vdash e : a^\prime.1 =_A b.1$.
        Applying \textit{1} to $\D{2}$ gives $\Gamma \vdash a^\prime : (x : A) \cap B$.
        The \textsc{Prm} rule concludes.
        \\ \\
        Suppose $b \betared b^\prime$.
        Then $a.1 =_A b.1 \equiv a.1 =_A b^\prime.1$ and the \textsc{Conv} rule yields $\Gamma \vdash e : a.1 =_A b^\prime.1$.
        Applying \textit{1} to $\D{3}$ gives $\Gamma \vdash b^\prime : (x : A) \cap B$.
        The \textsc{Prm} rule concludes.
        \\ \\
        Suppose wlog that $B \betared B^\prime$, the case when $A \betared A^\prime$ is similar.
        Then $(x : A) \cap B \equiv (x : A) \cap B^\prime$ and the \textsc{Conv} rule yields $\Gamma \vdash a : (x : A) \cap B^\prime$ and $\Gamma \vdash b : (x : A) \cap B^\prime$.
        Applying \textit{1} to $\D{1}$ yields $\Gamma \vdash (x : A) \cap B^\prime : \star$.
        The \textsc{Prm} rule concludes.
        \\ \\
        Suppose $e = \prefl(z;Z)$ and $\vartheta(e, a, b; (x:A) \cap B) \betared \prefl(a; (x : A) \cap B)$.
        By inversion $\Gamma \vdash \prefl(z;Z) : a.1 =_A b.1$, hence $z \equiv a.1$, $z \equiv b.1$.
        Thus, $a \equiv b$ and $\Gamma \vdash \prefl(a; (x : A) \cap B) : a =_{(x :A) \cap B} b$.
    \end{proofcase}

    $\text{Case: }\begin{array}{c} \CastRule[*] \end{array}$
    \begin{proofcase}
        Suppose $a \betared a^\prime$.
        Then $a =_A b.1 \equiv a^\prime =_A b.1$ and by \textsc{Conv} rule $\Gamma \vdash e : a^\prime =_A b.1$.
        Applying \textit{1} to $\D{1}$ yields $\Gamma \vdash a^\prime : A$.
        The \textsc{Cast} rule concludes.
        \\ \\
        Suppose $b \betared b^\prime$.
        Then $a =_A b.1 \equiv a =_A b^\prime.1$ and by \textsc{Conv} rule $\Gamma \vdash e : a =_A b^\prime.1$.
        Applying \textit{1} to $\D{2}$ yields $\Gamma \vdash b^\prime : (x : A) \cap B$.
        The \textsc{Cast} rule concludes.
        \\ \\
        Suppose $e \betared e^\prime$.
        Applying \textit{1} to $\D{3}$ yields $\Gamma \vdash e^\prime : a =_A b.1$
        The \textsc{Cast} rule concludes.
    \end{proofcase}

    \noindent \textbf{2.} Pattern-matching on $\Gamma \vdash t : A$.
    Note that except \textsc{Ax} and \textsc{Var} all the other cases are immediate by applying \textit{2} to all sub-derivations and using the associated rule.

    $\text{Case: }\begin{array}{c} \AxiomRule[*] \end{array}$
    \begin{proofcase}
        Immediate by the \textsc{Ax} rule, the context does not matter.
    \end{proofcase}

    $\text{Case: }\begin{array}{c} \VarRule[*] \end{array}$
    \begin{proofcase}
        Suppose $\Gamma_1 \betared \Gamma_1^\prime$.
        Reduction does not produce free variables, thus $x \notin FV(\Gamma^\prime_1;\Gamma_2)$.
        Applying \textit{1} to $\D{2}$ yields $\Gamma^\prime_1 \vdash A : K$.
        The \textsc{Var} rule concludes.
        \\ \\
        Suppose $\Gamma_2 \betared \Gamma_2^\prime$.
        As before $x \notin FV(\Gamma_1;\Gamma_2^\prime)$.
        The \textsc{Var} rule concludes.
        \\ \\
        Suppose $A \betared A^\prime$.
        Applying \textit{1} to $\D{2}$ gives $\Gamma_1 \vdash A^\prime : K$.
        The \textsc{Var} rule concludes.
    \end{proofcase}

    \noindent \textbf{3.} Pattern-matching on $\Gamma$.
    If $\Gamma$ is empty then $\varepsilon \betared \Gamma^\prime$ forces $\Gamma^\prime = \varepsilon$ and $\vdash \varepsilon$.
    Thus, let $\Gamma = \Delta; x_m : A$.
    \\ \\
    Suppose $\Delta; x_m : A \betared \Delta^\prime; x_m : A$.
    Then by \textit{3} applied to $\Delta$: $\vdash \Delta^\prime$.
    Now, because $\vdash \Delta; x_m : A$ it is the case that $\Delta \vdash A : K$.
    Using \textit{2} gives $\Delta^\prime \vdash A : K$.
    Therefore, $\vdash \Delta^\prime; x_m : A$.
    \\ \\
    Suppose $\Delta; x_m : A \betared \Delta; x_m : A^\prime$.
    Again $\vdash \Delta; x_m : A$ gives $\Delta \vdash A : K$.
    Using \textit{1} gives $\Delta \vdash A^\prime : K$.
    Thus, $\vdash \Delta; x_m : A^\prime$.
\end{proof}

\begin{lemma}
    \label{lem:2:preservation_no_type}
    \textcolor{white}{\_}
    \begin{enumerate}
        \item If $\Gamma \vdash t : A$ and $t \betastar t^\prime$ then $\Gamma \vdash t^\prime : A$
        \item If $\Gamma \vdash t : A$ and $\Gamma \betastar \Gamma^\prime$ then $\Gamma^\prime \vdash t : A$
        \item If $\vdash \Gamma$ and $\Gamma \betastar \Gamma^\prime$ then $\vdash \Gamma^\prime$
    \end{enumerate}
\end{lemma}
\begin{proof}
    For each property the proof proceeds by induction on multistep reduction using Lemma~\ref{lem:2:preservation_no_type_step} and the IH in the inductive case.
\end{proof}

\begin{lemma}
    \label{lem:2:preservation_type}
    If $\Gamma \vdash t : A$ and $A \betastar A^\prime$ then $\Gamma \vdash t : A^\prime$
\end{lemma}
\begin{proof}
    By classification: $\Gamma \vdash A : K$.
    Using Lemma~\ref{lem:2:preservation_no_type} gives $\Gamma \vdash A^\prime : K$.
    Note that $A \equiv A^\prime$.
    Thus, by the \textsc{Conv} rule $\Gamma \vdash t : A^\prime$.
\end{proof}

\begin{theorem}[Preservation]
    \label{lem:2:preservation}
    If $\Gamma \vdash t : A$, $\Gamma \betastar \Gamma^\prime$, $t \betastar t^\prime$, and $A \betastar A^\prime$ then $\Gamma^\prime \vdash t^\prime : A^\prime$
\end{theorem}
\begin{proof}
    Consequence of Lemma~\ref{lem:2:preservation_no_type} and Lemma~\ref{lem:2:preservation_type}.
\end{proof}

\section{Classification}


\begin{figure}
    \centering
    \begin{minipage}{0.5\textwidth}
        \begin{align*}
            \ptrunc{\!\!\pterm} &= x_\star \\
            \ptrunc{\!\!\ptype} &= x_\square
        \end{align*}
    \end{minipage}%
    \begin{minipage}{0.5\textwidth}
        \begin{align*}
            \ptrunc{\!\!\pkind} &= \star \\
            \ptrunc{\text{undefined}} &= \diamond
        \end{align*}
    \end{minipage}
    \caption{Domain and codomains for function types. The variable $K$ is either $\star$ or $\kind$.}
\end{figure}

\begin{figure}
    \centering
    \begin{align*}
        \mathcal{C}(x_\square) &=\!\! \ptype \\
        \mathcal{C}(x_\star) &=\!\! \pterm \\
        \mathcal{C}(\star) &=\!\! \pkind \\
        \mathcal{C}(\diamond) &=\!\! \ptype \\
        \mathcal{C}(\abs{\lambda_\tau}{x}{A}{t}) &=\!\! \ptype &\text{if }& (A\pkind \text{ or } A\ptype) \text{ and } t\ptype \\
        \mathcal{C}(\abs{\lambda_0}{x}{A}{t}) &=\!\! \pterm &\text{if }& (A\pkind \text{ or } A\ptype) \text{ and } t\pterm \\
        \mathcal{C}(\abs{\lambda_\omega}{x}{A}{t}) &=\!\! \pterm &\text{if }& A\ptype \text{ and } t\pterm \\
        \mathcal{C}((x : A) \to_\tau B) &=\!\! \pkind &\text{if }& (A\pkind \text{ or } A\ptype) \text{ and } B\pkind \\
        \mathcal{C}((x : A) \to_0 B) &=\!\! \ptype &\text{if }& (A\pkind \text{ or } A\ptype) \text{ and } B\ptype \\
        \mathcal{C}((x : A) \to_\omega B) &=\!\! \ptype &\text{if }& A\ptype \text{ and } B\ptype \\
        \mathcal{C}(\app{(\abs{\lambda_\tau}{x}{A}{t})}{\tau}{a}) &=\!\! \ptype &\text{if }& (A\pkind \text{ and } a\ptype) \text{ or } (A\ptype \text{ and } a\pterm) \\
            &&&\text{and } t\ptype \text{ and } [x := \ptrunc{\mathcal{C}(a)}]t\ptype \\
        \mathcal{C}(\app{f}{\tau}{a}) &=\!\! \ptype &\text{if }& (a\ptype \text{ or } a\pterm) \text{ and } f\ptype \\
        \mathcal{C}(\app{(\abs{\lambda_0}{x}{A}{t})}{0}{a}) &=\!\! \pterm &\text{if }& (A\pkind \text{ and } a\ptype) \text{ or } (A\ptype \text{ and } a\pterm) \\
            &&&\text{and } t\pterm \text{ and } [x := \ptrunc{\mathcal{C}(a)}]t\pterm \\
        \mathcal{C}(\app{f}{0}{a}) &=\!\! \pterm &\text{if }& (a\ptype \text{ or } a\pterm) \text{ and } f\pterm \\
        \mathcal{C}(\app{(\abs{\lambda_\omega}{x}{A}{t})}{\omega}{a}) &=\!\! \pterm &\text{if }& A\ptype \text{ and } a, t\pterm \text{ and } [x := \ptrunc{\mathcal{C}(a)}]t\pterm \\
        \mathcal{C}(\app{f}{\omega}{a}) &=\!\! \pterm &\text{if }& a\pterm \text{ and } f\pterm
    \end{align*}
    \caption{Domain and codomains for function types. The variable $K$ is either $\star$ or $\kind$.}
\end{figure}

\begin{figure}
    \centering
    \begin{align*}
        \mathcal{C}((x : A) \cap B) &=\!\! \ptype &\text{if }& A\ptype \text{ and } B\ptype \\
        \mathcal{C}([t_1, t_2; A]) &=\!\! \pterm &\text{if }& t_1, t_2\pterm \text{ and } A\ptype \\
        \mathcal{C}(t.1) &=\!\! \pterm &\text{if }& t\pterm \\
        \mathcal{C}(t.2) &=\!\! \pterm &\text{if }& t\pterm \\
        \mathcal{C}(a =_A b) &=\!\! \ptype &\text{if }& a, b\pterm \text{ and } A\ptype \\
        \mathcal{C}(\prefl(t; A)) &=\!\! \pterm &\text{if }& t\pterm \text{ and } A\ptype \\
        \mathcal{C}(\vartheta_i(e, a, b; T)) &=\!\! \pterm &\text{if }& e, a, b\pterm \text{ and } T\ptype \\
        \mathcal{C}(\psi(e, a, b; A, P)) &=\!\! \pterm &\text{if }& e, a, b\pterm \text{ and } A, P\ptype \\
        \mathcal{C}(\varphi(f, e; A, T)) &=\!\! \pterm &\text{if }& f, e\pterm \text{ and } A, T\ptype \\
        \mathcal{C}(\delta(e)) &=\!\! \pterm &\text{if }& e\pterm \\
        \mathcal{C}(t) &= \text{undefined} &&\text{otherwise}
    \end{align*}
    \caption{Domain and codomains for function types. The variable $K$ is either $\star$ or $\kind$.}
\end{figure}



Classification is a critical property of a system like CC with unified syntax.
It allows for the syntax to instead be stratified into levels which would enable an intrinsic presentation of the system.
For the core theory of Cedille2 there are only two universes like the original CC, thus the stratification places terms into three separate levels.
A term is either a \textit{kind} (thus $A = \kind$), a \textit{type} (thus $\Gamma \vdash A : \kind$), or a \textit{term} (thus $\Gamma \vdash A : \star$).
Determining the appropriate level for syntax is also decidable with a classification function defined in Figure~\ref{fig:2:classify}.
This function is crafted to maintain preservation of classification after both reduction and erasure.
Note that because the function is defined on syntax it is possible that there is no valid level because the syntax is not a proof, in these cases the syntax is given the classification \textit{undefined}.

\begin{definition}
    \textcolor{white}{\_}
    \begin{enumerate}
        \item $t\pterm$ iff $\mathcal{C}(t) =\!\!\pterm$
        \item $t\ptype$ iff $\mathcal{C}(t) =\!\!\ptype$
        \item $t\pkind$ iff $\mathcal{C}(t) =\!\!\pkind$
    \end{enumerate}
\end{definition}

Note that the condition $[x := \ptrunc{\mathcal{C}(a)}]t\ptype$ and others like it are necessary.
Take for example $\abs{\lambda_\tau}{x}{\star}{x_\star}$.
This is not well-typed and hence not a proof, but it also should not be a kind, type, or term because it will prevent preservation of classification during reduction.
If $a\pterm$ then the application will correctly produce a term, but if $a\ptype$ then an application will reduce to a type.

\begin{lemma}
    The definition of $\mathcal{C}(-)$ is terminating
\end{lemma}
\begin{proof}
    The definition is structural except application cases.
    In particular, application cases require evaluating $\mathcal{C}([x := \ptrunc{\mathcal{C}(a)}]t)$ for some subexpressions $a$ and $t$.
    Note that computing $\mathcal{C}(-)$ on subxpressions is of course terminating, but moreover $\ptrunc{-}$ is a constant function returning a constant syntactic form.
    Thus, a measure of size can be constructed such that the size of $\ptrunc{\mathcal{C}(a)}$ is always zero for any $a$.
    Substitution of syntactic forms of zero size do not change the size of the resulting term, therefore $\mathcal{C}([x := \ptrunc{\mathcal{C}(a)}]t)$ is a terminating invocation.
\end{proof}

\begin{lemma}
    \label{lem:2:classify_erase}
    If $\mathcal{C}(t)$ is defined then $\mathcal{C}(t) = \mathcal{C}(|t|)$
\end{lemma}
\begin{proof}
    By induction on $t$.
    Type-like syntax is homomorphic and thus the equation holds by the IH.
    Term-like syntax eliminates most of the extra structure leaving behind only another term-like syntax.
    A few cases are presented to illuminate both situations.

    $\text{Case: }t = a =_A b$
    \begin{proofcase}
        Have $|a =_A b| = |a| =_{|A|} |b|$, and because $\mathcal{C}(a =_A b)$ is defined it must be the case that $a, b\pterm$ and $A\ptype$.
        Applying the IH gives $\mathcal{C}(a) = \mathcal{C}(|a|)$, $\mathcal{C}(b) = \mathcal{C}(|b|)$, and $\mathcal{C}(A) = \mathcal{C}(|A|)$.
        Thus, $|a| =_{|A|} |b|\ptype$.
    \end{proofcase}

    $\text{Case: }t = \app{(\abs{\lambda_0}{x}{A}{t})}{0}{a}$
    \begin{proofcase}
        Have $|\app{(\abs{\lambda_0}{x}{A}{t})}{0}{a}| = |t|$ and $t\pterm$.
        Thus, by the IH $|t|\pterm$.
    \end{proofcase}
    
    $\text{Case: }t = \prefl(t; A)$
    \begin{proofcase}
        Have $|\prefl(t; A)| = \abs{\lambda}{x}{\diamond}{x_\star}$, and by computation $\abs{\lambda_\omega}{x}{\diamond}{x_\star}\pterm$.
    \end{proofcase}
\end{proof}

\begin{lemma}
    \label{lem:2:classify_subst}
    If $\mathcal{C}(t)$ and $\mathcal{C}(b)$ are defined then $$\mathcal{C}([x := t]b) = \mathcal{C}([x := \ptrunc{\mathcal{C}(t)}]b)$$
\end{lemma}
\begin{proof}
    If $\mathcal{C}(t)$ is defined then clearly $\mathcal{C}(t) = \mathcal{C}(\ptrunc{\mathcal{C}(t)})$ by definition.
    The lemma is then shown by induction on $b$.
\end{proof}

\begin{lemma}
    \label{lem:2:classify_preservation_step}
    If $\mathcal{C}(s)$ is defined and $s \betared t$ then $\mathcal{C}(s) = \mathcal{C}(t)$
\end{lemma}
\begin{proof}
    By induction on $s \betared t$, note that $\mathcal{C}(-)$ is structural making the inductive cases trivial.
    The first projection case is similar to the second projection case and thus omitted.

    $\text{Case: }\begin{array}{c} \app{(\abs{\lambda_m}{x}{A}{b})}{m}{t} \betared [x := t]b \end{array}$
    \begin{proofcase}
        Suppose wlog that $m = \tau$, then $(\app{(\abs{\lambda_\tau}{x}{A}{b})}{\tau}{t})\ptype$.
        Note that $t\ptype$ or $t\pterm$ by unraveling the previous definition.
        Now $[x := \ptrunc{\mathcal{C}(t)}]b\ptype$.
        By Lemma~\ref{lem:2:classify_subst} and the above observation: $[x := t]b\ptype$.
    \end{proofcase}

    $\text{Case: }\begin{array}{c} [t_1, t_2; A].2 \betared t_2 \end{array}$
    \begin{proofcase}
        Have $[t_1, t_2; A]\pterm$ and by deconstructing the definition $t_2\pterm$.
    \end{proofcase}

    $\text{Case: }\begin{array}{c} \app{\psi(\text{refl}(z; Z), a, b; A, P)}{\omega}{t} \betared t \end{array}$
    \begin{proofcase}
        Have $(\app{\psi(\text{refl}(z; Z), a, b; A, P)}{\omega}{t})\pterm$ and by deconstruction the definition $t\pterm$.
    \end{proofcase}

    $\text{Case: }\begin{array}{c} \vartheta(\text{refl}(z; Z), a, b; T) \betared \text{refl}(a; T) \end{array}$
    \begin{proofcase}
        Have $\vartheta(\text{refl}(z; Z), a, b; T)\pterm$ and by deconstruction the definition $a\pterm$ and $T\ptype$.
        Thus, $\text{refl}(a; T)\pterm$.
    \end{proofcase}
\end{proof}

\begin{lemma}
    \label{lem:2:classify_preservation}
    If $\mathcal{C}(s)$ is defined and $s \betastar t$ then $\mathcal{C}(s) = \mathcal{C}(t)$
\end{lemma}
\begin{proof}
    By induction on $s \betastar t$ and Lemma~\ref{lem:2:classify_preservation_step}.
\end{proof}

\begin{theorem}[Soundness of $\mathcal{C}(-)$]
    \label{lem:2:classify_soundness}
    \textcolor{white}{\_}
    \begin{enumerate}
        \item If $\Gamma \vdash t : A$ and $A = \kind$ then $t\pkind$
        \item If $\Gamma \vdash t : A$ and $\Gamma \vdash A : \kind$ then $t\ptype$
        \item If $\Gamma \vdash t : A$ and $\Gamma \vdash A : \star$ then $t\pterm$
    \end{enumerate}
\end{theorem}
\begin{proof}
    By induction on $\Gamma \vdash t : A$.
    The \textsc{Fst} amd \textsc{PrmFst} rules are omitted.

    $\text{Case: }\begin{array}{c} \AxiomRule[*] \end{array}$
    \begin{proofcase}
        Have $\star\pkind$ and $A = \kind$, hence trivial.
    \end{proofcase}

    $\text{Case: }\begin{array}{c} \VarRule[*] \end{array}$
    \begin{proofcase}
        If $K = \kind$ then $x_\square\ptype$ and $\Gamma \vdash A : \kind$.
        Otherwise, $K = \star$ and then $x_\star\pterm$ with $\Gamma \vdash A : \star$.
    \end{proofcase}

    $\text{Case: }\begin{array}{c} \PiRule[*] \end{array}$
    \begin{proofcase}
        Suppose wlog that $m = \tau$, now by the IH applied to $\D{1}$: $A\pkind$ or $A\ptype$.
        Applying the IH to $\D{2}$ gives $B\pkind$.
        Thus, $(x : A) \to_\tau B\pkind$.
    \end{proofcase}

    $\text{Case: }\begin{array}{c} \LambdaRule[*] \end{array}$
    \begin{proofcase}
        Suppose wlog that $m = \tau$.
        Applying the IH to $\D{1}$ gives $A\pkind$ or $A\ptype$.
        Note by $\D{2}$ that $\Gamma, x_\tau : A \vdash B : \kind$.
        Thus, applying the IH to $\D{2}$ yields $t\ptype$.
        Hence, $\abs{\lambda_\tau}{x}{A}{t}\ptype$.
    \end{proofcase}

    $\text{Case: }\begin{array}{c} \AppRule[*] \end{array}$
    \begin{proofcase}
        Suppose wlog that $m = \tau$.
        By classification and inversion with $\D{1}$: $\Gamma \vdash (x : A) \to_\tau B : \kind$.
        Deconstructing this judgment yields $\Gamma \vdash A : K$.
        Applying the IH to $\D{2}$ gives $a\ptype$ or $a\pterm$.
        Applying the IH to $\D{1}$ yields $f\ptype$.
        If $f$ is not an abstraction then the proof is done, thus suppose $f = \abs{\lambda}{x}{A}{t}$.
        Have $A\pkind$ or $A\ptype$, but note that $\Gamma \vdash A : K$ thus the classification of $a$ and $A$ must agree.
        Moreover, $t\pterm$.
        Suppose wlog that $a\ptype$ then $\ptrunc{\mathcal{C}(a)} = x_\square$.
        However, this means that $\Gamma \vdash A : \kind$ and that $\Gamma, x_\tau : A \vdash t : B$.
        Thus, the occurrences of $x$ in $t$ must be annotated as $x_\square$ otherwise the \textsc{Var} rule for $x$ would fail.
        Hence, $[x := x_\square]t = t$.
    \end{proofcase}

    $\text{Case: }\begin{array}{c} \IntersectionRule[*] \end{array}$
    \begin{proofcase}
        Applying the IH to $\D{1}$ and $\D{2}$ gives $A, B\ptype$.
        Hence, $(x : A) \cap B\ptype$.
    \end{proofcase}

    $\text{Case: }\begin{array}{c} \PairRule[*] \end{array}$
    \begin{proofcase}
        Deconstructing $\D{1}$ gives $\Gamma \vdash A : \star$ and $\Gamma, x : A \vdash B : \star$.
        Lemma~\ref{lem:2:subst} gives $\Gamma \vdash [x := t]B : \star$.
        Using the IH on $\D{1}$, $\D{2}$, and $\D{3}$ yields $(x : A) \cap B\ptype$ and $t, s\pterm$.
        Thus, $[t, s; (x : A) \cap B]\pterm$.
    \end{proofcase}

    $\text{Case: }\begin{array}{c} \SecondRule[*] \end{array}$
    \begin{proofcase}
        By classification and inversion on $\D{1}$: $\Gamma \vdash (x : A) \cap B : \star$.
        Using the IH on $\D{1}$ gives $t\pterm$.
        Hence, $t.2\pterm$.
    \end{proofcase}

    $\text{Case: }\begin{array}{c} \EqualityRule[*] \end{array}$
    \begin{proofcase}
        Applying the IH to $\D{1}$, $\D{2}$, and $\D{3}$ yields $A\ptype$ and $a, b\pterm$.
        Hence, $a =_A b\ptype$.
    \end{proofcase}

    $\text{Case: }\begin{array}{c} \ReflRule[*] \end{array}$
    \begin{proofcase}
        Applying the IH to $\D{1}$ and $\D{2}$ gives $A\ptype$ and $t\pterm$.
        Hence, $\prefl(t; A)\pterm$.
    \end{proofcase}

    $\text{Case: }\begin{array}{c} \SubstRule[*] \end{array}$
    \begin{proofcase}
        Classification and inversion on $\D{4}$ gives $\Gamma \vdash a =_A b : \star$.
        Likewise, $\Gamma \vdash (y : A) \to_\tau (p : a =_A y_\star) \to_\tau \star : \kind$.
        Applying the IH to all subderivations yields $A, P\ptype$ and $a, b, e\pterm$.
        Hence, $\psi(e, a, b; A, P)\pterm$.
    \end{proofcase}

    $\text{Case: }\begin{array}{c} \PromoteRule[*] \end{array}$
    \begin{proofcase}
        By classification, inversion and the IH used with $\D{4}$: $e\pterm$.
        The IH applied to $\D{1}$, $\D{2}$ and $\D{3}$ yields $a, b\pterm$ and $(x : A) \cap B\ptype$.
    \end{proofcase}

    $\text{Case: }\begin{array}{c} \CastRule[*] \end{array}$
    \begin{proofcase}
        By the IH applied to $\D{1}$: $(x : A) \cap B\ptype$.
        Thus, by the IH and inversion applied to the remaining derivations: $a, b, e\pterm$.
    \end{proofcase}

    $\text{Case: }\begin{array}{c} \SeparationRule[*] \end{array}$
    \begin{proofcase}
        Classification, inversion, and the IH applied to $\D{1}$ gives $e\pterm$.
        Hence, $\delta(e)\pterm$.
    \end{proofcase}

    $\text{Case: }\begin{array}{c} \ConvRule[*] \end{array}$
    \begin{proofcase}
        Classification, inversion, $\D{1}$ and $\D{3}$ yield $\Gamma \vdash B : K$.
        Suppose wlog that $K = \star$.
        Applying the IH to $\D{2}$ gives $t\pterm$.
    \end{proofcase}
\end{proof}

\section{Derivations}

\begin{align*}
    \pCast &= \abs{\lambda_\tau}{A}{\star}{
        \abs{\lambda_\tau}{B}{\star}{
            (f : A_\square \to_\tau A_\square \cap B_\square)
            \cap ((a:A_\square) \to_0 a_\star = (\app{f_\star}{0}{a_\star}).1)
        }
    } \\
    \pcastIrrel &= \abs{\lambda_0}{A}{\star}{
        \abs{\lambda_0}{B}{\star}{
            \abs{\lambda_0}{k}{\apptwo{\pCast}{\tau}{A_\square}{\tau}{B_\square}}{
                \\ &[\abs{\lambda_\omega}{a}{A_\square}{
                    \varphi(a_\star, \app{k_\star.1}{\omega}{a_\star}, \app{k_\star.2}{0}{a_\star})
                }, \abs{\lambda_0}{a}{A_\square}{
                    \prefl(a_\star; A_\square)
                };
                \apptwo{\pCast}{\tau}{A_\square}{\tau}{B_\square}]
            }
        }
    }
\end{align*}

\begin{theorem}
    \label{lem:2:cast_derivations}
    \textcolor{white}{\_}
    \begin{enumerate}
        \item $\pCast : \star \to_\tau \star \to_\tau \star$
        \item $\pcastIrrel : (A:\star) \to_0 (B:\star) \to_0 \apptwo{\pCast}{\tau}{A_\square}{\tau}{B_\square} \to_0 \apptwo{\pCast}{\tau}{A_\square}{\tau}{B_\square}$
    \end{enumerate}
\end{theorem}

\begin{align*}
    \pFalse &= (X:\star) \to_0 X_\square \\
    \pTop &= \pFalse \to_0 \pFalse \\
    \ptopInj &= \abs{\lambda_0}{A}{\star}{
        \abs{\lambda_\omega}{a}{A_\square}{
            \\ &[
                \abs{\lambda_0}{f}{\pFalse}{
                    \varphi(a_\star,
                        \app{f_\star}{0}{(A_\square \cap \pFalse)},
                        \app{f_\star}{0}{(a_\star = (\app{f_\star}{0}{(A_\square \cap \pFalse)}).1 )}.2 )
                },
                a_\star;
                A_\square \cap \pFalse
            ]
        }
    } \\
    \pView &= \abs{\lambda_\tau}{A}{\star}{
        \abs{\lambda_\tau}{x}{\pTop}{
            (z : (\pTop \cap A_\square)) \times (x_\star = z_\star.1)
        }
    } \\
    \pintrView &= \abs{\lambda_0}{A}{\star}{
        \abs{\lambda_\omega}{x}{\pTop}{
            \abs{\lambda_0}{a}{A}{
                \abs{\lambda_0}{e}{(x_\star = (\apptwo{\ptopInj}{0}{A_\square}{\omega}{a_\star}).1)}{
                    \\ &\apptwo{\psigma}
                        {\omega}{\varphi(x_\star, \apptwo{\ptopInj}{0}{A_\square}{\omega}{a_\star}, e_\star)}
                        {\omega}{\prefl(x_\star; \pTop)}
                }
            }
        }
    } \\
    \pelimView &= \abs{\lambda_0}{A}{\star}{
        \abs{\lambda_\omega}{b}{\pTop}{
            \abs{\lambda_0}{v}{\apptwo{\pView}{\tau}{A_\square}{\tau}{b_\star}}{
                \varphi(b_\star, \app{\pdfst}{\omega}{v}, \app{\pdsnd}{\omega}{v}).2
            }
        }
    }
\end{align*}

\begin{theorem}
    \label{lem:2:view_derivations}
    \textcolor{white}{\_}
    \begin{enumerate}
        \item $\pFalse : \star$
        \item $\pTop : \star$
        \item $\ptopInj : (A:\star) \to_0 A_\square \to_\omega \pTop \cap A_\square$
        \item $\pView : \star \to_\tau \pTop \to_\tau \star$
        \item {
            $
                \pintrView : 
                \begin{aligned}
                    &(A:\star) \to_0 (x:\pTop) \to_\omega (a:A_\square) \to_0
                        \\ &(x_\star = (\apptwo{\ptopInj}{0}{A_\square}{\omega}{a_\star}).1) \to_0 \apptwo{\pView}{\tau}{A_\square}{\tau}{x_\star}
                \end{aligned}
            $
        }
        \item $\pelimView : (A:\star) \to_0 (b:\pTop) \to_\omega \apptwo{\pView}{\tau}{A_\square}{\tau}{b_\star} \to_0 A$
    \end{enumerate}
\end{theorem}


\chapter{Proof Normalization}
\label{chap:3}

\Rule{\PiRuleFpOne}
    {
        \der{\D{1}}{\Gamma \vdash A : \kind}
        \\ \der{\D{2}}{\Gamma, x : A \vdash B : \kind}
    }
    {\Gamma \vdash (x : A) \to B : \kind}
    {Pi1}

\Rule{\PiRuleFpTwo}
    {
        \der{\D{1}}{\Gamma \vdash A : K}
        \\ \der{\D{2}}{\Gamma, x : A \vdash B : \star}
    }
    {\Gamma \vdash (x : A) \to B : \star}
    {Pi2}

\Rule{\LambdaRuleFp}
    {
        \der{\D{1}}{\Gamma \vdash (x:A) \to B : K}
        \\ \der{\D{2}}{\Gamma, x : A \vdash t : B}
    }
    {\Gamma \vdash \abs{\lambda}{x}{A}{t} : (x : A) \to B}
    {Lam}

\Rule{\AppRuleFp}
    {
        \der{\D{1}}{\Gamma \vdash f : (x : A) \to B} \\
        \der{\D{2}}{\Gamma \vdash a : A}
    }
    {\Gamma \vdash f\ a : [x := a]B}
    {App}

\Rule{\AxiomRuleFp}
    {\textcolor{white}{\_}}
    {\Gamma \vdash \star : \kind}
    {Axiom}

\Rule{\VarRuleFp}
    {
       \der{\D{1}}{(x : A) \in \Gamma}
    }
    {\Gamma \vdash x : A}
    {Var}

\Rule{\ConvRuleFp}
    {
        \der{\D{1}}{\Gamma \vdash A : K} \\
        \der{\D{2}}{\Gamma \vdash t : B} \\
        \der{\D{3}}{A \betaconv B}
    }
    {\Gamma \vdash t : A}
    {Conv}

\Rule{\ProductRuleFp}
    {
        \der{\D{1}}{\Gamma \vdash A : \star} \\
        \der{\D{2}}{\Gamma \vdash B : \star}
    }
    {\Gamma \vdash A \times B : \star}
    {Int}

\Rule{\PairRuleFp}
    {
        \der{\D{1}}{\Gamma \vdash A \times B : \star} \\
        \der{\D{2}}{\Gamma \vdash t : A} \\
        \der{\D{3}}{\Gamma \vdash s : B}
    }
    {\Gamma \vdash (t, s) : A \times B }
    {Pair}

\Rule{\FirstRuleFp}
    {\der{\D{1}}{\Gamma \vdash t : A \times B}}
    {\Gamma \vdash t.1 : A}
    {Fst}

\Rule{\SecondRuleFp}
    {\der{\D{1}}{\Gamma \vdash t : A \times B}}
    {\Gamma \vdash t.2 : B}
    {Snd}



There are several techniques for showing strong normalization of a PTS, including saturated sets \cite{geuvers1994_sn_satset}, model theory \cite{terlouw1995_sn}, realizability \cite{ong1993}, etc.
Geuvers and Nederhof describe a technique that models CC inside F$^\omega$ where term dependencies are all erased at the type level \cite{geuvers1991_sn_tof}.
In this chapter the technique of Geuvers and Nederhof will be adapted to show strong normalization of proof reduction.
Proof normalization ends up being a rather weak property, because it does not entail consistency.

\section{Model Description}

Figure~\ref{fig:syntax_fp} describes the syntax of System F$^\omega$ augmented with pairs.
The reduction relation for this system is presented in Figure~\ref{fig:reduction_fp} and the inference judgment in Figure~\ref{fig:typing_fp}.
System F$^\omega$ augmented with pairs is only slightly different from the original PTS description of F$^\omega$.
Moreover, it is a subsystem of the Calculus of Inductive Constructions and thus enjoys various metatheoretic properties such as substitution and weakening lemmas, preservation, strong normalization, and consistency.


\begin{figure}
    \centering
    \begin{align*}
        t &::= x\ |\ \mathfrak{b}(\kappa_1, x : t_1, t_2)\ |\ \mathfrak{c}(\kappa_2, t_1, \ldots, t_{\mathfrak{a}(\kappa_2)}) \\
        \kappa_1 &::= \lambda\ |\ \Pi \\
        \kappa_2 &::=\star\ |\ \kind\ |\ \text{app}\ |\ \text{prod}\ |\ \text{pair}\ |\ \text{fst}\ |\ \text{snd} \\
        &\mathfrak{a}(\star) = \mathfrak{a}(\kind) = 0 \\
        &\mathfrak{a}(\text{fst}) = \mathfrak{a}(\text{snd}) = 1 \\
        &\mathfrak{a}(\text{app}) = \mathfrak{a}(\text{prod}) = \mathfrak{a}(\text{pair}) = 2
    \end{align*}
    \vspace{-.4in}
    \begin{minipage}{0.5\textwidth}
        \begin{align*}
            \star &:= \mathfrak{c}(\star) \\
            \kind &:= \mathfrak{c}(\kind) \\
            \abs{\lambda}{x}{t_1}{t_2} &:= \mathfrak{b}(\lambda, x : t_1, t_2) \\
            (x : t_1) \to t_2 &:= \mathfrak{b}(\Pi, x : t_1, t_2) \\
            t_1\ t_2 &:= \mathfrak{c}(\text{app}, t_1, t_2)
        \end{align*}
    \end{minipage}%
    \begin{minipage}{0.5\textwidth}
        \begin{align*}
            t_1 \times t_2 &= \mathfrak{c}(\text{prod}, t_1, t_2) \\
            (t_1, t_2) &= \mathfrak{c}(\text{pair}, t_1, t_2) \\
            t.1 &= \mathfrak{c}(\text{fst}, t) \\
            t.2 &= \mathfrak{c}(\text{snd}, t)
        \end{align*}
    \end{minipage}
    \caption{Syntax for System F$^\omega$ with pairs.}
    \label{fig:syntax_fp}
\end{figure}


\begin{figure}
    \centering
    \begin{minipage}{0.5\textwidth}
        $$\BinderOneReduction$$
    \end{minipage}%
    \begin{minipage}{0.5\textwidth}
        $$\BinderTwoReduction$$
    \end{minipage}%
    $$\ConstructorReduction$$
    \begin{align*}
        (\abs{\lambda}{x}{A}{b})\ t &\betared [x := t]b \\
        [t_1, t_2].1 &\betared t_1 \\
        [t_1, t_2].2 &\betared t_2
    \end{align*}
    \caption{Reduction rules for System F$^\omega$ with pairs.}
    \label{fig:reduction_fp}
\end{figure}




\begin{figure}
    \centering
    \begin{minipage}{0.5\textwidth}
        $$\AxiomRuleFp$$
        $$\VarRuleFp$$
        $$\PiRuleFpOne$$
        $$\LambdaRuleFp$$
        $$\FirstRuleFp$$
        $$\SecondRuleFp$$
    \end{minipage}%
    \begin{minipage}{0.5\textwidth}
        $$\ConvRuleFp$$
        $$\PiRuleFpTwo$$
        $$\AppRuleFp$$
        $$\ProductRuleFp$$
        $$\PairRuleFp$$
    \end{minipage}%
    \caption{
        Typing rules for System F$^\omega$ with pairs.
    }
    \label{fig:typing_fp}
\end{figure}


The model follows all the same principles for the CC fragment of $\ced$.
For example, consider the \textsc{Lam} rule.
$$\LambdaRule$$
The goal is to find three semantic functions: one for kinds ($V(-)$); one for types ($\sema{-}$); and one for terms ($[-]$), such that:
    \begin{enumerate}
        \item $\sema{\Gamma} \vdash_\omega \sema{(x : A) \to_m B} : V(\pcodom(m))$
        \item $\sema{\Gamma; x_m : A} \vdash_\omega [t] : \sema{B}$
        \item $\sema{\Gamma} \vdash_\omega [\abs{\lambda_m}{x}{A}{t}] : \sema{(x : A) \to_m B}$
    \end{enumerate}
In order for this to work, term dependencies must all be dropped in function types.
Moreover, kinds are squished, such that $V(\kind) = V(\star) = \star$.
Thus, the judgment $\sema{\Gamma} \vdash_\omega \sema{(x : A) \to_m B} : V(\pcodom(m))$ must form an F$^\omega$ type.
The kind and type semantics is allowed to discard reductions because it only serves the purpose of maintaining a well-typed output.
Instead, it is the term semantics that must take care to preserve all possible reductions such that strong normalization is a consequence of the model.

For dependent intersections, the type semantics is the obvious one: $\sema{(x : A) \cap B} = \sema{A} \times \sema{B}$.
Note that because $A$ is a type, it must be the case that $x \notin FV(\sema{B})$ otherwise the resulting type is not well-formed in F$^\omega$.
This is true already for function types, thus this extension needs no special treatment.
For equality the situation is special, the approach taken is to interpret all equalities as the type of the identity function: $\sema{a =_A b} = \fpId$.
There does not appear to be a more sensible choice, as the dependencies $a$ and $b$ must be dropped.


\begin{figure}
    \centering
    \begin{align*}
        V(\kind) &= \star \\
        V(\star) &= \star \\
        V((x : A) \to_m B) &= V(A) \to V(B) &\text{if} \text{ $A$ kind} \\
        V((x : A) \to_m B) &= V(B) &\text{otherwise}
    \end{align*}
    \begin{align*}
        \sema{\kind} &= 0 \\
        \sema{\star} &= 0 \\
        \sema{x_\square} &= x \\
        \sema{(x : A) \to_m B} &= (x : V(A)) \to \sema{A} \to \sema{B} &\text{if}& \text{ $A$ kind} \\
        \sema{(x : A) \to_m B} &= (x : \sema{A}) \to \sema{B} &\text{if}& \text{ $A$ type} \\
        \sema{\abs{\lambda_\tau}{x}{A}{t}} &= \abs{\lambda}{x}{V(A)}{\sema{t}} &\text{if}& \text{ $A$ kind} \\
        \sema{\abs{\lambda_\tau}{x}{A}{t}} &= \sema{t} &\text{if}& \text{ $A$ type} \\
        \sema{\app{f}{\tau}{a}} &= \sema{f}\ \sema{a} &\text{if}& \text{ $a$ type} \\
        \sema{\app{f}{\tau}{a}} &= \sema{f} &\text{if}& \text{ $a$ term} \\
        \sema{(x : A) \cap B} &= \sema{A} \times \sema{B} \\
        \sema{a =_A b} &= \fpId
    \end{align*}
    \begin{align*}
        \sema{x_m : A} &= x : V(A), w_x : \sema{A} &\text{if}& \text{ $A$ kind} \\
        \sema{x_m : A} &= x : \sema{A} &\text{if}& \text{ $A$ type} \\
        \sema{\varepsilon} &= 0 : \star, \bot : (X : \star) \to X \\
        \sema{\Gamma, x_m : A} &= \sema{\Gamma}, \sema{x_m : A}
    \end{align*}
    \caption{
        Model for kinds and types, not that type dependencies are dropped.
        Define $\fpId := (X : \star) \to X \to X$.
    }
    \label{fig:3:model_types}
\end{figure}

\begin{figure}
    \centering
    \begin{align*}
        c^B &= \bot\ B \hspace{0.5in}\text{if $B$ type} \\
        c^\star &= 0 \\
        c^{(x : A) \to B} &= \abs{\lambda}{x}{A}{c^B}
    \end{align*}
    \begin{align*}
        [*] &= c^0 \\
        [ x_{\square} ] &= w_x \\
        [ x_\star ] &= x \\
        [(x : A) \to_m B] &= c^{0 \to 0 \to 0}\ [A]\ ([x := c^{V(A)}][w_x := c^{\sema{A}}][B]) &\text{if}& \text{ $A$ kind} \\
        [(x : A) \to_m B] &= c^{0 \to 0 \to 0}\ [A]\ ([x := c^{\sema{A}}][B]) &\text{if}& \text{ $A$ type} \\
        [\abs{\lambda_m}{x}{A}{t}] &= (\abs{\lambda}{y}{0}{
            \abs{\lambda}{x}{V(A)}{
                \abs{\lambda}{w_x}{\sema{A}}{[t]}
            }
        })\ [A] &\text{if}& \text{ $A$ kind} \\
        [\abs{\lambda_m}{x}{A}{t}] &= (\abs{\lambda}{y}{0}{
            \abs{\lambda}{x}{\sema{A}}{[t]}
        })\ [A] &\text{if}& \text{ $A$ type} \\
        [\app{f}{m}{a}] &= [f]\ \sema{a}\ [a] &\text{if}& \text{ $a$ type} \\
        [\app{f}{m}{a}] &= [f]\ [a] &\text{if}& \text{ $a$ term} \\
    \end{align*}
    \begin{align*}
        [(x : A) \cap B] &= c^{0 \to 0 \to 0}\ [A]\ ([x := c^{\sema{A}}][B]) \\
        [[t_1, t_2; A]] &= (\abs{\lambda}{y}{0}{([t_1], [t_2])})\ [A] \\
        [t.1] &= [t].1 \\
        [t.2] &= [t].2 \\
        [a =_A b] &= c^{0 \to \sema{A} \to \sema{A} \to 0}\ [A]\ [a]\ [b] \\
        [\text{refl}(t; A)] &= (\abs{\lambda}{y_1}{0}{
                \abs{\lambda}{y_2}{\sema{A}}{\fpid}
            })\ [A]\ [t] \\
        [\psi(e, a, b; A, P)] &= (\abs{\lambda}{y_1}{0}{
                \abs{\lambda}{y_2\ y_3}{\sema{A}}{
                    \abs{\lambda}{y_2}{\sema{A} \to \fpId \to 0}{
                        [e]\ \sema{P}
                    }
                }
            })\ [A]\ [a]\ [b]\ [P] \\
        [\vartheta(e, a, b; T)] &= (\abs{\lambda}{y_1}{\sema{T}}{
            \abs{\lambda}{y_2}{0} {
                \abs{\lambda}{y_3}{\sema{T}}{
                    [e]
                }
            }
        })\ [b]\ [T]\ [a] \\
        [\varphi(a, b, e)] &= (\abs{\lambda}{y}{\fpId}{
            ([a], [b].2)
        })\ [e] \\
        [\delta(e)] &= (\abs{\lambda}{y}{\fpId}{
            \bot
        })\ [e]
    \end{align*}
    \caption{
        Model for terms, note that critically every subexpression is represented in the model to make sure no reductions are potentially lost.
        The definition of $c$ is used to construct a canonical element for any kind or type.
        Define $\fpid := \abs{\lambda}{X}{\star}{
            \abs{\lambda}{x}{X}{x}
        }$.
    }
    \label{fig:3:model_terms}
\end{figure}


The model interpretation for contexts always introduces two fresh variables, $0 : \star$ which is a canonical type, and $\bot : (X : \star) \to X$ which is used to construct canonical inhabitants for a type.
Note that including $\bot$ prevents this model from entailing consistency for the source system.
Regardless, F$^\omega$ is strongly normalizing in all contexts, thus the addition of $\bot$ does not prevent the model from serving its current purpose.
Before exploring more in-depth examples of the model the reader is invited to skim the semantic functions in Figure~\ref{fig:3:model_types} and Figure~\ref{fig:3:model_terms}.

Consider the following examples to garner intuition for the semantic model:
\begin{enumerate}
    \item {
        Given $\varepsilon \vdash_{\ced} \abs{\lambda_0}{X}{\star}{
            \abs{\lambda_\omega}{x}{X_\square}{x_\star}
        } : (X : \star) \to_0 X_\square \to_\omega X_\square$
        then $$\sema{\varepsilon} = 0 : \star;\ \bot : (X : \star) \to X$$
        $$[\abs{\lambda_0}{X}{\star}{
            \abs{\lambda_\omega}{x}{X_\square}{x_\star}
        }] = (\abs{\lambda}{y}{0}{
            \abs{\lambda}{X}{\star}{
                \abs{\lambda}{w_X}{0}{
                    (\abs{\lambda}{y}{0}
                        \abs{\lambda}{x}{X}{x}
                    )\ w_X
                }
            }
        })\ c^0$$
        $$\sema{(X : \star) \to_0 X_\square \to_\omega X_\square}
            = (X : \star) \to 0 \to X \to X$$
    }
    \item {
        Given $\Gamma \vdash_{\ced} t : T$ where $\Gamma = A : \star;\ B : \star;\ a : A_\square;\ f : A_\square \to_\omega (x : A_\square) \cap B_\square$, $t = [(\app{f_\star}{\omega}{a_\star}).1, (\app{f_\star}{\omega}{a_\star}).2; (x : A_\square) \cap B_\square]$, and $T = (x : A_\square) \cap B_\square$ then
        \begin{align*}
            &\sema{A : \star;\ B : \star;\ a : A_\square;\ f : A_\square \to_\omega (x : A_\square) \cap B_\square} =\\
            &\quad 0 : \star;\ \bot : (X : \star) \to X;\ A : \star;\ w_A : 0;\ B : \star;\ w_B : 0; \\
            &\quad a : A;\ f : A \to A \times B
        \end{align*}
            $$[[(\app{f_\star}{\omega}{a_\star}).1, (\app{f_\star}{\omega}{a_\star}).2; (x : A_\square) \cap B_\square]] = (\abs{\lambda}{y}{0}{
                ((f\ a).1, (f\ a).2)
            })\ (c^{0 \to 0 \to 0}\ w_A\ w_B)$$
        $$\sema{(x : A_\square) \cap B_\square} = A \times B$$
    }
\end{enumerate}

Notice that from the perspective of the type semantics ($\sema{-}$) that term dependencies in predicates must be dropped, but that they are preserved in the term semantics ($[-]$).
Thus, extra layers of abstraction are added when interpreting function arguments that are kinds.

\section{Model Soundness}

With the model defined the next step is to prove it is sound.
The process begins by showing the interpretation of kinds ($V(-)$) is sound.
This is not particularly difficult as the kind interpretation is quite simple.
After, lemmas about substitution and conversion follow without difficulty.

\begin{theorem}[Soundness of $V$]
    \label{lem:3:soudness_of_v}
    If $\Gamma \vdash_{\ced} t : \kind$ then $\Delta \vdash_\omega V(t) : \kind$ for any $\Delta$
\end{theorem}
\begin{proof}
    By induction on $\Gamma \vdash_{\ced} t : \kind$.
    The cases: \textsc{Lam}, \textsc{App}, \textsc{Int}, \textsc{Pair}, \textsc{Fst}, \textsc{Snd}, \textsc{Eq}, \textsc{Refl}, \textsc{Subst}, \textsc{Prm}, \textsc{Cast}, \textsc{Sep}, and \textsc{Conv} are impossible by inversion.

    $\text{Case: }\begin{array}{c} \AxiomRule[*] \end{array}$
    \begin{proofcase}
        Trivial by the \textsc{Ax} rule.
    \end{proofcase}

    $\text{Case: }\begin{array}{c} \VarRule[*] \end{array}$
    \begin{proofcase}
        By $\D{2}$: $\Gamma \vdash \kind : K$ which is impossible.
    \end{proofcase}

    $\text{Case: }\begin{array}{c} \PiRule[*] \end{array}$
    \begin{proofcase}
        Suppose $A$ is a kind, then $\pdom(m, K) = \kind$ and $V((x : A) \to_m B) = V(A) \to V(B)$.
        Applying the IH to $\D{1}$ and $\D{2}$ gives $\Delta_1 \vdash_\omega V(A) : \kind$ and $\Delta_2 \vdash_\omega V(B) : \kind$.
        However, note that there are no variables in any well-defined $V(t)$ which $V(A)$ and $V(B)$ are.
        Thus, $\Delta \vdash_\omega V(A) : \kind$ and $\Delta, x : V(A) \vdash_\omega V(B) : \kind$ by properties of $F^\omega$.
        Now by the \textsc{Pi1} rule $\Delta \vdash_\omega V(A) \to V(B) : \kind$ as required.
        \\ \\
        Suppose $A$ is a type, then $\pdom(m, K) = \star$ and $V((x : A) \to_m B) = V(B)$.
        By the IH applied to $\D{2}$: $\Delta \vdash_\omega V(B) : \kind$.
    \end{proofcase}
\end{proof}

\begin{lemma}
    \label{lem:3:conv_of_v}
    If $\Gamma_1 \vdash_{\ced} A : \kind$, $\Gamma_2 \vdash_{\ced} B : \kind$, and $A \equiv B$ then $V(A) = V(B)$
\end{lemma}
\begin{proof}
    By induction on $\Gamma \vdash A : \kind$.
    Note that $A$ is either $\star$ or $(x : C) \to_\tau D$.
    Suppose $A = \star$, then because $\star \equiv B$ it must be that $B = \star$.
    Thus, $V(A) = \star = V(B)$.
    \\ \\
    Suppose $A = (x : C_1) \to_\tau D_1$, but this forces $B = (x : C_2) \to_\tau D_2$ where $C_1 \equiv C_2$ and $D_1 \equiv D_2$.
    Note that $\Gamma \vdash C_1 : K$ and $\Gamma, x : C_1 \vdash D_1 : \kind$.
    Now by the IH: $V(D_1) = V(D_2)$ (note that the contexts need not agree).
    If $C_1$ is a kind, then $V((x : C_1) \to_\tau D_1) = V(C_1) \to V(D_1)$ and by the IH $V(C_1) = V(C_2)$.
    Instead, if $C_1$ is a type then $V((x : C_1) \to_\tau D_1) = V(D_1)$, but $V(D_1) = V(D_2)$.
    Thus, $V(A) = V((x : C_1) \to_\tau D_1) = V((x : C_2) \to_\tau D_2) = V(B)$.
\end{proof}

\begin{lemma}
    \label{lem:3:v_subst}
    If $\Gamma \vdash_\omega V(t) : \kind$ then $[x := b]V(t) = V(t) = V([x := b]t)$
\end{lemma}
\begin{proof}
    By induction on $t$ and inversion on $\Gamma \vdash V(t) : \kind$.
    Note that there are only two possibilities:

    $\text{Case: } t = \star$
    \begin{proofcase}
        Have $[x := b]V(\star) = [x := b]\star = \star = V(\star) = V([x := b]\star)$.
    \end{proofcase}

    $\text{Case: }\begin{array}{c} t = (x : A) \to_m B \end{array}$
    \begin{proofcase}
        Note that $A$ must be a kind or a type because $\Gamma \vdash V(t) : \kind$.
        Suppose $A$ is a kind, then $V((x : A) \to_m B) = V(A) \to V(B)$.
        Destructing the judgment gives $\Gamma \vdash V(A) : \kind$ and $\Gamma, x : V(A) \vdash V(B) : \kind$.
        Thus, by the IH: $[x := b]V(A) = V(A) = V([x := b]A)$ and $[x := b]V(B) = V(B) = V([x := b]B)$.
        By computation, $V([x := b](x : A) \to_m B) = V((x : [x := b]A) \to_m [x := b]B) = V([x := b]A) \to V([x := b]B) = V(A) \to V(B) = V((x : A) \to_m B)$.
        Also, by computation $[x := b]V((x : A) \to_m B) = [x := b](V(A) \to V(B)) = [x := b]V(A) \to [x := b]V(B) = V(A) \to V(B) = V((x : A) \to_m B)$. 
        \\ \\
        Suppose $A$ is a type, then $V((x : A) \to_m B) = V(B)$.
        By the IH: $[x := b]V(B) = V(B) = V([x := b]B)$.
    \end{proofcase}
\end{proof}

Next is demonstrating soundness of the type semantics.
Note again that type variables cannot appear free in the result of a well-defined interpretation of types.
This is codified in the next lemma, and soundness follows from it and soundness of the model for kinds.
A standard substitution lemma is proven after.

\begin{lemma}
    \label{lem:3:var_not_in_sema_when_kind}
    Suppose $\Gamma \vdash_{\ced} t : A$, $x_m : B \in \Gamma$, and $B$ type, then $x \notin FV(\sema{t})$
    where $A = \kind$ or $\Gamma \vdash_{\ced} A : \kind$
\end{lemma}
\begin{proof}
    The restrictions on $A$ makes sure that $\sema{-}$ is well-defined.
    The definition of $\sema{-}$ intentionally throws away any dependence on terms.
    Thus, if $x$ is a term, because $B$ is a type, the only places where $x$ may appear in $t$ have all been thrown away.
    Therefore, $x \notin FV(\sema{t})$.
\end{proof}

\begin{theorem}[Soundness of $\sema{-}$]
    \label{lem:3:soundness_of_sema}
    If $\Gamma \vdash_{\ced} t : A$ then $\sema{\Gamma} \vdash_\omega \sema{t} : V(A)$
    where $A = \kind$ or $\Gamma \vdash_{\ced} A : \kind$
\end{theorem}
\begin{proof}
    By induction on $\Gamma \vdash_{\ced} t : A$.
    The cases: \textsc{Pair}, \textsc{Fst}, \textsc{Snd}, \textsc{Refl}, \textsc{Subst}, \textsc{Prm}, \textsc{Cast}, and \textsc{Sep} are impossible by inversion on $A = \kind$ or $\Gamma \vdash A : \kind$.

    $\text{Case: }\begin{array}{c} \AxiomRule[*] \end{array}$
    \begin{proofcase}
        By computation $\sema{\star} = 0$ and $V(\kind) = \star$.
        Note that $0 : \star \in \sema{\Gamma}$ thus this case is concluded by the \textsc{Var} rule.
    \end{proofcase}

    $\text{Case: }\begin{array}{c} \VarRule[*] \end{array}$
    \begin{proofcase}
        Note that $A \neq \kind$ by $\D{2}$, thus $K = \kind$.
        By computation $\sema{x_\square} = x$.
        Moreover, $A\pkind$ thus $x : V(A) \in \sema{\Gamma}$.
        Thus, $\sema{\Gamma} \vdash_\omega x : V(A)$
    \end{proofcase}

    $\text{Case: }\begin{array}{c} \PiRule[*] \end{array}$
    \begin{proofcase}
        By computation $V(\pcodom(m)) = V(\pdom(m, K)) = \star$.
        Applying the IH gives:
        \begin{enumerate}
            \item[$\D{1}$.] $\sema{\Gamma} \vdash_\omega \sema{A} : \star$
            \item[$\D{2}$.] $\sema{\Gamma, x_m : A} \vdash_\omega \sema{B} : \star$ 
        \end{enumerate}
        Suppose that $A$ is a kind.
        Then $\sema{(x : A) \to_m B} = (x : V(A)) \to \sema{A} \to \sema{B}$ and $\sema{\Gamma, x_m : A} = \sema{\Gamma}, x : V(A), w_x : \sema{A}$.\
        The \textsc{Pi2} rule applied with the results of the IH gives
            \begin{tightcenter} $\sema{\Gamma}, x : V(A) \vdash_\omega \sema{A} \to \sema{B} : \star$ \end{tightcenter}
        Now by Lemma~\ref{lem:3:soudness_of_v} applied to $\D{1}$: $\sema{\Gamma} \vdash_\omega V(A) : \kind$.
        Using the \textsc{Pi1} rule gives $\sema{\Gamma} \vdash_\omega V(A) \to \sema{A} \to \sema{B} : \star$.
        \\ \\
        Suppose that $A$ is a type.
        Then $\sema{(x : A) \to_m B} = (x : \sema{A}) \to \sema{B}$ and $\sema{\Gamma, x_m : A} = \sema{\Gamma}, x : \sema{A}$.
        Thus, by the \textsc{Pi2} rule $\sema{\Gamma} \vdash \sema{A} \to \sema{B} : \star$.
    \end{proofcase}

    $\text{Case: }\begin{array}{c} \LambdaRule[*] \end{array}$
    \begin{proofcase}
        It must be the case that $\Gamma \vdash (x : A) \to_m B : \kind$.
        Thus, $m = \tau$.
        Applying the IH gives:
        \begin{enumerate}
            \item[$\D{1}$.] $\sema{\Gamma} \vdash_\omega \sema{(x : A) \to_\tau B} : \star$
            \item[$\D{2}$.] $\sema{\Gamma, x_\tau : A} \vdash_\omega \sema{t} : V(B)$ 
        \end{enumerate}
        Suppose $A$ is a kind.
        Then $\sema{(x : A) \to_\tau B} = (x : V(A)) \to \sema{A} \to \sema{B}$, $\sema{\Gamma, x_m : A} = \sema{\Gamma}, x : V(A), w_x : \sema{A}$, and $\sema{\abs{\lambda_\tau}{x}{A}{t}} = \abs{\lambda}{x}{V(A)}{\sema{t}}$.
        Note that $\sema{\Gamma} \vdash c^{\sema{A}} : \sema{A}$.
        Thus, by substitution lemma for F$^\omega$: $\sema{\Gamma}, x : V(A) \vdash_\omega [w_x := c^{\sema{A}}]\sema{t} : [w_x := c^{\sema{A}}]V(B)$.
        However, because $A$ is kind and by Lemma~\ref{lem:3:var_not_in_sema_when_kind}: $[w_x := c^{\sema{A}}]\sema{t} = \sema{t}$.
        Note also that $FV(V(B))$ is empty, thus $[w_x := c^{\sema{A}}]V(B) = V(B)$.
        Thus, $\sema{\Gamma}, x : V(A) \vdash_\omega \sema{t} : V(B)$.
        Moreover, by Theorem~\ref{lem:3:soudness_of_v} it is the case that $\sema{\Gamma} \vdash V(A) : \kind$.
        Using the \textsc{Lam} rule gives $\sema{\Gamma} \vdash_\omega \abs{\lambda}{x}{V(A)}{\sema{t}} : V(A) \to V(B)$.
        \\ \\
        Suppose $A$ is a type.
        Then $\sema{\Gamma, x_m : A} = \sema{\Gamma}, x : \sema{A}$ and $\sema{\abs{\lambda_\tau}{x}{A}{t}} = \sema{t}$.
        Note additionally that $V((x : A) \to_m B) = V(B)$.
        Note that $\sema{\Gamma} \vdash c^{\sema{A}} : \sema{A}$.
        By substitution lemma, Lemma~\ref{lem:3:var_not_in_sema_when_kind}, and as above: $\sema{\Gamma} \vdash_\omega \sema{t} : V(B)$.
    \end{proofcase}

    $\text{Case: }\begin{array}{c} \AppRule[*] \end{array}$
    \begin{proofcase}
        It cannot be the case that $[x := a]B = \kind$ by inversion on $\D{1}$, thus $\Gamma \vdash [x := a]B : \kind$ which force $m = \tau$.
        Furthermore, by $\D{1}$: $\Gamma \vdash (x : A) \to_\tau B : \kind$.
        Applying the IH to $\D{1}$ thus gives $\sema{\Gamma} \vdash_\omega \sema{f} : V((x : A) \to_\tau B)$.
        \\ \\
        Suppose $A$ is a kind, then $a$ is a type.
        Thus, $V((x : A) \to_\tau B) = V(A) \to V(B)$ and $\sema{\app{f}{\tau}{a}} = \sema{f}\ \sema{a}$.
        Applying the IH to $\D{2}$ gives $\sema{\Gamma} \vdash_\omega \sema{a} : V(A)$.
        By the \textsc{App} rule: $\sema{\Gamma} \vdash \sema{f}\ \sema{a} : V(B)$.
        Now by Lemma~\ref{lem:3:v_subst}: $V(B) = V([x := a]B)$.
        \\ \\
        Suppose $A$ is a type, then $a$ is a term.
        Thus, $V((x : A) \to_\tau B) = V(B)$ and $\sema{\app{f}{\tau}{a}} = \sema{f}$.
        But, $\sema{\Gamma} \vdash_\omega \sema{f} : V(B)$ already.
        Now by Lemma~\ref{lem:3:v_subst}: $V(B) = V([x := a]B)$.
    \end{proofcase}

    $\text{Case: }\begin{array}{c} \IntersectionRule[*] \end{array}$
    \begin{proofcase}
        Applying the IH gives:
        \begin{enumerate}
            \item[$\D{1}$.] $\sema{\Gamma} \vdash_\omega \sema{A} : \star$
            \item[$\D{2}$.] $\sema{\Gamma, x_\tau : A} \vdash_\omega \sema{B} : \star$ 
        \end{enumerate}
        Note that $A$ is a type thus $\sema{\Gamma, x_\tau : A} = \sema{\Gamma}, x : \sema{A}$.
        Applying the \textsc{Lam} rule twice reduces the goal to $\sema{\Gamma}, \sema{A} : \star, \sema{B} : \star \vdash_\omega \sema{A} \times \sema{B} : \star$.
        However, the pair case is an otherwise simple F$^\omega$ type, thus a short sequence of rules concludes the case.
    \end{proofcase}

    $\text{Case: }\begin{array}{c} \EqualityRule[*] \end{array}$
    \begin{proofcase}
        By computation $\sema{a =_A b} = \fpId$ and $V(\star) = \star$.
        A short sequence of rules in F$^\omega$ yields $\sema{\Gamma} \vdash \fpId : \star$.
    \end{proofcase}

    $\text{Case: }\begin{array}{c} \ConvRule[*] \end{array}$
    \begin{proofcase}
        Note that $A \neq \kind$ by $\D{1}$, and furthermore that $K = \kind$.
        Now by classification and $\D{3}$: $\Gamma \vdash B : \kind$.
        Applying the IH to $\D{2}$ gives $\sema{\Gamma} \vdash_\omega \sema{t} : V(B)$.
        Using Lemma~\ref{lem:3:conv_of_v} with $\D{3}$ gives $V(A) = V(B)$.
        Thus, the \textsc{Conv} rule concludes the case.
    \end{proofcase}
\end{proof}

\begin{lemma}
    \label{lem:3:sema_subst}
    Suppose $\Gamma \vdash_\omega \sema{t} : T$ then $\sema{[x := b]t} = [x := \sema{b}]\sema{t}$
\end{lemma}
\begin{proof}
    By induction on $t$ and inversion on $\Gamma \vdash_\omega \sema{t} : T$.
    Thus, only the cases where $\sema{t}$ is well-defined need to be considered.

    $\text{Case: } t = \star \text{ or } t = \kind$
    \begin{proofcase}
        The situation is the same because $\sema{\star} = \sema{\kind}$.
        By computation $\sema{[x := b]\star} = \sema{\star} = 0$ and $[x := \sema{b}]\sema{\star} = [x := \sema{b}]0 = 0$.
    \end{proofcase}

    $\text{Case: } t = y_\square$
    \begin{proofcase}
        Suppose $x \neq y$, then by computation $\sema{[x := b]y_\square} = \sema{y_\square} = y$ and $[x := \sema{b}]\sema{y_\square} = [x := \sema{b}]y = y$.
        Suppose $x = y$, then $\sema{[x := b]y_\square} = \sema{b}$ and $[x := \sema{b}]\sema{y_\square} = [x := \sema{b}]y = \sema{b}$.
    \end{proofcase}

    $\text{Case: } t = (y : C) \to_m D$
    \begin{proofcase}
        Suppose $A$ is a kind.
        Then $\sema{[x := b](y : C) \to_m D} = \sema{(y : [x := b]C) \to_m ([x := b]D)} = (y : V([x := b]A)) \to \sema{[x := b]C} \to \sema{[x := b]D}$.
        By Lemma~\ref{lem:3:v_subst} and applying the IH:
        \begin{align*}
            (y &: V([x := b]A)) \to \sema{[x := b]C} \to \sema{[x := b]D} \\
            &= (y : [x := \sema{b}]V(A)) \to [x := \sema{b}]\sema{C} \to [x := \sema{b}]\sema{D} \\
            &= [x := \sema{b}]((y : V(A)) \to \sema{C} \to \sema{D}) \\
            &= [x := \sema{b}]\sema{(y : C) \to_m D}
        \end{align*}
        Suppose $A$ is a type.
        Then $\sema{[x := b](y : C) \to_m D} = \sema{(y : [x := b]C) \to_m ([x := b]D)} = (y : \sema{[x := b]C}) \to \sema{[x := b]D}$.
        Applying the IH and chasing similar computations as above concludes the case.
    \end{proofcase}

    $\text{Case: } t = \abs{\lambda_\tau}{C}{c}$
    \begin{proofcase}
        Suppose $C$ is a kind.
        Then $\sema{[x := b](\abs{\lambda_\tau}{x}{C}{c})} = \sema{\abs{\lambda_\tau}{x}{[x := b]C}{[x := b]c}} = \abs{\lambda}{x}{V([x := b]C)}{\sema{[x := b]c}}$.
        By Lemma~\ref{lem:3:v_subst} and the IH:
        \begin{align*}
            \abs{\lambda}{x}{V([x := b]C)}&{\sema{[x := b]c}} \\
            &= \abs{\lambda}{x}{[x := \sema{b}]V(C)}{[x := \sema{b}]\sema{c}} \\
            &= [x := \sema{b}](\abs{\lambda}{x}{V(C)}{\sema{c}}) \\
            &= [x := \sema{b}]\sema{\abs{\lambda}{x}{C}{c}} \\
        \end{align*}
        Suppose $C$ is a type.
        Then $\sema{[x := b](\abs{\lambda_\tau}{x}{C}{c})} = \sema{\abs{\lambda_\tau}{x}{[x := b]C}{[x := b]c}} = \sema{[x := b]c}$.
        By the IH: $\sema{[x := b]c} = [x := \sema{b}]\sema{c} = [x := \sema{b}]\sema{\abs{\lambda_\tau}{x}{C}{c}}$.
    \end{proofcase}

    $\text{Case: } t = \app{f}{\tau}{a}$
    \begin{proofcase}
        Suppose $a$ is a type.
        Then $\sema{[x := b](\app{f}{\tau}{a})} = \sema{(\app{[x := b]f}{\tau}{[x := b]a})} = \sema{[x := b]f}\ \sema{[x := b]a}$.
        Using the IH gives $\sema{[x := b]f}\ \sema{[x := b]a} = ([x := \sema{b}]\sema{f})\ ([x := \sema{b}]\sema{a}) = [x := \sema{b}](\sema{f}\ \sema{a}) = [x := \sema{b}]\sema{\app{f}{\tau}{a}}$.
        \\ \\
        Supppose $a$ is a term.
        Then $\sema{[x := b](\app{f}{\tau}{a})} = \sema{(\app{[x := b]f}{\tau}{[x := b]a})} = \sema{[x := b]f}$.
        Using the IH gives $\sema{[x := b]f} = [x := \sema{b}]\sema{f} = [x := \sema{b}]\sema{\app{f}{\tau}{a}}$.
    \end{proofcase}

    $\text{Case: } t = (y : C) \cap D$
    \begin{proofcase}
        By computation $\sema{[x := b]((y : C) \cap D)} = \sema{(y : [x := b]C) \cap [x := b]D} = \sema{[x := b]C} \times \sema{[x := b]D}$.
        Using the IH gives $\sema{[x := b]C} \times \sema{[x := b]D} = ([x := \sema{b}]\sema{C}) \times ([x := \sema{b}]\sema{D}) = [x := \sema{b}](\sema{C} \times \sema{D}) = [x := \sema{b}]\sema{(x : C) \cap D}$.
    \end{proofcase}

    $\text{Case: } t = c =_C d$
    \begin{proofcase}
        By computation $\sema{[x := b](c =_C d)} = \sema{([x := b]c) =_{[x := b]C} ([x := b]d)} = \fpId$.
        Again, by computation $[x := \sema{b}]\sema{c =_C d} = [x := \sema{b}]\fpId = \fpId$.
    \end{proofcase}
\end{proof}

Finally, soundness of the term semantics must be shown.
This is not as simple as the original argument for CC modelled in F$^\omega$ because conversion is relative to erasure.
Luckily, erasure is locally homomorphic on type-like structure and the type semantics drops all term dependencies.

\begin{lemma}
    \label{lem:3:vsema_erase}
    If $\Gamma \vdash_\omega V(t) : \kind$ then $V(t) = V(|t|)$
\end{lemma}
\begin{proof}
    By induction on $t$ and inversion on $\Gamma \vdash V(t) : \kind$.

    $\text{Case: } t = \star \text{ or } t = \kind$
    \begin{proofcase}
        By computation $V(|\kind|) = V(\kind) = V(\star) = V(|\!\star\!|)$.
    \end{proofcase}

    $\text{Case: } t = (x : A) \to_m B$
    \begin{proofcase}
        Suppose $A$ is a kind.
        By Lemma~\ref{lem:2:classify_erase}: $|A|$ kind.
        Then $V((x : A) \to_m B) = V(A) \to V(B)$.
        Note that the subexpressions are well-typed, thus by the IH $V(|A|) = V(A)$ and $V(|B|) = V(B)$.
        Now by computation $V(|(x : A) \to_m B|) = V((x : |A|) \to_m |B|) = V(|A|) \to V(|B|) = V(A) \to V(B)$.
        \\ \\
        Suppose $A$ is not a kind.
        Then $V((x : A) \to_m B) = V(B)$.
        By the IH $V(|B|) = V(B)$.
        Thus, by computation $V(|(x : A) \to_m B|) = V((x : |A|) \to_m |B|) = V(|B|) = V(B)$.
    \end{proofcase}
\end{proof}

\begin{lemma}
    \label{lem:3:sema_erase}
    If $\Gamma \vdash_\omega \sema{t} : T$ then $\sema{t} = \sema{|t|}$
\end{lemma}
\begin{proof}
    By induction on $t$ and inversion on $\Gamma \vdash \sema{t} : T$.
    Erasure is again homomorphic on all remaining syntactic forms after inversion, thus only two cases are presented.

    $\text{Case: } t = \star \text{ or } t = \kind \text{ or } t = x_\square$
    \begin{proofcase}
        In each case $|t| = t$ thus trivial.
    \end{proofcase}

    $\text{Case: } t = (x : A) \to_m B$
    \begin{proofcase}
        Have $|(x : A) \to_m B| = (x : |A|) \to_m |B|$.
        Suppose w.l.o.g. that $A$ is a kind.
        Then $\sema{(x : |A|) \to_m |B|} = (x : V(|A|)) \to \sema{|A|} \to \sema{|B|}$.
        By Lemma~\ref{lem:3:vsema_erase} and the IH $(x : V(|A|)) \to \sema{|A|} \to \sema{|B|} = (x : V(A)) \to \sema{A} \to \sema{B}$.
        Likewise, $\sema{(x : A) \to_m B} = (x : V(A)) \to \sema{A} \to \sema{B}$.
    \end{proofcase}
\end{proof}

The above lemmas demonstrate that if reduction happens in the erased term it should somehow be mirrored in reduction for the well-typed terms.
For kinds this turns out to be simple equality, as any possible dependence involving reduction are always dropped the structure of $V(t)$ for any $t$ is rigid.
The type semantics is slightly more complicated, but the same intuition holds: if a reduction were to occur in a term dependency then the result of the type semantics is equal, otherwise the reduction is exactly mirrored in the model.

\begin{lemma}
    \label{lem:3:vsema_erase_red_step}
    If $\Gamma \vdash_\omega V(s) : \kind$ and $|s| \betared t$ then $V(s) = V(t)$
\end{lemma}
\begin{proof}
    By induction on $|s| \betared t$.
    Note that only binder reduction is possible by inversion on $\Gamma \vdash V(s) : \kind$.

    $\text{Case: }\begin{array}{c} \BinderOneReduction[*] \end{array}$
    \begin{proofcase}
        Inversion on $\Gamma \vdash V(s) : \kind$ forces $s = (x : A) \to_m B$.
        Note that $|A| \betared A^\prime$.
        Suppose $A$ kind, then $V((x : A) \to_m B) = V(A) \to V(B)$.
        Now by the IH $V(A) = V(A^\prime)$ and $V((x : A^\prime) \to_m |B|) = V(A^\prime) \to V(B)$ by Lemma~\ref{lem:3:vsema_erase}.
        Suppose $A$ is not a kind, then $V((x : A) \to_m B) = V(B) = V((x : A^\prime) \to_m |B|)$.
    \end{proofcase}

    $\text{Case: }\begin{array}{c} \BinderTwoReduction[*] \end{array}$
    \begin{proofcase}
        Inversion on $\Gamma \vdash V(s) : \kind$ forces $s = (x : A) \to_m B$.
        Note that $|B| \betared B^\prime$.
        Suppose $A$ kind, then $V((x : A) \to_m B) = V(A) \to V(B)$.
        Now by the IH $V(B) = V(B^\prime)$ and $V((x : |A|) \to_m B^\prime) = V(A) \to V(B^\prime)$ by Lemma~\ref{lem:3:vsema_erase}.
        Suppose $A$ is not a kind, then $V((x : A) \to_m B) = V(B) = V(B^\prime) = V((x : |A|) \to_m B^\prime)$.
    \end{proofcase}
\end{proof}

\begin{lemma}
    \label{lem:3:sema_erase_red_step}
    If $\Gamma \vdash_\omega \sema{s} : T$ and $|s| \betared t$ then $\sema{s} \betared \sema{t}$ or $\sema{s} = \sema{t}$
\end{lemma}
\begin{proof}
    By induction on $|s| \betared t$.
    Note that only $\beta$-reduction is possible, as all other possible reduction steps are erased.

    $\text{Case: }\begin{array}{c} \app{(\abs{\lambda_m}{x}{A}{b})}{m}{t} \betared [x := t]b \end{array}$
    \begin{proofcase}
        By inversion on $\Gamma \vdash \sema{s} : T$ it must be the case that $m = \tau$.
        Thus, $|s| = \app{(\abs{\lambda_\tau}{x}{|A|}{|b|})}{\tau}{|t|}$ and $|s| \betared [x := |t|]|b|$.
        By Lemma~\ref{lem:2:erase_subst}: $[x := |t|]|b| = |[x := t]b|$.
        Now, Lemma~\ref{lem:3:sema_erase} yields $\sema{|[x := t]b|} = \sema{[x := t]b}$ and $\sema{|s|} = \sema{s}$.
        Using Lemma~\ref{lem:3:sema_subst} gives $\sema{[x := t]b} = [x := \sema{t}]\sema{b}$.
        Suppose $A$ is a kind, and thus $t$ is a type.
        Then $\sema{\app{(\abs{\lambda_\tau}{x}{A}{b})}{\tau}{t}} = (\abs{\lambda}{x}{V(A)}{\sema{b}})\ \sema{t} \betared [x := \sema{t}] \sema{b}$.
        Suppose $A$ is a type, and thus $t$ is a term.
        Then $\sema{\app{(\abs{\lambda_\tau}{x}{A}{b})}{\tau}{t}} = \sema{b}$, however this also means that $\Gamma \vdash \sema{b} : T$.
        The internally bound variable $x$ is thrown away, so it cannot be the case that $\sema{b}$ is well-typed in F$^\omega$ while $x \in FV(b)$ (Note that $x$ can be renamed to be disjoint from $\Gamma$), hence $x \notin FV(b)$.
        Thus, $[x := \sema{t}] \sema{b} = \sema{b}$ and the case is concluded.
    \end{proofcase}

    $\text{Case: }\begin{array}{c} \ConstructorReduction[*] \end{array}$
    \begin{proofcase}
        By inversion on $\Gamma \vdash \sema{s} : T$ it must be the case that $\kappa$ is $*$, $\kind$, $\bullet_\tau$, or eq.
        However, the cases $*$ and $\kind$ are impossible because they do not reduce.
        Suppose $|s| = \app{|f|}{\tau}{|a|}$ and assume w.l.o.g. that $|a| \betared a^\prime$.
        If $a$ is a term then $\sema{\app{|f|}{\tau}{|a|}} = \sema{|f|} = \sema{\app{|f|}{\tau}{|a^\prime|}}$ and $\sema{|f|} = \sema{f}$ by Lemma~\ref{lem:3:sema_erase}.
        Suppose $a$ is a type.
        Then, by the IH $\sema{a} \betared \sema{a^\prime}$ or $\sema{a} = \sema{a^\prime}$.
        Now $\sema{\app{|f|}{\tau}{|a|}} = \sema{|f|}\ \sema{|a|}$, but by Lemma~\ref{lem:3:sema_erase}: $\sema{|f|}\ \sema{|a|} = \sema{f}\ \sema{a}$.
        Thus, $\sema{f}\ \sema{a} \betared \sema{f}\ \sema{a^\prime}$ or $\sema{f}\ \sema{a} = \sema{f}\ \sema{a^\prime}$.
        \\ \\
        Suppose $|s| = |a| =_{|A|} |b|$.
        Note that $\sema{u =_U v} = \fpId$ for any $u, v, U$.
        Thus, $\sema{s} = \sema{|s|} = \sema{t}$.
    \end{proofcase}

    $\text{Case: }\begin{array}{c} \BinderOneReduction[*] \end{array}$
    \begin{proofcase}
        By inversion on $\Gamma \vdash \sema{s} : T$ it must be the case that $\kappa$ is $\Pi_m$, $\lambda_\tau$, or $\cap$.
        The $\cap$ and $\lambda_\tau$ cases are similar to the $\Pi_m$ case and thus omitted.
        Have $|s| = (x : |A|) \to_m |B|$ and note that $|A| \betared A^\prime$.
        Suppose w.l.o.g. that $A$ kind.
        Now $\sema{(x : |A|) \to_m |B|} = (x : V(|A|)) \to \sema{|A|} \to \sema{|B|}$.
        By the IH: $\sema{A} \betared \sema{A^\prime}$ or $\sema{A} = \sema{A^\prime}$.
        Suppose w.l.o.g. that $\sema{A} \betared \sema{A^\prime}$, then $(x : V(|A|)) \to \sema{|A|} \to \sema{|B|} \betared (x : V(A^\prime)) \to \sema{A^\prime} \to \sema{|B|}$ by Lemma~\ref{lem:3:vsema_erase_red_step}.
        Now $\sema{(x : A^\prime) \to_m |B|} = (x : V(A^\prime)) \to \sema{A^\prime} \to \sema{|B|}$.
    \end{proofcase}

    $\text{Case: }\begin{array}{c} \BinderTwoReduction[*] \end{array}$
    \begin{proofcase}
        By inversion on $\Gamma \vdash \sema{s} : T$ it must be the case that $\kappa$ is $\Pi_m$, $\lambda_\tau$, or $\cap$.
        The $\cap$ and $\lambda_\tau$ cases are similar to the $\Pi_m$ case and thus omitted.
        Have $|s| = (x : |A|) \to_m |B|$ and note that $|B| \betared B^\prime$.
        Suppose w.l.o.g. that $A$ kind.
        Now $\sema{(x : |A|) \to_m |B|} = (x : V(|A|)) \to \sema{|A|} \to \sema{|B|}$.
        By the IH: $\sema{B} \betared \sema{B^\prime}$ or $\sema{B} = \sema{B^\prime}$.
        Suppose w.l.o.g. that $\sema{B} \betared \sema{B^\prime}$, then $(x : V(|A|)) \to \sema{|A|} \to \sema{|B|} \betared (x : V(|A|)) \to \sema{|A|} \to \sema{B^\prime}$.
        Now $\sema{(x : |A|) \to_m B^\prime} = (x : V(|A|)) \to \sema{|A|} \to \sema{B^\prime}$.
    \end{proofcase}
\end{proof}

\begin{lemma}
    \label{lem:3:sema_erase_red}
    If $\Gamma \vdash_\omega \sema{s} : T$ and $|s| \betastar t$ then $\sema{s} \betastar \sema{t}$
\end{lemma}
\begin{proof}
    By induction on $|s| \betastar t$.
    The reflexivity case is trivial by Lemma~\ref{lem:3:sema_erase}.
    Suppose $|s| \betared z$ and $z \betastar t$.
    By Lemma~\ref{lem:3:sema_erase_red_step} either $\sema{s} \betared \sema{z}$ or $\sema{s} = \sema{z}$.
    If $\sema{s} \betared \sema{z}$ then by preservation $\Gamma \vdash \sema{z} : T$.
    Note that $|z| = z$ by Lemma~\ref{lem:2:erasure_idempotent} and because reduction does not introduce new syntactic forms.
    Applying the IH to $|z| \betastar t$ gives $\sema{z} \betastar \sema{t}$, thus $\sema{s} \betastar \sema{t}$.
    If $\sema{s} = \sema{z}$ then obviously $\Gamma \vdash \sema{z} : T$ and the same argument as above works.
\end{proof}

With the reduction lemmas the required lemma about conversion is provable.
Finally, soundness of the term semantics is proven by a straightforward induction on the inference judgment of $\ced$.

\begin{lemma}
    \label{lem:3:sema_conv}
    If $\Gamma \vdash_\omega \sema{A} : T$, $\Gamma \vdash_\omega \sema{B} : T$, $A, B$\pseobj, and $A \equiv B$ then $\sema{A} \betaconv \sema{B}$
\end{lemma}
\begin{proof}
    By Lemma~\ref{thm:2:beta_iff_conv} $|A| \betaconv |B|$.
    Deconstructing this gives $|A| \betastar z$ and $|B| \betastar z$.
    By Lemma~\ref{lem:3:sema_erase_red}: $\sema{A} \betastar \sema{z}$ and $\sema{B} \betastar \sema{z}$.
    Thus, $\sema{A} \betaconv \sema{B}$.
\end{proof}

\begin{lemma}
    \label{lem:3:fomega_eta}
    If $\Gamma \vdash_\omega t : T$ and $\Gamma \vdash_\omega a : A$ then $\Gamma \vdash (\abs{\lambda}{x}{A}{t})\ a : T$
\end{lemma}
\begin{proof}
    Have $\Gamma \vdash_\omega \abs{\lambda}{x}{A}{t} : A \to T$ because $x$ does not appear free in $t$.
    Thus, by the \textsc{App} rule $\Gamma \vdash (\abs{\lambda}{x}{A}{t})\ a : T$.
\end{proof}

\begin{lemma}
    \label{lem:3:canonical_element}
    If $\Gamma \vdash_\omega A : T$ and $(\bot : (X : \star) \to X) \in \Gamma$ then $\Gamma \vdash_\omega c^A : A$
\end{lemma}
\begin{proof}
    If $A$ type then the proof is trivial.
    If $A$ kind then the proof follows by induction on the depth of the function type.
\end{proof}

\begin{theorem}[Soundness of \text{[}$-$\text{]}]
    \label{lem:3:soundness}
    If $\Gamma \vdash_{\ced} t : A$ then $\sema{\Gamma} \vdash_\omega [t] : \sema{A}$
\end{theorem}
\begin{proof}
    By induction on $\Gamma \vdash_{\ced} t : A$.
    The \textsc{Fst} case is omitted because it is very similar to \textsc{Snd}.
    The cases \textsc{Ax}, \textsc{Var}, \textsc{Pi}, \textsc{Lam}, and \textsc{App} are the same as the translation from CC to F$^\omega$.

    $\text{Case: }\begin{array}{c} \IntersectionRule[*] \end{array}$
    \begin{proofcase}
        Applying the IH to subderivations:
        \begin{enumerate}
            \item[$\D{1}.$] $\sema{\Gamma} \vdash_\omega [A] : 0$
            \item[$\D{2}.$] $\sema{\Gamma, x_\tau : A} \vdash_\omega [B] : 0$
        \end{enumerate}
        Note that $\sema{\Gamma} \vdash_\omega 0 \to 0 \to 0 : \star$.
        Thus, $\sema{\Gamma} \vdash_\omega c^{0 \to 0 \to 0} : 0 \to 0 \to 0$.
        By $\D{1}$ it is the case that $A$ type, thus $\sema{\Gamma, x_\tau : A} = \sema{\Gamma}, x : \sema{A}$.
        Using Lemma~\ref{lem:3:soundness_of_sema} on $\D{1}$ gives $\sema{\Gamma} \vdash_\omega \sema{A} : \star$.
        The substitution lemma yields $\sema{\Gamma} \vdash_\omega [x := c^{\sema{A}}][B] : 0$.
        Now applying the \textsc{App} rule two times concludes the case.
    \end{proofcase}

    $\text{Case: }\begin{array}{c} \PairRule[*] \end{array}$
    \begin{proofcase}
        Applying the IH to subderivations:
        \begin{enumerate}
            \item[$\D{1}.$] $\sema{\Gamma} \vdash_\omega [(x : A) \cap B] : 0$
            \item[$\D{2}.$] $\sema{\Gamma} \vdash_\omega [t] : \sema{A}$
            \item[$\D{3}.$] $\sema{\Gamma} \vdash_\omega [s] : \sema{[x := t]B}$
        \end{enumerate}
        By Lemma~\ref{lem:3:sema_subst}: $\sema{[x := t]B} = [x := \sema{t}]\sema{B}$.
        However, $A$ is a type by $\D{1}$ and thus $x \notin FV(\sema{B})$, hence $[x := \sema{t}]\sema{B} = \sema{B}$.
        Now $\sema{\Gamma} \vdash_\omega ([t_1], [t_2]) : \sema{A} \times \sema{B}$ by the \textsc{Pair} rule.
        Applying \ref{lem:3:fomega_eta} concludes the case.
    \end{proofcase}

    $\text{Case: }\begin{array}{c} \SecondRule[*] \end{array}$
    \begin{proofcase}
        Note by $\D{1}$ that $A$ is a type, thus $x \notin FV(\sema{B})$.
        By Lemma~\ref{lem:3:sema_subst}: $\sema{[x := t.1]B} = [x := \sema{t.1}]\sema{B} = \sema{B}$.
        Applying the IH to $\D{1}$ gives $\sema{\Gamma} \vdash_\omega [t] : \sema{A} \times \sema{B}$.
        The \textsc{Snd} rule concludes the case.
    \end{proofcase}

    $\text{Case: }\begin{array}{c} \EqualityRule[*] \end{array}$
    \begin{proofcase}
        Applying the IH to subderivations:
        \begin{enumerate}
            \item[$\D{1}.$] $\sema{\Gamma} \vdash_\omega [A] : 0$
            \item[$\D{2}.$] $\sema{\Gamma} \vdash_\omega [a] : \sema{A}$
            \item[$\D{3}.$] $\sema{\Gamma} \vdash_\omega [b] : \sema{A}$
        \end{enumerate}
        Note that $\sema{\Gamma} \vdash_\omega 0 \to \sema{A} \to \sema{A} \to 0 : \star$.
        Thus, $\sema{\Gamma} \vdash_\omega c^{0 \to \sema{A} \to \sema{A} \to 0} : 0 \to \sema{A} \to \sema{A} \to 0$.
        Now applying the \textsc{App} rule three times concludes the case.
    \end{proofcase}

    $\text{Case: }\begin{array}{c} \ReflRule[*] \end{array}$
    \begin{proofcase}
        Applying the IH to subderivations:
        \begin{enumerate}
            \item[$\D{1}.$] $\sema{\Gamma} \vdash_\omega [A] : 0$
            \item[$\D{2}.$] $\sema{\Gamma} \vdash_\omega [t] : \sema{A}$
        \end{enumerate}
        Of course, $\sema{\Gamma} \vdash_\omega \fpid : \fpId$.
        Thus, applying Lemma~\ref{lem:3:fomega_eta} twice concludes the case.
    \end{proofcase}

    \begin{minipage}{.8\textwidth}$\text{Case: }\begin{array}{c} \SubstRule[*] \end{array}$\end{minipage}
    \begin{proofcase}
        Note that by classification and $\D{1}$ it is that case that $A$ type.
        Applying the IH to subderivations:
        \begin{enumerate}
            \item[$\D{1}.$] $\sema{\Gamma} \vdash_\omega [A] : 0$
            \item[$\D{2}.$] $\sema{\Gamma} \vdash_\omega [a] : \sema{A}$
            \item[$\D{3}.$] $\sema{\Gamma} \vdash_\omega [b] : \sema{A}$ 
            \item[$\D{4}.$] $\sema{\Gamma} \vdash_\omega [e] : \fpId$
            \item[$\D{5}.$] $\sema{\Gamma} \vdash_\omega [P] : \sema{A} \to \fpId \to 0$
        \end{enumerate}
        Now $\sema{\Gamma} \vdash_\omega [e]\ \sema{P} : \sema{P} \to \sema{P}$.
        Note also that $\sema{\apptwo{P}{\tau}{a}{\tau}{\text{refl}(a;A)} \to_\omega \apptwo{P}{\tau}{b}{\tau}{e}} = \sema{P} \to \sema{P}$ because $\apptwo{P}{\tau}{a}{\tau}{\text{refl}(a;A)}$ is a type by $\D{3}$ and $a, b, e, \text{refl}(a;A)$ are all terms.
        Applying Lemma~\ref{lem:3:fomega_eta} four times concludes the case.
    \end{proofcase}

    $\text{Case: }\begin{array}{c} \PromoteRule[*] \end{array}$
    \begin{proofcase}
        Applying the IH to subderivations:
        \begin{enumerate}
            \item[$\D{1}.$] $\sema{\Gamma} \vdash_\omega [(x : A) \cap B] : 0$
            \item[$\D{2}.$] $\sema{\Gamma} \vdash_\omega [a] : \sema{A} \times \sema{B}$
            \item[$\D{3}.$] $\sema{\Gamma} \vdash_\omega [b] : \sema{A} \times \sema{B}$
            \item[$\D{4}.$] $\sema{\Gamma} \vdash_\omega [e] : \fpId$
        \end{enumerate}
        Applying Lemma~\ref{lem:3:fomega_eta} three times concludes the case.
    \end{proofcase}

    $\text{Case: }\begin{array}{c} \CastRule[*] \end{array}$
    \begin{proofcase}
        Note by $\D{1}$ it is clear that $A$ is a type.
        Applying the IH to subderivations:
        \begin{enumerate}
            \item[$\D{1}.$] $\sema{\Gamma} \vdash_\omega [a] : \sema{A}$
            \item[$\D{2}.$] $\sema{\Gamma} \vdash_\omega [b] : \sema{A} \times \sema{B}$
            \item[$\D{3}.$] $\sema{\Gamma} \vdash_\omega [e] : \fpId$
        \end{enumerate}
        Note that $\sema{\Gamma} \vdash_\omega [b].2 : \sema{B}$.
        Thus, $\sema{\Gamma} \vdash_\omega ([a], [b].2) : \sema{A} \times \sema{B}$.
        Applying Lemma~\ref{lem:3:fomega_eta} concludes the case.
    \end{proofcase}

    $\text{Case: }\begin{array}{c} \SeparationRule[*] \end{array}$
    \begin{proofcase}
        By computation $[\delta(e)] = (\abs{\lambda}{x}{\mathcal{I}([e])}{\bot})\ [e]$ and $\sema{(X : \star) \to_0 X} = (X : \star) \to X$.
        Note that $\sema{\Gamma} \vdash_\omega \bot : (X : \star) \to X$ and by definition $\sema{\Gamma} \vdash_\omega [e] : \mathcal{I}([e])$.
        Thus, by Lemma~\ref{lem:3:fomega_eta}: $\sema{\Gamma} \vdash [\delta(e)] : \sema{(X : \star) \to_0 X}$.
    \end{proofcase}

    $\text{Case: }\begin{array}{c} \ConvRule[*] \end{array}$
    \begin{proofcase}
        By classification, $\D{1}$ and $\D{3}$: $\Gamma \vdash B : K$.
        Now using Theorem~\ref{lem:3:soundness_of_sema} gives $\sema{\Gamma} \vdash_\omega \sema{A} : \star$ and $\sema{\Gamma} \vdash_\omega \sema{B} : \star$.
        Note that $A,B\pseobj$ by Lemma~\ref{lem:2:infer_implies_pseobj_type} and $|A| \betaconv |B|$ by Lemma~\ref{thm:2:beta_iff_conv}.
        By Lemma~\ref{lem:3:sema_conv}: $\sema{A} \betaconv \sema{B}$.
        Applying the IH to $\D{2}$ gives $\sema{\Gamma} \vdash_\omega [t] : \sema{B}$.
        The \textsc{Conv} rule concludes the case.
    \end{proofcase}
\end{proof}

\section{Normalization}

With soundness of the model shown the normalization argument follows in the same way as for CC modelled in F$^\omega$.
That is, proof reduction in $\ced$ is bounded by reduction in F$^\omega$, and thus because F$^\omega$ is strongly normalizing it provides a maximum number of reduction steps for which any proof must normalize in $\ced$.
Note that some reduction steps are technical, especially $\vartheta$, but are not conceptually difficult.

\begin{lemma}
    \label{lem:3:canonical_subst}
    $[x := b]c^A = c^{[x := b]A}$
\end{lemma}
\begin{proof}
    Straightforward by unraveling the definition of canonical elements ($c$) and applying substitution computation rules.
\end{proof}

\begin{lemma}
    \label{lem:3:subst}
    If $\Gamma \vdash t : A$ and $(x : B) \in \Gamma$ then
    \begin{enumerate}
        \item $[[x := b]a] = [x := \sema{b}][w_x := [b]][a]$ if $B$ kind
        \item $[[x := b]a] = [x := [b]][a]$ if $B$ type
    \end{enumerate}
\end{lemma}
\begin{proof}
    By induction on $\Gamma \vdash t : A$.
    Substitution is structural and with Lemma~\ref{lem:3:sema_subst}, Lemma~\ref{lem:3:v_subst}, and Lemma~\ref{lem:3:canonical_subst} many cases are straightforward by induction.
    Thus, only the variable cases and the \textsc{Int} case are presented.

    $\text{Case: }\begin{array}{c} \VarRule[*] \end{array}$
    \begin{proofcase}
        Rename to $y$.
        Suppose $x \neq y$, then $[[x := b]y_\star] = y$, $[x := \sema{b}][w_x := [b]][y_\star] = y$, and $[x := [b]][y_\star] = y$.
        When $y_\square$ the situation is the same.
        Suppose $x = y$ and that $B$ kind.
        If $B$ is kind, then it must be the case that $y_\square$.
        Now $[[x := b]y_\square] = [b]$ and  $[x := \sema{b}][w_x := [b]][y_\square] = [x := \sema{b}][w_x := [b]]w_y = [b]$.
        Suppose instead that $B$ type, then $[[x := b]y_\star] = [b]$ and $[x := [b]][y_\star] = [x := [b]]y = [b]$.
    \end{proofcase}

    $\text{Case: }\begin{array}{c} \IntersectionRule[*] \end{array}$
    \begin{proofcase}
        Suppose w.l.o.g. that $B$ is a kind.
        Then $[[x := b](y : A) \cap B] = [(y : [x := b]A) \cap [x := b]B] = c^{0 \to 0 \to 0}[[x := b]A] ([y := c^{\sema{[x := b]A}}][[x := b]B])$.
        Now by the IH, Lemma~\ref{lem:3:sema_subst}, and the fact that $w_x \notin FV(\sema{A})$ the right-hand side is equal to $c^{0 \to 0 \to 0}[x := \sema{b}][w_x := [b]][A] ([y := c^{[x := \sema{b}][w_x := [b]]\sema{A}}][x := \sema{b}][w_x := [b]][B])$.
        Consider $[x := \sema{b}][w_x := [b]][(y : A) \cap B] = [x := \sema{b}][w_x := [b]]c^{0 \to 0 \to 0}[A] ([y := c^{\sema{A}}][B])$.
        Note that $x, w_x \notin FV(0 \to 0 \to 0)$, thus by Lemma~\ref{lem:2:subst_commute}, Lemma~\ref{lem:3:canonical_subst}, and computation rules of substitution this matches the previous right-hand side.
    \end{proofcase}
\end{proof}

\begin{lemma}
    \label{lem:3:reduction_bounded}
    If $\Gamma \vdash s : T$ and $s \betared t$ then $[s] \betastar_{\neq 0} [t]$
\end{lemma}
\begin{proof}
    By induction on $s \betared t$.
    The first projection case is very similar to the second projection case.
    Note by a simple observation that $[-]$ replicates every subexpression on the left-hand side with a matching invocation of $[-]$ on the right-hand side.
    Thus, if there is a reduction inside a subexpression it will always be tracked in the corresponding $[-]$ invocation via the inductive hypothesis.
    For this reason the structural reduction cases are omitted.

    $\text{Case: }\begin{array}{c} \app{(\abs{\lambda_m}{x}{A}{b})}{m}{t} \betared [x := t]b \end{array}$
    \begin{proofcase}
        Note by $\Gamma \vdash s : T$ that $A$ is either a kind or a type.
        Suppose $A$ is a kind and note that makes $t$ a type.
        Then $[\app{(\abs{\lambda_m}{x}{A}{b})}{m}{t}] = (\abs{\lambda}{y}{0}{\abs{\lambda}{x}{V(A)}{\abs{\lambda}{w_x}{\sema{A}}{[b]}}}) \ [A]\ \sema{t}\ [t]$.
        The variable $y$ is fresh thus after one $\beta$-reduction $(\abs{\lambda}{x}{V(A)}{\abs{\lambda}{w_x}{\sema{A}}{[b]}})\ \sema{t}\ [t]$.
        Now applying two more $\beta$-reductions yields $[x := \sema{t}][w_x := [t]][b]$.
        Note that $[[x := t]b] = [x := \sema{t}][w_x := [t]][b]$ by Lemma~\ref{lem:3:subst}.
        Thus, $[s] \betastar_{=3} [t]$, i.e. $[s]$ reduces to $[t]$ in three steps.
        \\ \\
        Suppose $A$ is a type and note that makes $t$ a term.
        Then $[\app{(\abs{\lambda_m}{x}{A}{b})}{m}{t}] = (\abs{\lambda}{y}{0}{\abs{\lambda}{w_x}{\sema{A}}{[b]}}) \ [A]\ [t]$.
        The variable $y$ is fresh thus after one $\beta$-reduction $(\abs{\lambda}{w_x}{\sema{A}}{[b]})\ [t]$.
        Applying one more $\beta$-reduction yields $[x := [t]][b]$.
        Note that $[[x := t]b] = [x := [t]][b]$ by Lemma~\ref{lem:3:subst}.
        Thus, $[s] \betastar_{=2} [t]$.
    \end{proofcase}

    $\text{Case: }\begin{array}{c} [t_1, t_2; A].2 \betared t_2 \end{array}$
    \begin{proofcase}
        Have $[[t_1, t_2; A].2] = ((\abs{\lambda}{y}{0}{([t_1], [t_2])})\ [A]).2$.
        Note that the variable $y$ is fresh and thus not in $FV([t_1])$ or $FV([t_2])$.
        A second projection and one $\beta$-reduction yields $[t_2]$.
        Thus, $[s] \betastar_{=2} [t_2]$.
    \end{proofcase}
    
    $\text{Case: }\begin{array}{c} \app{\psi(\text{refl}(t; A_1), a, b; A_2, P)}{\omega}{t} \betared t \end{array}$
    \begin{proofcase}
        Note that $t$ is a term by inversion on $\Gamma \vdash s : T$.
        Have $[\app{\psi(\text{refl}(t; A_1); A_2, P)}{\omega}{t}] =
            (\abs{\lambda}{y_1}{0}{
                \abs{\lambda}{y_2\ y_3}{\sema{A}}{
                    \abs{\lambda}{y_4}{\sema{A_2} \to \fpId \to 0}{
                        [\text{refl}(t; A_1)]\ \sema{P}
                    }
                }
            })\ [A_2]\ [a]\ [b]\ [P]\ [t]$.
        Applying four $\beta$-reductions yields $[\text{refl}(t; A_1)]\ \sema{P}\ [t]$.
        Now $[\text{refl}(t; A_1)] =
            (\abs{\lambda}{y_1}{0}{
                \abs{\lambda}{y_2}{\sema{A_1}}{\fpid}
            })\ [A_1]\ [t]$.
        Applying two more $\beta$-reductions gives $\fpid\ \sema{P}\ [t]$.
        Finally, applying two remaining $\beta$-reductions yields $[t]$.
        Thus, $[s] \betastar_{=8} [t]$.
    \end{proofcase}
    
    $\text{Case: }\begin{array}{c} \vartheta(\text{refl}(t; A), a, b; T) \betared \text{refl}(a; T) \end{array}$
    \begin{proofcase}
        Have:
        \begin{align*}
            [\vartheta(\text{refl}(t; A), a, b; T)] &= \\
            (\abs{\lambda}{y_1}{\sema{T}}{
                \abs{\lambda}{y_2}{0} {
                    \abs{\lambda}{y_3}{\sema{T}}{
                        &((\abs{\lambda}{y_1}{0}{
                            \abs{\lambda}{y_2}{\sema{A}}{\fpid}
                        })\ [A]\ [t])
                    }
                }
            })\ [b]\ [T]\ [a]
        \end{align*}
        Note that all $y_i$ are fresh and thus not in the free variables of any subexpressions.
        Performing two $\beta$-reductions on the interior (the result of $[\text{refl}(t_1;A)]$) and the outermost $\beta$-reduction yields:
            $(\abs{\lambda}{y_2}{0} {
                \abs{\lambda}{y_3}{\sema{T}}{
                    \fpid
                }
            })\ [T]\ [a]$.
        Now $[\text{refl}(a; T)] = (\abs{\lambda}{y_2}{0} {
                \abs{\lambda}{y_3}{\sema{T}}{
                    \fpid
                }
            })\ [T]\ [a]$.
        Thus, $[s] \betastar_{=3} [t]$.
    \end{proofcase}
\end{proof}

\begin{theorem}[Proof Normalization]
    \label{lem:3:proof_normalization}
    If $\Gamma \vdash t : A$ then $t$ is strongly normalizing and there exists a unique value $t_n$ such that $t \betastar t_n$
\end{theorem}
\begin{proof}
    Using Lemma~\ref{lem:3:soundness_of_sema} gives $\sema{\Gamma} \vdash_\omega [t] : \sema{A}$.
    Note that F$^\omega$ with pairs is strongly normalizing with a unique normal form (because it is also confluent).
    Thus, \textit{all} possible reduction paths to the normal form are terminating.
    Let $\partial$ be the \textit{maximum} number of reduction steps $[t]$ could take to reach a normal form.
    Note that this value is computable by brute force search.
    Pick any sequence of reductions in $t$ bounded by $\partial$.
    If this sequence concludes in a value then $t$ is strongly normalizing, because the sequence is arbitrary.
    If $t$ is not a value then $t \betastar_{>\partial} t^\prime$, but this is impossible by Lemma~\ref{lem:3:reduction_bounded}.
    Now by confluence of reduction, all values reached from any arbitrary reduction path must be joinable at a single value.
    Thus, $t \betastar t_n$ where $t_n$ is a unique value.
\end{proof}

\chapter{Consistency and Relationship to CDLE}

\Rule{\CedAxiomRule}
    {
        \textcolor{white}{\_}
    }
    {\Gamma \vdash \star}
    {}

\Rule{\CedTypeVarRule}
    {
        \der{\D{1}}{(x : \kappa) \in \Gamma}
    }
    {\Gamma \vdash x \infr \kappa}
    {}

\Rule{\CedTypeKindPiRule}
    {
        \der{\D{1}}{\Gamma \vdash A \infr \star}
        \\ \der{\D{2}}{\Gamma; x : A \vdash \kappa}
    }
    {\Gamma \vdash \abs{\Pi}{x}{A}{\kappa}}
    {}

\Rule{\CedKindKindPiRule}
    {
        \der{\D{1}}{\Gamma \vdash \kappa^\prime}
        \\ \der{\D{2}}{\Gamma; x : \kappa^\prime \vdash \kappa}
    }
    {\Gamma \vdash \abs{\Pi}{x}{\kappa^\prime}{\kappa}}
    {}

\Rule{\CedTypeTypePiRule}
    {
        \der{\D{1}}{\Gamma \vdash A \infr \star}
        \\ \der{\D{2}}{\Gamma; x : A \vdash B \infr \star}
    }
    {\Gamma \vdash \abs{\Pi}{x}{A}{B} \infr \star}
    {}

\Rule{\CedKindForallRule}
    {
        \der{\D{1}}{\Gamma \vdash \kappa}
        \\ \der{\D{2}}{\Gamma; x : \kappa \vdash B \infr \star}
    }
    {\Gamma \vdash \abs{\forall}{x}{\kappa}{B} \infr \star}
    {}

\Rule{\CedTypeForallRule}
    {
        \der{\D{1}}{\Gamma \vdash A \infr \star}
        \\ \der{\D{2}}{\Gamma; x : A \vdash B \infr \star}
    }
    {\Gamma \vdash \abs{\forall}{x}{A}{B} \infr \star}
    {}

\Rule{\CedKindTypeLambdaRule}
    {
        \der{\D{1}}{\Gamma \vdash \kappa^\prime}
        \\ \der{\D{2}}{\Gamma; x : \kappa^\prime \vdash t \infr \kappa}
    }
    {\Gamma \vdash \abs{\lambda}{x}{\kappa^\prime}{t} \infr \abs{\Pi}{x}{\kappa^\prime}{\kappa}}
    {}

\Rule{\CedTypeTypeLambdaRule}
    {
        \der{\D{1}}{\Gamma \vdash A \infr \star}
        \\ \der{\D{2}}{\Gamma; x : A \vdash t \infr \kappa}
    }
    {\Gamma \vdash \abs{\lambda}{x}{A}{t} \infr \abs{\Pi}{x}{A}{\kappa}}
    {}

\Rule{\CedTypeTypeApplyRule}
    {
        \der{\D{1}}{\Gamma \vdash f \infr \abs{\Pi}{x}{\kappa_1}{\kappa_2}}
        \\ \der{\D{2}}{\Gamma \vdash a \infr \kappa_1^\prime}
        \\ \der{\D{3}}{\kappa_1 \cong \kappa_1^\prime}
    }
    {\Gamma \vdash f \cdot a \infr [x := a]\kappa_2}
    {}

\Rule{\CedTermTypeApplyRule}
    {
        \der{\D{1}}{\Gamma \vdash f \infr \abs{\Pi}{x}{A}{\kappa}}
        \\ \der{\D{2}}{\Gamma \vdash a \chck A}
    }
    {\Gamma \vdash f\ a \infr [x := \chi\ A - a]\kappa}
    {}

\Rule{\CedIotaRule}
    {
        \der{\D{1}}{\Gamma \vdash A \infr \star}
        \\ \der{\D{2}}{\Gamma; x : A \vdash B \infr \star}
    }
    {\Gamma \vdash \abs{\iota}{x}{A}{B} \infr \star}
    {}

\Rule{\CedEqRule}
    {
        \der{\D{1}}{FV(a\ b) \subseteq dom(\Gamma)}
    }
    {\Gamma \vdash \{ a \simeq b \} \infr \star}
    {}

\Rule{\CedTermVarRule}
    {
        \der{\D{1}}{(x : A) \in \Gamma}
    }
    {\Gamma \vdash x \infr A}
    {}

\Rule{\CedTypeTermLambdaRule}
    {
        \der{\D{1}}{T \betastar_n \abs{\Pi}{x}{A}{B}}
        \\ \der{\D{2}}{\Gamma; x : A \vdash t \chck B}
    }
    {\Gamma \vdash \absu{\lambda}{x}{t} \chck T}
    {}

\Rule{\CedKindTermLambdaRule}
    {
        \der{\D{1}}{T \betastar_n \abs{\forall}{x}{\kappa}{B}}
        \\ \der{\D{2}}{\Gamma; x : \kappa \vdash t \chck B}
    }
    {\Gamma \vdash \absu{\Lambda}{x}{t} \chck T}
    {}

\Rule{\CedErasedLambdaRule}
    {
        \der{\D{1}}{T \betastar_n \abs{\forall}{x}{A}{B}}
        \\ \der{\D{2}}{\Gamma; x : A \vdash t \chck B}
        \\ \der{\D{3}}{x \notin FV(|t|)}
    }
    {\Gamma \vdash \absu{\Lambda}{x}{t} \chck T}
    {}

\Rule{\CedPairRule}
    {
        \der{\D{1}}{T \betastar_n \abs{\iota}{x}{A}{B}}
        \\ \der{\D{2}}{\Gamma \vdash t_1 \chck A}
        \\ \der{\D{3}}{\Gamma \vdash t_2 \chck [x := \chi\ A - t_1]B}
        \\ \der{\D{4}}{|t_1| \betaconv_\eta |t_2|}
    }
    {\Gamma \vdash [t_1, t_2] \chck T}
    {}

\Rule{\CedSecondProjectRule}
    {
        \der{\D{1}}{\Gamma \vdash t \cinfr \abs{\iota}{x}{A}{B}}
    }
    {\Gamma \vdash t.2 \infr [x := t.1]B}
    {}

\Rule{\CedDeltaRule}
    {
        \der{\D{1}}{\Gamma \vdash t \infr A}
        \\ \der{\D{2}}{A \cong \{ \absu{\lambda}{x\ y}{x} \simeq \absu{\lambda}{x\ y}{y} \}}
    }
    {\Gamma \vdash \delta - t \chck T}
    {}

\Rule{\CedChiRule}
    {
        \der{\D{1}}{\Gamma \vdash A \infr \star}
        \\ \der{\D{2}}{\Gamma \vdash t \chck A}
    }
    {\Gamma \vdash \chi\ A - t \infr A}
    {}

\Rule{\CedCheckRule}
    {
        \der{\D{1}}{\Gamma \vdash t \infr A}
        \\ \der{\D{2}}{A \cong B}
    }
    {\Gamma \vdash t \chck B}
    {}

\Rule{\CedTermTermApplyRule}
    {
        \der{\D{1}}{\Gamma \vdash f \cinfr \abs{\Pi}{x}{A}{B}}
        \\ \der{\D{2}}{\Gamma \vdash a \chck A}
    }
    {\Gamma \vdash f\ a \chck [x := \chi\ A - a]B}
    {}

\Rule{\CedTypeTermApplyRule}
    {
        \der{\D{1}}{\Gamma \vdash f \cinfr \abs{\Pi}{x}{\kappa_1}{B}}
        \\ \der{\D{2}}{\Gamma \vdash a \infr \kappa_2}
        \\ \der{\D{3}}{\kappa_1 \cong \kappa_2}
    }
    {\Gamma \vdash f \cdot a \chck [x := a]B}
    {}

\Rule{\CedErasedApplyRule}
    {
        \der{\D{1}}{\Gamma \vdash f \cinfr \abs{\forall}{x}{A}{B}}
        \\ \der{\D{2}}{\Gamma \vdash a \chck A}
    }
    {\Gamma \vdash f\ \edash a \chck [x := \chi\ A - a]B}
    {}

\Rule{\CedFirstProjectRule}
    {
        \der{\D{1}}{\Gamma \vdash f \cinfr \abs{\iota}{x}{A}{B}}
    }
    {\Gamma \vdash t.1 \infr A}
    {}

\Rule{\CedBetaRule}
    {
        \der{\D{1}}{T \betastar_n \{ a \simeq b \}}
        \\ \der{\D{2}}{FV(t) \subseteq dom(\Gamma)}
        \\ \der{\D{3}}{|a| \betaconv_\eta |b|}
    }
    {\Gamma \vdash \beta \{ t \} \chck T}
    {}

\Rule{\CedRewriteRule}
    {
        \der{\D{1}}{\Gamma \vdash e \cinfr \{ a \simeq b^\prime \}}
        \\ \der{\D{2}}{FV(b^\prime) \subseteq dom(\Gamma)}
        \\ \der{\D{3}}{|b^\prime| \betaconv_\eta |b|}
        \\ \der{\D{4}}{\Gamma \vdash [x := b]A \infr \star}
        \\ \der{\D{5}}{\Gamma \vdash t \chck [x := b]A }
        \\ \der{\D{6}}{[x := a]A \cong T}
    }
    {\Gamma \vdash \rho\ e\ @\ x\ \langle b \rangle.\ A - t \chck T}
    {}

\Rule{\CedPhiInferRule}
    {
        \der{\D{1}}{\Gamma \vdash e \chck \{ a \simeq b \}}
        \\ \der{\D{2}}{\Gamma \vdash a \infr A}
        \\ \der{\D{3}}{FV(b) \subseteq dom(\Gamma)}
    }
    {\Gamma \vdash \varphi\ e - a\ \{ b \} \infr A}
    {}

\Rule{\CedPhiCheckRule}
    {
        \der{\D{1}}{\Gamma \vdash e \chck \{ a \simeq b \}}
        \\ \der{\D{2}}{\Gamma \vdash a \chck A}
        \\ \der{\D{3}}{FV(b) \subseteq dom(\Gamma)}
    }
    {\Gamma \vdash \varphi\ e - a\ \{ b \} \chck A}
    {}

\Rule{\CedSymRule}
    {
        \der{\D{1}}{\Gamma \vdash e \cinfr \{ a \simeq b \}}
    }
    {\Gamma \vdash \text{\c{c}}\ e \infr \{ b \simeq a \}}
    {}

\Rule{\CedTypeCongRedRule}
    {
        \der{\D{1}}{A \betastar_n A^\prime \not\betared_n}
        \\ \der{\D{2}}{B \betastar_n B^\prime \not\betared_n}
        \\ \der{\D{2}}{A^\prime \cong^t B^\prime}
    }
    {A \cong B}
    {}

\Rule{\CedTypeCongVarRule}
    {
        \textcolor{white}{\_}
    }
    {x \cong^t x}
    {}

\Rule{\CedTypeCongTypeForallRule}
    {
        \der{\D{1}}{A_1 \cong A_2}
        \\ \der{\D{2}}{B_1 \cong B_2}
    }
    {\abs{\forall}{x}{A_1}{B_1} \cong^t \abs{\forall}{x}{A_2}{B_2}}
    {}

\Rule{\CedTypeCongTypeLambdaRule}
    {
        \der{\D{1}}{A_1 \cong A_2}
        \\ \der{\D{2}}{B_1 \cong B_2}
    }
    {\abs{\lambda}{x}{A_1}{B_1} \cong^t \abs{\lambda}{x}{A_2}{B_2}}
    {}

\Rule{\CedTypeCongIotaRule}
    {
        \der{\D{1}}{A_1 \cong A_2}
        \\ \der{\D{2}}{B_1 \cong B_2}
    }
    {\abs{\iota}{x}{A_1}{B_1} \cong^t \abs{\iota}{x}{A_2}{B_2}}
    {}

\Rule{\CedTypeCongTypeApplyRule}
    {
        \der{\D{1}}{A_1 \cong^t A_2}
        \\ \der{\D{2}}{B_1 \cong B_2}
    }
    {A_1 \cdot B_1 \cong^t A_2 \cdot B_2}
    {}

\Rule{\CedTypeCongKindForallRule}
    {
        \der{\D{1}}{\kappa_1 \cong \kappa_2}
        \\ \der{\D{2}}{B_1 \cong B_2}
    }
    {\abs{\forall}{x}{\kappa_1}{B_1} \cong^t \abs{\forall}{x}{\kappa_2}{B_2}}
    {}

\Rule{\CedTypeCongPiRule}
    {
        \der{\D{1}}{A_1 \cong A_2}
        \\ \der{\D{2}}{B_1 \cong B_2}
    }
    {\abs{\Pi}{x}{A_1}{B_1} \cong^t \abs{\Pi}{x}{A_2}{B_2}}
    {}
    
\Rule{\CedTypeCongKindLambdaRule}
    {
        \der{\D{1}}{\kappa_1 \cong \kappa_2}
        \\ \der{\D{2}}{B_1 \cong B_2}
    }
    {\abs{\lambda}{x}{\kappa_1}{B_1} \cong^t \abs{\lambda}{x}{\kappa_2}{B_2}}
    {}

\Rule{\CedTypeCongTermApplyRule}
    {
        \der{\D{1}}{A_1 \cong^t A_2}
        \\ \der{\D{2}}{|b_1| \betaconv_\eta |b_2|}
    }
    {A_1\ b_1 \cong^t A_2\ b_2}
    {}

\Rule{\CedTypeCongEqRule}
    {
        \der{\D{1}}{|a_1| \betaconv_\eta |a_2|}
        \\ \der{\D{2}}{|b_1| \betaconv_\eta |b_2|}
    }
    {\{ a_1 \simeq b_1 \} \cong^t \{ a_2 \simeq b_2 \}}
    {}

\Rule{\CedKindCongAxiomRule}
    {
        \textcolor{white}{\_}
    }
    {\star \cong \star}
    {}

\Rule{\CedKindCongTypePiRule}
    {
        \der{\D{1}}{A_1 \cong A_2}
        \\ \der{\D{2}}{\kappa_1 \cong \kappa_2}
    }
    {\abs{\Pi}{x}{A_1}{\kappa_1} \cong \abs{\Pi}{x}{A_2}{\kappa_2}}
    {}

\Rule{\CedKindCongKindPiRule}
    {
        \der{\D{1}}{\kappa^\prime_1 \cong \kappa^\prime_2}
        \\ \der{\D{2}}{\kappa_1 \cong \kappa_2}
    }
    {\abs{\Pi}{x}{\kappa^\prime_1}{\kappa_1} \cong \abs{\Pi}{x}{\kappa^\prime_2}{\kappa_2}}
    {}

\newcommand{\cBool}{\textnormal{Bool}}
\newcommand{\ctt}{\textnormal{tt}}
\newcommand{\cff}{\textnormal{ff}}

\newcommand{\cId}{\textnormal{Id}}
\newcommand{\crefl}{\textnormal{refl}}
\newcommand{\ctheta}{\textnormal{theta}}
\newcommand{\csubst}{\textnormal{subst}}
\newcommand{\cdelta}{\textnormal{delta}}
\newcommand{\cirrel}{\textnormal{irrel}}

The Calculus of Dependent Lambda Eliminations (CDLE) was first introduced in 2017 \cite{stump2017_cdle} as the core system for the in progress Cedille tool.
At that time, CDLE included complicated machinery for lifting lambda terms to the type-level enabling some large eliminations.
Over the years, the core system for the Cedille tool was still referred to as CDLE as it evolved culminating in the current core system used in Cedille version 1.1.2 \cite{stump2021_cedillecore}.
The ideas leading to CDLE grew over time with work on efficient lambda encodings in total theories \cite{stump2016_encodings}; self-types for encodings \cite{fu2014self}; and experiments involving irrelevance \cite{sjoberg2012irrelevance,sjoberg2011equality}.
Ultimately, the modern version of CDLE, as presented in this chapter, is the culmination of these efforts.

CDLE is an affirmative answer to the question: is lambda-encoded data enough for a proof assistant?
While there may be other philosophical objections, Mendler-style encodings have been shown to be efficient and support course-of-values induction \cite{firsov2018_mendler,firsov2018course}.
The $\varphi$ construct, an idea borrowed from the direct computation rule of Nuprl \cite{allen2000}, enabled several encodings including efficient Mendler encodings.
A non-exhaustive list of the successes of CDLE include: quotient subtypes \cite{marmaduke2020quotients}; coinductive data \cite{jenkins2020efficient}; zero-cost constructor subtypes \cite{marmaduke2020_constructor_subtyping}; monotonic recursive types \cite{jenkins2021monotone}; simulated large eliminations \cite{jenkins2021_large_elim}; and inductive-inductive data \cite{marmaduke2023_indind}.

CDLE commits to impredicative (i.e. parametric in the sense of $F^\omega$) quantification.
With that in mind the well-studied reader may not be surprised at the power and versatility of CDLE.
However, taming impredicative quantification without losing logical consistency is a difficult task.
Indeed, this is why several proof assistants have discarded impredicative quantification or relegated it to a universe of propositions.
A core philosophy behind both CDLE and $\ced$ is to try a different path and embrace impredicative quantification.
To achieve that goal a realizability model was developed for CDLE to demonstrate logical consistency \cite{stump2021_cedillecore}.
This chapter will describe a model of $\ced$ in CDLE to prove consistency.

\section{Calculus of Dependent Lambda Eliminations}

CDLE is described using an intrinsic style where syntax is presented directly with the typing derivation.
However, erasure is still a crucial part of CDLE which gives it an extrinsic philosophy.
Whether a system is intrinsic or extrinsic is perhaps not an interesting distinction.
Technically, $\ced$ is described extrinsically because syntax is defined independently of the typing relation, but there is no essential reason for this choice.
Moreover, any intrinsic system necessarily admits a projection of its raw syntax which enables an extrinsic presentation.
It is better to think about these details via their philosophical import.
An intrinsic system wishes to say that raw syntax has no meaning, or at the very least no meaning the designer cares about.
Alternatively, an extrinsic system wishes to say that types are in some sense only annotations with the raw syntax being primary.


\begin{figure}
    \centering
    $$\CedAxiomRule \CedTypeKindPiRule \CedKindKindPiRule$$
    \caption{
        TODO
    }
    \label{fig:4:kind}
\end{figure}


\begin{figure}
    \centering
    \begin{minipage}{0.5\textwidth}
        $$\CedTypeVarRule$$
        $$\CedTypeForallRule$$
        $$\CedTypeTypeLambdaRule$$
        $$\CedTermTypeApplyRule$$
        $$\CedIotaRule$$
    \end{minipage}%
    \begin{minipage}{0.5\textwidth}
        $$\CedKindForallRule$$
        $$\CedTypeTypePiRule$$
        $$\CedKindTypeLambdaRule$$
        $$\CedTypeTypeApplyRule$$
        $$\CedEqRule$$
    \end{minipage}%
    \caption{
        TODO
    }
    \label{fig:4:type}
\end{figure}


\begin{figure}
    \centering
    $$\Gamma \vdash t \cinfr A \text{ iff } \exists\ T.\ (\Gamma \vdash t \infr T) \wedge (T \betastar_n A)$$
    \begin{minipage}{0.5\textwidth}
        $$\CedTermVarRule$$
        $$\CedTypeTermLambdaRule$$
        $$\CedKindTermLambdaRule$$
        $$\CedErasedLambdaRule$$
        $$\CedPairRule$$
        $$\CedSecondProjectRule$$
        $$\CedDeltaRule$$
        $$\CedChiRule$$
        $$\CedSymRule$$
    \end{minipage}%
    \begin{minipage}{0.5\textwidth}
        $$\CedCheckRule$$
        $$\CedTermTermApplyRule$$
        $$\CedTypeTermApplyRule$$
        $$\CedErasedApplyRule$$
        $$\CedFirstProjectRule$$
        $$\CedBetaRule$$
        $$\CedRewriteRule$$
        $$\CedPhiInferRule$$
        $$\CedPhiCheckRule$$
    \end{minipage}%
    \caption{
        TODO
    }
    \label{fig:4:term}
\end{figure}


As one might guess these philosophical positions are not entirely black and white.
For example, Pfenning demonstrates how both methods can be combined \cite{pfenning2008church}.
Cedille has been historically described as an extrinsic system.
The type theory $\ced$ might best be described as a \textit{combined} system, both intrinsic and extrinsic.
That is, a \textit{proof} has no meaning as just syntax, but an \textit{object} discards the extra information as mere annotations.

The CLDE type system kind formation rules are presented in Figure~\ref{fig:4:kind}, type formation rules in Figure~\ref{fig:4:type}, and term annotation rules in Figure~\ref{fig:4:term}.
Lowercase letters are used to refer to metavariables of terms, uppercase letters for metavariables of types, and variations of $\kappa$ for metavariables of kinds.
Call-by-name reduction of the $\lambda$-calculus fragment is used in the rules for types and is written $A \betared_n B$.
The purpose of this relation is only to reveal a constructor for a type, thus weak-head normal form is sufficient.
Conversion for types is presented in Figure~\ref{fig:4:type_cong} and kind conversion in Figure~\ref{fig:4:kind_cong}.
Note that these conversion relations correspond to $\beta$-conversion for types and kinds.
Finally, erasure of terms is presented in Figure~\ref{fig:4:erasure}.
Erasure is only meaningful for terms in CDLE unlike in $\ced$ where it is defined for all syntax.

The presentation in this work deviates from other descriptions of CDLE by adding a symmetry rule for equality (\c{c}).
This rule is admissible using the rewrite rule ($\rho$), but it is convenient to have available for the model.
Otherwise, the presentation is identical to the one by Stump and Jenkins \cite{stump2021_cedillecore}.

\begin{figure}
    \centering
    $$\CedTypeCongRedRule$$
    \begin{minipage}{0.5\textwidth}
        $$\CedTypeCongVarRule$$
        $$\CedTypeCongTypeForallRule$$
        $$\CedTypeCongTypeLambdaRule$$
        $$\CedTypeCongIotaRule$$
        $$\CedTypeCongTypeApplyRule$$
    \end{minipage}%
    \begin{minipage}{0.5\textwidth}
        $$\CedTypeCongKindForallRule$$
        $$\CedTypeCongPiRule$$
        $$\CedTypeCongKindLambdaRule$$
        $$\CedTypeCongTermApplyRule$$
        $$\CedTypeCongEqRule$$
    \end{minipage}%
    \caption{
        Definition of conversion for types in CDLE.
    }
    \label{fig:4:type_cong}
\end{figure}

\begin{figure}
    \centering
    $$\CedKindCongAxiomRule$$
    \begin{minipage}{0.5\textwidth}
        $$\CedKindCongTypePiRule$$
    \end{minipage}%
    \begin{minipage}{0.5\textwidth}
        $$\CedKindCongKindPiRule$$
    \end{minipage}%
    \caption{
        Defintion of conversion for kinds in CDLE.
    }
    \label{fig:4:kind_cong}
\end{figure}



\begin{figure}
    \centering
    \begin{minipage}{0.5\textwidth}
        \begin{align*}
            |x| &= x \\
            |f\ a| &= |f|\ |a| \\
            |\absu{\Lambda}{x}{t}| &= |t| \\
            |[t_1, t_2]| &= |t_1| \\
            |t.2| &= |t| \\
            |\delta - t| &= \absu{\lambda}{x}{x} \\
            |\varphi\ e - a\ \{ b \}| &= |b|
        \end{align*}
    \end{minipage}%
    \begin{minipage}{0.5\textwidth}
        \begin{align*}
            |\absu{\lambda}{x}{t}| &= \absu{\lambda}{x}{|t|} \\
            |f \cdot a| &= |f| \\
            |f\ \edash a| &= |f| \\
            |t.1| &= |t| \\
            |\beta \{ t \}| &= |t| \\
            |\rho\ e\ @\ x\ \langle a \rangle.\ A - t| &= |t| \\
            |\chi\ A - t| &= |t|
        \end{align*}
    \end{minipage}%
    \caption{
        TODO
    }
    \label{fig:4:erasure}
\end{figure}

A few useful facts about CDLE are needed before defining the model.
First, some helpful terms are defined below.
Note that an annotation rule ($\chi$) is added to guarantee that each definition always infers a type.
The $\cBool$ definition is a standard Church encoded boolean type, with its two associated values: $\ctt$ and $\cff$.
An identity type, $\cId$, is defined as a desired output of the model for the equality of $\ced$.
Indeed, CDLEs equality is very flexible in comparison to $\ced$.
Not only is it untyped, but it allows for any well-scoped term to serve as the erasure (or object) of a reflexivity proof.
\begin{align*}
    \cBool &:= \abs{\forall}{X}{\star}{X \to X \to X} \\
    \ctt &:= \chi\ \cBool\ - \absu{\Lambda}{X}{
        \absu{\lambda}{x\ y}{x}
    } \\
    \cff &:= \chi\ \cBool\ - \absu{\Lambda}{X}{
        \absu{\lambda}{x\ y}{y}
    } \\
    \cId &:= \abs{\lambda}{A}{\star}{
        \abs{\lambda}{a\ b}{A}{
            \abs{\iota}{e}{\{ a \simeq b \}}{
                \abs{\iota}{y}{\{ (\absu{\lambda}{x}{x}) \simeq e \}}{
                    \abs{\forall}{X}{\star}{
                        X \to X
                    }
                }
            }
        }
    } \\
    \crefl &:= \chi\ \abs{\forall}{A}{\star}{
            \abs{\forall}{a}{A}{
                \cId \cdot A\ a\ a
            }
        }\ -
        \\ &\absu{\Lambda}{A\ a}{
        [\beta\{ \absu{\lambda}{x}{x} \}, [\beta\{ \absu{\lambda}{x}{x} \}, \absu{\Lambda}{X}{\absu{\lambda}{x}{x}}]]
    } \\
    \cdelta &:= \chi\ \cId \cdot \cBool\ \ctt\ \cff \to \abs{\forall}{X}{\star}{X}\ -
        \\ &\absu{\lambda}{e}{
        (\delta - e.1) \cdot (\cId \cdot \cBool\ \ctt\ \cff \to \abs{\forall}{X}{\star}{X})\ e
    }
\end{align*}
Aside from the previous terms it is also useful to have terms representing the target output of the substitution and promotion rules of $\ced$.
All of these terms are constructed to obtain specific erasures.
\begin{align*}
    \ctheta &:= \chi\ 
        \abs{\forall}{A}{\star}{
            \abs{\forall}{B}{A \to \star}{
                \abs{\forall}{a\ b}{(\abs{\iota}{x}{A}{B\ x})}{
                    \\ &\cId \cdot A\ a.1\ b.1 \to \cId \cdot (\abs{\iota}{x}{A}{B\ x})\ a\ b
                }
            }
        }\ -
        \\ &\absu{\Lambda}{A\ B\ a\ b}{
            \absu{\lambda}{e}{
                \\ &\varphi\ (\rho\ e.2.1\ @\ x\ \langle e \rangle.\ \{ x \cong e \}\ -\ \beta\{ \absu{\lambda}{x}{x} \})\ -
                \\ &(\rho\ e.1\ @\ x\ \langle b \rangle.\ \cId \cdot (\abs{\iota}{x}{A}{B\ x}))\ x\ b\ -\ \crefl \cdot (\abs{\iota}{x}{A}{B\ x})\ \edash b)
                \\ &\{ e \} 
            }
        }
\end{align*}
\begin{align*}
    \csubst &:= \chi\ 
        \abs{\forall}{A}{\star}{
            \abs{\forall}{a\ b}{A}{
                \abs{\forall}{P}{ (\abs{\Pi}{y}{A}{\cId \cdot A\ a\ y \to \star}) }{
                    \\ &\abs{\Pi}{e}{\cId \cdot A\ a\ b}{
                        P\ a\ (\crefl \cdot A\ \edash a) \to P\ b\ e
                    }
                }
            }
        }\ -
        \\ &\absu{\Lambda}{A\ a\ b\ P}{
            \absu{\lambda}{e}{
                \\ &\rho\ e.2.1\ @\ x\ \langle e \rangle.\ P\ a\ x \to P\ b\ e\ -\
                \\ &\rho\ e.1\ @\ x\ \langle b \rangle.\ P\ x\ e \to P\ b\ e\ -\
                \\ &e.2.2 \cdot (P\ b\ e)
            }
    }
\end{align*}
The erasure of each term is designed to match with the erasure of the associated construct in $\ced$.
While this might not be strictly necessary to obtain a model of $\ced$ inside CDLE it makes the process easier.
Moreover, carefully crafting terms with specific erasures is a trivial matter in CDLE because of the $\varphi$ rule.
\begin{align*}
    |\ctt| &= \absu{\lambda}{x\ y}{x} \\
    |\cff| &= \absu{\lambda}{x\ y}{y} \\
    |\crefl \cdot A\ \edash a| &= \absu{\lambda}{x}{x} \\
    |\cdelta\ e| &= |e| \\
    |\ctheta \cdot A \cdot B\ \edash a\ \edash b\ e| &= |e| \\
    |\csubst \cdot A\ \edash a\ \edash b \cdot P\ e| &= |e|
\end{align*}
Finally, each of these terms is shown to infer the desired type.
Note that for syntax that is type-like, such as $\cId$ and $\cBool$, there is no type-checking rule, only an inference judgment.
Moreover, the $\chi$ rule only works with term-like syntax.
Thus, for these definitions more care is needed to infer the correct kind.
\begin{lemma}
    \label{lem:4:c1_stuff}
    \textcolor{white}{\_}
    \begin{enumerate}
        \item $\vdash_{\oldced} \cBool \infr \star$
        \item $\vdash_{\oldced} \ctt \infr \cBool$
        \item $\vdash_{\oldced} \cff \infr \cBool$
        \item $\vdash_{\oldced} \cId \infr \abs{\Pi}{A}{\star}{A \to A \to \star}$
        \item {
            $
                \vdash_{\oldced} \crefl \infr \abs{\forall}{A}{\star}{
                    \abs{\forall}{a}{A}{
                        \cId \cdot A\ a\ a
                    }
                }
            $
        }
        \item $\vdash_{\oldced} \cdelta \infr \cId \cdot \cBool\ \ctt\ \cff \to \abs{\forall}{X}{\star}{X}$
        \item {
            $\vdash_{\oldced} \ctheta \infr
            \begin{aligned}
                 &\abs{\forall}{A}{\star}{
                    \abs{\forall}{B}{A \to \star}{
                        \abs{\forall}{a\ b}{(\abs{\iota}{x}{A}{B\ x})}{
                            \\ &\cId \cdot A\ a.1\ b.1 \to \cId \cdot (\abs{\iota}{x}{A}{B\ x})\ a\ b
                        }
                    }
                }
            \end{aligned}
            $
        }
        \item {
            $\vdash_{\oldced} \csubst \infr 
            \begin{aligned}
                &\abs{\forall}{A}{\star}{
                    \abs{\forall}{a\ b}{A}{
                        \abs{\forall}{P}{ (\abs{\Pi}{y}{A}{\cId \cdot A\ a\ y \to \star}) }{
                            \\ &\abs{\Pi}{e}{\cId \cdot A\ a\ b}{
                                P\ a\ (\crefl \cdot A\ \edash a) \to P\ b\ e
                            }
                        }
                    }
                }
            \end{aligned}
            $
        }
    \end{enumerate}
\end{lemma}
\begin{proof}
    Straightforward by applying a short sequence of CDLE rules in each case.
    These inferences are trivially formalized in the Cedille tool.
\end{proof}

A small collection of additional lemmas about CDLE is needed to prove soundness of the model and presented next.
These lemmas are standard: weakening, symmetry of conversion, and transitivity of conversion.
The only real difficulty is the bidirectional presentation which requires stating the desired lemma for each variation of judgment and using mutual recursion in the proof.

\begin{lemma}
    \label{lem:4:c1_weakening}
    Suppose $\Gamma \vdash_{\oldced} T \infr K$ and $x$ fresh
    \begin{enumerate}
        \item If $t$ is a kind and $\Gamma, \Delta \vdash_{\oldced} t$ then $\Gamma, x : T, \Delta \vdash_{\oldced} t$
        \item If $t$ is a type and $\Gamma, \Delta \vdash_{\oldced} t \infr K$ then $\Gamma, x : T, \Delta \vdash_{\oldced} t \infr K$
        \item If $t$ is a term and $\Gamma, \Delta \vdash_{\oldced} t \infr A$ then $\Gamma, x : T, \Delta \vdash_{\oldced} t \infr A$
        \item If $t$ is a term and $\Gamma, \Delta \vdash_{\oldced} t \chck A$ then $\Gamma, x : T, \Delta \vdash_{\oldced} t \chck A$
    \end{enumerate}
\end{lemma}
\begin{proof}
    Straightforward by mutual recursion on the associated judgments.
\end{proof}

\begin{lemma}
    \label{lem:4:c1_sym}
    \textcolor{white}{\_}
    \begin{enumerate}
        \item If $a, b$ are terms and $|a| \betaconv_\eta |b|$ then $|b| \betaconv_\eta |a|$
        \item If $A, B$ are types and values and $A \cong^t B$ then $B \cong^t A$
        \item If $A, B$ are types and $A \cong B$ then $B \cong A$
        \item If $A, B$ are kinds and $A \cong B$ then $B \cong A$
    \end{enumerate}
\end{lemma}
\begin{proof}
    Note that \textit{1.} holds because $|a|$ and $|b|$ are untyped $\lambda$-calculus terms.
    For \textit{2.} through \textit{4.} mutual recursion and pattern match on $A$ is sufficient.
\end{proof}

\begin{lemma}
    \label{lem:4:c1_trans}
    \textcolor{white}{\_}
    \begin{enumerate}
        \item If $a, b, c$ are terms, $|a| \betaconv_\eta |b|$, and $|b| \betaconv_\eta |c|$ then $|a| \betaconv_\eta |c|$
        \item If $A, B, C$ are types and values, $A \cong^t B$, and $B \cong^t C$ then $A \cong^t C$
        \item If $A, B, C$ are types, $A \cong B$, and $B \cong C$ then $A \cong C$
        \item If $A, B, C$ are kinds, $A \cong B$, and $B \cong C$ then $A \cong C$
    \end{enumerate}
\end{lemma}
\begin{proof}
    Note that \textit{1.} holds because $|a|$ and $|b|$ are untyped $\lambda$-calculus terms and reduction is confluent.
    The remainder are proved by mutual recursion.
    Note that in \textit{3.} the types $A, B$, and $C$ are reduced using call-name to a weak-head normal form.
    In particular, this reduction strategy is deterministic, thus $B \betastar_n B^\prime$ for a unique $B^\prime$.
    This combined with using \textit{2.} is sufficient for the \textit{3.} case.
    The other two cases follow by pattern matching on $B$, inversion on the respective conversions, and applying the IH.
\end{proof}

\section{Counterexamples to Decidability of Type Checking in CDLE}

It is well-known that Cedille does not enjoy decidability of type checking.
However, it might not be clear exactly how this property fails.
Below is a series of formalized examples that will loop when attempting to check using the Cedille tool.
\begin{enumerate}
    \item {
        First, there is an obvious problem caused by equality and reflexivity witnesses being untyped $\lambda$-calculus terms.
        Under this regime $\Omega = (\absu{\lambda}{x}{x\ x}) (\absu{\lambda}{x}{x\ x})$ is easily inserted.
\begin{minted}{Lean}
    bad : { (λ x. x x) (λ x. x x) ≃ λ x. x } = β.

    omega : { λ x. x ≃ λ x. x } = β{(λ x. x x) (λ x. x x)}.
    bad : { omega ≃ λ x. x } = β.
\end{minted}
        This is an unsurprising consequence of the design of the CDLE equality type and certainly the least interesting instance of non-termination.
    }
    \item {
        To fix the previous case the equality could instead be \textit{annotated}.
        Note, this does not mean that equality is typed, but only that the indices and reflexivity witnesses must infer a type.
        With this change the rewrite rule ($\rho$) allows annotated substitutions.
        As it turns out, these kinds of rewrites also enable non-termination.
\begin{minted}{Lean}
    Id : Π A:★. Π B:★. A ➔ B ➔ ★ = λ A:★. λ B:★. λ x:A. λ y:B. {x ≃ y}.
    Unit : ★ = ∀ X:★. X ➔ X.
    unit : Unit = Λ X. λ x. x.
    self : Unit ➔ Unit = λ u. u u.
    False : ★ = ∀ X:★. X.
    bad : ∀ P:False ➔ ★. Π f:False. P f
    = Λ P. λ f. {e1 = f·(Id·False·(Unit ➔ Unit) f self)}
        - {e2 = f·(Id·False·False f (f·(False ➔ False) f))}
        - ρ e2 - ρ e1 - (f·(P f)).
\end{minted}
        The equality described in the example should be typed at False, but an equality casting False to $\textnormal{Unit} \to \textnormal{Unit}$ is used to generate $\Omega$ in the resulting predicate.
        Thus, when the tool attempts to check the convertibility of f with $\Omega$ the type checking algorithm loops.
    }
    \item {
        Even when equality is typed the rewrite rule can cause non-termination.
        For this example, it is necessary that there is some method to encode equality of types.
        In the formalization below a simulated large elimination on booleans is used.
        Note that the only way to construct this large elimination that is currently known is by using the $\varphi$ construct.
        However, any extension or feature that enables discussing type equalities would unlock this example (e.g. universe hierarchies with an equality type at each level).
\begin{minted}{Lean}
    False : ★ = ∀ X:★. X.
    Not : ★ ➔ ★ = λ A:★. A ➔ False.
    True : ★ = Not·False.
    self : True = λ f. f·(False ➔ False) f.
    Bool : ★ = ∀ X:★. X ➔ X ➔ X.
    tt : Bool = Λ X. λ x. λ y. x.
    ff : Bool = Λ X. λ x. λ y. y.
    Id : Π A:★. A ➔ A ➔ ★ = λ A:★. λ a:A. λ b:A. {b ≃ a}.
    subst : ∀ A:★. ∀ a:A. ∀ b:A. ∀ P:A ➔ ★. P a ➔ Id·A a b ➾ P b
    = Λ A. Λ a. Λ b. Λ P. λ p. Λ i. ρ i - p.
    
    elim : ★ ➔ ★ ➔ Bool ➔ ★
    = λ A:★. λ B:★. λ x:Bool. ι _:{x ≃ tt} ➾ A. {x ≃ ff} ➾ B.
    in1 : ∀ A:★. ∀ B:★. A ➔ elim·A·B tt
    = Λ A. Λ B. λ a. [Λ e. a
        , Λ e. {f:False = δ - e} - φ (f·{f ≃ a}) - (f·B) {a}].
    
    cast : ∀ A:★. ∀ B:★. ∀ a:Bool. ∀ b:Bool.
        Id·Bool a b ➾ elim·A·B a ➔ elim·A·B b
    = Λ A. Λ B. Λ a. Λ b. Λ e. λ p. subst·Bool -a -b ·(elim·A·B) p -e.
    omega : Not·(∀ a:Bool. ∀ b:Bool. Id·Bool a b)
    = λ x. (cast·True·False -tt -ff -(x -tt -ff) (in1 self)).2 -β.
    Omega : Not·(∀ a:Bool. ∀ b:Bool. Id·Bool a b)
    = λ x. self (omega x).
    bad : {Omega ≃ λ x. x} = β.
\end{minted}
        This example is a direct adaptation of Abel's work \cite{abel2020_normalization}.
        It depends on the elimination form of equality being irrelevant, an impredicative universe, and some method of discussing equality of types.
        This is the first of three examples of irrelevance in equality eliminators causing non-termination.
    }
    \item {
        The separation rule ($\delta$) is also irrelevant and can cause non-termination.
        Constructing an $\Omega$ is easy as $\delta$ erases to an identity function, thus applying two self terms directly to a $\delta$ immediately yields $\Omega$.
\begin{minted}{Lean}
    False : ★ = ∀ X:★. X.
    Unit : ★ = ∀ X:★. X ➔ X.
    self : Unit ➔ Unit = λ u. u u.
    Bool : ★ = ∀ X:★. X ➔ X ➔ X.
    tt : Bool = Λ X. λ x. λ y. x.
    ff : Bool = Λ X. λ x. λ y. y.
    Id : Π A:★. A ➔ A ➔ ★ = λ A:★. λ x:A. λ y:A. {y ≃ x}.
    omega : Id·Bool tt ff ➾ False
    = Λ e. (δ - e)·((Unit ➔ Unit) ➔ (Unit ➔ Unit) ➔ False) self self.
    bad : {omega ≃ λ x. x} = β.
\end{minted}
    }
    \item {
        Finally, the $\varphi$ construct is capable of giving a term a recursive type in an inconsistent context.
        Constructing $\Omega$ is a trivial consequence.
\begin{minted}{Lean}
    False : ★ = ∀ X:★. X.
    Unit : ★ = ∀ X:★. X ➔ X.
    self : Unit ➔ Unit = λ u. u u.
    b : False ➔ ι _:Unit ➔ Unit. Unit = λ f. [f·(Unit ➔ Unit), f·Unit].
    e : Π f:False. {b f ≃ self} = λ f. f·{b f ≃ self}.
    omega : False ➾ Unit = Λ f. self (φ (e f) - (b f).2 {self}).
    bad : {omega ≃ λ x. x} = β.
\end{minted}
    }
\end{enumerate}

\section{Model}

Figure~\ref{fig:4:model} describes the model of $\ced$ in CDLE.
This model is straightforward: abstractions to abstractions, applications to applications, pairs to pairs, etc.
The complicated part is the equality type, however all the necessary work to find suitable terms for the rules of equality was already completed above.
There is one hiccup involving the promotion ($\vartheta$) rule.
In order to have a fully applied $\ctheta$ it must be the case that the annotation for $\vartheta$ is an intersection type.
For proofs this will always be the case, but for arbitrary syntax it is not necessarily true.
To work around this a catch-all case is defined where the model only interprets the equality proof $e$.
This choice is largely arbitrary, but it is picked to make sure that one critical property is preserved: erasure.


\begin{figure}
    \centering
    \begin{minipage}{0.5\textwidth}
        \begin{align*}
            \sema{(x : A) \to_\tau B} &= \abs{\Pi}{x}{\sema{A}}{\sema{B}} \\
            \sema{(x : A) \to_\omega B} &= \abs{\Pi}{x}{\sema{A}}{\sema{B}} \\
            \sema{(x : A) \to_0 B} &= \abs{\forall}{x}{\sema{A}}{\sema{B}} \\
            \sema{\abs{\lambda_\tau}{x}{A}{t}} &= \abs{\lambda}{x}{\sema{A}}{\sema{t}} \\
            \sema{\abs{\lambda_\omega}{x}{A}{t}} &= \absu{\lambda}{x}{\sema{t}}\\
            \sema{\abs{\lambda_0}{x}{A}{t}} &= \absu{\Lambda}{x}{\sema{t}}
        \end{align*}
    \end{minipage}%
    \begin{minipage}{0.5\textwidth}
        \begin{align*}
            \sema{\star} &= \star \\
            \sema{x_K} &= x \\
            \sema{\app{f}{\tau}{a}} &= \sema{f}\ \sema{a} \quad&\text{if }& a\pterm \\
            \sema{\app{f}{\tau}{a}} &= \sema{f} \cdot \sema{a} \quad&\text{if }& a\ptype \\
            \sema{\app{f}{\omega}{a}} &= \sema{f}\ \sema{a} \\
            \sema{\app{f}{0}{a}} &= \sema{f}\ \edash \sema{a} \quad&\text{if }& a\pterm \\
            \sema{\app{f}{0}{a}} &= \sema{f} \cdot \sema{a} \quad&\text{if }& a\ptype
        \end{align*}
    \end{minipage}
    \begin{minipage}{0.5\textwidth}
        \begin{align*}
            \sema{(x : A) \cap B} &= \abs{\iota}{x}{\sema{A}}{\sema{B}} \\
            \sema{[t_1, t_2, A]} &= [\sema{t_1}, \sema{t_2}]
        \end{align*}
    \end{minipage}%
    \begin{minipage}{0.5\textwidth}
        \begin{align*}
            \sema{t.1} &= \sema{t}.1 \\
            \sema{t.2} &= \sema{t}.2
        \end{align*}
    \end{minipage}
    \begin{align*}
        \sema{a =_A b} &= \cId \cdot \sema{A}\ \sema{a}\ \sema{b} \\
        \sema{\text{refl}(t; A)} &= \crefl \cdot \sema{A}\ \edash \sema{t} \\
        \sema{\vartheta(e, a, b; (x : A) \cap B)} &= \ctheta \cdot \sema{A} \cdot \sema{B}\ \edash \sema{a}\ \edash \sema{b}\ \sema{e} \\
        \sema{\vartheta(e, a, b; T)} &= \sema{e} \\
        \sema{\psi(e, a, b; A, P)} &= \csubst  \cdot \sema{A}\ \edash \sema{a}\ \edash \sema{b} \cdot \sema{P}\ \sema{e} \\
        \sema{\varphi(a, b, e)} &= \varphi\ \text{\c{c}}\ \sema{e}.1 - \sema{b}\ \{ \sema{a} \} \\
        \sema{\delta(e)} &= \cdelta\ \sema{e}
    \end{align*}
    \begin{align*}
        \sema{\varepsilon} &= \varepsilon \\
        \sema{\Gamma, x : A} &= \sema{\Gamma}, x : \sema{A}
    \end{align*}
    \caption{
        Model definition interpreting $\ced$ in CDLE.
    }
    \label{fig:4:model}
\end{figure}



\begin{lemma}
    \label{lem:4:sema_erasure}
    If $t\pterm$ then $\sema{|t|} = |\sema{t}|$
\end{lemma}
\begin{proof}
    By induction on $t$ and inversion on $t\pterm$.
    The case of first projection and first equality promotion cases are omitted.

    $\text{Case: }t = x_\star$
    \begin{proofcase}
        Have $\sema{|x_\star|} = \sema{x_\star} = x$ and $|\sema{x_\star}| = |x| = x$, hence trivial.
    \end{proofcase}

    $\text{Case: }t = \abs{\lambda_0}{x}{A}{b}$
    \begin{proofcase}
        Have $\sema{|\abs{\lambda_0}{x}{A}{b}|} = \sema{|b|}$ and $|\sema{\abs{\lambda_0}{x}{A}{b}}| = |\absu{\Lambda}{x}{\sema{b}}| = |\sema{b}|$.
        Note that $b\pterm$, hence by the IH $\sema{|b|} = |\sema{b}|$.
    \end{proofcase}

    $\text{Case: }t = \abs{\lambda_\omega}{x}{A}{b}$
    \begin{proofcase}
        Have $\sema{|\abs{\lambda_\omega}{x}{A}{b}|} = \absu{\lambda}{x}{\sema{|b|}}$ and $|\sema{\abs{\lambda_\omega}{x}{A}{b}}| = |\absu{\lambda}{x}{\sema{b}}| = \absu{\lambda}{x}{|\sema{b}|}$.
        Note that $b\pterm$, hence by the IH $\sema{|b|} = |\sema{b}|$.
    \end{proofcase}

    $\text{Case: }t = \app{f}{0}{a}$
    \begin{proofcase}
        Have $\sema{|\app{f}{0}{a}|} = \sema{|f|}$ and $|\sema{\app{f}{0}{a}}| = |\sema{f}\ \edash \sema{a}| = |\sema{f}|$.
        Given $\app{f}{0}{a}\pterm$ it is always the case that $f\pterm$.
        Thus, by the IH $\sema{|f|} = |\sema{f}|$.
    \end{proofcase}

    $\text{Case: }t = \app{f}{\omega}{a}$
    \begin{proofcase}
        Have $\sema{|\app{f}{\omega}{a}|} = \sema{|f|}\ \sema{|a|}$ and $|\sema{\app{f}{\omega}{a}}| = |\sema{f}|\ |\sema{a}|$.
        Note that $f, a\pterm$ because the mode is $\omega$ there is no possibility of $a\ptype$.
        Hence, by the IH $\sema{|f|} = |\sema{f}|$ and $\sema{|a|} = |\sema{a}|$.
    \end{proofcase}

    $\text{Case: }t = [t_1, t_2; A]$
    \begin{proofcase}
        Have $\sema{|[t_1, t_2; A]|} = \sema{|t_1|}$ and $|\sema{[t_1, t_2; A]}| = |[\sema{t_1}, \sema{t_2}]| = |\sema{t_1}|$.
        By the IH applied to $t_1\pterm$: $\sema{|t_1|} = |\sema{t_1}|$.
    \end{proofcase}

    $\text{Case: }t = t.2$
    \begin{proofcase}
        Have $\sema{|t.2|} = \sema{|t|}$ and $|\sema{t.2}| = |\sema{t}.2| = |\sema{t}|$.
        By the IH applied to $t\pterm$: $\sema{|t|} = |\sema{t}|$.
    \end{proofcase}

    $\text{Case: }t = \prefl(a; A)$
    \begin{proofcase}
        Have $\sema{|\prefl(a; A)|} = \sema{\abs{\lambda}{x}{\diamond}{x_\star}} = \absu{\lambda}{x}{x}$ and $|\sema{\prefl(a; A)}| = |\crefl \cdot \sema{A}\ \edash \sema{a}| = \absu{\lambda}{x}{x}$.
    \end{proofcase}

    $\text{Case: }t = \vartheta(e, a, b; T)$
    \begin{proofcase}
        Have $\sema{|\vartheta(e, a, b; T)|} = \sema{|e|}$.
        Suppose $T = (x : A) \cap B$ then $|\sema{\vartheta(e, a, b; (x : A) \cap B)}| = |\ctheta \cdot \sema{A} \cdot \sema{B}\ \edash \sema{a}\ \edash \sema{b}\ \sema{e}| = |\sema{e}|$.
        Otherwise, $|\sema{\vartheta(e, a, b; T)}| = |\sema{e}|$.
        By the IH applied to $e\pterm$: $\sema{|e|} = |\sema{e}|$.
    \end{proofcase}

    $\text{Case: }t = \psi(e, a, b; A, P)$
    \begin{proofcase}
        Have $\sema{|\psi(e, a, b; A, P)|} = \sema{|e|}$ and $|\sema{\psi(e, a, b; A, P)}| = |\csubst \cdot \sema{A}\ \edash \sema{a}\ \edash \sema{b} \cdot \sema{P}\ \sema{e}| = |\sema{e}|$.
        By the IH applied to $e\pterm$: $\sema{|e|} = |\sema{e}|$.
    \end{proofcase}

    $\text{Case: }t = \varphi(a, b, e)$
    \begin{proofcase}
        Have $\sema{|\varphi(a, b, e)|} = \sema{|a|}$ and $|\sema{\varphi(a, b, e)}| = |\varphi\ \text{\c{c}}\ \sema{e}.1 - \sema{b}\ \{ \sema{a} \}| = |\sema{a}|$.
        By the IH applied to $a\pterm$: $\sema{|a|} = |\sema{a}|$.
    \end{proofcase}

    $\text{Case: }t = \delta(e)$
    \begin{proofcase}
        Have $\sema{|\delta(e)|} = \sema{|e|}$ and $|\sema{\delta(e)}| = |\cdelta\ \sema{e}| = |\sema{e}|$.
        By the IH applied to $e\pterm$: $\sema{|e|} = |\sema{e}|$.
    \end{proofcase}
\end{proof}

To obtain soundness we first need to know that conversion is preserved.
Luckily, because $\ced$ terms are closely matched with CDLE terms lemmas involving reduction can be precise.

\begin{lemma}
    \label{lem:4:sema_subst}
    $\sema{[x := b]t} = [x := \sema{b}]\sema{t}$
\end{lemma}
\begin{proof}
    Straightforward by induction on $t$, substitution is structural with the only exception being variables, but $\sema{x_K} = x$.
\end{proof}

\begin{lemma}
    \label{lem:4:sema_term_step}
    If $t\pterm$ and $|t| \betared t^\prime$ then $|\sema{t}| \betared \sema{t^\prime}$
\end{lemma}
\begin{proof}
    By induction on $t$ and inversion on $t\pterm$.
    The cases: erased lambda, pair, first projection, second projection, promotion ($\vartheta$), substitution ($\psi$), and separation ($\delta$) all erase to a subexpression that is a term.
    Hence, these cases are very similar to the erased application case and omitted.
    The erasure of the variable, reflexivity, and cast cases are values and thus do not reduce.

    $\text{Case: }t = \abs{\lambda_\omega}{x}{A}{b}$
    \begin{proofcase}
        Have $|\abs{\lambda_\omega}{x}{A}{b}| = \abs{\lambda_\omega}{x}{\diamond}{|b|}$ which means $\abs{\lambda_\omega}{x}{\diamond}{|b|} \betared \abs{\lambda_\omega}{x}{\diamond}{b^\prime}$.
        Now $b\pterm$ and $|b| \betared b^\prime$, applying the IH gives $|\sema{b}| \betared \sema{b^\prime}$.
        Note that $|\sema{\abs{\lambda_\omega}{x}{A}{b}}| = \absu{\lambda}{x}{|\sema{b}|} \betared \absu{\lambda}{x}{|\sema{b^\prime}|}$.
        By Lemma~\ref{lem:4:sema_subst}: $|\sema{b^\prime}| = \sema{|b^\prime|}$.
        However, $b^\prime$ is the result of a contracted redex in an already erased term, hence $|b^\prime| = b^\prime$.
        Thus, $|\sema{\abs{\lambda_\omega}{x}{A}{b}}| \betared \sema{\abs{\lambda_\omega}{x}{\diamond}{b^\prime}}$.
    \end{proofcase}

    $\text{Case: }t = \app{f}{0}{a}$
    \begin{proofcase}
        Have $|\app{f}{0}{a}| = |f|$, thus $|f| \betared t^\prime$.
        Applying the IH gives $|\sema{f}| \betared \sema{t^\prime}$.
        Note that $|\sema{\app{f}{0}{a}}| = |\sema{f}\ \edash \sema{a}| = |\sema{f}|$.
        Thus, $|\sema{\app{f}{0}{a}}| \betared \sema{t^\prime}$.
    \end{proofcase}

    $\text{Case: }t = \app{f}{\omega}{a}$
    \begin{proofcase}
        Have $|\app{f}{\omega}{a}| = \app{|f|}{\omega}{|a|}$.
        Suppose $|f| = \abs{\lambda_\omega}{x}{\diamond}{b}$ and $\app{|f|}{\omega}{|a|} \betared [x := |a|]b$.
        Now $|\sema{\app{f}{\omega}{a}}| = |\sema{f}|\ |\sema{a}|$.
        By Lemma~\ref{lem:4:sema_erasure}: $|\sema{f}| = \sema{|f|} = \absu{\lambda}{x}{\sema{b}}$.
        Thus, $(\absu{\lambda}{x}{\sema{b}})\ |\sema{a}| \betared [x := |\sema{a}|]\sema{b}$.
        Using Lemma~\ref{lem:4:sema_erasure} and Lemma~\ref{lem:4:sema_subst} gives $[x := |\sema{a}|]\sema{b} = \sema{[x := |a|]b}$.
        \\ \\
        Suppose w.l.o.g. that $|f| \betared f^\prime$ (the case of $|a| \betared a^\prime$ is very similar).
        Note that $f\pterm$, applying the IH gives $|\sema{f}| \betared \sema{f^\prime}$.
        Now $|\sema{\app{f}{\omega}{a}}| = |\sema{f}|\ |\sema{a}| \betared \sema{f^\prime}\ |\sema{a}| = \sema{\app{f^\prime}{\omega}{|a|}}$.
        The final equality uses Lemma~\ref{lem:4:sema_erasure}.
    \end{proofcase}
\end{proof}

\begin{lemma}
    \label{lem:4:sema_term_multistep}
    If $t\pterm$ and $|t| \betastar t^\prime$ then $|\sema{t}| \betastar \sema{t^\prime}$
\end{lemma}
\begin{proof}
    By induction on $|t| \betastar t^\prime$ using Lemma~\ref{lem:4:sema_term_step}, Lemma~\ref{lem:2:classify_preservation}, and Lemma~\ref{lem:2:classify_erase}.
\end{proof}

\begin{lemma}
    \label{lem:4:sema_term_conv}
    If $a, b\pterm$ and $|a| \betaconv |b|$ then $|\sema{a}| \betaconv |\sema{b}|$
\end{lemma}
\begin{proof}
    Deconstructing $|a| \betaconv |b|$ gives $|a| \betastar z$ and $|b| \betastar z$.
    Applying Lemma~\ref{lem:4:sema_term_multistep} gives $|\sema{a}| \betastar \sema{z}$ and $|\sema{b}| \betastar \sema{z}$.
    Thus, $|\sema{a}| \betaconv |\sema{b}|$.
\end{proof}

\begin{lemma}
    \label{lem:4:sema_type_step}
    If $s\ptype$ and $s \betared_n t$ then $\sema{s} \betared_n \sema{t}$
\end{lemma}
\begin{proof}
    By induction on $s$ and inversion on $s\ptype$.
    Note that only the case where $s$ is a redex is important as all other cases are in weak-head normal form.
    Thus, suppose $s = \app{f}{\tau}{a}$,  $f = \abs{\lambda_\tau}{x}{A}{b}$, and $\app{f}{\tau}{a} \betared_n [x := a]b$.
    Suppose w.l.o.g. that $a\pterm$.
    Now $\sema{\app{f}{\tau}{a}} = \sema{f}\ \sema{a} = (\abs{\lambda}{x}{\sema{A}}{\sema{b}})\ \sema{a} \betared [x := \sema{a}]\sema{b}$.
    Using Lemma~\ref{lem:4:sema_subst} gives $[x := \sema{a}]\sema{b} = \sema{[x := a]b}$.
\end{proof}

\begin{lemma}
    \label{lem:4:sema_type_multistep}
    If $s\ptype$ and $s \betastar_n t$ then $\sema{s} \betastar_n \sema{t}$
\end{lemma}
\begin{proof}
    By induction on $s \betastar_n t$ using Lemma~\ref{lem:4:sema_type_step} and Lemma~\ref{lem:2:classify_preservation}.
\end{proof}

\begin{lemma}
    \label{lem:4:sema_type_conv}
    \textcolor{white}{\_}
    \begin{enumerate}
        \item If $A, B\ptype$, $A\ B$ are values, and $A \equiv B$ then $\sema{A} \cong^t \sema{B}$
        \item If $A, B\ptype$ and $A \equiv B$ then $\sema{A} \cong \sema{B}$
        \item If $A, B\pkind$ and $A \equiv B$ then $\sema{A} \cong \sema{B}$
    \end{enumerate}
\end{lemma}
\begin{proof}
    By mutual recursion.

    \noindent \textbf{1.}
    By induction on $A$ and inversion on $A$ being a value and $A \equiv B$ (hence $B$ must match $A$).
    Conversion in $\oldced$ is structural over weak-head normal forms and in this case $A$ and $B$ must be weak-head normal.
    Thus, a combination of \textit{1.}, \textit{2.}, \textit{3.}, and Lemma~\ref{lem:4:sema_term_conv} on subexpressions in each case is sufficient.

    \noindent \textbf{2.}
    By Theorem~\ref{lem:3:proof_normalization}, $\exists\ A^\prime, B^\prime$ such that $A \betastar A^\prime$, $B \betastar B^\prime$ and $A^\prime, B^\prime$ are values.
    Lemma~\ref{lem:2:classify_preservation} gives that $A^\prime, B^\prime\ptype$.
    Lemma~\ref{lem:2:conv_red_f} gives that $A^\prime \equiv B^\prime$.
    Thus, applying \textit{1.} concludes.

    \noindent \textbf{3.}
    By induction on $A$ and inversion on $A \equiv B$.
    Again, conversion of kinds is structural in $\oldced$.
    Thus, a combination of \textit{2.} and \textit{3.} on subexpressions in each case is sufficient.
\end{proof}

\begin{theorem}[Soundness of $\sema{-}$]
    \label{lem:4:soudness}
    Suppose $\Gamma \vdash_{\ced} t : A$
    \begin{enumerate}
        \item if $A = \kind$ then $\sema{\Gamma} \vdash_{\oldced} \sema{t}$
        \item if $\Gamma \vdash_{\ced} A : \kind$ then $\sema{\Gamma} \vdash_{\oldced} \sema{t} \infr T$ and $T \cong \sema{A}$
        \item if $\Gamma \vdash_{\ced} A : \star$ then $\sema{\Gamma} \vdash_{\oldced} \sema{t} \chck \sema{A}$
    \end{enumerate}
\end{theorem}
\begin{proof}
    By induction on $\Gamma \vdash_{\ced} t : A$.
    Note that each case is mutually exclusive by classification.

    $\text{Case: }\begin{array}{c} \AxiomRule[*] \end{array}$
    \begin{proofcase}
        Have $A = \kind$ and $\Gamma \vdash_{\oldced} \star$, hence trivial.
    \end{proofcase}

    $\text{Case: }\begin{array}{c} \VarRule[*] \end{array}$
    \begin{proofcase}
        Let $\Gamma = \Gamma_1; x : A; \Gamma_2$.
        Have $(x : \sema{A}) \in \sema{\Gamma}$.
        Now $\sema{\Gamma_1} \vdash_{\oldced} x \infr \sema{A}$ by the IH and $\sema{\Gamma} \vdash_{\oldced} x \infr \sema{A}$ by Lemma~\ref{lem:4:c1_weakening}.
        Suppose $K = \kind$ then $\sema{A} \cong \sema{A}$ and $\sema{\Gamma} \vdash_{\oldced} x \infr \sema{A}$.
        Suppose $K = \star$ then $\sema{\Gamma} \vdash_{\oldced} x \chck \sema{A}$.
    \end{proofcase}

    $\text{Case: }\begin{array}{c} \PiRule[*] \end{array}$
    \begin{proofcase}
        Suppose $m = \tau$, then $\pdom(m, K) = K$ and $\pcodom(m) = \kind$.
        Applying the IH gives:
        \begin{enumerate}
            \item[$\D{1}$.] $\sema{\Gamma} \vdash_{\oldced} \sema{A}$ if $K = \kind$
            \item[$\D{1}$.] $\sema{\Gamma} \vdash_{\oldced} \sema{A} \infr \star$ if $K = \star$
            \item[$\D{2}$.] $\sema{\Gamma}, x : \sema{A} \vdash_{\oldced} \sema{B}$
        \end{enumerate}
        The corresponding $\Pi$ rule for the two possibilities of $K$ concludes the case.
        \\ \\
        Suppose $m = 0$, then $\pdom(m, K) = K$ and $\pcodom(m) = \star$.
        Applying the IH gives:
        \begin{enumerate}
            \item[$\D{1}$.] $\sema{\Gamma} \vdash_{\oldced} \sema{A}$ if $K = \kind$
            \item[$\D{1}$.] $\sema{\Gamma} \vdash_{\oldced} \sema{A} \infr \star$ if $K = \star$
            \item[$\D{2}$.] $\sema{\Gamma}, x : \sema{A} \vdash_{\oldced} \sema{B} \infr \star$
        \end{enumerate}
        The corresponding $\forall$ rule for the two possibilities of $K$ concludes the case.
        \\ \\
        Suppose $m = \omega$, then $\pdom(m, K) = \star$ and $\pcodom(m) = \star$.
        Applying the IH gives:
        \begin{enumerate}
            \item[$\D{1}$.] $\sema{\Gamma} \vdash_{\oldced} \sema{A} \infr \star$
            \item[$\D{2}$.] $\sema{\Gamma}, x : \sema{A} \vdash_{\oldced} \sema{B} \infr \star$
        \end{enumerate}
        The corresponding $\Pi$ rule concludes the case.
    \end{proofcase}

    $\text{Case: }\begin{array}{c} \LambdaRule[*] \end{array}$
    \begin{proofcase}
        Suppose $m = \tau$, then $\pcodom(m) = \kind$.
        Note that this means that $t\ptype$.
        Applying the IH gives:
        \begin{enumerate}
            \item[$\D{1}$.] $\sema{\Gamma} \vdash_{\oldced} \abs{\Pi}{x}{\sema{A}}{\sema{B}}$
            \item[$\D{2}$.] $\sema{\Gamma}, x : \sema{A} \vdash_{\oldced} \sema{t} \infr T$ and $T \cong \sema{B}$
        \end{enumerate}
        Suppose $\sema{\Gamma} \vdash_{\oldced} \sema{A}$, then $\sema{\Gamma} \vdash_{\oldced} \abs{\lambda}{x}{\sema{A}}{\sema{t}} \infr \abs{\Pi}{x}{\sema{A}}{T}$.
        By rules of conversion for kinds yields $\abs{\Pi}{x}{\sema{A}}{T} \cong \abs{\Pi}{x}{\sema{A}}{\sema{B}}$.
        The case where $\sema{A}$ is a type instead of a kind is similar.
        \\ \\
        Suppose $m = 0$, then $\pcodom(m) = \star$.
        Note that this means $t\pterm$.
        Applying the IH gives:
        \begin{enumerate}
            \item[$\D{1}$.] $\sema{\Gamma} \vdash_{\oldced} \abs{\Pi}{x}{\sema{A}}{\sema{B}} \infr \star$
            \item[$\D{2}$.] $\sema{\Gamma}, x : \sema{A} \vdash_{\oldced} \sema{t} \chck \sema{B}$
        \end{enumerate}
        Note that $FV(|\sema{t}|) \subseteq FV(|t|)$, thus $x \notin FV(|\sema{t}|)$.
        Using the corresponding $\Lambda$ rule based on the classification of $\sema{A}$ concludes the case.
        \\ \\
        Suppose $m = \omega$, then $\pcodom(m) = \star$.
        This case is omitted because the previous case is a more general version of it.
    \end{proofcase}

    $\text{Case: }\begin{array}{c} \AppRule[*] \end{array}$
    \begin{proofcase}
        Suppose $m = \tau$.
        Classification forces $f\ptype$, but $a$ is either a term or a type.
        Applying the IH gives:
        \begin{enumerate}
            \item[$\D{1}$.] $\sema{\Gamma} \vdash_{\oldced} \sema{f} \infr T$ with $T \cong \abs{\Pi}{x}{\sema{A}}{\sema{B}}$
            \item[$\D{2}$.] $\sema{\Gamma} \vdash_{\oldced} \sema{a} \infr T_2$ with $T_2 \cong \sema{A}$ if $a\ptype$
            \item[$\D{2}$.] $\sema{\Gamma} \vdash_{\oldced} \sema{a} \chck \sema{A}$ if $a\pterm$
        \end{enumerate}
        Note that because kinds cannot reduce, it must be the case that $\exists\ C, D$ such that $T = \abs{\Pi}{x}{C}{D}$.
        Moreover, $C \cong \sema{A}$ and $D \cong \sema{B}$ by the conversion rules.
        Suppose $a\ptype$ then using the associated rule yields $\sema{\Gamma} \vdash_{\oldced} \sema{f} \cdot \sema{a} \infr [x := \sema{a}]D$.
        Now, $[x := \sema{a}]D \cong [x := \sema{a}]\sema{B}$ and the case is concluded.
        Suppose $a\pterm$ then using the associated rule yields $\sema{\Gamma} \vdash_{\oldced} \sema{f}\ \sema{a} \infr [x := \chi\ C - \sema{a}]D$.
        Again, $[x := \chi\ C - \sema{a}]D \cong [x := \chi\ C - \sema{a}]\sema{B}$ and the case is concluded.
        Note that $[x := \chi\ C - \sema{a}]\sema{B} \cong [x := \sema{a}]\sema{B}$ because the $\chi$ is only well-typed in term positions, where it is promptly erased during conversion checking.
        \\ \\
        Suppose $m = 0$.
        Classification forces $f\pterm$, but $a$ is either a term or a type.
        Applying the IH gives:
        \begin{enumerate}
            \item[$\D{1}$.] $\sema{\Gamma} \vdash_{\oldced} \sema{f} \chck \abs{\forall}{x}{\sema{A}}{\sema{B}}$
            \item[$\D{2}$.] $\sema{\Gamma} \vdash_{\oldced} \sema{a} \infr T_2$ with $T_2 \cong \sema{A}$ if $a\ptype$
            \item[$\D{2}$.] $\sema{\Gamma} \vdash_{\oldced} \sema{a} \chck \sema{A}$ if $a\pterm$
        \end{enumerate}
        Deconstructing the checking judgment for $\sema{f}$ yields $\exists\ C, D$ such that $\sema{\Gamma} \vdash_{\oldced} \sema{f} \cinfr \abs{\forall}{x}{C}{D}$ and $C \cong \sema{A}$ and $D \cong \sema{B}$.
        Suppose $a\ptype$ then the associated judgment gives $\sema{\Gamma} \vdash_{\oldced} \sema{f} \cdot \sema{a} \infr [x := \sema{a}]D$.
        Now, $[x := \sema{a}]D \cong [x := \sema{a}]\sema{B}$ and the case is concluded.
        Suppose $a\pterm$ then the associated judgment gives $\sema{\Gamma} \vdash_{\oldced} \sema{f}\ \edash \sema{a} \infr [x := \chi\ C - \sema{a}]D$.
        Again, $[x := \chi\ C - \sema{a}]D \cong [x := \chi\ C - \sema{a}]\sema{B}$ and the case is concluded.
        \\ \\
        Suppose $m = \omega$
        Classification forces $f, a\pterm$.
        Applying the IH gives:
        \begin{enumerate}
            \item[$\D{1}$.] $\sema{\Gamma} \vdash_{\oldced} \sema{f} \chck \abs{\Pi}{x}{\sema{A}}{\sema{B}}$
            \item[$\D{2}$.] $\sema{\Gamma} \vdash_{\oldced} \sema{a} \chck \sema{A}$ if $a\pterm$
        \end{enumerate}
        As with the previous case, $\exists\ C, D$ such that $\sema{\Gamma} \vdash_{\oldced} \sema{f} \cinfr \abs{\Pi}{x}{C}{D}$ and $C \cong \sema{A}$ and $D \cong \sema{B}$.
        Applying the associated rule yields $\sema{\Gamma} \vdash_{\oldced} \sema{f}\ \sema{a} \infr [x := \chi\ C - \sema{a}]D$.
        Now, $[x := \chi\ C - \sema{a}]D \cong [x := \chi\ C - \sema{a}]\sema{B}$ and the case is concluded.
    \end{proofcase}

    $\text{Case: }\begin{array}{c} \IntersectionRule[*] \end{array}$
    \begin{proofcase}
        Applying the IH gives:
        \begin{enumerate}
            \item[$\D{1}$.] $\sema{\Gamma} \vdash_{\oldced} \sema{A} \infr \star$
            \item[$\D{2}$.] $\sema{\Gamma}, x : \sema{A} \vdash_{\oldced} \sema{B} \infr \star$
        \end{enumerate}
        Thus, $\sema{\Gamma} \vdash_{\oldced} \abs{\iota}{x}{\sema{A}}{\sema{B}} \infr \star$ as required.
    \end{proofcase}

    $\text{Case: }\begin{array}{c} \PairRule[*] \end{array}$
    \begin{proofcase}
        Note by classification and $\D{1}$: $\Gamma \vdash A : \star$ and $\Gamma, x : A \vdash B : \star$.
        Applying the IH gives:
        \begin{enumerate}
            \item[$\D{1}$.] $\sema{\Gamma} \vdash_{\oldced} \abs{\iota}{x}{\sema{A}}{\sema{B}} \infr \star$
            \item[$\D{2}$.] $\sema{\Gamma} \vdash_{\oldced} \sema{t} \chck \sema{A}$
            \item[$\D{3}$.] $\sema{\Gamma} \vdash_{\oldced} \sema{s} \chck [x := \sema{t}]\sema{B}$
        \end{enumerate}
        Note that $\sema{\Gamma} \vdash_{\oldced} \sema{t} \chck \sema{A}$ so clearly $\sema{\Gamma} \vdash_{\oldced} \sema{s} \chck [x := \chi\ \sema{A}\ -\ \sema{t}]\sema{B}$ as the $\chi$ merely adds extra typing information.
        Lemma~\ref{lem:4:sema_term_conv} applied to $\D{4}$ and using the fact that $t, s\pterm$ gives $|\sema{t}| \betaconv |\sema{s}|$.
        Combining this information yields $\sema{\Gamma} \vdash \sema{[t, s; A]} \chck \sema{(x : A) \cap B}$.
    \end{proofcase}

    $\text{Case: }\begin{array}{c} \FirstRule[*] \end{array}$
    \begin{proofcase}
        By classification $t\pterm$.
        Applying the IH to $\D{1}$ gives $\sema{\Gamma} \vdash_{\oldced} \sema{t} \chck \abs{\iota}{x}{\sema{A}}{\sema{B}}$.
        Deconstruct this checking rule and notice that either the inferred type is already an intersection or it must reduce to an intersection.
        Thus, $\exists\ C\ D$ such that $\sema{\Gamma} \vdash_{\oldced} \sema{t} \cinfr \abs{\iota}{x}{\sema{C}}{\sema{D}}$ and $\abs{\iota}{x}{\sema{C}}{\sema{D}} \cong \abs{\iota}{x}{\sema{A}}{\sema{B}}$.
        Deconstructing the congruence yields $\sema{C} \cong \sema{A}$.
        Thus, $\sema{\Gamma} \vdash_{\oldced} \sema{t}.1 \chck \sema{A}$
    \end{proofcase}

    $\text{Case: }\begin{array}{c} \SecondRule[*] \end{array}$
    \begin{proofcase}
        By classification $t\pterm$.
        Applying the IH to $\D{1}$ gives $\sema{\Gamma} \vdash_{\oldced} \sema{t} \chck \abs{\iota}{x}{\sema{A}}{\sema{B}}$.
        Deconstruct this checking rule and notice that either the inferred type is already an intersection or it must reduce to an intersection.
        Thus, $\exists\ C\ D$ such that $\sema{\Gamma} \vdash_{\oldced} \sema{t} \cinfr \abs{\iota}{x}{\sema{C}}{\sema{D}}$ and $\abs{\iota}{x}{\sema{C}}{\sema{D}} \cong \abs{\iota}{x}{\sema{A}}{\sema{B}}$.
        Deconstructing the congruence yields $\sema{D} \cong \sema{B}$ and thus $[x := \sema{t}.1]\sema{D} \cong [x := \sema{t}.1]\sema{B}$.
        Now $\sema{\Gamma} \vdash_{\oldced} \sema{t}.2 \infr [x := \sema{t}.1]\sema{D}$.
        Thus, $\sema{\Gamma} \vdash_{\oldced} \sema{t}.2 \chck [x := \sema{t}.1]\sema{B}$.
    \end{proofcase}

    $\text{Case: }\begin{array}{c} \EqualityRule[*] \end{array}$
    \begin{proofcase}
        Note that $a, b\pterm$ by $\D{1}$.
        Applying the IH gives:
        \begin{enumerate}
            \item[$\D{1}$.] $\sema{\Gamma} \vdash_{\oldced} \sema{A} \infr \star$
            \item[$\D{2}$.] $\sema{\Gamma} \vdash_{\oldced} \sema{a} \chck \sema{A}$
            \item[$\D{3}$.] $\sema{\Gamma} \vdash_{\oldced} \sema{b} \chck \sema{A}$
        \end{enumerate}
        By Lemma~\ref{lem:4:c1_stuff}, Lemma~\ref{lem:4:c1_weakening}, and the application rule for $\oldced$: $\sema{\Gamma} \vdash_{\oldced} \cId \cdot \sema{A}\ \sema{a}\ \sema{b} \infr \star$.
    \end{proofcase}

    $\text{Case: }\begin{array}{c} \ReflRule[*] \end{array}$
    \begin{proofcase}
        Note that $t\pterm$ by $\D{1}$.
        Applying the IH gives:
        \begin{enumerate}
            \item[$\D{1}$.] $\sema{\Gamma} \vdash_{\oldced} \sema{A} \infr \star$
            \item[$\D{2}$.] $\sema{\Gamma} \vdash_{\oldced} \sema{t} \chck \sema{A}$
        \end{enumerate}
        By Lemma~\ref{lem:4:c1_stuff}, Lemma~\ref{lem:4:c1_weakening}, and the application rule for $\oldced$: $\sema{\Gamma} \vdash_{\oldced} \crefl \cdot \sema{A}\ \edash \sema{t} \infr \cId \cdot \sema{A}\ \sema{t}\ \sema{t}$.
    \end{proofcase}

    \begin{minipage}{.8\textwidth}$\text{Case: }\begin{array}{c} \SubstRule[*] \end{array}$\end{minipage}
    \begin{proofcase}
        Note by $\D{1}$ that $a, b\pterm$ and by classification $e\pterm$ with $A, P\ptype$.
        Applying the IH gives:
        \begin{enumerate}
            \item[$\D{1}$.] $\sema{\Gamma} \vdash_{\oldced} \sema{A} \infr \star$
            \item[$\D{2}$.] $\sema{\Gamma} \vdash_{\oldced} \sema{a} \chck \sema{A}$
            \item[$\D{3}$.] $\sema{\Gamma} \vdash_{\oldced} \sema{b} \chck \sema{A}$
            \item[$\D{4}$.] $\sema{\Gamma} \vdash_{\oldced} \sema{e} \chck \cId \cdot \sema{A}\ \sema{a}\ \sema{b}$
            \item[$\D{5}$.] $\sema{\Gamma} \vdash_{\oldced} \sema{P} \infr T$ and $T \cong \abs{\forall}{y}{\sema{A}}{\cId \cdot \sema{A}\ \sema{a}\ \sema{y} \to \star}$
        \end{enumerate}
        By Lemma~\ref{lem:4:c1_stuff}, Lemma~\ref{lem:4:c1_weakening}, and the application rule for $\oldced$:
            $\sema{\Gamma} \vdash_{\oldced} \csubst \cdot \sema{A}\ \edash \sema{a}\ \edash \sema{b} \cdot \sema{P}\ \sema{e} \infr \sema{P}\ \sema{a}\ (\crefl \cdot \sema{A}\ \edash \sema{a}) \to \sema{P}\ \sema{b}\ \sema{e}$.
    \end{proofcase}

    $\text{Case: }\begin{array}{c} \PromoteRule[*] \end{array}$
    \begin{proofcase}
        Note by $\D{1}$ that $a, b\pterm$ and by classification $e\pterm$ with $(x : A) \cap B\ptype$.
        Applying the IH gives:
        \begin{enumerate}
            \item[$\D{1}$.] $\sema{\Gamma} \vdash_{\oldced} \abs{\iota}{x}{\sema{A}}{\sema{B}} \infr \star$
            \item[$\D{2}$.] $\sema{\Gamma} \vdash_{\oldced} \sema{a} \chck \abs{\iota}{x}{\sema{A}}{\sema{B}}$
            \item[$\D{3}$.] $\sema{\Gamma} \vdash_{\oldced} \sema{b} \chck \abs{\iota}{x}{\sema{A}}{\sema{B}}$
            \item[$\D{4}$.] $\sema{\Gamma} \vdash_{\oldced} \sema{e} \chck \cId \cdot \sema{A}\ \sema{a}.1\ \sema{b}.1$
        \end{enumerate}
        Note that $\sema{\Gamma}, x : \sema{A} \vdash_{\oldced} \sema{B} \infr \star$ which means $\sema{\Gamma} \vdash_{\oldced} \sema{B} \infr A \to \star$.
        By Lemma~\ref{lem:4:c1_stuff}, Lemma~\ref{lem:4:c1_weakening}, and the application rule for $\oldced$:
            $\sema{\Gamma} \vdash_{\oldced} \ctheta \cdot \sema{A} \cdot \sema{B}\ \edash \sema{a}\ \edash \sema{b}\ \sema{e} \infr \cId \cdot (\abs{\iota}{x}{\sema{A}}{\sema{B}})\ \sema{a}\ \sema{b}$.
    \end{proofcase}

    $\text{Case: }\begin{array}{c} \CastRule[*] \end{array}$
    \begin{proofcase}
        Note by soundness of classification that $a, b, e\pterm$.
        Applying the IH gives:
        \begin{enumerate}
            \item[$\D{1}$.] $\sema{\Gamma} \vdash_{\oldced} \sema{a} \chck \sema{A}$
            \item[$\D{2}$.] $\sema{\Gamma} \vdash_{\oldced} \sema{b} \chck \abs{\iota}{x}{\sema{A}}{\sema{B}}$
            \item[$\D{3}$.] $\sema{\Gamma} \vdash_{\oldced} \sema{e} \chck \cId \cdot \sema{A}\ \sema{a}\ \sema{b}.1$
        \end{enumerate}
        By the application and first projection rule and some maneuvering of type conversion: $\sema{\Gamma} \vdash_{\oldced} \text{\c{c}}\ \sema{e}.1 \chck \{ \sema{b} \cong \sema{a}\}$.
        Note that $FV(\sema{a}) \subseteq dom(\Gamma)$ because otherwise $\D{1}$ is not a proof.
        Thus, the goal is obtained by the $\varphi$ rule of $\oldced$.
    \end{proofcase}

    $\text{Case: }\begin{array}{c} \SeparationRule[*] \end{array}$
    \begin{proofcase}
        Applying the IH to $\D{1}$ yields $\sema{\Gamma} \vdash_{\oldced} \cId \cBool \ctt \cff$.
        By Lemma~\ref{lem:4:c1_stuff}, Lemma~\ref{lem:4:c1_weakening}, and the application rule for $\oldced$:
            $\sema{\Gamma} \vdash_{\oldced} \cdelta\ \sema{e} \infr \abs{\forall}{X}{\star}{X}$.
    \end{proofcase}

    $\text{Case: }\begin{array}{c} \ConvRule[*] \end{array}$
    \begin{proofcase}
        Suppose $K = \kind$.
        Then by classification and $\D{3}$: $\Gamma \vdash B : \kind$.
        Applying the IH to $\D{2}$ gives $\sema{\Gamma} \vdash \sema{t} \infr T$ with $T \cong \sema{B}$.
        By Lemma~\ref{lem:4:sema_type_conv}: $\sema{A} \cong \sema{B}$.
        Now by Lemma~\ref{lem:4:c1_trans} and Lemma~\ref{lem:4:c1_sym}: $T \cong \sema{B}$.
        \\ \\
        Suppose $K = \star$.
        Then by classification and $\D{3}$: $\Gamma \vdash B : \star$.
        Applying the IH to $\D{2}$ gives $\sema{\Gamma} \vdash \sema{t} \infr \sema{B}$.
        By Lemma~\ref{lem:4:sema_type_conv} and Lemma~\ref{lem:4:c1_sym}: $\sema{B} \cong \sema{A}$.
        Applying the checking rule of $\oldced$ yields $\sema{\Gamma} \vdash \sema{t} \chck \sema{A}$.
    \end{proofcase}
\end{proof}

\begin{theorem}[Logical Consistency]
    \label{lem:4:logical_consistency}
    $\neg (\vdash_{c_2} t : (X : \star) \to_0 X_\square)$
\end{theorem}
\begin{proof}
    Proceed using proof by negation.
    Suppose $\vdash_{c_2} t : (X : \star) \to_0 X_\square$.
    By Theorem~\ref{lem:4:soudness}: $\vdash_{\oldced} \sema{t} \chck \abs{\forall}{X}{\star}{X}$.
    However, this is impossible by consistency of $\oldced$.
\end{proof}

\begin{corollary}[Equational Consistency]
    \label{lem:4:eq_consistency}
    $\neg (\vdash_{c_2} t : \ptt =_{\pBool} \pff)$
\end{corollary}

\chapter{Object Normalization and \texorpdfstring{$\varphi$}{Phi} the Foil}
\label{chap:5}

Consistency guarantees that the logic and equational theory of $\ced$ is non-trivial.
Proof normalization guarantees that, at least, inference for kinds and types is decidable.
Neither of these properties are strong enough on their own to guarantee decidability of type checking.
To obtain decidability of type checking it must be the case that objects are normalizing.
Unfortunately, object normalization does not hold when the \textsc{Cast} rule is used, and it is not clear how the rule may be repaired to acquire object normalization.
A proof is \textbf{strict} if it does not use the \textsc{Cast} rule in its derivation.
While strict proofs do have normalizing objects the technique described below to prove this fact depends on both proof normalization and consistency.
This is suggestive of how difficult a property object normalization is to show.

\section{Normalization for Strict Proofs}

The core observation is that proof reduction in strict proofs upper-bounds reduction in their corresponding objects.
Thus, if a strict object steps, and note that this must be a $\beta$-step, then there is some strict proof such that the original strict proof reduces to it and the erasures match.
There could be many more reductions in the strict proof because syntax forms for equality and intersections are all mostly erased.
However, none of these forms will block a $\beta$-redex because the proof is well-typed.
Note that this property hinges on both proof normalization and equational consistency.
Proof normalization is used to eliminate any extraneous redexes that would otherwise be erased.
Consistency is used to eliminate the $\delta$ case as it could theoretically generate a $\beta$-redex after erasure if the theory was not equationally consistent.
Of course, $\varphi$ could also generate a $\beta$-redex after erasure, but this is impossible because the syntax under consideration is strict.

\begin{definition}
    $\Gamma \vdash_{\ced^\text{-}} t : A$ iff $\mathcal{D} : \Gamma \vdash t : A$ and the \textnormal{\textsc{Cast}} rule is not used in $\mathcal{D}$
\end{definition}

\begin{lemma}
    \label{lem:5:strict_erase_red_step}
    If $\Gamma \vdash_{\ced^\text{-}} s : A$ and $|s| \betared t$ then $\exists\ t^\prime$ such that $s \betastar_{\neq 0} t^\prime$ and $|t^\prime| = t$
\end{lemma}
\begin{proof}
    By induction on $\Gamma \vdash_{\ced^\text{-}} s : A$.
    The erasure of the \textsc{Ax}, \textsc{Var}, and \textsc{Refl}, cases are values and thus do not reduce.
    The \textsc{Cast} case is impossible because it is intentionally excluded.
    First projection is very similar to second projection case.
    The \textsc{Int} and \textsc{Eq} cases are structural in erasure and are thus very similar to the \textsc{Pi} case.

    $\text{Case: }\begin{array}{c} \PiRule[*] \end{array}$
    \begin{proofcase}
        Have $|(x : A) \to_m B| = (x : |A|) \to_m |B|$.
        Suppose that $|A| \betared t$.
        By the IH applied to $\D{1}$: $\exists\ t^\prime$ such that $A \betastar_{\neq 0} t^\prime$ and $|t^\prime| = t$.
        Thus, $(x : A) \to_m B \betastar_{\neq 0} (x : t^\prime) \to_m B$ and $|(x : t^\prime) \to_m B| = (x : t) \to_m |B|$.
        The case where a reduction happens in $|B|$ is similar.
    \end{proofcase}

    $\text{Case: }\begin{array}{c} \LambdaRule[*] \end{array}$
    \begin{proofcase}
        Suppose $m = 0$.
        Have $|\abs{\lambda_0}{x}{A}{b}| = |b|$ with $|b| \betared t$.
        Applying the IH to $\D{2}$ concludes the case.
        \\ \\
        Suppose that $m = \omega$, note that $m = \tau$ is very similar and thus omitted.
        Have $|\abs{\lambda_\omega}{x}{A}{b}| = \abs{\lambda_\omega}{x}{\diamond}{|b|}$ and $|b| \betared t$.
        Applying the IH to $\D{2}$ yields $\exists\ t^\prime$ such that $b \betastar_{\neq 0} t^\prime$ and $|t^\prime| = t$.
        Now $\abs{\lambda_\omega}{x}{A}{b} \betastar_{\neq 0} \abs{\lambda_\omega}{x}{A}{t^\prime}$ and $|\abs{\lambda_\omega}{x}{A}{t^\prime}| = \abs{\lambda_\omega}{x}{\diamond}{t}$.
    \end{proofcase}

    $\text{Case: }\begin{array}{c} \AppRule[*] \end{array}$
    \begin{proofcase}
        If $m = 0$ then the proof follows by a straightforward application of the IH to $\D{1}$.
        \\ \\
        Suppose that $m = \omega$.
        Let $|f| = \abs{\lambda_\omega}{x}{\diamond}{v}$ and $\app{|f|}{\omega}{|a|} \betared [x := |a|]v$.
        By Theorem~\ref{lem:3:proof_normalization} $f$ is strongly normalizing in proof reduction.
        If $f$ contains a projection redex, promotion redex, or erased application redex then produce $f_i$ by contracting that redex.
        Continue contracting these redexes until none remain, assume $k$ such redexes are contracted, thus $f \betastar f_k$.
        Note that none of these redexes affect the erasure of $f$, thus $|f| = |f_k|$.
        Now $f_k$ has only three possibilities: $f_k = \abs{\lambda_\omega}{x}{A}{b}$, or $f_k = \psi(\prefl(z; Z), a, b; A, P)$, or $f_k = \delta(\prefl(t; A))$.
        The $\varphi$ case is impossible by the restriction of the judgment and by Theorem~\ref{lem:4:eq_consistency} the $\delta$ case is impossible.
        \begin{itemize}
            \item {
                Suppose $f_k = \abs{\lambda_\omega}{x}{A}{b}$.
                Now $\app{f_k}{\omega}{a} \betared [x := a]b$ and $|[x := a]b| = [x := |a|]v$.
            }
            \item {
                Suppose $f_k = \psi(\prefl(z; Z), a, b; A, P)$.
                Now $\app{\psi(\prefl(z; Z), a, b; A, P)}{\omega}{a} \betared a$.
                Note that $|f_k| = |f|$, but $|\psi(\prefl(z; Z), a, b; A, P)| = \abs{\lambda_\omega}{x}{\diamond}{x}$ and $|f| = \abs{\lambda_\omega}{x}{\diamond}{v}$.
                Thus, $v = x$ and $|a| = [x := |a|]v$.
            }
        \end{itemize}
        \vspace{.15in}
        Suppose $m = \omega$ and $|f| \betared t$.
        Note that the case where $|a| \betared t$ is very similar and thus omitted.
        Applying the IH to $\D{1}$ gives $\exists\ t^\prime$ such that $f \betastar_{\neq 0} t^\prime$ and $|t^\prime| = t$.
        Now $\app{f}{\omega}{a} \betastar_{\neq 0} \app{t^\prime}{\omega}{a}$ and $|\app{t^\prime}{\omega}{a}| = \app{t}{\omega}{|a|}$.
        \\ \\
        Suppose $m = \tau$ then erasure is structural.
        Thus, a $\beta$-redex is tracked exactly and any structural redexes are very similar to the $m = \omega$ case.
    \end{proofcase}

    $\text{Case: }\begin{array}{c} \PairRule[*] \end{array}$
    \begin{proofcase}
        Have $|[t_1, t_2; A]| = |t_1|$ and $|t_1| \betared t$.
        Applying the IH to $\D{1}$ yields $\exists\ t^\prime$ such that $t_1 \betastar_{\neq 0} t^\prime$ and $|t^\prime| = t$.
        Now $[t_1, t_2; A] \betastar_{\neq 0} [t^\prime, t_2; A]$ and $|[t^\prime, t_2; A]| = t$.
    \end{proofcase}

    $\text{Case: }\begin{array}{c} \SecondRule[*] \end{array}$
    \begin{proofcase}
        Have $|b.2| = |b|$ and $|b| \betared t$.
        Applying the IH to $\D{1}$ gives $\exists\ t^\prime$ such that $b \betastar_{\neq 0} t^\prime$ and $|t^\prime| = t$.
        Now $b.2 \betastar_{\neq 0} t^\prime.2$ and $|t^\prime.2| = t$.
    \end{proofcase}

    $\text{Case: }\begin{array}{c} \SubstRule[*] \end{array}$
    \begin{proofcase}
        Have $|\psi(e, a, b; A, T)| = |e|$ and $|e| \betared t$.
        Applying the IH to $\D{4}$ yields $\exists\ t^\prime$ such that $e \betastar_{\neq 0} t^\prime$ and $|t^\prime| = t$.
        Now $\psi(e, a, b; A, T) \betastar_{\neq 0} \psi(t^\prime, a, b; A, T)$
            and $|\psi(t^\prime, a, b; A, T)| = t$.
    \end{proofcase}

    $\text{Case: }\begin{array}{c} \PromoteRule[*] \end{array}$
    \begin{proofcase}
        Have $|\vartheta(e, a, b; (x : A) \cap B)| = |e|$ and $|e| \betared t$.
        Applying the IH to $\D{4}$ gives $\exists\ t^\prime$ where $e \betastar_{\neq 0} t^\prime$ and $|t^\prime| = t$.
        Now $\vartheta(e, a, b; (x : A) \cap B) \betastar_{\neq 0} \vartheta(t^\prime, a, b; (x : A) \cap B)$
            and $|\vartheta(t^\prime, a, b; (x : A) \cap B)| = t$.
    \end{proofcase}

    $\text{Case: }\begin{array}{c} \SeparationRule[*] \end{array}$
    \begin{proofcase}
        Have $|\delta(e)| = |e|$ and $|e| \betared t$.
        Applying the IH to $\D{1}$ gives $\exists\ t^\prime$ where $e \betastar_{\neq 0} t^\prime$ and $|t^\prime| = t$.
        Now $\delta(e) \betastar_{\neq 0} \delta(t^\prime)$ and $|\delta(t^\prime)| = t$.
    \end{proofcase}

    $\text{Case: }\begin{array}{c} \ConvRule[*] \end{array}$
    \begin{proofcase}
        Immediate by the IH applied to $\D{2}$.
    \end{proofcase}

\end{proof}

\begin{theorem}[Strict Object Normalization]
    \label{lem:5:strict_object_sn}
    If $\Gamma \vdash_{\ced^\text{-}} t : A$ then $|t|$ is strongly normalizing
\end{theorem}
\begin{proof}
    By Theorem~\ref{lem:3:proof_normalization}: $t$ is strongly normalizing wrt proof reduction.
    Let $\partial$ be the maximum length reduction sequence $t$ could take to reach the unique value.
    Suppose wlog that $|t|$ contains a redex.
    Contract this redex giving $|t| \betared e_1$.
    By Lemma~\ref{lem:5:strict_erase_red_step}: $\exists\ t_1$ such that $t \betastar_{\neq 0} t_1$ and $|t_1| = e_1$.
    Using preservation of proof reduction: $\Gamma \vdash_{\ced^\text{-}} t_1 : A$.
    Let the number of contracted redexes by the reduction $t \betastar_{\neq 0} t_1$ be $k$, then there is a maximum of $\partial - k$ redexes in $t_1$.
    If redexes remain in $e_1$ than the process can be repeated because $t_1$ is a strict proof whose erasure is $e_1$.
    However, eventually the number of steps taken must run out, because $\partial$ is a finite value.
    Thus, the procedure may be repeated as many times as desired, but $e_i$, the value after $i$ iterations of this process, must eventually run out of redexes by Lemma~\ref{lem:5:strict_erase_red_step}.
    Therefore, $|t|$ is strongly normalizing.
\end{proof}

Strong normalization of strict objects leads to an interesting observation.
Recall the definition of conversion: $a \equiv b$ if and only if $\exists\ u, v$ such that $a \betastar u$, $b \betastar v$ and $|u| \betaconv |v|$.
An observant reader may wonder why reduction is allowed after two candidates objects, $|u|$ and $|v|$ are obtained.
In other words, why not merely compare for equality: $|u| = |v|$.
The answer is because $\varphi$ may generate $\beta$-redexes after erasure, and it turns out that it is the only syntax form for which this is possible.
Thus, if $\varphi$ was removed from the system then conversion \textit{could} be defined using equality of objects instead of reduction convertibility of objects.
The $\varphi$ form is unique amongst all the other syntax.

Another question that the reader may have is why not represent the reduction of $\varphi$ in the proof system.
The answer is that there is no obvious way to make the reduction well-typed, thus preservation would be lost.
Indeed, the proof witness of a $\varphi(a, b, e)$ form, $b$, is allowed to be as complicated as required to produce the subtype $(x : A) \cap B$.
However, the object, $|a|$, is typed at the super-type $A$.
To make it possible to type this term in the proof system some notion of subtyping would have to be added directly into the rules.
It is not immediately clear how to make this move without producing a radically different system.
Yet, it does hint that the $\varphi$ rule is, in some sense, expanding a semantic subtyping relation that is later realized internally via a notion of casts.
Indeed, it may be fruitful to view the proof-object distinction as being fundamentally related to subtyping.

\section{Observational Equivalence of Objects}

Unfortunately, proofs involving the $\varphi$ form do not have normalizing objects.
While it is not clear how to augment the proof system to enforce normalization it is possible to describe an external condition on proofs that would guarantee object normalization for any arbitrary proof.
The idea is to observe that each $\varphi(a, b, e)$ form has some associated proof witness ($b$) and some object witness ($a$).
Evidence ($e$) is also provided that these two witnesses are equal at type $A$.
If $e$ reduces to a value, then that implies $|a| \betaconv |b|$, but if this holds than whatever usage of $\varphi$ should be normalizing.
However, the evidence produced in a proof need not ever reduce to a value, yet it will still be discarded by the erasure of $\varphi$.

Observational (or contextual) equivalence of objects gives a strong enough claim to transfer the normalization property from one object to another.
Objects being the concept of interest means that contexts need to be well-typed because an object is only the erasure of a proof.
To make contexts the inductive structure of syntax is reused with a unique fresh free variable, labelled $h$, that represents a hole.
The variable is unique meaning it occurs only once in the given syntax, but it can be trivially duplicated by an abstraction.
Context structure could be defined inductively, but this methodology allows reuse of erasure and substitution.
\begin{definition}
    A \textbf{context} $\gamma : (\Gamma, A) \to (\Delta, B)$ is a syntactic form with a unique free variable $h$ representing a hole such that if $\Gamma \vdash t : A$ then $\Delta \vdash [h := t]\gamma : B$.
\end{definition}

Observational equivalence is then defined to be logical equivalence of divergence of the associated objects substituted for $h$ in the given context.
There are several possible ways to defined observational equivalence including the choice of what counts as an observation.
For the purposes of this chapter divergence is the only observation of interest.
Note that it is easy to see that observational equivalence forms an equivalence relation relative to the parameters $\Gamma$ and $A$.

\begin{definition}
    The syntax $a$ and $b$ are \textbf{observationally equivalent} at $A$ in $\Gamma$ (written: $\Gamma \vdash a \approx_A b$) iff
    for any context $\gamma : (\Gamma, A) \to (\varepsilon, \pUnit)$ with unique fresh variable $h$: $|[h := a]\gamma|$ normalizes iff $|[h := b]\gamma|$ normalizes
\end{definition}

\begin{lemma}
    \label{lem:5:obs_refl}
    $\Gamma \vdash a \approx_A a$
\end{lemma}
\begin{proof}
    Immediate by definition.
\end{proof}

\begin{lemma}
    \label{lem:5:obs_sym}
    If $\Gamma \vdash a \approx_A b$ then $\Gamma \vdash b \approx_A a$
\end{lemma}
\begin{proof}
    By definition the stated condition holds via an if-and-only-if.
    Hence, observational equivalence is symmetric.
\end{proof}

\begin{lemma}
    \label{lem:5:obs_trans}
    If $\Gamma \vdash a \approx_A b$ and $\Gamma \vdash b \approx_A c$ then $\Gamma \vdash a \approx_A c$
\end{lemma}
\begin{proof}
    Let $\gamma : (\Gamma, A) \to (\varepsilon, \pUnit)$ be an arbitrary context with unique fresh variable $h$.
    Suppose $|[h := b]\gamma|$ diverges, then by $\Gamma \vdash b \approx_A c$ it must be the case that $|[h := c]\gamma|$ diverges.
    By Lemma~\ref{lem:5:obs_sym}: $\Gamma \vdash b \approx_A a$ and thus as above $|[h := a]\gamma|$ diverges.
    Suppose $|[h := b]\gamma|$ normalizes, then by $\Gamma \vdash b \approx_A c$: $|[h := c]\gamma|$ normalizes.
    Likewise, using symmetry and the same reasoning: $|[h := a]\gamma|$ normalizes.
    Hence, $|[h := a]\gamma|$ normalizes if and only if $|[h := c]\gamma|$ normalizes.
\end{proof}

\begin{definition}
    A proof is $\varphi$-\textbf{safe} iff for every usage of $\varphi$ with $\Gamma \vdash \varphi(a, b, e) : (x : A) \cap B$ then $\Gamma \vdash \varphi(a, b, e) \approx_{(x : A) \cap B} b$
\end{definition}

\begin{theorem}
    If $\Gamma \vdash t : A$ and $t$ is $\varphi$-safe then $|t|$ is strongly normalizing
\end{theorem}
\begin{proof}
    By lexicographic induction on the nesting count of $\varphi$ in $t$ and the inference judgment $\Gamma \vdash t : A$.
    If $t$ does not contain any $\varphi$ subexpressions then it is a strict proof and thus $|t|$ is strongly normalizing by Theorem~\ref{lem:5:strict_object_sn}.
    Thus, suppose $t$ has $i + 1$ nested $\varphi$ expressions.
    For every case except the \textsc{App} case $|t|$ is strongly normalizing by the IH.
    The \textsc{App} case is special because the function-part could be a $\varphi$ and thus generate a $\beta$-redex in erasure that is not tracked by proof reduction.

    $\text{Case: }\begin{array}{c} \AppRule[*] \end{array}$
    \begin{proofcase}
        Suppose wlog that $f = \varphi(a^\prime, b, e).2$ and thus $\app{|f|}{\omega}{|a|} = \app{|a^\prime|}{\omega}{|a|}$.
        By the IH both $|a^\prime|$ and $|a|$ are strongly normalizing.
        Note that $t$ is $\varphi$-safe thus $\Gamma \vdash \varphi(a^\prime, b, e) \approx_{(x : A) \cap B} b$.
        Thus, it must be the case that $\Gamma \vdash \app{\varphi(a^\prime, b, e).2}{\omega}{a} \approx_{[x := a]B} \app{b.2}{\omega}{a}$.
        Hence, for context $\gamma$ with hole $h$: $[h := \app{|a^\prime|}{\omega}{|a|}]\gamma$ is normalizing if and only if $[h := \app{|b|}{\omega}{|a|}]\gamma$ is normalizing.
        However, $\app{b.2}{\omega}{a}$ has a smaller nesting level of $\varphi$ expressions, thus $\app{|b|}{\omega}{|a|}$ is strongly normalizing.
    \end{proofcase}
\end{proof}

Characterizing when $\varphi$ does not introduce diverging objects is useful because it enables, at the bare minimum, an external validation of each usage.
It is not clear how this requirement may be internalized in the system.
First, a logical relation capturing observational equivalence would likely need to be developed, but because this relation needs to capture equivalence of objects it is not obvious how to adapt existing approaches.
Moreover, that relation would have to be bolted on an as auxiliary proof system in order to prove $\varphi$-safety.
At least, the evidence required to use a \textsc{Cast} rule is a sanity check.
Indeed, if this evidence is ``morally'' true then contextual equivalence will hold by the Leibniz Law.

\begin{conjecture}
    $\Gamma \vdash \varphi(a, b, e) \approx_{(x : A) \cap B} b$ iff $\Gamma \vdash a \approx_{A} b.1$
\end{conjecture}

Note that while the evidence for $\varphi(a, b, e)$ has the type $e : a =_A b.1$ it is easy to use this evidence to construct a proof $e^\prime : \varphi(a, b, e) =_{(x : A) \cap B} b$.
Just eliminate $e$ using $\psi$ and the objects will match.
Going the opposite direction is just as simple, as $b$ may be substituted with the left-hand side and the objects will again be identical.
However, it is not clear that a first projection expressed via observational equivalence is logically equivalent to $\varphi$-safety.
The primary obstacle is determining if the erasure of every $\gamma : ((x : A) \cap B, \Gamma) \to (\varepsilon, \pUnit)$ context can be computed via a first projection operation on contexts to obtain $\gamma.1 : (A, \Gamma) \to (\varepsilon, \pUnit)$ with the same erasure.
Demonstrating this conjecture holds would be the first important step to defining a logical relation for contextual equivalence, because it would mean that $\varphi$ terms could be removed entirely from the definition.

\section{Counterexamples with \texorpdfstring{$\varphi$}{Phi}}

It does not take much effort to produce an example of divergence using $\varphi$.
Note, however, that all examples require a context where $\pFalse$ is derivable.
The first example uses $\varphi$ to give self a recursive type: $\pself : \pUnit$ and $\pself : \pUnit \to_\omega \pUnit$ simultaneously.
Divergence is a trivial consequence.
In this example, the False premise is completely erased.
This is not really a problem as a proof assistant needs to reduce under binders anyway and the erased argument blocks logical issues regardless.
\begin{align*}
    \pFalse &= (X : \star) \to_0 X_\square \\
    \pself &= \abs{\lambda_\omega}{x}{\pUnit}{
        \apptwo{x}{0}{\pUnit}{\omega}{x}
    } \\
    |\pself| &= \abs{\lambda_\omega}{x}{\diamond}{\app{x}{\omega}{x}} \\
    b &= \abs{\lambda_\omega}{f}{\pFalse}{
        [
            \app{f}{0}{(\pUnit \to_\omega \pUnit)},
            \app{f}{0}{\pUnit}
        ]
    } \\
    e &= \abs{\lambda_\omega}{f}{\pFalse}{
        \app{f}{0}{
            (\pself =_{\pUnit \to_\omega \pUnit} (\app{b}{\omega}{f}).1)
        }
    } \\
    bad &= \abs{\lambda_0}{f}{\pFalse}{
        \app{\pself}{\omega}{
            (
                \varphi(\pself, \app{b}{\omega}{f}, \app{e}{\omega}{f})
            ).2
        }
    } \\
    |bad| &= \app{|\pself|}{\omega}{|\pself|}
\end{align*}
What one can learn from the above example is that the hypothetical evidence is problematic for using $\varphi$.
Restricting the context is one idea to make all usages $\varphi$-safe.
Unfortunately, the restriction that $FV(|e|)$ is empty is too strong, it prevents all interesting usages because $b.1 \betastar a$ in all cases as a result.
Instead, the reader might imagine that the context is \textit{partially} restricted.
For example, suppose $b : (a : A) \to (x : A) \cap B$ and $e : (a : A) \to a_\star =_A (\app{b}{\omega}{a}).1$ with $FV(|e|)$ empty.
With this setup, $e$ depends only on the single input and expresses only the fact that $b$ is extensionally an identity function.
The object witness term $a$ can then be dropped and the object for the $\varphi$ term would be: $|\varphi(b, e)| = \abs{\lambda_\omega}{x}{\diamond}{x}$.
Unfortunately, this idea fails as enough of the context may be uncurried into the type of $A$ to construct a divergent term.
\begin{align*}
    A &= (\pUnit \to_\omega \pUnit) \times \pFalse \\
    T &= (A \to_\omega \pUnit \to_\omega \pUnit) \to_\omega (\pUnit \to_\omega \pUnit) \to_\omega \pUnit \\
    b &= \abs{\lambda_\omega}{w}{A}{
        \app{(\app{\psnd}{\omega}{w})}{0}{(A \cap T)}
    } \\
    e &= \abs{\lambda_\omega}{x}{A}{
        \app{(\app{\psnd}{\omega}{x})}{0}{(x =_A (\app{b}{\omega}{x}).1)}
    } \\
    phi &= \abs{\lambda_\omega}{a}{A}{\varphi(a, \app{b}{\omega}{a}, \app{e}{\omega}{a})} \\
    p1 &= \abs{\lambda_\omega}{f}{\pFalse}{\apptwo{\ppair}{\omega}{\pself}{\omega}{f}} \\
    p2 &= \abs{\lambda_\omega}{x}{A}{\app{\pfst}{\omega}{x}} \\
    p3 &= \abs{\lambda_\omega}{f}{\pFalse}{
        \apptwo{(\app{phi}{\omega}{(\app{p1}{\omega}{f})}).2}
            {\omega}{p2}
            {\omega}{\pself}
    } \\
    bad &= \abs{\lambda_\omega}{f}{\pFalse}{
        \apptwo{(\appthr{p3}{\omega}{f}{0}{\pUnit}{\omega}{\punit})}
            {0}{(\pUnit \to_\omega \pUnit)}
            {\omega}{\pself}
    } \\
    |bad| &= \abs{\lambda_\omega}{f}{\diamond}{\app{|\pself|}{\omega}{|\pself|}}
\end{align*}
This counterexample requires a relevant abstraction, but this could probably be avoided by a more sophisticated formulation.
Again, it also does not really matter as proof assistants reduce under binders anyway.
This example demonstrates that finding a balance between usability and restriction of the context is very difficult, if not simply impossible.

Another option is to remove $\varphi$ altogether from the system.
It is a significant source of complexity because it demands reduction after erasure in the definition of conversion and is the \textit{only} source of divergence in $\ced$.
To contrast, $\oldced$ has the following sources of divergence:
\begin{enumerate}
    \item terms on the left-hand and right-hand side of an equality are untyped-$\lambda$-calculus syntax;
    \item the Kleene Trick allows for untyped-$\lambda$-calculus syntax as witnesses of trivial equalities;
    \item the rewrite rule, $\rho$, is erased and thus enables non-termination by Abel and Coquand \cite{abel2020_normalization};
    \item the $\varphi$ rule, for the same reason as $\ced$.
\end{enumerate}
The first three sources are eliminated by the design of $\ced$, yet the last remains.
Ultimately, the \textsc{Cast} rule is too important to not only the spirit of Cedille but its capability.
Losing $\varphi$, as far as the current research shows, would prevent almost all existing encodings.
The cost to be paid would be too much.


% \appendix

% \chapter{Stuff you should know already}
% \lipsum[100-120]

% \chapter{Stuff no one ever thought to tell you}
% \lipsum[121-140]

\backmatter

\printbibliography

\end{document}
