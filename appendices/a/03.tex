
\begin{lemma}
    The following hold:
    \begin{enumerate}
        \item $\Gamma \vdash t \infr A$ implies $\Gamma \vdash t \cinfr A$
        \item $\Gamma \vdash t \cinfr A$ implies $\Gamma \vdash t \chck A$
        \item $\Gamma \vdash t \chck A$ implies $\exists B,\ \Gamma \vdash t \infr B$
    \end{enumerate}
\end{lemma}
\begin{proof}
    We prove each individually:
    \begin{enumerate}
        \item Immediate by reflexivity of multistep reduction.
        \item {
            By inspecting the derivation we know that there exist $B$ such that $\Gamma \vdash t \infr B$ and $B \betastar A$.
        }
        \item Immediate by inspecting the derivation
    \end{enumerate}
\end{proof}

\begin{theorem}
    If $\Gamma \vdash t \chck A$ then 
    \begin{enumerate}
        \item {
            $t \betared t^\prime$ implies $\Gamma \vdash t^\prime \chck A$, and
        }
        \item {
            $\Gamma \betared \Gamma^\prime$ implies $\Gamma^\prime \vdash t \chck A$
        }
    \end{enumerate}
\end{theorem}
\begin{proof}
    By induction on the proof derivation of $t$ with motives
    \begin{itemize}
        \item {
            $\Gamma \vdash t \infr A$ motive is
            $(t \betared t^\prime \to \Gamma \vdash t^\prime \chck A) \wedge (\Gamma \betared \Gamma^\prime \to \Gamma^\prime \vdash t \chck A)$
        }
        \item {
            $\Gamma \vdash t \cinfr A$ motive is
            $(t \betared t^\prime \to \Gamma \vdash t^\prime \chck A) \wedge (\Gamma \betared \Gamma^\prime \to \Gamma^\prime \vdash t \chck A)$
        }
        \item {
            $\Gamma \vdash t \chck A$ motive is
            $(t \betared t^\prime \to \Gamma \vdash t^\prime \chck A) \wedge (\Gamma \betared \Gamma^\prime \to \Gamma^\prime \vdash t \chck A)$
        }
        \item {
            $\vdash \Gamma$ motive is
            $\Gamma \betared \Gamma^\prime \to\ \vdash \Gamma^\prime$
        }
    \end{itemize}

$\text{Case: }\begin{array}{c} \AxiomRule[*] \end{array}$
\begin{proofcase}
    \begin{enumerate}
        \item Impossible by inversion on reduction.
        \item {
            By the IH we know that $\vdash \Gamma^\prime$ which is sufficient to construct the proof $\Gamma^\prime \vdash \star \infr \kind$.
        }
    \end{enumerate}
\end{proofcase}

$\text{Case: }\begin{array}{c} \VarRule[*] \end{array}$
\begin{proofcase}
    \begin{enumerate}
        \item Impossible by inversion on reduction.
        \item {
            By the IH we know that $\vdash \Gamma^\prime$.
            Either $(x : A) \in \Gamma$ or $(x : A^\prime) \in \Gamma$ where $A \betared A^\prime$.
            The first case is trivial.
            We know that $A$ is either $\star$ or $\kind$, but by inversion on the reduction $A \betared A^\prime$ the second case is impossible.        
        }
    \end{enumerate}
\end{proofcase}

$\text{Case: }\begin{array}{c} \PiRule[*] \end{array}$
\begin{proofcase}
    
\end{proofcase}

$\text{Case: }\begin{array}{c} \LambdaRule[*] \end{array}$
\begin{proofcase}
    Obvious.
\end{proofcase}

$\text{Case: }\begin{array}{c} \AppRule[*] \end{array}$
\begin{proofcase}
    Obvious.
\end{proofcase}

$\text{Case: }\begin{array}{c} \IntersectionRule[*] \end{array}$
\begin{proofcase}
    Obvious.
\end{proofcase}

$\text{Case: }\begin{array}{c} \PairRule[*] \end{array}$
\begin{proofcase}
    Obvious.
\end{proofcase}

$\text{Case: }\begin{array}{c} \FirstRule[*] \end{array}$
\begin{proofcase}
    Obvious.
\end{proofcase}

$\text{Case: }\begin{array}{c} \SecondRule[*] \end{array}$
\begin{proofcase}
    Obvious.
\end{proofcase}

$\text{Case: }\begin{array}{c} \EqualityRule[*] \end{array}$
\begin{proofcase}
    Obvious.
\end{proofcase}

$\text{Case: }\begin{array}{c} \ReflRule[*] \end{array}$
\begin{proofcase}
    Obvious.
\end{proofcase}

$\text{Case: }\begin{array}{c} \EqualityInductionRule[*] \end{array}$
\begin{proofcase}
    Obvious.
\end{proofcase}

$\text{Case: }\begin{array}{c} \PromoteRule[*] \end{array}$
\begin{proofcase}
    Obvious.
\end{proofcase}

$\text{Case: }\begin{array}{c} \CastRule[*] \end{array}$
\begin{proofcase}
    Obvious.
\end{proofcase}

$\text{Case: }\begin{array}{c} \SeparationRule[*] \end{array}$
\begin{proofcase}
    Obvious.
\end{proofcase}

$\text{Case: }\begin{array}{c} \HeadInferenceRule[*] \end{array}$
\begin{proofcase}
    Obvious.
\end{proofcase}

$\text{Case: }\begin{array}{c} \CheckRule[*] \end{array}$
\begin{proofcase}
    Obvious.
\end{proofcase}

$\text{Case: }\begin{array}{c} \ContextEmptyRule[*] \end{array}$
\begin{proofcase}
    The goal in the well-formed context cases is not to produce a proof of a syntactic term, but instead to construct a well-formed context, which is trivial in this case.
\end{proofcase}

$\text{Case: }\begin{array}{c} \ContextAppendRule[*] \end{array}$
\begin{proofcase}
    Either $\Gamma \betared \Gamma^\prime$ or $A \betared A^\prime$.
    In the first case the IH gives us $\vdash \Gamma^\prime$ and $\Gamma^\prime \vdash A \cinfr K$ and because reduction does not alter free variables we know that $x \notin \text{FV}(\Gamma^\prime)$.
    Thus, $\vdash \Gamma^\prime, x : A$ is constructed in the first case.

    In the second case the IH gives us $\Gamma \vdash A^\prime \cinfr K$.
    Which is sufficient to construct $\vdash \Gamma, x : A^\prime$.
\end{proofcase}
\end{proof}
