
\begin{theorem}
    If $\Gamma \vdash A \cinfr K$ then 
    \begin{enumerate}
        \item {
            $A \betared A^\prime$ implies $\Gamma \vdash A^\prime \cinfr K$, and
        }
        \item {
            $\Gamma \betared \Gamma^\prime$ implies $\Gamma^\prime \vdash A \cinfr K$,
        }
    \end{enumerate}
    where $K$ is either $\star$ or $\kind$
\end{theorem}
\begin{proof}
    By induction on the proof derivation of $A$ with motives
    \begin{itemize}
        \item {
            $\Gamma \vdash A \infr K$ motive is
            $$(\forall A^\prime,\ A \betared A^\prime \to \Gamma \vdash A^\prime \infr K) \wedge (\forall\ \Gamma^\prime,\ \Gamma \betared \Gamma^\prime \to \Gamma^\prime \vdash A \infr K)$$
        }
        \item {
            $\Gamma \vdash A \cinfr K$ motive is
            $$(\forall A^\prime,\ A \betared A^\prime \to \Gamma \vdash A^\prime \cinfr K) \wedge (\forall\ \Gamma^\prime,\ \Gamma \betared \Gamma^\prime \to \Gamma^\prime \vdash A \cinfr K)$$
        }
        \item {
            $\Gamma \vdash A \chck K$ motive is
            $$(\forall A^\prime,\ A \betared A^\prime \to \Gamma \vdash A^\prime \chck K) \wedge (\forall\ \Gamma^\prime,\ \Gamma \betared \Gamma^\prime \to \Gamma^\prime \vdash A \chck K)$$
        }
        \item {
            $\vdash \Gamma$ motive is
            $\forall\ \Gamma^\prime,\ \Gamma \betared \Gamma^\prime \to\ \vdash \Gamma^\prime$
        }
    \end{itemize}

$\text{Case: }\begin{array}{c} \AxiomRule[*] \end{array}$
\begin{proofcase}
    \begin{enumerate}
        \item Impossible by inversion on reduction.
        \item {
            By the IH we know that $\vdash \Gamma^\prime$ which is sufficient to construct the proof $\Gamma^\prime \vdash \star \infr \kind$. Finally, we have $\kind \betastar \kind$ to construct $\Gamma^\prime \vdash \star \cinfr \kind$.
        }
    \end{enumerate}
\end{proofcase}

$\text{Case: }\begin{array}{c} \VarRule[*] \end{array}$
\begin{proofcase}
    \begin{enumerate}
        \item Impossible by inversion on reduction.
        \item {
            By the IH we know that $\vdash \Gamma^\prime$.
            Either $(x : A) \in \Gamma$ or $(x : A^\prime) \in \Gamma$ where $A \betared A^\prime$.
            The first case is trivial.
            We know that $A$ is either $\star$ or $\kind$, but by inversion on the reduction $A \betared A^\prime$ the second case is impossible.        
        }
    \end{enumerate}
\end{proofcase}

$\text{Case: }\begin{array}{c} \PiRule[*] \end{array}$
\begin{proofcase}
    \begin{enumerate}
        \item {
            If $A \betared A^\prime$ then by IH on $\D{1}$ we know that $\Gamma \vdash A^\prime \cinfr \text{dom}_\Pi(m, K)$.
            Now, we have $\Gamma,x:A \betared \Gamma,x:A^\prime$ and by IH on $\D{2}$ we have $\Gamma,x:A^\prime \vdash B \cinfr \text{codom}_\Pi(m)$.
            Reconstructing the function type yields $\Gamma \vdash (x:A^\prime) \to_m B \infr \text{codom}_\Pi(m)$.
            If $B \betared B^\prime$ then by IH on $\D{2}$ we construct the goal again.
        }
        \item {
            If $\Gamma \betared \Gamma^\prime$ then by the IH on $\D{1}$ we have $\Gamma^\prime \vdash A \cinfr \text{dom}_\Pi(m, K)$.
            Now we know that $\Gamma,x:A \betared \Gamma^\prime, x:A$ by an inference rule of context reduction.
            Thus, by IH on $\D{2}$ we have $\Gamma^\prime, x:A \vdash B \cinfr \text{codom}_\Pi(m)$.
            Reconstructing the function type yields $\Gamma^\prime \vdash (x:A) \to_m \infr \text{codom}_\Pi(m)$.
        }
    \end{enumerate}
\end{proofcase}

$\text{Case: }\begin{array}{c} \LambdaRule[*] \end{array}$
\begin{proofcase}
    Immediate because $(x : A) \to_m B$ is not equal to $\star$ or $\kind$.
\end{proofcase}

$\text{Case: }\begin{array}{c} \AppRule[*] \end{array}$
\begin{proofcase}
    First observe that $[x := a]B \neq \kind$ because $\star$ cannot unify with an application syntactic form.
    Assume $[x := a]B = \star$ which means $B = \star$.
    \begin{enumerate}
        \item {
            If $f \betared f^\prime$ or $a \betared a^\prime$ then in either case the IH for the respective derivation yields the necessary judgments to reconstruct the application inference.
        }
        \item {
            If $\Gamma \betared \Gamma^\prime$ then the result is straightforward by the IH.
        }
    \end{enumerate}
\end{proofcase}

$\text{Case: }\begin{array}{c} \IntersectionRule[*] \end{array}$
\begin{proofcase}
    \begin{enumerate}
        \item {
            If $A \betared A^\prime$ then by IH on $\D{1}$ we know that $\Gamma \vdash A^\prime \cinfr \star$.
            Now, we have $\Gamma,x:A \betared \Gamma,x:A^\prime$ so by the IH on $\D{2}$ we know that $\Gamma,x:A^\prime \vdash B \cinfr \star$.
            If $B \betared B^\prime$ then the IH on $\D{2}$ is sufficient to obtain $\Gamma,x:A \vdash B^\prime \cinfr \star$.
            Reconstructing the intersection type in both cases concludes this case.
        }
        \item {
            Straightforward by applying the IH.
        }
    \end{enumerate}
\end{proofcase}

$\text{Case: }\begin{array}{c} \PairRule[*] \end{array}$
\begin{proofcase}
    Immediate because $(x : A) \cap B$ is not equal to $\star$ or $\kind$.
\end{proofcase}

$\text{Case: }\begin{array}{c} \FirstRule[*] \end{array}$
\begin{proofcase}
    If $A = \kind$ then the judgment is impossible because only $\star$ infers $\kind$.
    If $A = \star$ then $t.1$ must be either a variable (impossible), a function type (impossible), an application (impossible), an intersection type (impossible), a second projection (impossible), or an equality type (impossible).

    In other words, while it is true that $A$ could theoretically be $\star$, because the proof term is $t.1$ the syntactic form does not match any rule that could infer $\star$.
\end{proofcase}

$\text{Case: }\begin{array}{c} \SecondRule[*] \end{array}$
\begin{proofcase}
    By similar reasoning to the first projection case, it is not possible for $B = \star$ or $B = \kind$.
\end{proofcase}

$\text{Case: }\begin{array}{c} \EqualityRule[*] \end{array}$
\begin{proofcase}
    \begin{enumerate}
        \item {
            There are three possibilities: $a \betared a^\prime$, or $A \betared A^\prime$ or $b \betared b^\prime$.
            In any case the IH yields the corresponding derivation for the reduced proof.
        }
        \item By the IH and reconstructing the equality type rule.
    \end{enumerate}
\end{proofcase}

$\text{Case: }\begin{array}{c} \ReflRule[*] \end{array}$
\begin{proofcase}
    Immediate because $t =_A t$ is not equal to $\star$ or $\kind$.
\end{proofcase}

$\text{Case: }\begin{array}{c} \EqualityInductionRule[*] \end{array}$
\begin{proofcase}
    Immediate because $\app{U}{m}{V}$ is not equal to $\star$ or $\kind$ (for any $U, V, m$).
    Note, while this expression may \textit{reduce} to $\star$ the restriction placed is syntactic identity.
\end{proofcase}

$\text{Case: }\begin{array}{c} \PromoteRule[*] \end{array}$
\begin{proofcase}
    Immediate because $a =_T b$ is not equal to $\star$ or $\kind$ (for any $T$).
\end{proofcase}

$\text{Case: }\begin{array}{c} \CastRule[*] \end{array}$
\begin{proofcase}
    Immediate because $(x : A) \cap B$ is not equal to $\star$ or $\kind$.
\end{proofcase}

$\text{Case: }\begin{array}{c} \SeparationRule[*] \end{array}$
\begin{proofcase}
    Immediate because $(X : \star) \to_0 X$ is not equal to $\star$ or $\kind$.
\end{proofcase}

$\text{Case: }\begin{array}{c} \HeadInferenceRule[*] \end{array}$
\begin{proofcase}
    We know that $B$ is either $\star$ or $\kind$, but by inversion of $\D{2}$ it must be the case that $A = B$.
    By the IH on $\D{1}$ we obtain the goal.
\end{proofcase}

$\text{Case: }\begin{array}{c} \CheckRule[*] \end{array}$
\begin{proofcase}
    We know that $B$ is either $\star$ or $\kind$, but by inversion of $\D{2}$ it cannot be $\kind$.
    Thus, $B = \star$ and moreover it must be that $A = \star$ to satisfy $\D{3}$.
    By the IH on $\D{1}$ we have:
    $(t \betared t^\prime \to \Gamma \vdash t^\prime \infr \star) \wedge (\Gamma \betared \Gamma^\prime \to \Gamma^\prime \vdash t \infr \star)$.
    Both cases of the goal follow from the IH.
\end{proofcase}

$\text{Case: }\begin{array}{c} \ContextEmptyRule[*] \end{array}$
\begin{proofcase}
    By inversion on the reduction relation the context cannot be empty.
\end{proofcase}

$\text{Case: }\begin{array}{c} \ContextAppendRule[*] \end{array}$
\begin{proofcase}
    Either $\Gamma \betared \Gamma^\prime$ or $A \betared A^\prime$.
    In the first case the IH gives us $\vdash \Gamma^\prime$ (from $\D{2}$) and $\Gamma^\prime \vdash A \cinfr K$ (from $\D{3}$) and because reduction does not alter free variables we know that $x \notin \text{FV}(\Gamma^\prime)$.
    Thus, $\vdash \Gamma^\prime, x : A$ is constructed in the first case.

    In the second case the IH gives us $\Gamma \vdash A^\prime \cinfr K$ (from $\D{3}$).
    Which is sufficient to construct $\vdash \Gamma, x : A^\prime$.
\end{proofcase}
\end{proof}
