\chapter{Proof of Preservation}
\label{ap:b}

\begin{lemma}
    \textcolor{white}{\_}
    \begin{enumerate}
        \item If $\Gamma \vdash s \infr A$, $s \betared t$, and $\Gamma \betared \Delta$ then $\exists\ B.$ $A \betastar B$ and $\Delta \vdash t \infr B$
        \item If $\Gamma \vdash s \infr A$ and $\Gamma \betared \Delta$ then $\exists\ B.$ $A \betastar B$ and $\Delta \vdash s \infr B$
        \item If $\Gamma \vdash s \cinfr A$, $s \betared t$, and $\Gamma \betared \Delta$ then $\exists\ B.$ $A \betastar B$ and $\Delta \vdash t \cinfr B$
        \item If $\Gamma \vdash s \cinfr A$ and $\Gamma \betared \Delta$ then $\exists\ B.$ $A \betastar B$ and $\Delta \vdash s \cinfr B$
        \item If $\Gamma \vdash s \chck A$, $s \betared t$, and $\Gamma \betared \Delta$ then $\Delta \vdash t \chck A$
        \item If $\Gamma \vdash s \chck A$ and $\Gamma \betared \Delta$ then $\Delta \vdash s \chck A$
        \item If $\vdash \Gamma$ and $\Gamma \betared \Delta$ then $\vdash \Delta$
    \end{enumerate}
    \label{lem:b:preservation_step}
\end{lemma}
\begin{proof}
    By mutual induction, cases 1 and 2 are merged into the inductive motive of $\Gamma \vdash s \infr A$, likewise for cases 3 and 4, and 5 and 6 for their respective relations.

    $\text{Case: }\begin{array}{c} \AxiomRule[*] \end{array}$
    \begin{proofcase}
        By case analysis, $\star \betared t$ is impossible.
        \\ \\
        Suppose $\Gamma \betared \Delta$.
        By the IH applied to $\D{1}$: $\vdash \Delta$.
        Thus, $\Delta \vdash \star \infr \kind$.
    \end{proofcase}

    $\text{Case: }\begin{array}{c} \VarRule[*] \end{array}$
    \begin{proofcase}
        By case analysis, $x \betared t$ is impossible.
        \\ \\
        Suppose $\Gamma \betared \Delta$.
        By the IH applied to $\D{1}$: $\vdash \Delta$.
        Note that context reduction does not alter associated variables.
        Suppose that $A \betared A^\prime$ within $\Gamma \betared \Delta$.
        Then $(x : A^\prime) \in \Delta$ and $\Delta \vdash x \infr A^\prime$.
        Suppose instead that $A$ is unchanged by $\Gamma \betared \Delta$.
        Then $(x : A) \in \Delta$ and $\Delta \vdash x \infr A$.
    \end{proofcase}

    $\text{Case: }\begin{array}{c} \PiRule[*] \end{array}$
    \begin{proofcase}
        Suppose $s \betared t$.
        There are two possible cases:

        $\text{Case: }\begin{array}{c} \BinderReductionOne[*] \end{array}$
        \begin{proofcase}
            Applying the IH yields $\D{1}$ $\exists\ T_1$ such that $\pdom(m, K) \betastar T_1$ and $\Delta \vdash A^\prime \cinfr T_1$.
            However, $\pdom(m,K)$ cannot reduce, thus $T_1 = \pdom(m,K)$.
            Note that $\Gamma, x : A \betared \Delta, x : A^\prime$.
            Thus, by the IH applied to $\D{2}$ $\exists\ T_2$ such that $\Delta, x : A^\prime \vdash B \cinfr T_2$.
            However, $\pcodom(m)$ cannot reduce, thus $T_2 = \pcodom(m)$.
            Therefore, $\Delta \vdash (x : A^\prime) \to_m B \infr \pcodom(m)$.
        \end{proofcase}

        $\text{Case: }\begin{array}{c} \BinderReductionTwo[*] \end{array}$
        \begin{proofcase}
            By the IH on $\D{2}$: $\exists\ T$ such that $\pcodom(m) \betastar T$ and $\Delta, x : A \vdash B^\prime \cinfr T$.
            However, $\pcodom(m)$ cannot reduce, thus $T = \pcodom(m)$.
            Thus, $\Delta \vdash (x : A) \to_m B^\prime \infr \pcodom(m)$.
        \end{proofcase}
        Suppose $\Gamma \betared \Delta$.
        Then, also: $\Gamma, x : A \betared \Delta, x : A$.
        Note that because this is mutual induction, the context argument is generalized in the inference cases.
        Using the IH yields: $\exists\ T_1, T_2$ such that $\pdom(m, K) \betastar T_1$, $\pcodom(m) \betastar T_2$, $\Delta \vdash A \cinfr T_1$, and $\Delta, x : A \vdash B \cinfr T_2$.
        As before, $T_1 = \pdom(m, K)$ and $T_2 = \pcodom(m)$.
        Thus, $\Delta \vdash (x : A) \to_m B \infr \pcodom(m)$.
    \end{proofcase}

    $\text{Case: }\begin{array}{c} \LambdaRule[*] \end{array}$
    \begin{proofcase}
        Suppose $s \betared t$.
        There are two possible cases:

        $\text{Case: }\begin{array}{c} \BinderReductionOne[*] \end{array}$
        \begin{proofcase}
            Exactly as the \textsc{Pi} case, the IH applied to $\D{1}$ yields: $\Delta \vdash A^\prime \cinfr \pdom(m, K)$.
            Now $\Gamma, x : A \betared \Delta, x : A^\prime$.
            Using the IH with $\D{2}$ means $\exists\ B^\prime$ such that $B \betastar B^\prime$ and $\Delta, x : A^\prime \vdash t \infr B^\prime$.
            Applying the IH to $\D{3}$ with
        \end{proofcase}

        $\text{Case: }\begin{array}{c} \BinderReductionTwo[*] \end{array}$
        \begin{proofcase}
        \end{proofcase}
    \end{proofcase}

    $\text{Case: }\begin{array}{c} \AppRule[*] \end{array}$
    \begin{proofcase}
    \end{proofcase}

    $\text{Case: }\begin{array}{c} \IntersectionRule[*] \end{array}$
    \begin{proofcase}
    \end{proofcase}

    $\text{Case: }\begin{array}{c} \PairRule[*] \end{array}$
    \begin{proofcase}
    \end{proofcase}

    $\text{Case: }\begin{array}{c} \FirstRule[*] \end{array}$
    \begin{proofcase}
    \end{proofcase}

    $\text{Case: }\begin{array}{c} \SecondRule[*] \end{array}$
    \begin{proofcase}
    \end{proofcase}

    $\text{Case: }\begin{array}{c} \EqualityRule[*] \end{array}$
    \begin{proofcase}
    \end{proofcase}

    $\text{Case: }\begin{array}{c} \ReflRule[*] \end{array}$
    \begin{proofcase}
    \end{proofcase}

    $\text{Case: }\begin{array}{c} \SubstRule[*] \end{array}$
    \begin{proofcase}
    \end{proofcase}

    $\text{Case: }\begin{array}{c} \PromoteFstRule[*] \end{array}$
    \begin{proofcase}
    \end{proofcase}

    $\text{Case: }\begin{array}{c} \CastRule[*] \end{array}$
    \begin{proofcase}
    \end{proofcase}

    $\text{Case: }\begin{array}{c} \SeparationRule[*] \end{array}$
    \begin{proofcase}
        Suppose $s \betared t$.
        Note that this can only mean $e \betared e^\prime$ in this case.
        By the IH: $\Gamma \vdash e^\prime \chck \text{ctt} =_{\text{cBool}} \text{cff}$.
        Thus, $\Gamma \vdash \delta(e^\prime) \infr (X : \star) \to_0 X$.
        \\ \\
        Suppose $\Gamma \betared \Delta$.
        By the IH: $\Delta \vdash e \chck \text{ctt} =_{\text{cBool}} \text{cff}$.
        Thus, $\Delta \vdash \delta(e) \infr (X : \star) \to_0 X$.
    \end{proofcase}

    $\text{Case: }\begin{array}{c} \HeadInferenceRule[*] \end{array}$
    \begin{proofcase}
        Suppose that $s \betastar t$.
        Proceed by induction on $s \betastar t$.
        Note that the reflexivity case is trivial.
        Suppose $s \betared z$ and $z \betastar t$.
        By the outer IH with $s \betared z$ and $\D{1}$, $\exists\ T_1$ such that $A \betastar T_1$ and $\Gamma \vdash z \infr T_1$.
        By confluence, $\exists\ B_1$ such that $T_1 \betastar B_1$ and $B \betastar B_1$.
        Now, applying the inner IH with $z \betastar t$ gives $\exists\ B_2$ with $B_1 \betastar B_2$ and $\Gamma \vdash t \cinfr B_2$.
        Note that $B \betastar B_1 \betastar B_2$.
        % By the IH applied to $\D{1}$ $\exists\ A^\prime$ such that $A \betastar A^\prime$ and $\Gamma \vdash t \infr A^\prime$.
        % Using confluence, $\exists\ B^\prime$ such that $A^\prime \betastar B^\prime$ and $B \betastar B^\prime$.
        % Thus, $\Gamma \vdash t \cinfr B^\prime$.
        % \\ \\
        % Suppose that $\Gamma \betared \Delta$.
        % By the IH applied to $\D{1}$ $\exists\ A^\prime$ such that $A \betastar A^\prime$ and $\Delta \vdash s \infr A^\prime$.
        % Using confluence, $\exists\ B^\prime$ such that $A^\prime \betastar B^\prime$ and $B \betastar B^\prime$.
        % Thus, $\Delta \vdash s \cinfr B^\prime$.
    \end{proofcase}

    $\text{Case: }\begin{array}{c} \CheckRule[*] \end{array}$
    \begin{proofcase}
        Suppose that $s \betared t$.
        By the IH applied to $\D{1}$ $\exists\ A^\prime$ such that $A \betastar A^\prime$ and $\Gamma \vdash t \infr A^\prime$.
        Using Lemma~\ref{lem:2:infer_type_is_pseobj} on $\D{1}$ yields $A\pseobj$.
        Applying Lemma~\ref{lem:2:conv_red_f} to $\D{2}$ yields $A^\prime \equiv B$.
        Thus, $\Gamma \vdash t \chck B$.
        \\ \\
        Suppose that $\Gamma \betared \Delta$.
        By the IH applied to $\D{1}$ $\exists\ A^\prime$ such that $A \betastar A^\prime$ and $\Delta \vdash s \infr A^\prime$.
        Using Lemma~\ref{lem:2:infer_type_is_pseobj} on $\D{1}$ yields $A\pseobj$.
        Applying Lemma~\ref{lem:2:conv_red_f} to $\D{2}$ yields $A^\prime \equiv B$.
        By the IH applied to $\D{2}$ $\exists\ T$ such that $K \betastar T$ and $\Delta \vdash B \cinfr T$.
        However, $K = \star$ or $K = \kind$ cannot step, therefore $T = K$.
        Thus, $\Delta \vdash s \chck B$.
    \end{proofcase}

    $\text{Case: }\begin{array}{c} \ContextEmptyRule[*] \end{array}$
    \begin{proofcase}
        Impossible by case analysis of $\Gamma \betared \Delta$ the context cannot be empty.
    \end{proofcase}

    $\text{Case: }\begin{array}{c} \ContextAppendRule[*] \end{array}$
    \begin{proofcase}
        By cases on $\Gamma, x : A \betared \Delta, x : A$.

        % $\text{Case: }\begin{array}{c} \ContextReductionHeadRule[*] \end{array}$
        % \begin{proofcase}
        %     Necessarily $\Gamma = \Delta$.
        %     By the IH applied to $\D{3}$ using $A \betared A^\prime$: $\Gamma \vdash A^\prime \cinfr K$.
        %     The \textsc{CtxApp} rule concludes the case.
        % \end{proofcase}

        % $\text{Case: }\begin{array}{c} \ContextReductionTailRule[*] \end{array}$
        % \begin{proofcase}
        %     By the IH $\vdash \Delta$ and $\Delta \vdash A \cinfr K$.
        %     Note that reduction does not introduce free variables, so $x \notin FV(\Delta)$.
        %     The \textsc{CtxApp} rule concludes the case.
        % \end{proofcase}
    \end{proofcase}
\end{proof}