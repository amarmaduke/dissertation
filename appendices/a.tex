\chapter{Proof of Confluence}
\label{ap:a}


\Rule{\ParBinderReduction}
    {
        \der{\D{1}}{t_1 \parred t_1^\prime} \\
        \der{\D{2}}{t_2 \parred t_2^\prime}
    }
    {\mathfrak{b}(\kappa, x : t_1, t_2) \parred \mathfrak{b}(\kappa, x : t_1^\prime, t_2^\prime) }
    {ParBind}

\Rule{\ParConstructorReduction}
    {
        \der{\D{i}}{t_i \parred t_i^\prime \quad \forall\ i \in \{1, \ldots, \mathfrak{a}(\kappa) \}}
    }
    {\mathfrak{c}(\kappa, t_1, \ldots, t_i, \ldots, t_{\mathfrak{a}(\kappa)}) \parred \mathfrak{c}(\kappa, t_1^\prime, \ldots, t_i^\prime, \ldots, t^\prime_{\mathfrak{a}(\kappa)}) }
    {ParCtor}

\Rule{\ParBetaReduction}
    {
        \der{\D{1}}{t_1 \parred t_1^\prime} \\
        \der{\D{2}}{t_2 \parred t_2^\prime} \\
        \der{\D{3}}{t_3 \parred t_3^\prime}
    }
    { \app{(\abs{\lambda_m}{x}{t_1}{t_2})}{m}{t_3} \parred [x := t_3^\prime]t_2^\prime }
    {ParBeta}

\Rule{\ParFstReduction}
    {
        \der{\D{1}}{t_1 \parred t_1^\prime} \\
        \der{\D{2}}{t_2 \parred t_2^\prime} \\
        \der{\D{3}}{t_3 \parred t_3^\prime}
    }
    { [t_1, t_2; t_3].1 \parred t_1^\prime }
    {ParFst}

\Rule{\ParSndReduction}
    {
        \der{\D{1}}{t_1 \parred t_1^\prime} \\
        \der{\D{2}}{t_2 \parred t_2^\prime} \\
        \der{\D{3}}{t_3 \parred t_3^\prime}
    }
    { [t_1, t_2; t_3].2 \parred t_2^\prime }
    {ParSnd}

\Rule{\ParPrmFstReduction}
    {
        \der{\D{1}}{t_1 \parred t_1^\prime} \\
        \der{\D{2}}{t_2 \parred t_2^\prime} \\
        \der{\D{3}}{t_3 \parred t_3^\prime}
    }
    { \vartheta_1(\text{refl}(t_1), t_2, t_3) \parred \text{refl}(t_2^\prime) }
    {ParPrmFst}

\Rule{\ParPrmSndReduction}
    {
        \der{\D{1}}{t_1 \parred t_1^\prime} \\
        \der{\D{2}}{t_2 \parred t_2^\prime} \\
        \der{\D{3}}{t_3 \parred t_3^\prime}
    }
    { \vartheta_2(\text{refl}(t_1), t_2, t_3) \parred \text{refl}(t_2^\prime) }
    {ParPrmSnd}

\Rule{\ParSubstReduction}
    {
        \der{\D{1}}{t_1 \parred t_1^\prime} \\
        \der{\D{2}}{t_2 \parred t_2^\prime}
    }
    { \psi(\text{refl}(t_1), t_2) \parred \abs{\lambda_\omega}{x}{\app{t_2^\prime}{\tau}{t_1^\prime}}{x}}
    {ParSubst}

\Rule{\ParVarReduction}
    {\color{white}{\_}}
    { x \parred x}
    {ParVar}

\begin{figure}
    \centering
    $$\ParVarReduction$$
    $$\ParConstructorReduction$$
    $$\ParBinderReduction$$
    $$\ParBetaReduction$$
    $$\ParSubstReduction$$
    $$\ParFstReduction$$
    $$\ParSndReduction$$
    $$\ParPrmFstReduction$$
    $$\ParPrmSndReduction$$
    \caption{Parallel reduction rules for arbitrary syntax.}
\end{figure}



\begin{figure}
    \centering
    \begin{align*}
        \parp{\app{(\abs{\lambda_m}{x}{t_1}{t_2})}{m}{t_3}} &= [x := \parp{t_3}]\parp{t_2} \\
        \parp{\app{\psi(\text{refl}(t_1; t_2), t_3, t_4; t_5, t_6)}{\omega}{t_7}} &= \parp{t_7} \\
        \parp{[t_1, t_2; t_3].1} &= \parp{t_1} \\
        \parp{[t_1, t_2; t_3].2} &= \parp{t_2} \\
        \parp{\vartheta(\text{refl}(t_1; t_2), t_3, t_4; t_5)} &= \text{refl}(\parp{t_3}; \parp{t_5}) \\
        \parp{\mathfrak{c}(\kappa, t_1, \ldots, t_{\mathfrak{a}(\kappa)})} &= \mathfrak{c}(\kappa, \parp{t_1}, \ldots, \parp{t_{\mathfrak{a}(\kappa)}}) \\
        \parp{\mathfrak{b}(\kappa, (x : t_1), t_2)} &= \mathfrak{b}(\kappa, (x : \parp{t_1}), \parp{t_2}) \\
        \parp{x_K} &= x_K
    \end{align*}
    \caption{
        Definition of a reduction completion function $\parp{-}$ for parallel reduction.
        Note that this function is defined by pattern matching, applying cases from top to bottom.
        Thus, the cases at the very bottom are catch-all for when the prior cases are not applicable.
    }
    \label{fig:par-triangle}
\end{figure}


\begin{lemma}
    For any $t$, $t \parred t$
    \label{lem:a:par_refl}
\end{lemma}
\begin{proof}
    Straightforward by induction on $t$.
\end{proof}

\begin{lemma}
    If $s \betared t$ then $s \parred t$
    \label{lem:a:beta_implies_par_step}
\end{lemma}
\begin{proof}
    By induction on $s \betared t$.

    $\text{Case: }\begin{array}{c} \app{(\abs{\lambda_m}{x}{A}{b})}{m}{t} \betared [x := t]b \end{array}$
    \begin{proofcase}
        By Lemma~\ref{lem:a:par_refl}: $t \parred t$ and $b \parred b$.
        Applying the \textsc{ParBeta} rule concludes the case.
    \end{proofcase}

    $\text{Case: }\begin{array}{c} [t_1, t_2; A].1 \betared t_1 \end{array}$
    \begin{proofcase}
        As above, $t_1 \parred t_1$, applying the \textsc{ParFst} rule concludes the case.
    \end{proofcase}

    $\text{Case: }\begin{array}{c} [t_1, t_2; A].2 \betared t_2 \end{array}$
    \begin{proofcase}
        Same as previous case using \textsc{ParSnd} rule.
    \end{proofcase}

    $\text{Case: }\begin{array}{c} \psi(\text{refl}(t),P) \betared \abs{\lambda_\omega}{x}{\app{P}{\tau}{t}}{x} \end{array}$
    \begin{proofcase}
        Using Lemma~\ref{lem:a:par_refl}: $t \parred t$ and $P \parred P$.
        Applying the \textsc{ParSubst} rule concludes the case.
    \end{proofcase}

    $\text{Case: }\begin{array}{c} \vartheta_1(\text{refl}(t_1), t_2, t_3) \betared \text{refl}(t_2) \end{array}$
    \begin{proofcase}
        As with previous cases, $t_2 \parred t_2$.
        Applying the \textsc{ParPrmFst} rule concludes the case.
    \end{proofcase}

    $\text{Case: }\begin{array}{c} \vartheta_2(\text{refl}(t_1), t_2, t_3) \betared \text{refl}(t_2) \end{array}$
    \begin{proofcase}
        Same as above using \textsc{ParPrmSnd}.
    \end{proofcase}

    $\text{Case: }\begin{array}{c} \NonbinderReduction[*] \end{array}$
    \begin{proofcase}
        By the IH applied to $\D{1}$: $t_i \parred t_i^\prime$.
        Note that there is only one subderivation.
        For all $j \neq i$ $t_j \parred t_j$ by Lemma~\ref{lem:a:par_refl}.
        Using the \textsc{ParCtor} rule concludes the case.
    \end{proofcase}

    $\text{Case: }\begin{array}{c} \BinderReductionOne[*] \end{array}$
    \begin{proofcase}
        Applying the IH to $\D{1}$ yields $t_1 \parred t_1^\prime$.
        By Lemma~\ref{lem:a:par_refl}: $t_2 \parred t_2$.
        Using the \textsc{ParBind} rule concludes the case.
    \end{proofcase}

    $\text{Case: }\begin{array}{c} \BinderReductionTwo[*] \end{array}$
    \begin{proofcase}
        Similar to previous case.
    \end{proofcase}
\end{proof}

\begin{lemma}
    If $s \betastar t$ then $s \parstar t$
    \label{lem:a:beta_implies_par}
\end{lemma}
\begin{proof}
    By induction on $s \betastar t$ applying Lemma~\ref{lem:a:beta_implies_par_step} in the inductive case.
\end{proof}

\begin{lemma}
    If $s \parred t$ then $s \betastar t$
    \label{lem:a:par_implies_beta_step}
\end{lemma}
\begin{proof}
    By induction on $s \parred t$.

    $\text{Case: }\begin{array}{c} \ParVarReduction[*] \end{array}$
    \begin{proofcase}
        By reflexivity of reduction.
    \end{proofcase}

    $\text{Case: }\begin{array}{c} \ParConstructorReduction[*] \end{array}$
    \begin{proofcase}
        By the IH applied to each $\D{i}$: $t_i \betastar t_i^\prime$ for all $i$.
        Applying Lemma~\ref{lem:2:beta_par} concludes the case.
    \end{proofcase}

    $\text{Case: }\begin{array}{c} \ParBinderReduction[*] \end{array}$
    \begin{proofcase}
        As the previous case, the IH yields $t_1 \betastar t_1$ and $t_2 \betastar t_2^\prime$.
        Again using Lemma~\ref{lem:2:beta_par} concludes the case.
    \end{proofcase}

    $\text{Case: }\begin{array}{c} \ParBetaReduction[*] \end{array}$
    \begin{proofcase}
        Applying the IH to all available derivations and using Lemma~\ref{lem:2:beta_par} gives $\app{(\abs{\lambda_m}{x}{t_1}{t_2})}{m}{t_3} \betastar \app{(\abs{\lambda_m}{x}{t_1^\prime}{t_2^\prime})}{m}{t_3^\prime}$.
        Applying the beta rule of reduction with transitivity concludes the case.
    \end{proofcase}

    $\text{Case: }\begin{array}{c} \ParSubstReduction[*] \end{array}$
    \begin{proofcase}
        Same as the previous case but using the substitution rule.
    \end{proofcase}

    $\text{Case: }\begin{array}{c} \ParFstReduction[*] \end{array}$
    \begin{proofcase}
        Same as the previous case but using the first rule.
    \end{proofcase}

    $\text{Case: }\begin{array}{c} \ParSndReduction[*] \end{array}$
    \begin{proofcase}
        Same as the previous case but using the second rule.
    \end{proofcase}

    $\text{Case: }\begin{array}{c} \ParPrmFstReduction[*] \end{array}$
    \begin{proofcase}
        Same as the previous case but using the first promotion rule.
    \end{proofcase}

    $\text{Case: }\begin{array}{c} \ParPrmSndReduction[*] \end{array}$
    \begin{proofcase}
        Same as the previous case but using the second promotion rule.
    \end{proofcase}
\end{proof}

\begin{lemma}
    If $s \parstar t$ then $s \betastar t$
    \label{lem:a:par_implies_beta}
\end{lemma}
\begin{proof}
    By induction on $s \parstar t$ applying Lemma~\ref{lem:a:par_implies_beta_step} in the inductive case.
\end{proof}

\begin{lemma}
    If $s \parred s^\prime$ and $t \parred t^\prime$ then $[x := s]t \parred [x := s^\prime]t^\prime$
    \label{lem:a:par_subst}
\end{lemma}
\begin{proof}
    By induction on $t \parred t^\prime$.

    $\text{Case: }\begin{array}{c} \ParVarReduction[*] \end{array}$
    \begin{proofcase}
        Rename to $y$.
        If $x = y$ then $s \parred s^\prime$ which is a premise.
        If $x \neq y$ then no substitution is performed and $y \parred y$.
    \end{proofcase}

    $\text{Case: }\begin{array}{c} \ParConstructorReduction[*] \end{array}$
    \begin{proofcase}
        Applying the IH to $\D{i}$ yields $[x := s]t_i \parred [x := s^\prime]t_i^\prime$ for all $i$.
        Unfolding substitution for $\mathfrak{c}$ and applying the \textsc{ParCtor} rule concludes the case.
    \end{proofcase}

    $\text{Case: }\begin{array}{c} \ParBinderReduction[*] \end{array}$
    \begin{proofcase}
        As above the IH gives $[x := s]t_i \parred [x := s^\prime]t_i^\prime$ for $i = 1$ and $i = 2$.
        Unfolding substitution for $\mathfrak{b}$ and applying the \textsc{ParBind} rule concludes.
    \end{proofcase}

    $\text{Case: }\begin{array}{c} \ParBetaReduction[*] \end{array}$
    \begin{proofcase}
        By the IH: $[x := s]t_i \parred [x := s^\prime]t_i^\prime$ for $i = 1, 2, 3$.
        The \textsc{ParBeta} rule gives the following:
        $[x := s]\app{(\abs{\lambda_m}{y}{t_1}{t_2})}{m}{t_3} = \app{(\abs{\lambda_m}{y}{[x := s]t_1}{[x := s]t_2})}{m}{[x := s]t_3} \parred [y := t_3^\prime][x := s^\prime]t_2^\prime$.
        Note that $y$ is bound and thus not a free variable in $s^\prime$ and, moreover, by implicit renaming $x \neq y$.
        Thus, by Lemma~\ref{lem:2:subst_commute} $[y := t_3^\prime][x := s^\prime]t_2^\prime = [x := s^\prime][y := t_3^\prime]t_2^\prime$.
    \end{proofcase}

    $\text{Case: }\begin{array}{c} \ParSubstReduction[*] \end{array}$
    \begin{proofcase}
        By the IH: $[x := s]t_i \parred [x := s^\prime]t_i^\prime$ for $i = 1, 2$.
        The \textsc{ParSubst} rule gives:
        $[x := s](\psi(\text{refl}(t_1), t_2)) = \psi(\text{refl}([x := s]t_1), t_2) \parred \abs{\lambda_\omega}{x}{\app{[x := s^\prime]t_2^\prime}{\tau}{[x := s^\prime]t_1^\prime}}{x} = [x := s^\prime]\abs{\lambda_\omega}{x}{\app{t_2^\prime}{\tau}{t_1^\prime}}{x}$.
    \end{proofcase}

    $\text{Case: }\begin{array}{c} \ParFstReduction[*] \end{array}$
    \begin{proofcase}
        By the IH: $[x := s]t_i \parred [x := s^\prime]t_i^\prime$ for $i = 1, 2, 3$.
        The \textsc{ParFst} rule gives:
        $[x := s][t_1, t_2; t_3].1 = [[x := s]t_1, [x := s]t_2; [x := s]t_3].1 \parred [x := s^\prime]t_1^\prime$.
    \end{proofcase}

    $\text{Case: }\begin{array}{c} \ParSndReduction[*] \end{array}$
    \begin{proofcase}
        Similar to previous case.
    \end{proofcase}

    $\text{Case: }\begin{array}{c} \ParPrmFstReduction[*] \end{array}$
    \begin{proofcase}
        By the IH: $[x := s]t_i \parred [x := s^\prime]t_i^\prime$ for $i = 1, 2, 3$.
        The \textsc{ParFst} rule gives:
        $[x := s]\vartheta_1(\text{refl}(t_1), t_2, t_3) = \vartheta_1(\text{refl}([x := s]t_1), [x := s]t_2, [x := s]t_3) \parred \text{refl}([x := s^\prime]t_2^\prime) = [x := s^\prime]\text{refl}(t_2^\prime)$.
    \end{proofcase}

    $\text{Case: }\begin{array}{c} \ParPrmSndReduction[*] \end{array}$
    \begin{proofcase}
        Similar to previous case.
    \end{proofcase}
\end{proof}

\begin{lemma}[Parallel Triangle]
    If $s \parred t$ then $t \parred \parp{s}$
    \label{lem:a:par_triangle}
\end{lemma}
\begin{proof}
    By induction on $s \parred t$.

    $\text{Case: }\begin{array}{c} \ParVarReduction[*] \end{array}$
    \begin{proofcase}
        Have $\parp{x} = x$.
        Thus, this case is trivial.
    \end{proofcase}

    $\text{Case: }\begin{array}{c} \ParConstructorReduction[*] \end{array}$
    \begin{proofcase}
        By the IH applied to $\D{i}$: $t_i^\prime \parred \parp{t_i}$ for all $i$.
        Proceed by cases of $\parp{\mathfrak{c}(\kappa, t_1, \ldots t_{\mathfrak{a}(\kappa)})}$.

        $\text{Case: }\begin{array}{c} \parp{\app{(\abs{\lambda_m}{x}{t_1}{t_2})}{m}{t_3}} = [x := \parp{t_3}]\parp{t_2} \end{array}$
        \begin{proofcase}
            Note that $\mathfrak{c}(\kappa, t_1^\prime, \ldots t^\prime_{\mathfrak{a}(\kappa)}) = \app{(\abs{\lambda_m}{x}{t_1^\prime}{t_2^\prime})}{m}{t_3^\prime}$.
            Using the \textsc{ParBeta} rule yields $\app{(\abs{\lambda_m}{x}{t_1^\prime}{t_2^\prime})}{m}{t_3^\prime} \parred [x := \parp{t_3}]\parp{t_2}$.
        \end{proofcase}

        $\text{Case: }\begin{array}{c} \parp{\psi(\text{refl}(t_1), t_2)} = \abs{\lambda_\omega}{x}{\app{\parp{t_2}}{\tau}{\parp{t_1}}}{x} \end{array}$
        \begin{proofcase}
            Note that $\mathfrak{c}(\kappa, t_1^\prime, \ldots t^\prime_{\mathfrak{a}(\kappa)}) = \psi(\text{refl}(t_1^\prime), t_2^\prime)$.
            Using the \textsc{ParSubst} rule yields $\psi(\text{refl}(t_1^\prime), t_2^\prime) \parred \abs{\lambda_\omega}{x}{\app{\parp{t_2}}{\tau}{\parp{t_1}}}{x}$.
        \end{proofcase}

        $\text{Case: }\begin{array}{c} \parp{[t_1, t_2; t_3].1} = \parp{t_1} \end{array}$
        \begin{proofcase}
            Note that $\mathfrak{c}(\kappa, t_1^\prime, \ldots t^\prime_{\mathfrak{a}(\kappa)}) = [t_1^\prime, t_2^\prime; t_3^\prime].1$.
            Using the \textsc{ParFst} rule yields $[t_1^\prime, t_2^\prime; t_3^\prime].1 \parred \parp{t_1}$.
        \end{proofcase}

        $\text{Case: }\begin{array}{c} \parp{[t_1, t_2; t_3].2} = \parp{t_2} \end{array}$
        \begin{proofcase}
            Similar to previous case.
        \end{proofcase}

        $\text{Case: }\begin{array}{c} \parp{\vartheta_1(\text{refl}(t_1), t_2, t_3)} = \parp{t_2} \end{array}$
        \begin{proofcase}
            Note that $\mathfrak{c}(\kappa, t_1^\prime, \ldots t^\prime_{\mathfrak{a}(\kappa)}) = \vartheta_1(\text{refl}(t_1^\prime), t_2^\prime, t_3^\prime)$.
            Using the \textsc{ParPrmFst} rule yields $\vartheta_1(\text{refl}(t_1^\prime), t_2^\prime, t_3^\prime) \parred \parp{t_2}$.
        \end{proofcase}

        $\text{Case: }\begin{array}{c} \parp{\vartheta_2(\text{refl}(t_1), t_2, t_3)} = \parp{t_2} \end{array}$
        \begin{proofcase}
            Similar to previous case.
        \end{proofcase}

        $\text{Case: }\begin{array}{c} \parp{\mathfrak{c}(\kappa, t_1, \ldots t_{\mathfrak{a}(\kappa)})} = \mathfrak{c}(\kappa, \parp{t_1}, \ldots \parp{t_{\mathfrak{a}(\kappa)}}) \end{array}$
        \begin{proofcase}
            Using the \textsc{ParCtor} rule concludes the case.
        \end{proofcase}
    \end{proofcase}

    $\text{Case: }\begin{array}{c} \ParBinderReduction[*] \end{array}$
    \begin{proofcase}
        Note that $\parp{\mathfrak{b}(\kappa, (x : t_1), t_2)} = \mathfrak{b}(\kappa, (x : \parp{t_1}), \parp{t_2})$.
        By the IH applied to $\D{i}$: $t_i^\prime \parred \parp{t_i}$ for $i = 1, 2$.
        Thus, by the \textsc{ParBind} rule $\mathfrak{b}(\kappa, (x : t_1^\prime), t_2^\prime) \parred \mathfrak{b}(\kappa, (x : \parp{t_1}), \parp{t_2})$.
    \end{proofcase}

    $\text{Case: }\begin{array}{c} \ParBetaReduction[*] \end{array}$
    \begin{proofcase}
        Note that $\parp{\app{(\abs{\lambda_m}{x}{t_1}{t_2})}{m}{t_3}} = [x := \parp{t_3}]\parp{t_2}$.
        By the IH applied to $\D{i}$: $t_i^\prime \parred \parp{t_i}$ for $i = 1, 2, 3$.
        Thus, by Lemma~\ref{lem:a:par_subst} $[x := t_3^\prime]t_2^\prime \parred [x := \parp{t_3}]\parp{t_2}$.
    \end{proofcase}

    $\text{Case: }\begin{array}{c} \ParSubstReduction[*] \end{array}$
    \begin{proofcase}
        Note that $\parp{\psi(\text{refl}(t_1), t_2)} = \abs{\lambda_\omega}{x}{\app{\parp{t_2}}{\tau}{\parp{t_1}}}{x}$.
        By the IH applied to $\D{i}$: $t_i^\prime \parred \parp{t_i}$ for $i = 1, 2$.
        Applying the \textsc{ParBind} rule yields $\abs{\lambda_\omega}{x}{\app{t_2^\prime}{\tau}{t_1^\prime}}{x} \parred \abs{\lambda_\omega}{x}{\app{\parp{t_2}}{\tau}{\parp{t_1}}}{x}$.
    \end{proofcase}

    $\text{Case: }\begin{array}{c} \ParFstReduction[*] \end{array}$
    \begin{proofcase}
        Note that $\parp{[t_1, t_2; t_3].1} = \parp{t_1}$.
        By the IH applied to $\D{i}$: $t_i^\prime \parred \parp{t_i}$ for $i = 1, 2, 3$.
        Thus, $t_1^\prime \parred \parp{t_1}$.
    \end{proofcase}

    $\text{Case: }\begin{array}{c} \ParSndReduction[*] \end{array}$
    \begin{proofcase}
        Similar to previous case.
    \end{proofcase}

    $\text{Case: }\begin{array}{c} \ParPrmFstReduction[*] \end{array}$
    \begin{proofcase}
        Note that $\parp{\vartheta_1(\text{refl}(t_1), t_2, t_3)} \parred \parp{t_2}$.
        By the IH applied to $\D{i}$: $t_i^\prime \parred \parp{t_i}$ for $i = 1, 2, 3$.
        Thus, $t_2^\prime \parred \parp{t_2}$.
    \end{proofcase}

    $\text{Case: }\begin{array}{c} \ParPrmSndReduction[*] \end{array}$
    \begin{proofcase}
        Similar to previous case.
    \end{proofcase}
\end{proof}

\begin{lemma}[Paralell Strip]
    If $s \parred t_1$ and $s \parstar t_2$ then $\exists\ t$ such that $t_1 \parstar t$ and $t_2 \parred t$
    \label{lem:a:par_strip}
\end{lemma}
\begin{proof}
    By induction on $s \parstar t_2$, pick $t = t_1$ for the reflexivity case.
    Consider the transitivity case, $\exists\ z_1$ such that $s \parred z_1$ and $z_1 \parstar t_2$.
    Applying Lemma~\ref{lem:a:par_triangle} to $s \parred z_1$ yields $z_1 \parred \parp{s}$.
    By the IH with $z_1 \parred \parp{s}$: $\exists\ z_2$ such that $\parp{s} \parstar z_2$ and $t_2 \parred z_2$.
    Using Lemma~\ref{lem:a:par_triangle} again on $s \parred t_1$ yields $t_1 \parred \parp{s}$.
    Now by transitivity $t_1 \parstar z_2$.
\end{proof}

\begin{lemma}[Parallel Confluence]
    If $s \parstar t_1$ and $s \parstar t_2$ then $\exists\ t$ such that $t_1 \parstar t$ and $t_2 \parstar t$
    \label{lem:a:par_confluence}
\end{lemma}
\begin{proof}
    By induction on $s \parstar t_1$, pick $t = t_2$ for the reflexivity case.
    Consider the transitivity case, $\exists\ z_1$ such that $s \parred z_1$ and $z_1 \parstar t_1$.
    By Lemma~\ref{lem:a:par_strip} applied with $s \parred z_1$ and $s \parstar t_2$ yields $\exists\ z_2$ such that $z_1 \parstar z_2$ and $t_2 \parred z_2$.
    Using the IH with $z_1 \parred z_2$ gives $\exists\ z_3$ such that $t_1 \parstar z_3$ and $z_2 \parstar z_3$.
    By transitivity $t_2 \parstar z_3$.
\end{proof}

\begin{lemma}[Confluence]
    If $s \betastar t_1$ and $s \betastar t_2$ then $\exists\ t$ such that $t_1 \betastar t$ and $t_2 \betastar t$
\end{lemma}
\begin{proof}
    By Lemma~\ref{lem:a:beta_implies_par} applied twice: $s \parstar t_1$ and $s \parstar t_2$.
    Now by parallel confluence (Lemma~\ref{lem:a:par_confluence}) $\exists\ t$ such that $t_1 \parstar t$ and $t_2 \parstar t$.
    Finally, two applications of Lemma~\ref{lem:a:par_implies_beta} conclude the proof.
\end{proof}
